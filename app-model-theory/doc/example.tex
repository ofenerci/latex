\chapter{\docTitle}
%
\docepigraph{The sciences do not try to explain, they hardly even try to interpret, they mainly make models. By a model is meant a mathematical construct which, with the addition of certain verbal interpretations, describes observed phenomena. The justification of such a mathematical construct is solely and precisely that it is expected to work.}{von Neumann}{\citep{vonneumann:wikiquote}}
%
To illustrate various problem solving techniques, we
%
\mquote{Since I like to keep things informal, if I say \scare{we} in the text, I really mean you and I.}{\cite[p. 2]{gleich:2005}}
%
will analyze the motion of a charged particle using Newtonian Physics. We will do so by showing various math and physics methods in different levels of sophistication: guessing, dimensional analysis, approximations and analytic techniques. Finally, we present a final wrapped-up solution.

one has to assume, derive and test the model. The issue is strengthened in design.


\section{Problem statement}
%
In professional work, seldom does one find a \scare{well-posed problem}, where all the problem data, the information regarding...  one has to work towards that goal: to well pose a problem (well posed problems have more chances to find a solution).

However, herein we will not be concerned by posing a problem, we will rather take one from a book (the reference exercise) and illustrate the ideas on how to solve problems by applying different techniques to solve such an exercise.
%
\mquote{He who seeks for methods without having a definite problem in mind seeks in the most part in vain.}{\cite{hilbert:quotes}}
%


\subsection{Reference exercise}\label{sec:referenceexercise}
%
As a \lingo{reference exercise}, I chose a nice one presented in the \book{Particle Kinetics and Lorentz Force in Geometric Language} section in \cite[chap. 1, p. 8]{thorne:2011}. There, geometric ideas, \via the \fact{geometric principle},
%
~\mnote{The laws of physics must all be expressible as geometric [\dots] relationships between geometric objects [\dots], which represent physical entities. \cite[part I, p. iii]{thorne:2013}}
%
are applied to Newtonian physics.

Now, to save your time for finding it, I quote the exercise verbatim:
%
\begin{description}
%
\item[Energy change for charged particle] \theorem{Without introducing any coordinates or basis vectors, show that, when a particle with charge $q$ interacts with electric and magnetic fields, its energy changes at a rate}
%
\begin{equation}\label{eq:referenceequation}
  dE/dt = \mathbf{v}\iprod\mathbf{E}\,.
\end{equation}
%
\end{description}
%
In \cref{eq:referenceequation}, $E$ represents the particle's kinetic energy, $t$ (Newton's) universal time, $\mathbf{v}$ the particle's velocity and $\mathbf{E}$ an electric field.


\subsection{Reference exercise analysis}\label{sec:referenceanalysis}
%
In the reference exercise, we are asked to derive (match) a given formula. As a healthy advice, always check if a formula, specially one to match, is correct (in this case, derivable). But, how to know if a formula is correct without a formal derivation? Catch-22! Well, not really. We have a simple (but powerful) method to analyze formula correctness without the need of long computations: \lingo{dimensional analysis}
%
\mquote{Upon seeing any equation, first check its dimensions [\dots]. If all terms do not have identical dimensions, the equation is not worth solving -- a great savings of effort.}{\cite[p. 42]{sanjoy:2010}}
%
-- in a correct equation, all of its terms have the same dimensions. Let's see if \cref{eq:referenceequation} passes this test.

Since we are dealing with electrodynamics, we choose the dimensions of force $\phdim F$, length $\phdim L$, electric charge $\phdim Q$ and time $\phdim T$ as base dimensions for the analysis. Then, for \cref{eq:referenceequation}, we have that
%
\begin{align*}
%
  \dfrac{\dim dE}{\dim dt} &= \dfrac{\phdim{FL}}{\phdim{T}} & \eqtxt{LHS of \cref{eq:referenceequation}} \\
  %
  &\not= & \\
  %
  \dim{\mathbf{v}}\iprod\dim{\mathbf{E}} &= \dfrac{\phdim{FL}}{\phdim{QT}}\,. & \eqtxt{RHS of \cref{eq:referenceequation}}
%
\end{align*}
%
Note the additional $\phdim Q$ in the RHS of \cref{eq:referenceequation} (or the lack thereof in the LHS). Dimensions do not match, thus the formula is false! Then, we could well stop here and move on. But, we are a bit curious: what went wrong? Misprint, mistype or bad derivation are possible causes. But, as an error pointer, we realize that in the exercise statement the particle's electric charge $q$ is given and a magnetic field is mentioned; however, neither appear in \cref{eq:referenceequation}. Let's use this observation to guess that
%
\begin{equation}\label{eq:plausibleequation}
  dE/dt = q\mathbf{v}\iprod\mathbf{E}\,.
\end{equation}
%
The last equation is dimensionally homogeneous and, thus, plausible.


\subsection{Reference exercise reformulation}
%
In the last section, we briefly analyzed the reference exercise, found out its conclusion is incorrect and guessed a plausible formula. But, we are far from the end: a plausible equation is not our final answer.

The next step is to reform the exercise statement itself. I think the reference exercise aim was not to asked us to \scare{show} a wrong conclusion. We replace then the \emph{show} bit for \emph{find}. In this way, we do not have to worry about any formula to match. We will have to derive one. Let's try:
%
\begin{description}
%
\item[Energy change for charged particle] \theorem{Without introducing any coordinates or basis vectors, find the energy change rate $dE/dt$ of a particle with charge $q$ when it interacts with an electric field $\mathbf{E}$ and a magnetic field.}
%
\end{description}
%


\subsection{Working exercise}\label{sec:workingexercise}
%
The reference exercise was analyzed and reformulated to avoid its wrong conclusion. We could work with the reformulated reference exercise. However, there are some additional changes I would like to make before having a \emph{working exercise}:
%
\begin{itemize}
%
\item I will generalize the statement by relaxing hypotheses and removing data. Specifically, I will remove the recommendation of not using coordinates nor basis vectors;
%
~\mnote{although this is an important reminder of working with geometric objects rather than with coordinates, it limits generalization.}
%
will relax the \scare{particle} model by hypothesizing a \scare{body}, $\body$, instead; will replace the data \scare{charge $q$}, \scare{electric field $\mathbf{E}$} and \scare{magnetic field} with \scare{electrically charged}. These replacements will correct the lack of $q$ and the magnetic field in \cref{eq:referenceequation} -- perhaps they are not needed.
%
\item Notice that in the reference exercise the particle interacts with fields. However, it is not mentioned the \emph{agent} that creates the fields.
%
\mquote{[On Newton's laws] the laws fail to explicitly state that \emph{every force has an agent}, that every force is a binary function describing the action of an agent on an object.}{\cite{hestenes:1987}}
%
This is a serious omission that perpetuates the \lingo{impetus believe}: \scare{that a force can be imparted to an object and act on it independently of any agent} \cite{hestenes:1987}. It is particularly notorious in the reference exercise: we are told, in electromagnetic theories, that any charged particle creates electric and magnetic fields. So, is the $\mathbf{E}$ in the reference exercise due to the moving particle or to another one? We \emph{interpret} the statement as \scare{there is a particle, different from the moving one, that creates the electric and magnetic fields with which the moving particle interacts}. We correct this lack of agent by adding a second electrically charged body, $\body'$.
%
\item Finally, I like how mathematicians present propositions. They explicitly write the \lingo{premises} (or hypotheses) and the \lingo{conclusion} (or question) in a way that nothing is left to interpretation. In such a fashion, the problem becomes \lingo{self-contained}. Self-contained statements tend to sound a bit pedantic; but, the result pays off in understanding.
%
\end{itemize}

With these changes in mind, we present a working version of the exercise:
%
\begin{description}
%
\item[Charged body energy change] \theorem{Consider a massive, electrically charged body $\body$ moving toward an electrically charged body $\body'$. Then, find $\body$ temporal change of energy.}
%
\end{description}


\section{Working exercise solution}
%
Now we are ready to work on the exercise. But, wait! Without you noticing it, we already stared working on it. We started by applying two of my favorite methods: \lingo{dimensional analysis} and \lingo{abstraction}. 
%
% --------------------------------------------------------------- Figure
%
\begin{figure}[bt]
  \capstart
  \begin{center}
  \footnotesize
  %
    \Tree [.{how to handle complexity} 
            %
            [.{organizing complexity} 
              [.{abstraction} ] [.{divide and conquer} ] 
            ]
            %
            [.{discarding \emph{fake} complexity (symmetry)} 
              [.{dimensional analysis} ]
             ] 
           ]
  %
  \normalsize
  \end{center}
  \caption[Handling complexity]
    {Dimensional analysis and abstraction as methods to handle complexity when solving physics or math problems, adapted from \cite[p. 2]{sanjoy:2008}}
  \label{fig:handlingcomplexity}
\end{figure}
%
% ------------------------------------------------------------ EndFigure
%
% --------------------------------------------------------------- Figure
% pos.: bthH. size: width=0.5\textwidth. file:./graph/fname.pdf
% caption. label:fig:wec
%
%\docfigure{bt}{height=0.8\textheight}{./graph/complexity.pdf}{How to handle complexity}{Various techniques used to handle complexity when solving physics (or math) problems \cite[p. 2]{sanjoy:2008}. Two of them are used in the present text: dimensional analysis and abstraction.}{fig:handlingcomplexity}
%
% ------------------------------------------------------------ EndFigure
%
These methods are ideal to handle complexity, \vide \cref{fig:handlingcomplexity}. While dimensional analysis discards \scare{fake} complexity by compressing information, abstraction organizes it.

Dimensional analysis was briefly treated in \cref{sec:referenceanalysis} and will be shown more fully in \cref{sec:dimanalysis}. In the next section, I will explain abstraction.


\subsection{Abstraction}
%
As seen in \cref{sec:workingexercise},
\mquote{The problem statement should be very general and free of as much data as possible, as later stages in the modelling process will consider and gather what is needed.}{\cite[p. 8]{nsw:2000}}
%
abstraction is the process of relaxing hypotheses and removing data to leave only the bare bones of an statement. There are some reasons to do this:
%
\begin{enumerate}
%
\item a neater, crisper, easier to picture exercise statement, due to the lack of data and suggested notation -- \confer the reference exercise statement, \cref{sec:referenceexercise}, with the working exercise statement, \cref{sec:workingexercise};
%
\item as the statement becomes more abstract, keywords that show governing effects begin to emerge;
%
\item as the governing effects appear, theories can be proposed to model those effects. Then, we have to make assumptions and filled out data to satisfy theories frameworks. These latter steps engage us in a better understanding of the physics behind the model and its limitations.
%
\end{enumerate}

For instance, in the case of the working exercise, note the keywords \emph{electrically charged} and \emph{moving}. They point to a theory of electromagnetism and to a theory of motion: an electrodynamic theory. We can now choose any of the available ones. For motion, we could choose Newton's, Lagrange's, Hamilton's, Einstein's or quantum theories; for electromagnetism, classical, quantum, field theories and so on.
%
% --------------------------------------------------------------- Figure
%
% position: bthH. size:width=0.5\textwidth. file:location+filename.pdf
% caption. label:fig:wec
% use: \docfloatwidth whenever possible!
\docfigure{bt}{width=0.9\textwidth}{./graph/class-quantum.pdf}{Classical and quantum physics}
{The relationship of the three frameworks for classical physics (on right) to four frameworks for quantum physics (on left).         Each arrow indicates an approximation. All other frameworks are approximations to the ultimate laws of quantum gravity (whatever   they may be -- perhaps a variant of string theory). \cite[chap. 1, p. iv]{thorne:2013}}
{fig:classicalquantumphysics}
%
% ------------------------------------------------------------ EndFigure
%
%
% --------------------------------------------------------------- Figure
%
% position: bthH. size:width=0.5\textwidth. file:location+filename.pdf
% caption. label:fig:wec
% use: \docfloatwidth whenever possible!
%\docfigure{bt}{width=1.2\textwidth}{./graph/class-physics.pdf}{Classical physics}
%{The three frameworks and arenas for the classical laws of physics, and their relationship to each other. \cite[chap. 1, p. iii]{thorne:2013}}{fig:classicalphysics}
%
% ------------------------------------------------------------ EndFigure
%

For the present case, in order to retain the spirit of the reference exercise, we choose Newtonian physics as the main physical framework in which to work. (The relationships among several physical theories is presented in \cref{fig:classicalquantumphysics}.)


\subsection{Notation}
%
Often,
%
\mquote{A name is not the same as an explanation. Do not expect the structure of a name or symbol to tell you everything you need to know. Most of what you need to know belongs in the legend. The name or symbol should allow you to look up the explanation in the legend.}{\cite{denker:notation}}
%
scalars, vectors and other mathematical objects are typeset with different font faces for each math type. Although this convention works fine for printed texts, it posses issues when working on paper. See, for instance, that, in the reference exercise, $E$ is used to represent energy and $\mathbf{E}$ to represent electric field. Now, how to distinguish between $E$ and $\mathbf{E}$ with pen on paper without both e's getting confused?

The alternatives, then, are to decorate objects, like using arrows on top of letters for vectors -- $\vec E$ for electric field -- or to use majuscules and minuscules to distinguish objects. Herein I \emph{could} follow any of such conventions, but I am not going to. I do not like how arrows, bold typefaces or majuscules look like. Instead I will use different symbols for different quantities and minuscules to typeset variables; for instance, $\force$ would represent force, $\efield$ electric field, $\ekin$ kinetic energy and so forth. Even though this latter convention appears error prone, it constantly reminds me to be careful when working with mixed types of math objects, for I do not rely on typographical decoration anymore.

Finally,
%
\mquote{Typography exists to honor content.}{\cite[p. 17]{brinhurst:2004}}
%
in the writer's eyes, besides honoring math objects by quenching notation, this flat, undecorated typography seems to give equations an air of elegance and simplicity unmatchable by heavy decoration. Compare, for instance, the undecorated version of Newton's second law of motion 
%
\mquote{But in our opinion truths of this kind should be drawn from notions rather than from notations.}{\cite{gauss:wikiquote}}
%
\beq
  \force = \mass\acc\,,
\eeq
%
with $\vec F = m\vec a$ or $\mathbf{F} = m\mathbf{a}$, its decorated and bold counterparts.


\subsection{Adoption of physical framework -- model theory}\label{sec:physicalframework}
%
\via the model theory \cite{hestenes:1987}.
%
\mquote{Before developing the necessary mathematics, survey the crucial physics.}{\cite[p. 11]{lindner:2011}}
%
\begin{description}
%
\item[Theory] Newtonian electrodynamics: (Newtonian physics)
%
\item[Object] body $\body$ modeled as a moving particle: (state variables, object variables, ...) with charge $\echarge$ and mass $\mass$.
%
\item[Agent] body $\body'$ modeled as a stationary particle.
%
\item[Dynamic laws]
%
\begin{align}
  %
  \mom   &= \mass\vel   & \eqtxt{def. momentum} \\
  \force &= \dt\mom     & \eqtxt{Newton's second law} \\
  2\ekin &= \mass\vel^2 & \eqtxt{def. kinetic energy}
  %
\end{align}
%
\item[Interaction laws]
%
\begin{align}
  \force' &= \echarge\parth{\efield' + \vel\cprod\mfield'} & \eqtxt{Lorentz force}
\end{align}
%
\item[Interpretation] ...
%
\item[QED]
%
\end{description}

Notice that interpretation forms a part of the proof!


\subsection{Approximate solution}\label{sec:approxsolution}
%
Now that we have the physical framework in place, it is time to mathematically model the phenomenon using laws, definitions and theorems from the framework. However, we will not use the full-blown formulas; we will use approximations instead.
%
\mquote{Too much mathematical rigor teaches \lingform{rigor mortis}: the fear of making an unjustified leap even when it lands on a correct result. Instead of paralysis, have courage -- shoot first and ask questions later. Although unwise as public policy, it is a valuable problem-solving philosophy.}{\cite[p. viii]{sanjoy:2010}}
%
% --------------------------------------------------------------- Figure
%
\begin{figure}[bt]
  \capstart
  \begin{center}
  \footnotesize
  %
    \Tree [.{how to handle complexity} 
            %
            [.{discarding \emph{fake} complexity (symmetry)} 
              [.{dimensional analysis} ]
            ] 
            [.{discarding \emph{actual} complexity} 
              [.{approximation} ] 
            ]
           ]
  %
  \normalsize
  \end{center}
  \caption[Handling complexity, again]
    {Approximation as methods to handle complexity when solving physics or math problems, adapted from \cite[p. 2]{sanjoy:2008}}
  \label{fig:handlingcomplexitytwo}
\end{figure}
%
% ------------------------------------------------------------ EndFigure

We do this because we do not want to be distracted by fancy math, detailed calculations, extra accurate results, unnecessary math factors (like $\tau$ or $\pi$) in our first contact with the model formulation, \vide \cref{fig:handlingcomplexitytwo}. We want understanding first, then we polish the model little by little; \ie, we will firstly focus on estimating the backbone effects influencing the phenomenon. There are some recommended estimations available \cite{francis:1999}:
%
\begin{itemize}
%
\item discarding unnecessary factors;
%
\item number guessing;
%
\item geometry tinkering (everything is a cube or a sphere);
%
\item usage of ratios;
%
\item usage of conservation laws;
%
\item dimensional analysis;
%
\item plausibility checks.
%
\end{itemize}
%
In our case, we will mainly discard unnecessary factors and use the \lingo{secant method} for approximating derivatives \cite[p. 38]{sanjoy:2010}.

The secant method for approximating derivatives consists in replacing derivatives by divisions:
%
\beq
  \dfrac{\dx\psi}{\dx x} \sim \dfrac{\psi}{x}\,,
\eeq
%
where $\psi$ is a function whose derivative with respect to $x$ exists and $\sim$ means \lingo{is similar to}.

Discarding factors and approximating derivatives by the secant can be illustrated by estimating the kinetic energy temporal change of a moving particle:
%
\beq
  \ekin = \dfrac{1}{2}\mass\vel^2
        \sim\mass\vel^2
  \implies
  \dt\ekin \sim \dfrac{\ekin}{t} 
           \sim \dfrac{\mass\vel^2}{t}\,,
\eeq
%
where we have discarded the factor of $1/2$ and approximated the kinetic energy time derivative.

Working back on the exercise, first, we write the set of equations obtained in \cref{sec:physicalframework}:
%
\begin{align*}
  2\ekin  &= \mass\vel^2\,,                                   &\eqtxt{kinetic energy} \\
  \force  &= \dt\mom = \mass\dt\vel\,.                        &\eqtxt{Newton's second law} \\
  \force' &= \echarge\parth{\efield' + \vel\cprod\mfield'}\,, &\eqtxt{Lorentz force}
\end{align*}
%
where unprimed quantities represent $\point$ (moving body -- \emph{object} -- modeled as particle) properties, while primed quantities $\point'$ (stationary body -- \emph{agent} -- modeled as particle) properties.

Then, we drop numeric factors, product between vectors and use the secant method to approximate derivatives to find
%
\begin{align*}
  \ekin    &\sim \mass\vel^2\,, \\
  \dt\ekin &\sim \ekin/t \sim \mass\vel^2/t \sim \parth{\mass\vel/t}\vel\,, \\
  \force   &\sim \mass\vel/t\,, \\
  \force'  &\sim \echarge\parth{\efield' + \vel\mfield'}\,.
\end{align*}

We begin the analysis by \lingo{dividing and conquering}, \vide \cref{fig:handlingcomplexity}, the last set. We first analyze the electric field effect, then the magnetic field effect on the particle motion and finally we join them both.

\subsubsection{Electric field effect}
%
To understand the electric field effect, we use a \lingo{tree diagram} that contains all the computational work with the quantities and the formulas governing the particle's motion, \vide \cref{fig:electricfieldeffect}.
%
% --------------------------------------------------------------- Figure
%
\begin{figure}[bt]
  \capstart
  \begin{center}
  \begingroup
  \footnotesize
  %
  \begin{tikzpicture}[grow'=up]
    %
    %       % conclusion
    \Tree [.{$\ekin/t\sim\echarge\efield'\vel$} 
            %
            % kin energy branch
            [.{$\ekin/t\sim\mass\vel\vel/t$} [.{$\ekin\sim\mass\vel\vel$} ] ] 
            %
            % forces branch
            [.{$\mass\vel\vel/t\sim\echarge\efield'\vel$} 
              [.{$\mass\vel/t\sim\echarge\efield'$} 
                  {$\force\sim\mass\vel/t$} 
                  {$\force'\sim\echarge\efield'$} 
                ] 
              ] 
            ] 
          ] 
  %
  \end{tikzpicture}
  \endgroup
  \end{center}
  \caption[Electric field effect]
    {Effect of electric field on particle $\point$}
  \label{fig:electricfieldeffect}
\end{figure}
%
% ------------------------------------------------------------ EndFigure

Atop \cref{fig:electricfieldeffect}, we wrote the three main dynamic laws governing the particle's motion: kinetic energy definition, Newton's second law of motion and Lorentz force law. On the LHS, we begun with the kinetic energy definition and then approximated its time derivative. On the RHS, we begun with Newton's second law and Lorentz force law, then equated them and multiplied the result by $\vel$ (to match the $\vel\vel$ in the LHS). Finally, we equated the LHS to the RHS to find that
%
\begin{equation}\label{eq:kinenergyelectricfield}
  \dt\ekin \sim \ekin/t 
           \sim \echarge\efield'\vel\,,
\end{equation}
%
which gives the effect of the electric field created by $\point'$ on $\point$ motion. This equation shows that $\dt\ekin$ depends neither on forces nor on masses. A fact we will use a bit later.

Notice that we were careful when writing vector multiplications, since there are some: inner product, cross product, outer product and geometric product. However, using this very same information, we can go a bit further with \cref{eq:kinenergyelectricfield}. We know that $\efield'$ and $\vel$ are vectors, so we need a product between them, and we also know that $\dt\ekin$ is a scalar. Thus, for the two sides of \cref{eq:kinenergyelectricfield} to agree, the only product between $\efield'$ and $\vel$ is the inner product, since is the only one that returns a scalar. With this, \cref{eq:kinenergyelectricfield} becomes:
%
\beq
  \dt\ekin \sim \echarge\efield'\iprod\vel\,,
\eeq
%
which agrees nicely with \cref{eq:plausibleequation}; the equation we guessed when correcting the reference exercise. 

It is worth to mention that \cref{eq:plausibleequation} and \cref{eq:kinenergyelectricfield} illustrate the power of dimensional analysis and approximations. With these two, plus some insights, we were able to independently derive meaningful expressions without the need of full-blow calculations! Plausible equations with little job done.


\subsubsection{Magnetic field effect}
%
To understand the magnetic field effect, we repeat the same methodology used in the previous section: a tree diagram, \vide \cref{fig:magneticfieldeffect}.
%
% --------------------------------------------------------------- Figure
%
\begin{figure}[bt]
  \capstart
  \begin{center}
  \begingroup
  \footnotesize
  %
  \begin{tikzpicture}[grow'=up]
    %
    %       % conclusion
    \Tree [.{$\ekin/t\sim\echarge\vel\mfield'\vel$} 
            %
            % kin energy branch
            [.{$\ekin/t\sim\mass\vel^2/t$} [.{$\ekin\sim\mass\vel^2$} ] ] 
            %
            % forces branch
            [.{$\mass\vel^2/t\sim\echarge\vel\mfield'\vel$} 
              [.{$\mass\vel/t\sim\echarge\vel\mfield'$} 
                  {$\force\sim\mass\vel/t$} 
                  {$\force'\sim\echarge\vel\mfield'$} 
                ] 
              ] 
            ] 
          ] 
  %
  \end{tikzpicture}
  \endgroup
  \end{center}
  \caption[Magnetic field effect]
    {Effect of magnetic field on particle $\point$}
  \label{fig:magneticfieldeffect}
\end{figure}
%
% ------------------------------------------------------------ EndFigure
%
The result is that
%
\beq
  \dt\ekin \sim \ekin/t 
           \sim \echarge\vel\mfield'\vel\,,
\eeq
%
which gives the effect of the electric field created by $\point'$ on $\point$ motion. This equation shows again that $\dt\ekin$ does not depend on $\force$, $\force'$ or $\mass$.

In this case, note the triple product of the vectors $\vel\mfield'\vel$. It can be traced back to the dynamic laws as
%
\beq
  \vel\mfield'\vel \sim \parth{\vel\cprod\mfield'}\iprod\vel\,.
\eeq
%
Again, the right hand side $\vel$ must enter as an inner product, since the LHS of the equation is a scalar, $\dt\ekin$, and the result of $\vel\cprod\mfield'$ is a vector. Using vector algebra, it is possible to show that the triple product vanishes. We could show it that way, but we prefer to argue geometrically. The product $\vel\cprod\mfield'$ returns a vector perpendicular to the plane formed by $\vel$ and $\mfield'$. Then, the triple product \emph{must} vanish, for the angle formed by the vector resulting from $\vel\cprod\mfield'$ and $\vel$ is
%
~\mnote{For convenience, define $\tau\defas 2\pi$. \cite{hartl:2010}}
%
$\tau/4$ or \SI{90}{\degree}. Thus,
%
\begin{equation}\label{eq:kinenergymagneticfield}
  \dt\ekin \sim 0\,.
\end{equation}

The last equation agrees with the physics of the phenomenon. Call $\force\txt m$ the force due to the magnetic field; \ie, the RHS term of Lorentz force: $\echarge\parth{\vel\cprod\mfield}$. Note that $\force\txt m$ is always perpendicular to both the $\vel$ and the $\mfield$ that created it -- mathematically expressed by the (cross) product $\vel\cprod\mfield$. Then, when a charged particle moves through the field, it traces an helical path in which the helix axis is parallel to the field and where $\vel$ remains constant. Because the magnetic force is \emph{always} perpendicular to the motion, then $\mfield$ can do \emph{no} work. No work, no change of kinetic energy. Thus, $\dt\ekin\sim 0$, or, accordingly, $\dt\ekin$ must not depend on $\mfield$. (Indeed, $\dt\ekin$ does not depend on $\mfield'$ \cite[p. 142]{tong:2013}.)

Finally, we see again that approximate methods and some physical insights yield meaningful results.


\subsubsection{Electromagnetic field effect}
%
Once we have understood a bit more on the physics of the phenomenon, we can join the electric and magnetic effects into an electromagnetic effect, \vide \cref{fig:electromagneticfieldeffect}.
%
% --------------------------------------------------------------- Figure
%
\begin{figure}[bt]
  \capstart
  \begin{center}
  \begingroup
  \footnotesize
  %
  \begin{tikzpicture}[grow'=up]
    %
    %       % conclusion
    \Tree [.{$\ekin/t\sim\echarge\parth{\efield' + \vel\mfield'}\vel$} 
            %
            % kin energy branch
            [.{$\ekin/t\sim\mass\vel^2/t$} [.{$\ekin\sim\mass\vel^2$} ] ] 
            %
            % forces branch
            [.{$\mass\vel^2/t\sim\echarge\parth{\efield' + \vel\mfield'}\vel$} 
              [.{$\mass\vel/t\sim\echarge\parth{\efield' + \vel\mfield'}$} 
                  {$\force\sim\mass\vel/t$} 
                  {$\force'\sim\echarge\parth{\efield' + \vel\mfield'}$} 
                ] 
              ] 
            ] 
          ] 
  %
  \end{tikzpicture}
  \endgroup
  \end{center}
  \caption[Electromagnetic effect]
    {Effect of electromagnetic field on particle $\point$}
  \label{fig:electromagneticfieldeffect}
\end{figure}
%
% ------------------------------------------------------------ EndFigure
%
The result is that
%
\beq
  \dt\ekin \sim \echarge\vel\iprod\efield'\,,
\eeq
%
which gives the effect of the electric field created by $\point'$ on $\point$ motion. This equation shows that $\dt\ekin$ does not depend on $\force$, $\force'$, $\mass$ or $\mfield'$.

One more thing. The last equation is fine, but we can do better. See that both sides have the same dimensions and that the inner product of $\efield$ and $\vel$, two vectors, is a scalar. Then, we can present the last formula as
%
\begin{equation}\label{eq:scaledkinenergyelectrofield}
  \dt\ekin/\echarge\efield\iprod\vel\sim 1\,.
\end{equation}

The presentation of \cref{eq:scaledkinenergyelectrofield} has two advantages: it is manifestly \lingo{scaled}
%
~\mnote{dimensionless and of order unity}
%
and it stresses the scalar character the solution.


\subsection{Dimensional analysis}\label{sec:dimanalysis}
%
We
%
\mquote{No investigator should allow himself to proceed to the detailed solution of a problem until he has made a dimensional analysis of the nature of the solution which will be obtained, and convinced himself by appeal to experiment that the points of view embodied in the underlying equations are sound.}{\cite[p. 88]{bridgman:1920}}
%
have already used dimensional analysis when checking the reference exercise. We did it, however, for an elementary task: formula checking. This time, we based the whole model on this tool.

When using dimensional analysis as the basis of the model, the goal is to determine the \emph{functional form} of the model formula; \ie, dimensionless relationships among dimensionless quantities. This is better understood by means of using the tool. 

To find the functional form of a model, some steps can be given:
%
\begin{description}
%
\item[Relevant quantities] We write down the set of the quantities, we think, govern the phenomenon under analysis. In the present case, we choose the set $\elset{\echarge, \ekin, \vel, \efield', \mfield', t}$ of \emph{six} quantities. Such a set was obtained in previous sections. Note that we have \emph{not} included forces nor masses in the set, due to the information we got in \cref{sec:approxsolution}. According to this set, we propose the phenomenon to be modeled by a (so far \emph{unknown}) function $\phi$ of the form
%
\begin{equation}\label{eq:dimensionalmodel}
  \phi\vat{\echarge, \ekin, \vel, \efield', \mfield', t} = 0 \,.
\end{equation}
%
\item[Base dimensions set] Since we are dealing with electromagnetism, we need to choose a suitable set of dimensions on which base the relevant physical quantities. This comes from a chain of reasoning: to analyze geometric problems, we need only the dimension of length, $\phdim L$; to analyze kinematics, we add time $\phdim T$; to analyze dynamics, we can add mass $\phdim M$, force $\phdim F$ or energy $\phdim E$; to analyze electrodynamics, we add electric charge $\phdim Q$. We choose, for solving the present exercise, the set $\elset{\phdim E, \phdim L, \phdim T, \phdim Q}$ of \emph{four} base dimensions. Again, this set was chosen based on the realizations gained in \cref{sec:approxsolution}.
%
\item[Physical model] We then write down a list of the dimensions of the relevant quantities with respect to the set of the base dimensions. We call this list the \lingo{physical model}: \vide \cref{tab:physicalmodelelectricparticle}, \cite[p. 4]{price:2006}.
%
%
% --------------------------------------------------------------- Table
%
\begin{table}\capstart\begingroup\footnotesize\begin{center}
  \begin{tabularx}{0.60\textwidth}{lcX}
%
\toprule
%
\tabhead{Physical quantity} & \tabhead{Symbol} & \tabhead{Dimensions} \\
%
\midrule
\multicolumn{3}{c}{Object properties} \\
\midrule
%
$\point$ electric charge  & $\echarge$ & $\phdim Q$ \\
$\point$ kinetic energy   & $\ekin$    & $\phdim E$ \\
$\point$ velocity         & $\vel$     & $\phdim{L/T}$ \\
%
\midrule
\multicolumn{3}{c}{Agent properties} \\
\midrule
%
$\point'$ electric field  & $\efield'$ & $\phdim{E/QL}$ \\
$\point'$ magnetic field  & $\mfield'$ & $\phdim{ET/L^2Q}$ \\
%
\midrule
%
Time & $t$ & $\phdim T$ \\
%
\bottomrule
%
  \end{tabularx}\end{center}\endgroup\normalsize
  \caption[Physical model: electric phenomenon]
  {Physical model for an electrically charged particle $\point$ moving towards an electrically charged particle $\point'$}
  \label{tab:physicalmodelelectricparticle}
\end{table}
%
% ------------------------------------------------------------- EndTable
%
%
\item[Dimensionless model] According to the Buckinham's theorem, \cref{eq:dimensionalmodel} can be rewritten by grouping its six arguments into $6 - 4 = 2$ dimensionless quantities; \ie, \cref{eq:dimensionalmodel} becomes
%
\begin{equation}
  \fdim'\vat{\kdim_1, \kdim_2} = 0 \,,
\end{equation}
%
where the dimensionless quantities $\kdim_1$ and $\kdim_2$ and the dimensionless function $\fdim'$ are still unknown.
%
\item[Dimensionless quantities] Now, it is time to find out $\kdim_1$ and $\kdim_2$. There are several methods to do this (like linear algebra), we choose the method of \lingo{repeating variables}. According to this method, the first step is to choose a subset of the set of relevant quantities as a \emph{new} base dimensions set. There are four dimensions in our base dimensions set, so we need four relevant quantities. See that $t$ brings the dimension of $\phdim T$, $\vel$ the dimension of $\phdim L$, $\echarge$ the dimension of $\phdim Q$ and $\efield'$ the dimension of $\phdim E$. 

The second step is to find dimensionless combinations of the physical model quantities in the new set of base dimensions. We find out the set by physical arguments. The first dimensionless quantity $\kdim_1$ \emph{must} include the quantity we are interested in (sometimes called the \lingo{independent variable}); in our case, $\dt\ekin\sim\ekin/t$. Thus, we need to choose quantities that make the dimensions of $\ekin/t$ equal to one. Since $\dim \ekin/t = \phdim E/\phdim T$ (dimensions of \lingo{power}), we form a power using $\efield$ (which brings $\phdim E$), $\vel$ (which brings $\phdim L$) and $\echarge$ (which brings $\phdim Q$); \ie,
%
\begin{equation}
  \kdim_1 = \dfrac{\ekin}{t\efield\vel\echarge}\,.
\end{equation}

Following a similar reasoning, we find that $\kdim_2 = \mfield\vel/\efield$.
%
\item[Dimensionless formula] Now, we replace $\kdim_1$ and $\kdim_2$ into \cref{eq:dimensionalmodel}, to have
%
\beq
  \fdim'\vat{\kdim_1, \kdim_2} = \fdim'\vat{\dfrac{\ekin}{t\efield\vel\echarge}, \dfrac{\mfield\vel}{\efield}} 
                               = 0\,,
\eeq
%
from where we could isolate $\kdim_1$:
%
\beq
  \dfrac{\ekin}{t\efield\vel\echarge} = \fdim\vat{\dfrac{\mfield\vel}{\efield}}\,,
\eeq
%
or, by replacing, $\ekin/t\sim\dt\ekin$:
%
\begin{equation}\label{eq:dimlessmodelfinal}
  \dfrac{\dt\ekin}{\efield\vel\echarge} = \fdim\vat{\dfrac{\mfield\vel}{\efield}}\,,
\end{equation}
%
where $\fdim$ is our desired result: the dimensionless form of \cref{eq:dimensionalmodel} -- the equation thought to model the phenomenon under analysis.\qed
%
\end{description}
%
A closed form for the function $\fdim$ in \cref{eq:dimlessmodelfinal} cannot be found by dimensional analysis alone. It must be determined by experimentation, other analytic means or physical reasoning. However, dimensional analysis confirms our suspicion: $\dt\ekin\sim\ekin/t\sim\echarge\efield\vel$. 

Finally, note that $\dt\ekin$ should increase with increasing $\echarge$, increasing $\efield$ and increasing $\vel$. \Cref{eq:dimlessmodelfinal} passes these \lingo{plausibility checks}.


\section{Proof sketch work}
%
In the last sections, we have applied several approximate methods to find the solution to the working exercise. Not only did these approximations provide solutions, but also a method to be used in finding a more formal answer.

The sketch for the derivation can be found in \cref{fig:electromagneticfieldeffect}, where we analyzed the electric and magnetic combined (electromagnetic) effect on the moving particle. The proof path is as follows:
%
\begin{itemize}
%
\item state the definition of the particle kinetic energy and differentiate it with respect to time (first equation);
%
\item find the particle equation of motion by stating Newton's second law for a constant mass particle subject to a Lorentz force originated by another electrically charged and motionless particle.
%
\item right inner multiply the equation of motion by the particle velocity (second equation) and
%
\item finally, join the two equations to have the time derivative of the kinetic energy as function of the electromagnetic field.
%
\end{itemize}


\subsection{Kinetic energy time derivative}
%
Let's start the proof sketch by finding the kinetic energy time derivative. We do that by using a \lingo{natural deduction proof tree} (Gentzen style):
%
\begingroup\small
\begin{prooftree}
  %
  \AxiomC{$2\ekin = \mass\vel^2$}                                              \ndtlabel{$\vel\parallel\vel$}
  \UnaryInfC{$2\ekin = \mass\vel\iprod\vel$}                                   \ndtlabel{$\iod t{}$}
  \UnaryInfC{$2\dt\ekin = \iod t\parth{\mass\vel\iprod\vel}$}                  \ndtlabel{$\dt\mass = 0$}
  \UnaryInfC{$2\dt\ekin = \mass\iod t\parth{\vel\iprod\vel}$}                  \ndtlabel{prod. rule}
  \UnaryInfC{$2\dt\ekin = \mass\parth{\dt\vel\iprod\vel + \vel\iprod\dt\vel}$} \ndtlabel{$\dt\vel\iprod\vel = \vel\iprod\dt\vel$}
  \UnaryInfC{$2\dt\ekin = 2\mass\dt\vel\iprod\vel$}                            \ndtlabel{${}\cdot 1/2$}
  \UnaryInfC{$\dt\ekin  = \mass\dt\vel\iprod\vel$}
  %
\end{prooftree}
\endgroup
%
which results in
%
\begin{equation}\label{eq:kinentimederformal}
  \dt\ekin = \mass\dt\vel\iprod\vel\,.
\end{equation}
%
The last equation gives the kinetic energy time derivative.


\subsection{Equation of motion}
%
Now, we repeat the work for the equation of motion. Let's find out Newton's second law of motion, then multiply it by $\vel$ (LHS of the tree) and equal it to Lorentz force law also multiplied it by $\vel$ (RHS of the tree):
%
%
\begingroup\small
\begin{prooftree}
  %
  % Newton's second law of motion
  %
  \AxiomC{$\mom = \mass\vel$}                              \ndtlabel{$\iod t{}$}
  \UnaryInfC{$\dt\mom = \iod t\parth{\mass\vel}$}          \ndtlabel{$\dt\mass = 0$}
  \UnaryInfC{$\dt\mom = \mass\dt\vel$}                     \ndtlabel{Newton's law}
  \UnaryInfC{$\force = \mass\dt\vel$}                      \ndtlabel{${}\iprod\vel$}
  \UnaryInfC{$\force\iprod\vel = \mass\dt\vel\iprod\vel$}
  %
  % Lorentz force law
  %
  \AxiomC{$\force' = \echarge\parth{\efield' + \vel\cprod\mfield'}$}                        \ndtlabel{${}\iprod\vel$}
  \UnaryInfC{$\force'\iprod\vel = \echarge\parth{\efield' + \vel\cprod\mfield'}\iprod\vel$} \ndtlabel{Expand parenth.}
  \UnaryInfC{$\force'\iprod\vel = \echarge\efield'\iprod\vel + \echarge\vel\cprod\mfield'\iprod\vel$} 
    \ndtlabel{$\echarge\vel\cprod\mfield'\iprod\vel = 0$}
  %
  \UnaryInfC{$\force'\iprod\vel = \echarge\efield'\iprod\vel$} \ndtlabel{$\force\iprod\vel = \force'\iprod\vel$}
  %
  % Equate Newton's to Lorentz
  \BinaryInfC{$\mass\dt\vel\iprod\vel = \echarge\efield'\iprod\vel$}
  %
\end{prooftree}
\endgroup
%
%
which results in
%
\begin{equation}\label{eq:eqnofmotionformal}
  \mass\dt\vel\iprod\vel = \echarge\efield'\iprod\vel\,.
\end{equation}
%
The last equation gives the particle equation of motion.


\subsection{Kinetic energy time derivative and equation of motion}
%
Finally, let's join the kinetic energy time derivative, \cref{eq:kinentimederformal}, to the equation of motion, \cref{eq:eqnofmotionformal}, to find:
%
\beq
 \dt\ekin = \echarge\efield'\iprod\vel\,,
\eeq
%
which upon rearrangement gives
%
\begin{equation}
  \dfrac{\dt\ekin}{\echarge\efield'\iprod\vel} = 1\,.
\end{equation}
%
The last equation gives the desired result: the particle kinetic energy time derivative due to an interaction with an electromagnetic field.

This (analytic) solution confirms our guessed model and the approximate solutions. Thus, it creates confidence, not only on our intuition, but also on the efficacy of approximate methods.


\section{Formal solution}
%
In this section, we will write up the analytic solution to make it ready for presentation.


\subsection{Statement}
%
\theorem{Consider a massive, electrically charged body $\body$ moving toward an electrically charged body $\body'$. Then, find $\body$ temporal change of energy.}


\subsection{Solution}
%
Agree on the given hypotheses and on the symbols and notation previously established (sections...).

Write first the particle's kinetic energy as $2\ekin = \mass\vel\iprod\vel$ and then calculate its temporal change $\dt\ekin$:
%
\begin{equation}\label{eq:timederivkinenergy}
  \dt\ekin = \mass\dt\vel\iprod\vel\,.
\end{equation}

Model the movement of the particle (equation of motion) by equating Newton's second law to Lorentz force law and right inner multiply both sides by $\vel$:
%
\begin{equation}\label{eq:chargedparticleeqofmotion}
  \mass\dt\vel\iprod\vel = \echarge\parth{\efield' + \vel\cprod\mfield'}\iprod\vel\,.
\end{equation}


Equate \cref{eq:chargedparticleeqofmotion} and \cref{eq:timederivkinenergy} to find: $\dt\ekin/\echarge = \efield\iprod\vel + \vel\cprod\mfield\iprod\vel$. Since the scalar triple product vanishes, the model is then $\dt\ekin = \echarge\efield\iprod\vel$, which can be scaled to 
\beq
  \dfrac{\dt\ekin}{\echarge\efield\iprod\vel} = 1\,.
\eeq
%
The last formula models the temporal change of kinetic energy of a charged particle moving through a constant electromagnetic field.


\subsection{Notes}
%
Note that in the formal solution
%
\begin{itemize}
%
\item the presentation is brief, concise, straight to the point, but not incomplete. It only leaves \scare{obvious details} to be filled in; \eg, nowhere it is written that $\dt\mom = \dt\mass\vel + \mass\dt\vel = \mass\dt\vel$, because under hypotheses, $\mass$ is constant, so it is \scare{well-known} that $\dt\mom = \mass\dt\vel$ in such a case;
%
\item equations are referred to by proper, technical names: Newton's second law of motion, scalar triple product and so on;
%
\item only \scare{important} equations, derivations and results are displayed, whereas small equations, non-trivial, but small, derivations and partial results are presented in-line -- with the running text;
%
\item verbs are used in imperative voice to avoid the use of personal grammar forms -- we, us, one and so on -- and of the passive voice.
%
\end{itemize}


\section{Model interpretation}
%


\section{Conclusions}
%
the answer to the problem was wrong in \cite{thorne:2011} and then corrected in \cite{thorne:2013}.

quote: nuilluis in verba :).


\section{Final remarks}
%
The method herein presented is far from being perfect. But it has worked nicely for me, not only when solving textbook exercises, but also in personal research and professional work. In textbooks, authors can write shorter, open statements, because the context given by the surrounding text allows them to do so. In real research, however, one never finds a textbook problem with a back-of-the-book solution.

The most important aspects on solving exercises are, according to my experience:
%
\begin{itemize}
%
\item having a problem that interests me;
%
\item working hard on having a good description of the problem;
%
\item making assumptions.
%
\end{itemize}
%
If all of the previous premises are satisfied, I ....

My working methodology is heavily influenced by John Denker, David Hestenes and Sanjoy Mahajan's ideas.

David Hestenes and Kip Thorne ideas on working with the geometric principle...

Math proofs from Houston.

the writing style comes from AIP Style Manual, Denker, Linder, Mahajan and mainly Fourier (theory of heat!).

I like to write in first person (singular and plural)
%
~\mquote{The old taboo against using the first person in formal prose has long been deplored by the best authorities and ignored by some of the best writers. [\dots] A single author should also use \scare{we} in the common construction that includes the reader.}{\cite[p. 14]{aip:1990}}

