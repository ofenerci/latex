\chapter{Things to organize}


\subsection{Interpretation of the solution}
%
[Apply the case to the electron-proton. Only e-field interaction needed :)]

%The proton cre- ates an electric field due to its charge qp. Lorentz force states that the proton’s field strength |ep| is given by |ep| ∝ qp/r2, where r is the dis- tance from the proton’s center. Geometrically, this means that |ep| cre- ates concentric surfaces of equal electric potential in E3, called isoelectric surfaces, just like a static « heat source » forms concentric isothermal sur- faces around its center. When something moves towards the proton, it will « pierce » such surfaces. Note that the field strength scales inversely with the squared distance: for instance, if the distance is halved, the field strengthens by a factor of four. In other words, the closer to the proton’s center, the stronger the interaction with its field becomes. On the other hand,whenanelectron,withchargeqe <0,entersthefield,itis«attracted» to the proton’s center as the force between them, |fp-e| ∝ −qeqp/r2, in- creases with decreasing distance. In turn, the electron’s velocity ve in- creases and so does its kinetic energy ke ∝ ve2. Therefore, we expect k ̇e ∼ −qeepve. Notice the negative sign in the expression: it says that the electron looses energy as it falls into the proton! Also, see that k ̇ e does not depend on the electron’s mass.


\section{Things}
%
Integral: $\lint{a,b}{ax + b}{\dx x}$, 124.2134 -- $124.2134$.


\section{Dim analysis}
%
[taken from \cite[p. 97]{bhaskar:1990}]

A crucial step in computing the $\kdim$s is the selection of $r$ basis variables. In principle there are $\binom nr$ choices; however, many of these do not yield an ensemble, \ie there are fewer equations than there are unknowns.

We have found the following \lingo{basis selection heuristics} to be useful. These heuristics assume that for all candidates for membership in the basis, the property that basis dimensions cover the dimensional space of the problem holds.
%
\begin{itemize}
%
\item A variable of interest, whose behavior is to be reasoned about, should not be included in the basis.
%
\item Exogenous variables, \emph{whenever possible}, should be included in the basis. 
%
\item Other things being equal, dimensional richness (\eg $\phdim{MLT^{-2}}$ is richer than $\phdim L$) is the criterion for including a variable in the basis.
%
\item Given several variables with the same dimensional representation, only one should be included in the basis.
%
\end{itemize}

Implementing a system for dimensional reasoning is largely a matter of selecting the input variables and output variables that characterize a particular device or process. The heuristics that we have listed above are guides to implementing device models rather than heuristics that can be used by a system to select basis variables.


\section{Test}
%
Let $u$ and $v$ be two vectors in $\nespace n$. Then, define their inner product by
%
\beq
  u\iprod v = \ifvec k\ivec uk\iprod\ifvec l\ivec vl
            = \ivec uk\ivec vl\ifvec k\iprod\ifvec l
            = \ivec uk\ivec vl\ifmet kl\,.
\eeq

Consider two space vectors $u$ and $v$. Define their \lingo{inner product} as
%
\beq
  u\iprod v = \ifvec k\ivec uk\iprod\ifvec l\ivec vl
            = \ivec uk\ivec vl\ifvec k\iprod\ifvec l
            = \ivec uk\ivec vl\ifmet kl\,.
\eeq

Geometric algebra in 3-dims: $\set G^3$, $\nespace 3$.

Cliffs (multivectors) in space:
%
\begin{align*}
  c &\defas s + v + b + t \\
  c &\equiv s + v + b + t \,.
\end{align*}

A quantity which changes continuously in value is called a fluent. [clifford, elements of dynamic. vol. i, p. 62] (good book). add classification of kinematics. bodies: particle, elastic, rigid...

This rate of change of a fluent quantity is called its fluxion, or sometimes, more shortly, its flux. It appears from the above considerations that a flux is always to be conceived as a velocity ; because a quantity must be continuous to be fluent, must therefore be specified either by a line or an angle (which may be placed at the centre of a standard circle and measured by its arc) and rate of change of a length measured on a straight line or circle means velocity of one end of it (if the other be still) or difference of velocity of the two ends.

operational def. conceptual def.

evolution. state var. and so on.


\section{Components}
%
Given a frame $\seq{\ifvec k}{k = 0}{3}$, we can write any arbitrary vector $v$ as a linear combination of the frame elements:
%
\beq
  v = \ivec v0\ifvec 0 + \ivec v1\ifvec 1 + \ivec v2\ifvec 2 + \ivec v3\ifvec 3
    = \ivec vk\ifvec k\,,
\eeq
%
for suitable scalars $\seq{\ivec vk}{k = 0}{3}$.

Terminology: These scalars ($\ivec vk$) are sometimes called the \lingo{components of $v$ in the chosen frame}. They can also be called the \lingo{matrix elements of $v$ in the chosen frame}.

Terminology: The vector $\ivec v0\ifvec 0$ is sometimes called the \lingo{component of $v$ in the chosen $\ifvec 0$ direction} (and similarly for the other terms on the RHS of the last equations). Such a vector can also be called the \lingo{projection of $v$ onto the chosen directions}.

It is usually obvious from context which definition of \scare{component} is intended. If you want to avoid ambiguity in your writing, you can avoid the word \scare{component} and instead say \scare{matrix element} or \scare{projection} as appropriate.

Beware: Even though $\ivec vk$ is a component of the vector $v$ and is a scalar, please do not think of $\ifvec k$ in the same way. Each $\ifvec k$ is a vector unto itself, not a scalar. The $k$ in $\ifvec k$ tells which vector, whereas the $k$ in $\ivec vk$ tells which component of the vector. There is no advantage in imagining some super-vector that has the $\ifvec k$ vectors as its components.


\section{Functions}
%
$\lfdef{f}{N}{Z}{x}{4-x}$. $f$ is a function from the set $\set N$ \emph{to} the set $\set Z$ that \emph{maps} $x$ to $4 - x$. The formula $f\vat x = 4 - x$ is the value of $f$ under the argument $x$.

In general, $f:\set D\to\set C$, $x\mapsto y$; \ie, $f$ is a function from the set $\set D$ (domain) to the set $\set C$ (codomain) that maps $x$ to $y$ or, succinctly, $y = f\vat x$.


\section{Physical laws}
%
laws are 

