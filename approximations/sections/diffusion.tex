\section{Diffusion equation}
%
%%%
\newcommand{\krcoeff}{\gamma} % reaction coefficient


[Lee DeVille. Math 558 - Methods of Applied Mathematics.]


\subsection{Dim analysis}
If we consider the density of, for instance, a chemical in solution, where we denote said concentration at $\lpos$ and $t$ by $\conc\vat{\lpos, t}$, then the diffusion equation is given by
\begin{equation}\label{eq:dimensionaldiffusioneqn}
\ipd t\conc = \kmdiff\ipd{\lpos\lpos}\conc\,,
\end{equation}
where $\kmdiff$ is the mass diffusion coefficient.

Let us imagine that we post this problem on the domain $\ooineq{0}{\lpos}{\infty}$ %$\lpos\in \ooint{0, \infty}$ 
and $t > 0$. Moreover, we assume that $\conc\vat{\lpos, 0} = 0$ for all $\lpos$ (zero concentration at time zero) and we inject the chemical at $\lpos = 0$, so that $\conc\vat{0, t} = \conc_0$. We also append the boundary condition $\conc\vat{\infty, t} = 0$.

Now, assuming that the concentration is a function of $\elset{\lpos, t, \kmdiff, \conc_0}$, we obtain
\beq
\dfrac{\conc}{\conc_0} = \kdimf\vat{\dfrac{\lpos}{\sqrt{\kmdiff t}}}\,.
\eeq
Thus we have the nondim quantity $\kdim = \lpos/\sqrt{\kmdiff t}$ and we have $\conc/\conc_0 = \kdimf\vat\kdim$.

Plugging this expression, $\conc/\conc_0 = \kdimf\vat\kdim$, into \cref{eq:dimensionaldiffusioneqn} and its boundary and initial conditions, we obtain
\beq
\kdimf\vat{\kdim} = 1 - \dfrac{1}{\sqrt{\pi}}\int_0^\kdim \exp\vat{-s^2/4}\,\dx s
\eeq
and therefore we have the general solution
\beq
\conc\vat{\lpos, t} = \conc_0\left(1 - \dfrac{1}{\sqrt{\pi}}\int_0^{\lpos/\sqrt{\kmdiff t}}\exp\vat{-s^2/4}\,\dx s\right)\,.
\eeq


\subsection{Scaling}
We can add a nonlinear term to the diffusion equation as follows:
\beq
\ipd t\conc = \kmdiff\ipd{\lpos\lpos}\conc + \krcoeff\conc^3\quad\text{and}\quad
\conc\vat{\lpos, 0} = \conc_0\vat{\lpos}\,.
\eeq
This is an example of a \lingo{reaction-diffusion} equation; the polynomial term is a (local) reaction of the substance whose concentration we are tracking. Rescaling with
\beq
\lpos = \chpq\lpos\scpq\lpos\,,\quad
t = \chpq t\scpq t\quad\text{and}\quad
\conc = \chpq\conc\scpq\conc
\eeq
we obtain
\beq
\ipd{\scpq t}\scpq\conc = \kmdiff\dfrac{\chpq t}{\chpq\lpos^2}\ipd{\scpq\lpos\scpq\lpos}\scpq\conc 
                          + \krcoeff\chpq t\chpq\conc^2\scpq\conc\,.
\eeq
Thus, we have
\beq
\kdim_1 = \dfrac{\kmdiff\chpq t}{\chpq\lpos^2}\quad\text{and}\quad
\kdim_2 = \krcoeff\chpq t\chpq\conc^2
\eeq
as our nondim quantities. This lets us know what parameters we would choose to have (relative) small diffusion or (relative) large diffusion. If $\kdim_1\ll\kdim_2$; \ie, $\kmdiff\chpq\conc/\krcoeff\chpq\lpos^2\ll 1$, then we can write $\kdim_1 = \smpq$, $\kdim_2 = 1$ and we have the PDE
\beq
\ipd{\scpq t}\scpq\conc = \smpq\ipd{\scpq\lpos\scpq\lpos}\scpq\conc + \scpq\conc^3\,,
\eeq
or the \lingo{small diffusion} scaling regime. This lets us know how we would make this small; \eg, say the constants $\elset{\kmdiff, \krcoeff}$, we could either take $\chpq\conc$ small (small concentrations) or $\chpq\lpos$ (long lengthscales) to get the small diffusion regime.

Similarly, if $\kdim_1\gg\kdim_2$, or $\kmdiff\chpq\conc/\krcoeff\chpq\lpos^2\gg 1$, then we can write $\kdim_1 = 1$, $\kdim_2 = \smpq$ and we have the PDE
\beq
\ipd{\scpq t}\scpq\conc = \ipd{\scpq\lpos\scpq\lpos}\scpq\conc + \smpq\scpq\conc^3\,,
\eeq
or the \lingo{small reaction} scaling regime. This can be obtained by looking at large concentrations or really small lengthscales.
