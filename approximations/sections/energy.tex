%%%
%
\newcommand{\atemp}{\temp\infty}    % ambient temperature
\newcommand{\wtemp}{\temp\txt{w}}   % wall temperature
\newcommand{\mtemp}{\temp\txt{max}} % max temperature

\section{Energy}
Consider a large, thin concrete slab of thickness $\length$ that is \lingo{setting}. Setting is an exothermic process that releases $\then$, where $\dim\then = [\phdim E/\phdim T\phdim L^3]$ -- thermal power density. The outside surfaces are kept at the ambient temperature, so the temperature of the walls, $\wtemp$, equal the ambient temperature: $\wtemp = \atemp$. What is the maximum internal temperature?


\begin{guess}
Since the walls are kept at constant temperature, the process is at steady state. However, temperature ranges spatially through the slab thickness. If one measures the spatial variation by $\pos$, then the slab temperature satisfies $\temp = \temp\vat\pos$.

By symmetry, the center temperature coincides with $\mtemp$ at the slab center, $\pos = 1/2$, and decreases smoothly to a minimum at the walls, $\pos = 0$ and $\pos = \length$. This symmetry gives room to think about an inverted parabolic temperature distribution inside the slab with the parabola vertex at $\mtemp$.
\end{guess}


\begin{dimensional}
Place a Cartesian coordinate axis running from one wall to the other covering the slab thickness. Let $\pos$ measure position within $0\leq\pos\length$. Thus, since the process is at steady state, $\temp = \temp\vat\pos$.

Choose the dimensional set to be $\elset{\phdim E, \phdim L, \phdim T, \phdimtemp}$. Hypothesize the quantities governing the phenomenon to be those listed in \cref{tab:settingslabquantities}. 
%%% Tables
%
% ------------------------------------------------------------- PreTable
\docpretable{bt}{0.9\textwidth}{lcc}%
% position: bthH. size: 0.9\textwidth. cols: llcp{6mm}
% use: \docfloatwidth whenever possible!
\toprule
Quantity    & Symbol    & Dimension \\
\midrule
Slab temperature                     & $\temp$     & $\phdimtemp$ \\
Slab thermal conduction coefficient  & $\kthcond$  & $\phdimtemp$ \\
Slab thickness                       & $\length$   & $\phdim L$ \\
Wall temperature                     & $\wtemp$    & $\phdimtemp$ \\
Setting power density                & $\then$     & $\phdim E/\phdim T\phdim L^3$ \\
Position within the slab             & $\pos$      & $\phdim L$ \\
\bottomrule
% ------------------------------------------------------------ PostTable
\end{tabularx}
\docposttable{Quantities: slab problem}{Quantities and dimensions affecting the thermal conduction of the concrete slab setting.}{tab:settingslabquantities}
% ------------------------------------------------------------- EndTable

As seen in \cref{tab:settingslabquantities}, according to the Pi-theorem, $6 - 4 = 2$ dimensionless quantities can be constructed. The first one:
\beq
\kdim_1 = \dfrac{\kthcond\left(\temp - \wtemp\right)}{\then\length^2}\,,
\eeq
which measures the relationship between energy conduction and energy production. The second dimensionless quantity:
\beq
\kdim_2 = \dfrac{\pos}{\length}\,,
\eeq
which is a geometric ratio.

With both dimensionless quantities, one can apply the principle of dimensional homogeneity for physical laws to find
\beq
\kdim_1 = \kdimf\vat{\kdim_2}
\implies
\dfrac{\kthcond\left(\temp - \wtemp\right)}{\then\length^2} = \kdimf\vat{\dfrac{\pos}{\length}}\,.
\eeq

Scale temperatures by means of $\kdim_1$ and lengths by $\kdim_2$; \ie,
\beq
\scpq\temp = \kdim_1\temp\qquad\text{and}\qquad
\scpq\pos  = \kdim_2\pos\,.
\eeq

Hence, finally, the equation governing the phenomenon can be written as
\begin{equation}\label{eq:slabsettingdimlessequation}
\scpq\temp = \kdimf\vat{\scpq\pos}\,.
\end{equation}
where the function $\kdimf$ cannot be further determined by dimensional analysis.
\end{dimensional}


\begin{approximation}
Assuming a parabolic distribution of temperatures, $\kdimf$ in \cref{eq:slabsettingdimlessequation} can be hypothesize to satisfy
\beq
\scpq\temp = a\scpq\pos^2 + b\scpq\pos + c\,,
\eeq
where $\elset{a,b,c}$ are dimensionless quantities to be determined.

Now, we can use a theorem in geometry that states that three points uniquely determine a parabola. Two of these points can be found from the problem statement:
\beq
\begin{dcases}
\tuple{\scpq\pos = 0, \scpq\temp = 0} \\
\tuple{\scpq\pos = 1, \scpq\temp = 0}\,.
\end{dcases}
\eeq
Setting $a = -1/2$ (an inverted parabola) and solving the previous systems of equations, one finds that
\beq
a = -\dfrac{1}{2}\,,\qquad
b = \dfrac{1}{2}\qquad\text{and}\qquad
c = 0\,.
\eeq
Replacing these values in the hypothesized $\kdimf$, one has
\beq
2\scpq\temp = \scpq\pos\left(1 - \scpq\pos\right)\,.
\eeq

Now, using symmetry, when $\scpq\pos = 1/2$, then $\scpq\temp = \scpq\mtemp$:
\beq
\scpq\mtemp = \dfrac{1}{8}
\eeq
or, returning to the dimensional quantities, $\mtemp$ can be found by
\beq
\dfrac{\kthcond\left(\mtemp - \wtemp\right)}{\then\length^2} = \dfrac{1}{8}\,.\mqed
\eeq
\end{approximation}
