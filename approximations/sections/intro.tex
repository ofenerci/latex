\section{Introduction}
%
%%%
\epigraph{[...] the principle known as Occam's Razor: \latin{essentia non sunt multiplicanda praeter necessitatem} (hypotheses are not to be multiplied without necessity).}{attributed to William of Ockham, quoted by R.V. Jones}{\citep[p.95]{gibbings:2011}.}
%
\epigraph{Only wimps specialize on the general case. Real scientists pursue examples.}{Beresford Parlett}{\citep{berry:1995}}.
%%%

[The Art of Approximation in Science and Eng. Sanjoy]

An approximate model can be better than an exact model! An approximate answer is all that we can understand because our minds are a small part of the world itself. So when we represent or model the world, we have to throw away aspects of the world in order for our minds to contain the model. Thus, making useful models means discarding less important information so that our minds may grasp the important features that remains.



