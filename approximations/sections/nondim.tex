\section{Nondimensionalization}


\subsection{Free fall}
Nondimensionalze the equation of free fall:
\begin{equation}\label{eq:dimensionalfreefallmodel}
\pos = \pos_0 + \vel_0 t + \dfrac{1}{2}\grav t^2\,.
\end{equation}


Solution:
\begin{itemize}
%
\item Problem domain: kinematics. Set of dim. ind. dims: $\elset{\phdim L, \phdim T}$.
%
\item Physical model: see \cref{tab:free-fall}.
%
% ------------------------------------------------------------- PreTable
\docpretable{bt}{0.55\textwidth}{ccc}%
% position: bthH. size: 0.9\textwidth. cols: llcp{6mm}
\toprule
%
Quantity    &     Symbol &    Dimension \\
%
\midrule
%
Vertical displacement         & $\pos$   & $\phdim L$ \\
Initial vertical displacement & $\pos_0$ & $\phdim L$ \\
Time                          & $t$      & $\phdim T$ \\
Velocity                      & $\vel$   & $\phdim L/\phdim T$ \\
Initial velocity              & $\vel_0$ & $\phdim L/\phdim T$ \\
Free fall acceleration        & $\grav$  & $\phdim L/\phdim T^2$ \\
%
\bottomrule
%
\end{tabularx}
\docposttable{Free fall}{Physical model of an object free falling}{tab:free-fall}
% ------------------------------------------------------------- EndTable
%
\item Dimensional physical model:
\beq
f\vat{\pos, \pos_0, t, \vel, \vel_0, \grav} = 0\,.
\eeq
%
\item Number of physical quantities: 6.
%
\item Number of scaling quantities: 2 (cardinality of the set of dim. ind. dims).
%
\item Number of control quantities: 2 (dependent and independent quantities).
%
\item Number of dimless quantities: $6 - 2 = 4$. 
%
\item Choose $\elset{\pos_0, \vel_0}$ as scaling quantities, to know the effect of gravity on the vertical displacement. Then,
\begin{equation}\label{eq:dimlessfreefallphysicalmodel}
\kdim_\pos  = \dfrac{\pos}{\pos_0}\,, \quad
\kdim_t     = \dfrac{\vel_0 t}{\pos_0}\,, \quad
\kdim_\vel  = \dfrac{\vel}{\vel_0}\quad\text{and}\quad
\kdim_\grav = \dfrac{\pos_0\grav}{\vel_0}\,.
\end{equation}
%
\item Dimless physical model (always possible, guarantee by the $\kdim$-theorem):
\beq
\kdimf\vat{\kdim_\pos, \kdim_t, \kdim_\vel, \kdim_\grav} = 0\,.
\eeq
%
\item Using \cref{eq:dimensionalfreefallmodel} and \cref{eq:dimlessfreefallphysicalmodel} find
\beq
\scpq\pos = 1 + \scpq t + \dfrac{1}{2}\kdim_\grav \scpq t^2\,,
\eeq
where all the barred quantities are dimensionless! and $\kdim_\grav = \pos_0\grav/\vel_0$.
%
\end{itemize}
%
The process can be repeated if the effect of $\pos_0$ on $\pos$ is required, by using $\elset{\vel_0, \grav}$ as scaling parameters to find
\beq
\scpq\pos = \kdim + t + \dfrac{1}{2}\scpq t^2\,,
\eeq
where all the barred quantities are dimensionless and $\kdim = \grav\pos_0/\vel_0^2$.


\subsection{Drag}
\begin{itemize}
%
\item Problem domain: physics -- fluid dynamics. Set of dim. ind. dims: $\elset{\phdim L, \phdim M, \phdim T}$.
%
\item Physical model: see \cref{tab:drag-physical-model}.
%
% ------------------------------------------------------------- PreTable
\docpretable{bt}{0.55\textwidth}{ccc}%
% position: bthH. size: 0.9\textwidth. cols: llcp{6mm}
\toprule
%
Quantity    &     Symbol &    Dimension \\
%
\midrule
%
Drag             & $\drag$  & $\phdim M\phdim L/\phdim T^2$ \\
Velocity         & $\vel$   & $\phdim L/\phdim T$ \\
Sectional area   & $\area$  & $\phdim L^2$ \\
Fluid density    & $\dens$  & $\phdim M/\phdim L^3$ \\
%
\bottomrule
%
\end{tabularx}
\docposttable{Drag}{Physical model of an moving object with drag}{tab:drag-physical-model}
% ------------------------------------------------------------- EndTable
%
\item Dimensional physical model:
\beq
f\vat{\drag, \vel, \area, \dens} = 0\,.
\eeq
%
\item Number of physical quantities: 4.
%
\item Number of scaling quantities: 3 (cardinality of the set of dim. ind. dims).
%
\item Number of control quantities: 2 (dependent and independent quantities).
%
\item Number of dimless quantities: $4 - 3 = 1$. 
%
\item Choose $\elset{\vel, \area, \dens}$ as scaling quantities. Then,
\begin{equation}\label{eq:dimlessdragmodel}
\kdim_\drag  = \dfrac{\drag}{\dens\area\vel^2}\,.
\end{equation}
%
\item Dimless physical model:
\beq
\kdimf\vat{\kdim_\drag} = 0\,.
\eeq
%
\item Assume-derive-calculate-check. Assume a functional form for the last eqn. Begin with the simplest: a monomial of degree one:
\beq
\dfrac{\drag}{\dens\area\vel^2} = \kdim\,,
\eeq
where $\dim\kdim = 1$.
%
\item $\kdim$ has to be either derived by theory or by experiment.
%
\item Finally, any functional form has to be tested against experiment.
%
\end{itemize}
