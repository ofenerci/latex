\section{Pressure}
Consider a piston of volume $\vol$ and sectional area $\area$ holding an amount $\amount$ of an ideal gas and consider a force $\force$ being applied on $\area$ that compresses the fluid. 

Such a force generates a pressure $\press$ on the gas. See now that $\press$ can be viewed as energy density instead of force per unit area. With this view, one finds the external energy applied to the system $\ener\txt{ext}$ by
\beq
\ener\txt{ext}\sim\press\vol\,.
\eeq

This external stimulus makes the gas to perform $\press\vol$ work, the gas internal response $\ener\txt{int}$, given by
\beq
\ener\txt{int}\sim\amount\kgas\temp\,,
\eeq
where $\kgas$ represents the gas constant and $\temp$ the gas temperature.

Thus, according to the energy conservation principle, the external stimulus must be balanced by the gas internal response:
\beq
\ener\txt{ext}\sim\ener\txt{int}
              \sim\amount\kgas\temp
\implies
\kgas\temp\sim\dfrac{\ener\txt{ext}}{\amount}\,.
\eeq
That is, $\kgas\temp$ is a measure of the external energy distributed per amount of gas -- molar energy.

On the other hand, since by definition an ideal gas does not interact, its total internal energy equals its kinetic energy alone:
\beq
\ener\txt{int}\sim\mass\vel^2\,,
\eeq
where $\mass$ represents the gas mass and $\vel$ the average velocity of the gas particles. Thus, one finds
\beq
\mass\vel^2\sim\amount\kgas\temp
\implies
\temp\sim\dfrac{\mass\vel^2}{\amount\kgas}
     \propto\vel\,.
\eeq
Hence, temperature can also be viewed as a measure of the mean particle velocity of the gas particles.


