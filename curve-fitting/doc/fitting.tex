\chapter*{Fitting data by the method of least squares}


\section*{Reference exercise}
%
[Taken from \citep[p. 25.]{cohen:2010}]

Given $n$ experimental data points $\tuple{x_k, y_k}$ and a \lingo{fitting equation}, \aka model equation, $f\vat{x_k}$ with unknown coefficients $\elset{\alpha_k}$ with $1 \leq k \leq n$, the method of least squares consists on minimizing the \lingo{square of the root mean square, rms, error} 
%
\beq
    e = \ssum{k}{\epsilon_k}^2 = \ssum{k}{\left(f_k - y_k\right)^2}
\eeq
%
with respect to the coefficients $\elset{\alpha_k}$.


For instance, consider that a theory predicts the data in \cref{tab:data} decreasing with increasing $x$ as a quadratic in $1/x$.
%
% ------------------------------------------------------------- PreTable
\docpretable{bt}{0.18\textwidth}{cc}%
% position: bthH. size: 0.9\textwidth. cols: llcp{6mm}
% use: \docfloatwidth whenever possible!
% NOTE: does not include \toprule
%
\toprule
%
$x_k$ & $y_k$ \\
%
\midrule
%
1.3 & 5.42 \\
2.2 & 4.28 \\
3.7 & 3.81 \\
4.9 & 3.62 \\
%
\bottomrule
%
% ------------------------------------------------------------ PostTable
\end{tabularx}
\docposttable{Fitting data}{Data with inverse power of $x$ decrease}{tab:data}
% include: \end{tabularx}%
% ------------------------------------------------------------- EndTable

Begin by defining the model to fit the data:
%
\begin{equation}\label{eq:fittingequation}
    f\vat x = \alpha_1 + \alpha_2 \dfrac{1}{x} + \alpha_3 \dfrac{1}{x^2}\,.
\end{equation}
%
Then, find the square of the rms error of the model and the data:
%
\beq
    e = \ssum{k}{\left(\alpha_1 + \alpha_2\dfrac{1}{x} + \alpha_3\dfrac{1}{x^2} - y_k\right)^2}\,.    
\eeq
%
Minimize $e$ with respect to the coefficients $\elset{\alpha_k}$:
%
\begin{align*}
%
\ipd 1e &= 2\ssum{k}{\left(\alpha_1 + \alpha_2\dfrac{1}{x_k} + \alpha_3\dfrac{1}{x_k^2} - y_k\right)} = 0\,,\\
%
\ipd 2e &= 2\ssum{k}{\dfrac{1}{x_k}\left(\alpha_1 + \alpha_2\dfrac{1}{x_k} + \alpha_3\dfrac{1}{x_k^2} - y_k\right)} = 0\,,\\
%
\ipd 3e &= 2\ssum{k}{\dfrac{1}{x^2_k}\left(\alpha_1 + \alpha_2\dfrac{1}{x_k} + \alpha_3\dfrac{1}{x_k^2} - y_k\right)} = 0\,.
%
\end{align*}
%
Distribute the sums in every term, perform algebra and replace the values of \cref{tab:data} to have
%
\begin{align*}
    4.000\alpha_1 + 1.698\alpha_2 + 0.913\alpha_3 &= 17.130\,,\\
    1.698\alpha_1 + 0.913\alpha_2 + 0.577\alpha_3 &= 7.883\,,\\
    0.913\alpha_1 + 0.577\alpha_2 + 0.400\alpha_3 &= 4.52\,.
\end{align*}
%
Solve the system of equations to find $\elset{\alpha_1 = 3.261, \alpha_2 = 1.480, \alpha_3 = 1.722}$. The model thus becomes
%
\beq
    f\vat x = 3.261 + 1.480\dfrac{1}{x} + 1.722\dfrac{1}{x^2}\,,
\eeq
%
with a fitting error
%
\beq
    e = \sum_{k = 1}^4\left( 3.261 + 1.480\dfrac{1}{x} + 1.722\dfrac{1}{x^2} - y_k\right)^2 
      = 0.001\,. \mqed
\eeq


\section*{Data analysis and regression analysis}


\subsection*{Procedure}
%
Follow the procedure to fit data:
%
\begin{itemize}
%
\item plot data to see trends, NaNs, missing values, outliers, \etc. Consider fitting equations if there is not a subjacent theoretical equation for the data;
%
\item preprocess data: deal with missing values (interpolation), carefully remove NaNs and outliers, consider filtering and detrending data if appropriate;
%
\item summarize data with descriptive statistics;
%
\item normalize or standardize data to find more accurate fitting coefficients;
%
\item perform regression analysis with the norm. or std. data: find the fitting equation, perform error analysis (correlation coefficients, residuals);
%
\item denormalize or destandardize the coefficients and the fitting equation to give them the original dimensions;
%
\item analyze dimensions, specially for the fitting equation coefficients; \ie, give context to abstract fitting equations; non-dimensionalyze the results if possible;
%
\item present the fitting equation and its coefficients with proper dimensions, limits of applicability and errors. Follow Sonin's advice for result presentation, \cite[p. 23]{sonin:2001}.
%
\end{itemize}


\subsection*{Example revisited}
%
Let's revisit the reference exercise, but this time let's apply the procedure to analyze data and then perform regression analysis. 

To give a physical context to the data, assume that the impulse is temperature and the response mass density; \ie, $x$ becomes temperature $\temp/\si{\celsius}$ and $y$ mass density $\dens/\si{kg.cm^{-3}}$.


\subsubsection*{Data preprocessing}
%
To begin with, plot the unprocessed data to see trends, NaNs and so on.
%
% --------------------------------------------------------------- Figure
%
% position: bthH. size:width=0.5\textwidth. file:location+filename.pdf
% caption. label:fig:wec
% use: \docfloatwidth whenever possible!
 \docfigure{bt}{width=1.25\textwidth}{./graphs/raw-data.pdf}%
   {Raw data}%
   {Unprocessed experimental data}%
   {fig:rawdata}%
%
% ------------------------------------------------------------ EndFigure

As seen in \cref{fig:rawdata}, there are no NaNs, missing values or outliers. There is no need to filter, nor to detrend data either.

Since there is a \scare{theory} behind data, assume the fitting equation \cref{eq:fittingequation}; \ie,
%
\beq
    \dens\vat\temp = \alpha_1 + \alpha_2 \dfrac{1}{\temp} + \alpha_3 \dfrac{1}{\temp^2}\,.
\eeq


\subsubsection*{Descriptive statistics}
%


\subsubsection*{Data scaling}
%
Scale (normalize and non-dimensionalize) data by dividing the values by their maximum; \ie, by dividing temperature values by \SI{4.9}{\celsius} and mass density values by \SI{5.42}{g/cm^3}; \ie, transform data as
%
\begin{equation}\label{eq:scalefactors}
    \bar\temp = \dfrac{\temp}{\temp\txt{max}}\qquad\text{and}\qquad
    \bar\dens = \dfrac{\dens}{\dens\txt{max}}\,, 
\end{equation}
%
where the bars represent scaled quantities.

The result is presented in ...


\subsubsection*{Regression analysis}
%
Perform the regression analysis as explained in the \emph{Reference exercise} section with the scaled data to find:
%
\beq
    \bar\alpha_1 = 0.601703\,,\qquad
    \bar\alpha_2 = 0.055723\qquad\text{and}\qquad
    \bar\alpha_3 = 0.0132309\,,
\eeq
%
where the bars represent scaled coefficients.


\subsubsection*{Rescaling model equation}
%
Rescale the model equation by applying the inverse transform of \cref{eq:scalefactors}:
%
\beq
    \dfrac{\dens}{\dens\txt{max}} = \bar\alpha_1 
                                    + \bar\alpha_2\dfrac{\temp\txt{max}}{\temp} 
                                    + \bar\alpha_3\dfrac{\temp^2\txt{max}}{\temp^2}\,.
\eeq
%
By rescaling the model, the coefficients become
%
\beq
    \alpha_1 = 3.2612306\,,\qquad
    \alpha_2 = 1.479891434\qquad\text{and}\qquad
    \alpha_3 = 1.72179258678\,.
\eeq

Finally, find the rms error as in table ...
%
\beq
    e = 0.001\,.
\eeq
%
\Cref{fig:crosscheck} depicts a cross check between data and model prediction. The straight line indicates agreement between the two; \ie, the model fits experimental data.
%
% --------------------------------------------------------------- Figure
%
% position: bthH. size:width=0.5\textwidth. file:location+filename.pdf
% caption. label:fig:wec
% use: \docfloatwidth whenever possible!
 \docfigure{bt}{width=1.25\textwidth}{./graphs/cross-check.pdf}%
   {Cross-check}%
   {Cross-check between data and model prediction values}%
   {fig:crosscheck}%
%
% ------------------------------------------------------------ EndFigure


\subsubsection*{Results}
%
Regression of the data presented in ... results in the model
%
\beq
    \dens = \alpha_1 + \dfrac{\alpha_2}{\temp} + \dfrac{\alpha_3}{\temp^2}\,,\quad
            \begin{cases}
                \alpha_1/\si{kg.cm^{-3}} &= 3.2612306\,,\\
                \alpha_2/\si{kg.\celsius.cm^{-3}} &= 1.479891434\,,\\
                \alpha_3/\si{kg.\celsius^2.cm^{-3}} &= 1.72179258678\,,
            \end{cases}
\eeq
which is valid for $1.3\leq\temp/\si{\celsius}\leq 4.9$ and $3.62\leq\dens/\si{kg.cm^{-3}}\leq 5.42$, with an rms error of 0.001.

Finally, the model equation can be presented in dimless form:
%
\beq
    {\bar\dens} = 0.602 
                + \dfrac{0.0557}{\bar\temp} 
                + \dfrac{0.0132}{\bar\temp^2}\quad
    \begin{cases}
        \bar\dens = \dens/\SI{5.42}{kg.cm^{-3}}\,,\\
        \bar\temp = \temp/\SI{4.9}{\celsius}\,,
    \end{cases}
\eeq
which is valid for $0.265\leq\bar\temp\leq 1.000$ and $0.668\leq\bar\dens\leq 1.0000$, with an rms error of 0.001.
