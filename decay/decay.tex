\documentclass[10pt]{article}

%%% Packages
%
\RequirePackage{amsmath}  % maths
\RequirePackage{amsfonts} % maths
\RequirePackage{amssymb}  % maths
\RequirePackage{amsthm}   % maths (theorems) (qedsymbol)
%
%%% Math
%
\providecommand*{\Diff}{\Delta}              % difference
\providecommand*{\vat}[1]{\!\left[#1\right]} % value at
%
%%% Physics
%
\providecommand*{\kdecay}{k}      % decay coefficient
\providecommand*{\npart}{n}       % number of particles
\providecommand*{\dt}{\text{d}_t} % time derivative

\begin{document}

Let $\npart = \npart\vat t$ be the number of radioactive nuclei at time $t$ and let $\Diff t$ be a small change in time $t$. We know that the change in the number of nuclei is proportional to the number of nuclei at the start of the time period. Hence, the word equation translates to
%
\begin{equation}
  \dt\npart = -\kdecay\npart\,,
\end{equation}
%
where $\kdecay$ is a positive coefficient indicating the rate of decay per nucleus in unit time. We write $\npart$ in the right-hand side for $\npart\vat t$ as the dependance on $t$ is implied by the derivative $\dt$. We assume $\kdecay$ to be fixed although it will have a different value for different elements or isotopes.

\end{document}
