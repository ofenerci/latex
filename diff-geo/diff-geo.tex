\RequirePackage{fix-cm}

\documentclass[aps,pra,10pt,a4paper]{revtex4-1}

% packages
%
\usepackage{amsmath}
\usepackage{amsfonts}
\usepackage{amssymb}
\usepackage{amsthm}
\usepackage{cleveref}
\usepackage{fixltx2e}
\usepackage[fixamsmath,disallowspaces]{mathtools}
\usepackage{microtype}
\usepackage{mleftright}
\usepackage{tensor}
\usepackage[inline]{showlabels}
\usepackage[svgnames]{xcolor}
\usepackage{xspace}

% showlabels
%
\renewcommand{\showlabelfont}{\footnotesize\sffamily\color{VioletRed}}

% fonts
%
% Minion Pro with old style numerals
%
\RequirePackage[english]{babel}    % language support
\RequirePackage[no-math]{mathspec} % oldstyle nums in math mode
\RequirePackage{xunicode}          % xetex
\RequirePackage{xltxtra}           % xetex
%
\setmathsfont(Digits)[Numbers={OldStyle}]{MinionPro-Regular} %
\setsansfont[Mapping=tex-text,Scale=0.9]{Myriad Pro}         % sans font
\setmonofont[Mapping=tex-text,Scale=0.8]{Menlo}              % mono font
\defaultfontfeatures{Numbers=OldStyle}                       % before setting the roman font
\setmainfont[Mapping=tex-text]{MinionPro-Regular}            % roman font
\addfontfeature{Kerning=Uppercase}                           % space in Uppercase

% definitions, theorems, ...
%
\theoremstyle{plain}
\newtheorem{theorem}{Proposition}
%
\theoremstyle{definition}
\newtheorem{definition}{Definition}
\newtheorem{example}{Example}
\newtheorem{exercise}{Exercise}
%
\theoremstyle{remark}
\newtheorem*{remark}{Remark}
\newtheorem*{note}{Note}
%
\newenvironment{solution}{\begin{proof}[Solution]}{\end{proof}}

% linguistics
%
\providecommand*{\lingo}[1]{\textsl{#1}}
%
% linguistic forms
%
\providecommand*{\lingform}[1]{\textit{#1}}
%
\providecommand*{\aka}{\lingform{aka}}
\providecommand*{\eg}{\lingform{e.g.}}
\providecommand*{\ie}{\lingform{i.e.}}
%
% quotation marks
%
\providecommand*{\scare}[1]{\guillemotleft\,{#1}\,\guillemotright}

% geometric objects
%
% spaces
%
\providecommand*{\set}[1]{\mathbb{#1}}      % set
\providecommand*{\espace}{\set{E}}          % euclidean space
\providecommand*{\nespace}[1]{\espace^{#1}} % n-dim euclidean space
%
\providecommand*{\xset}[1]{\mleft\lbrace{#1}\rbrace\mright} % expanded set
%
% mechanics
%
\providecommand*{\acc}{a} % acceleration
\providecommand*{\pos}{x} % position
\providecommand*{\vel}{v} % velocity
%
% derivatives
%
\providecommand*{\dt}[1]{\dot{#1}}          % dot time derivative
\providecommand*{\ddt}[1]{\ddot{#1}}        % dot, dot time derivative
\providecommand*{\bder}{D_t}                % Bianchi derivative
\providecommand*{\dbder}[1]{\dot{\bar{#1}}} % dot Bianchi derivative
\providecommand*{\ipd}[1]{\partial_{#1}}    % index partial derivative

% index notation
%
\providecommand*{\iacood}[1]{\bar{#1}}         % index of an alterate coordinate system
\providecommand*{\cvec}[2]{\tensor{#1}{^{#2}}} % contravariant vector component
\providecommand*{\ccov}[2]{\tensor{#1}{_{#2}}} % covariant vector component
%
\providecommand*{\skchris}[3]{\tensor{\Gamma}{^{#1}_{#2}_{#3}}} % Christoffel symbols -- second kind
%
% coordinates
%
% rectangular coordinates
%
\providecommand*{\xpos}{x}
\providecommand*{\ypos}{y}
\providecommand*{\zpos}{z}
%
% cylindrical-polar coordinates
%
\providecommand*{\pxpos}{r}      % polar 'x' position
\providecommand*{\pypos}{\theta} % polar 'y' position
\providecommand*{\pzpos}{z}      % polar 'y' position
%
% spherical coordianates
%
\providecommand*{\sxpos}{\rho}   % spherical 'x' position
\providecommand*{\sypos}{\theta} % spherical 'y' position
\providecommand*{\szpos}{\phi}   % spherical 'z' position

% various
%
\providecommand*{\eqtext}[1]{\text{[\footnotesize\textsf{#1}]}} % text in equation
\providecommand*{\parens}[1]{\mleft({#1}\mright)}               % parenthesis
\providecommand*{\tuple}[1]{\parens{#1}}                        %
\providecommand*{\sbtxt}[1]{_\text{\textsf{#1}}}                % subscript text
\providecommand*{\spexp}[1]{^{\parens{#1}}}                     % superscript exponent
\providecommand*{\vat}[1]{\parens{#1}}                          % value of a function at a point


\begin{document}
%
\title{Exercises on differential geometry}
\author{dfhe}
\affiliation{ferrodecont}
%
%\maketitle
%

\section{Particle mechanics}
%
\begin{definition}[Particle]
  A \lingo{particle} is an ideal minute fragment of matter bearing physical or chemical properties; \eg, volume or mass.
\end{definition}

\begin{definition}[Point particle]
  A \lingo{point particle} is zero-dimensional particle.
\end{definition}

\begin{note}
  As being zero-dimensional, a point particle lacks spatial extension. Thus, such a representation of an object is appropriate when size, shape, and structure are irrelevant in a given context.
\end{note}

\begin{definition}[Point mass]
  A \lingo{point mass} is a point particle with nonzero mass and no other properties or structure.
\end{definition}

\begin{exercise}
  Consider an object moving around a point. Find the object acceleration as it moves.
\end{exercise}

\begin{solution}
  Model the moving object as a (point) particle. Measure the particle position, $\pos$, with two \emph{local coordinates}: $\tuple{\pxpos, \pypos}$, where $\pxpos$ represents the distance from the particle to the point and $\pypos$ the angle. Thus, as given by the coordinate choice, the chosen positive orientation is \lingo{counterclockwise}. Assume the particle moving in this direction.

In coordinate basis and using , find the differential of the particle position, then its velocity, and finally its acceleration in a coordinate basis.

Find the differential of the particle position, $d\pos$, by differentiating the particle position:
%
\begin{equation*}
  d\pos = \tuple{d\cvec\pos\pxpos, d\cvec\pos\pypos} 
        = \tuple{d\pxpos, d\pypos}\,.
\end{equation*}

Then, the particle velocity, $\vel = \dt\pos$, is
%
\begin{equation*}
  \vel = \tuple{\cvec\vel\pxpos, \cvec\vel\pypos}
       = \tuple{\dt\pxpos, \dt\pypos}\,.
\end{equation*}

To calculate the particle acceleration, $\acc$, apply the \lingo{temporal covariant derivative} (\aka, Bianchi's derivative) to the particle velocity: $\acc = \bder\vel = \dbder\vel$, or, in index notation together with Einstein's summation convention,
%
\begin{equation*}
  \cvec\acc i = \cvec{\dbder\vel}{i} 
                = \cvec{\dt\vel}{i} + \skchris ijk\cvec\vel j\cvec{\dt\pos}{k}\,,
\end{equation*}
%
where the indices $i,j$ run from $\pxpos$ to $\pypos$; \ie,
%
\begin{align*}
  \cvec\acc\pxpos &= \cvec{\dt\vel}{\pxpos} + \skchris{\pxpos}{\pxpos}{\pxpos}\cvec\vel\pxpos\cvec{\dt\pos}{\pxpos} + \dotsb \qquad\text{and} \\
  \cvec\acc\pypos &= \cvec{\dt\vel}{\pypos} + \skchris{\pypos}{\pxpos}{\pypos}\cvec\vel\pypos\cvec{\dt\pos}{\pxpos} + \dotsb \,.
\end{align*}
%
Now, considering that the only nonzero Christoffel's symbols are $\skchris{\pxpos}{\pypos}{\pypos} = -\pxpos$ and $\skchris{\pypos}{\pxpos}{\pypos} = \skchris{\pypos}{\pypos}{\pxpos} = 1/r$ and considering that $\cvec{\dt\vel}{i} = \cvec{\ddt\pos}{i}$, the particle acceleration becomes
%
\begin{equation*}
  \cvec\acc\pxpos = \ddt\pxpos - \pxpos\dt\pypos\spexp 2 \qquad\text{and}\qquad
  \cvec\acc\pypos = \pxpos\ddt\pypos + \dt\pxpos\dt\pypos \,.
\end{equation*}
%
Since $\cvec\acc\pxpos$ and $\cvec\acc\pypos$ are in coordinate basis, they should be transformed to physical basis via the coefficients of the metric, giving
%
\begin{equation*}
  \cvec\acc\pxpos\sbtxt{phy} = \ddt\pxpos - \pxpos\dt\pypos\spexp 2 \qquad\text{and}\qquad
  \cvec\acc\pypos\sbtxt{phy} = \pxpos\dt\pypos + \dt\pxpos\dt\pypos                 \,,
\end{equation*}
%
where $\cvec\acc\pxpos\sbtxt{phy}$ is called the \lingo{radial acceleration} component and $\cvec\acc\pypos\sbtxt{phy}$ the \lingo{circumferential acceleration} component.
%
\end{solution}


\section{Tensors: Local Machinery of Differential Geometry}
%
Physical and engineering laws must be independent of any particular coordinate systems used in describing them mathematically, if they are to be valid. In other words, all physical and engineering equations need to be tensorial or covariant. Therefore, for the reference purpose, in this section, we give the basic formulas from the standard tensor calculus, which is used throughout the text. The basic notational convention used in tensor calculus is Einstein's summation convention over repeated indices.


\subsection{Transformation of Coordinates}
%
Suppose that we have two sets of curvilinear coordinates that are single-valued, continuous and smooth functions of time, $\cvec\pos j = \cvec\pos j\vat t$, $j = 1, \dotsc, m$, and $\cvec{\pos}{\iacood i} = \cvec{\pos}{\iacood i}\vat t$, $i = 1, \dotsc, n$, representing trajectories of motion of some physical or engineering system. Then a general $m\times n$-dimensional transformation (\ie, a nonlinear map) $\cvec\pos j\mapsto\cvec{\pos}{\iacood i}$ is defined by the set of transformation equations
%
\begin{equation}\label{eq:coordinatetransformations}
  \cvec{\pos}{\iacood i} = \cvec{\pos}{\iacood i}\vat{\cvec\pos i}\,,\qquad\text{with}\qquad
  i = 1, \dotsc, n\,;
  j = 1, \dotsc, m\,.
\end{equation}
%
Now, if \lingo{Jacobi's determinant} of this \lingo{coordinate transformations} does not vanish, 
%
\begin{equation*}
  \mleft\vert \ipd j\cvec\pos{\iacood i} \mright\vert \neq 0 \,,
\end{equation*}
%
then the transformation \cref{eq:coordinatetransformations} is reversible and the inverse transformation,
%
\begin{equation*}
  \cvec\pos j = \cvec\pos j\vat{\cvec{\pos}{\iacood i}}\,,
\end{equation*}
%
exists as well.

\begin{example}
In three-dimensional Euclidean space, $\nespace 3$, transformation from rectangular (Cartesian) coordinates $\cvec\pos i$ into spherical coordinates $\cvec{\pos}{\iacood i}$ is given by
%
\begin{equation}\label{eq:sphericalcoordinatetransformations}
  \cvec\pos 1 = \cvec\pos{\iacood 1}\cos\cvec\pos{\iacood 2}\cos\cvec\pos{\iacood 3} \,,\quad
  \cvec\pos 2 = \cvec\pos{\iacood 1}\sin\cvec\pos{\iacood 2}\cos\cvec\pos{\iacood 3} \,,\quad\text{and}\quad
  \cvec\pos 3 = \cvec\pos{\iacood 1}\sin\cvec\pos{\iacood 3}\,.
\end{equation}
%
\end{example}


\end{document}
