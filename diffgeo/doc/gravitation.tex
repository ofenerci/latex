
\section{Geometrodynamics in brief}

In physics, we don't study objects by themselves, but rather \lingo{measurable} properties of the objects and their interrelations (\lingo{physical processes}). Thus, we don't study a body, but rather its electric charge, mass, orbit, or size and their relationships, such as law of motions. 

In this book, we use \lingo{geometric objects} to model object properties and to measure such properties. Additionally, we use \lingo{geometric transformations} to quantify physical processes.

Our main tool for studying nature is the \lingo{geometric principle}, which states that
%
\begin{quotation}
the laws of physics must all be expressible as geometric (coordinate-independent and reference-frame-independent) relationships between geometric objects (scalars, vectors, tensors), which represent physical entitities.
\end{quotation}

Don't describe motion by farewell objects. \lingo{Physics is simple only when analyzed locally}.

Space tells matter how to move. Matter tells space how to curve.


\subsection{Spacetime with and without coordinates}

Problem: how to measure in curved spacetime. Resolution: characterize events by what happens there. Events are points in spacetime. The name of an event can even be arbitrary. Coordinates provide a convenient naming system. To order events, introduce coordinates.

Coordinates generally do \emph{not} measure length. They are more like telephone numbers: they serve to localize an phone-line owner, but not to measure the distance between owners. Several coordinate systems can be used at once.

Choose two neighbor events, known as such by the nearness of their coordinate values in a smooth coordinate system. Draw an arrow from one to another, call the arrow a \lingo{vector}. Vectors can be given a convenient name, components, but it is still a convenience; the vector exists even without it.

Coordinate singularities are normally unavoidable. The solution is to use \lingo{coordinate patches}.

Spacetime is assumed to be continuous -- to apply calculus, just like the continuum hypothesis in continuum mechanics. The math framework to analyze the physics of spacetime is the math of manifolds (differentiable manifolds). Spacetime smoothness, continuity, breaks down at Planck length.


\subsection{Weightlessness}

Free fall is the natural state of motion. All objects fall with the same acceleration. Eliminate acceleration by use of a local inertial frame of reference.

For Newton space is absolute and independent on anything external to it. Newton's space is unobservable, nonexistent. But Einstein's local inertial frames exist and are simple. In local inertial frames, physics is Lorentzian.


\subsection{Local Lorentzian geometry with and without coordinates}

Local Lorentz geometry is the analog of local Euclidean geometry.


\subsection{Time}

The time coordinate of a Local Lorentz frame is so define that motion looks simple. Relative to a local Lorentz frame, a particle moves in a straight line with constant velocity.

Good clocks make spacetime trajectories of free particles look straight.


\subsection{Curvature}

Gravitation is manifest in relative acceleration of neighboring test particles. 

Relative acceleration is caused by (spacetime) curvature. Consider a sphere of radius $a$. The separation of nearby geodesics satisfies the \lingo{equation of geodesic deviation},
%
\begin{equation}\label{eq:geodesicdeviationsphere}
  d_{ss}\svec + R\svec = 0\,.
\end{equation}
%
Here, $R = 1/a^2$ is the so-called \lingo{Gaussian curvature of the surface}. For the surface of an apple, the same equation applies, with the one difference that the curvature $R$ varies from place to place.

In a space of more than two dimensions, an equation of the same general form applies, with several differences. In two dimensions, the \emph{direction} of acceleration of one geodesic relative to a nearby, fiducial geodesic is fixed uniquely by the demand that their separation vector, $\svec$, be perpendicular to the fiducial geodesics. Not so in three dimension or higher. There $\svec$ can remain perpendicular to the fiducial geodesic but rotate about it. Thus, to specify the relative acceleration, one must give not only its magnitude, but also its direction.

The relative acceleration in three dimensions and higher, then, is a vector. Call it $D_{ss}\svec$, and call its four components $D_{ss}\tvec\svec\alpha$. Why the capital $D$? Why not $d_{ss}\svec$? Because our coordinate separation separation is completely arbitrary. The twisting and turning of the coordinate lines can induce changes from point to point in the components $\tvec\svec\alpha$ of $\svec$, even if the vector $\svec$ is not changing at all. Consequently, the acceleration of the components $D_{ss}\svec$ are generally not equal to the components $D_{ss}\tvec\pos\alpha$ of the acceleration.

How, then, in curved spacetime, can one determine the components $D_{ss}\tvec\svec\alpha$ of the relative acceleration? By a more complicated version of the equation of the geodesic deviation \cref{eq:geodesicdeviationsphere}. Differential geometry provides us with a geometric object called \lingo{Riemann's curvature tensor}, $\riemann$. Riemann's tensor is the higher-dimensional analog of the Gaussian curvature $R$ of our apple's surface. Riemann's tensor is the math embodiment of the bends and warps in spacetime. And $\riemann$ is the agent by which those bends and warps (\lingo{curvature of spacetime}) produce the relative acceleration of geodesics; that is,
%
\begin{quotation}
  Curvature is characterized by \lingo{Riemann's curvature tensor}. Riemann tensor, through the equation of geodesic deviation, produces relative accelerations.
\end{quotation}
%
Riemann's tensor, like the metric tensor, can be thought of as a family of machines, one machinery residing at each event in spacetime. Each machine has three slots for the insertion of three vectors:
%
\begin{equation*}
  \riemann\vat{\slot,\slot,\slot}\,.
\end{equation*}
%
Choose a fiducial geodesic (free-particle world line) passing through an event $\event Q$, and denote its unit tangent vector (particle velocity) there by
%
\begin{equation*}
  \vel = d_\tau\pos\,;\qquad\text{components,}\qquad \tvec\vel\alpha = d_\tau\tvec\pos\alpha\,.
\end{equation*}
%
Choose another, neighboring geodesic, and denote by $\svec$ its perpendicular separation from the fiducial geodesic. There, insert $\vel$ into the first slot of Riemann's tensor $\riemann$ at $\event Q$, $\svec$ into the second slot, and $\vel$ into the third. Riemann's tensor will grind for awhile; then out will pop a new vector, $\riemann\vat{\vel,\svec,\vel}$.

The equation of geodesic deviation states that this new vector is the negative of the relative acceleration of the two geodesics:
%
\begin{equation*}
  D_{\tau\tau}\svec + \riemann\vat{\vel,\svec,\vel} = 0\,.
\end{equation*}
%
Riemann's tensor, like the metric tensor, and like all other tensor, is a linear machine. The vector it puts out is a linear function of each vector inserted into a slot:
%
\begin{equation*}
  \riemann\vat{2u, aw + bv, 3z} = 2\times a\times 3\,\riemann\vat{u,w,z} + 2\times b\times 3\,\riemann\vat{u,v,z}\,.
\end{equation*}
%
Consequently, in any coordinate system, the components of the vector put out in:
%
\begin{equation*}
  \rho = \riemann\vat{\vel,\svec,\vel}
  \iff
  \tvec\rho\alpha = \triemann\alpha\beta\gamma\delta\,\tvec\vel\beta\,\tvec\svec\gamma\,\tvec\vel\delta\,.
\end{equation*}
%
(Here there is an implied summation in the indices $\beta$, $\gamma$, and $\delta$.) The $4\times 4\times 4\times 4 = 256$ numbers $\triemann\alpha\beta\gamma\delta$ are called the \lingo{components of Riemann's tensor in the given coordinate system}. In term of components, the equation of geodesic deviation states
%
\begin{equation*}
  D_{\tau\tau}\tvec\svec\alpha + \triemann\alpha\beta\gamma\delta\,d_\tau\tvec\pos\beta\,\tvec\svec\gamma\,d_\tau\tvec\pos\beta = 0\,.
  \qquad
  \mleft\lbrack
    \tvec{\ddt\svec}\alpha + \triemann\alpha\beta\gamma\delta\,\tvec{\dt\pos}\beta\,\tvec\svec\gamma\,\tvec{\dt\pos}\beta = 0
  \mright\rbrack
\end{equation*}

In Einstein's geometric theory of gravity, this equation of geodesic deviation summarizes the entire effect of geometry on matter. It does for gravitation physics what Lorentz's force equation,
%
\begin{equation*}
  D_{\tau\tau}\tvec\pos\alpha - \dfrac{\echarge}{\mass}\tensor{\farad}{^\alpha_\beta}\,d_\tau\tvec\pos\beta = 0\,,
  \qquad
  \mleft\lbrack
    \tvec{\ddt\pos}\alpha - \dfrac{\echarge}{\mass}\tensor{\farad}{^\alpha_\beta}\,\tvec{\dt\pos}\beta = 0
  \mright\rbrack
\end{equation*}
%
does for electromagnetism.


\section{Box: 1.1. Mathematical notation for events, coordinates, and vectors}
%
\lingo{Events} are represented by \lingo{geometric points} written in capital letters, $\event P$, $\event Q$, $\event R$, sometimes with subscripts added, $\event P_1$, $\event P2$, $\event P3$.

\lingo{Coordinates of an event $\event P$} are denoted by $t\vat{\event P}$, $\xpos\vat{\event P}$, $\ypos\vat{\event P}$, $\zpos\vat{\event P}$, or indexed by $\tvec{\pos}{0}\vat{\event P}$, $\tvec{\pos}{1}\vat{\event P}$, $\tvec{\pos}{2}\vat{\event P}$, $\tvec{\pos}{3}\vat{\event P}$, or more abstractly by $\tvec{\pos}{\mu}\vat{\event P}$ or $\tvec{\pos}{\alpha}\vat{\event P}$, where it is to be understood that \emph{Greek indices} take on any value 0, 1, 2, or 3.

\lingo{Time coordinate}: when one of the former is picked to play this role, $\tvec{\pos}{0}\vat{\event P}$.

\lingo{Space coordinates} are denoted by $\tvec{\pos}{1}\vat{\event P}$, $\tvec{\pos}{2}\vat{\event P}$, $\tvec{\pos}{3}\vat{\event P}$, and sometimes denoted by $\tvec{\pos}{i}\vat{\event P}$ or $\tvec{\pos}{k}\vat{\event P}$. It is to be understood that \emph{Latin indices} take on values 1, 2, or 3.

\lingo{Shorthand notation}: one soon gets tired of writing explicitly the functional dependence of the coordinates, $\tvec{\pos}{\beta}\vat{\event P}$; so one adopts the shorthand notation $\tvec\pos\beta$ for the coordinate event $\event P$, and $\tvec\pos j$ for the space coordinate. One even begin to think of $\tvec\pos\beta$ as representing the event itself, but one must remind oneself that the values of $\tvec\pos 0$, $\tvec\pos 1$, $\tvec\pos 2$, $\tvec\pos 3$ depend not only on the choice of $\event P$ but also on the \emph{arbitrary} choice of coordinates.

\lingo{Other coordinates}, for the \emph{same} event $\event P$ may be denoted $\atvec\pos\alpha\vat{\event P}$ or just $\atvec\pos\alpha$, $\tvec\pos{\alpha\prime}\vat{\event P}$ or just $\tvec\pos{\alpha\prime}$, $\tvec\pos{\hat\alpha}\vat{\event P}$ or just $\tvec\pos{\hat\alpha}$. The bars, hats, and primes distinguish one coordinate system from another; by putting them on the indices rather than on the $\pos$'s, we simplify later notation. For instance, we may refer to the \emph{same} event by $\tuple{\tvec\pos 0, \tvec\pos 1} = \tuple{77.2, 22.6}$ in a coordinate system and by $\tuple{\tvec\pos{\bidx 0}, \tvec\pos{\bidx 1}} = \tuple{18.5, 51.4}$ in another coordinate system.

\lingo{Transformations}, from one coordinate system to another is achieved by the four functions
%
\begin{equation*}
  \atvec\pos 0\vat{\tvec\pos 0, \tvec\pos 1, \tvec\pos 2, \tvec\pos 3} \,,\quad
  \atvec\pos 1\vat{\tvec\pos 0, \tvec\pos 1, \tvec\pos 2, \tvec\pos 3} \,,\quad
  \atvec\pos 2\vat{\tvec\pos 0, \tvec\pos 1, \tvec\pos 2, \tvec\pos 3} \,,\quad\text{and}\quad
  \atvec\pos 3\vat{\tvec\pos 0, \tvec\pos 1, \tvec\pos 2, \tvec\pos 3} \,,
\end{equation*}
%
which are denoted more succintly $\atvec\pos\alpha\vat{\tvec\pos\beta}$.

\lingo{Separation vector} (little arrow) reaching from one event $\event Q$ to a neighboring event $\event P$ can be denoted abstractly by 
$u$, or $v$, or $\svec$, or $\event P - \event Q$. It can also be characterized by the coordinate value differences between $\event P$ and $\event Q$ (called \lingo{components} of the vector)
%
\begin{equation*}
  \tvec\svec\alpha \defas \tvec\pos\alpha\vat{\event P} - \tvec\pos\alpha\vat{\event Q}     \quad\text{and}\quad
  \atvec\svec\alpha \defas \atvec\pos\alpha\vat{\event P} - \atvec\pos\alpha\vat{\event Q}  \,.
\end{equation*}

\lingo{Transformation of components} of a vector from one coordinate system to another is achieved by \lingo{partial derivatives of the transformation equations}
%
\begin{equation*}
  \atvec\svec\alpha = \dfrac{\partial\atvec\pos\alpha}{\partial\tvec\pos\beta}\tvec\svec\beta
                    = \ipd\beta\atvec\pos\alpha\,\tvec\svec\beta \,,
\end{equation*}
%
since $\atvec\svec\alpha \defas \atvec\pos\alpha\vat{\event P} - \atvec\pos\alpha\vat{\event Q}$.

\lingo{Einstein's summation convention} is used here: any index that is repeated in a product is automatically summed on
%
\begin{equation*}
  \ipd\beta\atvec\pos\alpha\,\tvec\svec\beta = \dfrac{\partial\atvec\pos\alpha}{\partial\tvec\pos\beta}\tvec\svec\beta
                                             = \sum_{\beta = 0}^{3}\dfrac{\partial\atvec\pos\alpha}{\partial\tvec\pos\beta}\tvec\svec\beta\,.
\end{equation*}


\section{Box: 1.3. Local Lorentz geometry and local Euclidean geometry with and without coordinates}
%
\subsection{Local geometry in the language of modern mathematics}
%
\subsubsection{The metric for any manifold}
%
At each event of spacetime, indeed, at each point of any \lingo{Riemannian manifold}, there exists a geometrical object called the \lingo{metric tensor} $\metric$. It is a machine with two input slots for the insertion of two \emph{vectors}:
%
\begin{equation*}
  \metric\vat{\slot, \slot}\,.
\end{equation*}
%
If one inserts the same vector $u$ into both slots, then one gets out the square of the length of $u$, $\metric\vat{u,u} = \parens{u}^2$. If, on the other hand, one inserts two different vectors, $u$ and $v$, (it matters \emph{not} in which order!), then one gets out a number called the \lingo{scalar product of $u$ on $v$} and denoted $u\iprod v$:
%
\begin{equation*}
  \metric\vat{u,v} = \metric\vat{v,u}
                   = u\iprod v
                   = v\iprod u\,.
\end{equation*}
%
The metric is a \lingo{linear machine}:
%
\begin{equation*}
  \metric\vat{u, av + bw} = a\metric\vat{u, v} + b\metric\vat{u, v}\,.
\end{equation*}
%
Consequently, in a given (arbitrary) coordinate system, its operation on two vectors can be written in terms of their components as a bilinear expression:
%
\begin{equation*}
  \metric\vat{u,v} = \tcovmet\alpha\beta\tvec u\alpha\tvec v\beta
                   = \tcovmet 11\tvec u1\tvec v1 + \tcovmet 12\tvec u1\tvec v2 + \tcovmet 21\tvec u2\tvec v1 + \dotsb\,.
\end{equation*}
%
The quantities $\tcovmet\alpha\beta = \tcovmet\beta\alpha$ ($\alpha,\beta$ running from 0 to 3 in spacetime) are called the \lingo{components of $\metric$ in the given coordinate system}.


\subsubsection{Components of the metric in local Lorentz and local Euclidean frames}
%
To connect the metric with our previous description of local geometry, introduce local Euclidean coordinates (on apple) or local Lorentz coordinates (in spacetime). Let $\svec$ be the separation vector reaching from $\point A$ to $\point B$. Its components in the local Euclidean (Lorentz) coordinates are
%
\begin{equation*}
  \tvec\svec\alpha = \tvec\pos\alpha\vat{\point B} \tvec\pos\alpha\vat{\point A}\,.
\end{equation*}
%
Then, the squared length of $\pos_{\point A\point B}$, which is the same as the squared distance from $\point A$ to $\point B$, must be
%
\begin{equation*}
  \svec\iprod\svec = \metric\vat{\svec, \svec}
                   = \tcovmet\alpha\beta\tvec\pos\alpha\tvec\pos\beta
                   = \parens{\pos_{\point A\point B}}^2
                   = -\parens{\tvec\svec 0}^2 + \parens{\tvec\svec 1}^2 + \parens{\tvec\svec 2}^2 + \parens{\tvec\svec 3}^2\,.
\end{equation*}
%
Consequently, the components of the metric are
%
\begin{equation*}
  \tcovmet\alpha\beta = \diag\tuple{1,1,1,1}\,;
\end{equation*}
%
on apple, in local Euclidean coordinates;
%
\begin{equation*}
  \tcovmet{\hidx\alpha}{\hidx\beta} = \diag\tuple{-1,1,1,1}\,;
\end{equation*}
%
in spacetime, in local Lorentz coordinates.

These special components of the metric in local Lorentz coordinates are written here and hereafter as $\tcovmet{\hidx\alpha}{\hidx\beta}$.


\subsection{Statements of facts}
%
The geometry of an apple's surface is locally Euclidean everywhere. The geometry of spacetime is locally Lorentzian everywhere.


\section{Test for flatness}
%
\begin{itemize}
  \item Specify the extension in space $\length$ (\si{cm} or \si{m}) and extension in time $t$ (\si{cm} of \si{m} of light travel distance) of the region under study.
  \item Specify the precision $\delta\svec$ with which we can measure the separation of test particles in this region.
  \item Follow the motion of test particles moving along initially parallel world lines through the region of spacetime.
  \item When the world lines remain parallel to the precision $\delta\svec$ for all directions of travel, then one says that \scare{in a region so limited and to a precision so specified, spacetime is flat}.
\end{itemize}



\subsection{Modern thermodynamics pearls!}
%
\subsubsection{Conservation of energy [p. 25-26]}
%
The first law of thermodynamics states that energy obeys a local conservation law. By this we mean something very specific:
%
\begin{quotation}
  Any decrease in the amount of energy in a given region of space must be exactly balanced by a simultaneous increase in the amount of energy in an adjacent region of space.
\end{quotation}
%
Note the adjectives \emph{simultaneous} and \emph{adjacent}. The laws of physics do not permit energy to disappear now and reappear later. Similarly the laws do not permit energy to disappear from here and reappear at some distant place. \emph{Energy is conserved right here, right now}.

It is usually possible to observe and measure the physical processes whereby energy is transported from one region to the next. This allows us to express the energy-conservation law as an equation:
%
\begin{equation*}
  \text{change in energy (inside boundary)} = \text{net flow of energy (inward minus outward across boundary)}\,.
\end{equation*}

The word flow in this expression has a very precise technical meaning, closely corresponding to one of the meanings it has in everyday life.

There is also a \lingo{global law of conservation of energy}: The total energy in the universe cannot change. The local law implies the global law but not conversely. The global law is interesting, but not nearly as useful as the local law, for the following reason: suppose I were to observe that some energy has vanished from my laboratory. It would do me no good to have a global law that asserts that a corresponding amount of energy has appeared \scare{somewhere} else in the universe. There is no way of checking that assertion, so I would not know and not care whether energy was being globally conserved. Also it would be very hard to reconcile a non-local law with the requirements of special relativity.

[...], there is an important distinction between the notion of conservation and the notion of constancy. Local conservation of energy says that the energy in a region is constant except insofar as energy flows across the boundary.


\subsubsection{General Case : Some Energy Not Available [p. 31]}
%
This sheds an interesting side-light on the energy-conservation law. As with most laws of physics, this law, by itself, does not tell you what will happen; it only tells you what \emph{cannot} happen: you cannot have any process that fails to conserve energy. To say the same thing another way: 
%
\begin{quotation}
  if something is prohibited by the energy-conservation law, the prohibition is absolute, whereas if something is permitted by the energy-conservation law, the permission is conditional, conditioned on compliance with all the other laws of physics. 
\end{quotation}
%
In particular, [...], you have to comply with all the laws, not just conservation of energy. You also have to conserve angular momentum. You also have to comply with the second law of thermodynamics.
