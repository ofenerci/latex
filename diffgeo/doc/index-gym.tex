\section{Index gymnastics}

\begin{remark}[Index gymnastics]
  Consider Riemann's tensor, a $\torder 40$ tensor. When Riemann is \scare{empty} (no vectors inserted into it), the correspondence between its slot and index representations is
  %
  \begin{equation*}
    \riemann\vat{\vslot,\vslot,\vslot,\vslot} \iff \tensor{\riemann}{^\alpha^\beta^\gamma^\delta}\,.
  \end{equation*}
  %
  Note that, since all of the vector slots are empty, all of the indices are up. 
  
  However, as soon as a vector is inserted into a Riemann's slot, its correspondent index lowers to obey Einstein's summation convention. For instance, when a velocity $\vel$ is inserted in the second and fourth slots, the slot and index representations become
  %
  \begin{equation*}
    \riemann\vat{\vslot,\vel,\vslot,\vel} \iff \tensor{\riemann}{^\alpha_\beta^\gamma_\delta}\tvec\vel\beta\tvec\vel\delta\,,
  \end{equation*}
  %
  where the value of $\riemann\vat{\vslot,\vel,\vslot,\vel}$ is a $\torder 20$ tensor: two vector slots available, in slot parlance, and two contravariant indices available, in index notation parlance.
  
  With this observation in mind, the celebrated \lingo{equation of geodesic deviation} can be written as
  %
  \begin{equation*}
    D_{\ptime\ptime}\svec + \riemann\vat{\vslot,\vel,\svec,\vel} = 0
    \iff
    D_{\ptime\ptime}\tvec\svec\alpha + \triemann\alpha\beta\gamma\delta\tvec\vel\beta\tvec\svec\gamma\tvec\vel\delta = 0
    \,,
  \end{equation*}
  %
  after the separation (vector) $\svec$ has been plugged into Riemann's third slot. Since only one vector slot is available, $\riemann\vat{\vslot,\vel,\svec,\vel}$ returns a $\torder 10$ tensor (a vector or first-order contravariant tensor), which agrees with the nature of $D_{\ptime\ptime}\svec$, an acceleration.
  
  Finally, if, for whatever reason, another vector, say $\svec$ again, where to be inserted into Riemann's last available slot, then all of Riemann's indices were in the low position and the result will be a scalar value:
  %
  \begin{equation*}
    \riemann\vat{\svec,\vel,\svec,\vel} 
      \iff 
    \tensor{\riemann}{_\alpha_\beta_\gamma_\delta}\tvec\svec\alpha\tvec\vel\beta\tvec\svec\gamma\tvec\vel\delta\,.
      \qed
  \end{equation*}
\end{remark}

\begin{example}
  The metric tensor $\metric$ is a $\torder 20$ tensor; \ie, it accepts two vectors. When $\metric$ is empty, its representations are
  %
  \begin{equation*}
    \metric\vat{\vslot,\vslot}\iff\tvecmet ij\,.
  \end{equation*}
  %
  When two vectors, say $\vec a,\vec b$ are inserted into $\metric$, both of the $\metric$ indices lower producing a scalar
  %
  \begin{equation*}
    \metric\vat{\vec a, \vec b}\iff\tcovmet ij\tvec ai\tvec bj\,. \qed
  \end{equation*}
  %
\end{example}
