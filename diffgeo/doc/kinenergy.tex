
\section{Applications}
%
\subsection{Kinetic energy}
%
\begin{example}
  Express the kinetic energy $\ekin$ of a moving particle of mass $\mass$ traveling with velocity $\vel$.
\end{example}
%
\begin{solution}
  In component-free form, and thus according to the geometric principle, the kinetic energy of the particle is defined as
  %
  \begin{equation*}
    2\ekin = \mass\br{\vel}^2\,.
  \end{equation*}
  %
  In abstract-index form, the kinetic energy becomes
  %
  \begin{equation*}
    2\ekin = \mass\metric\vat{\vel,\vel}\,.
  \end{equation*}
  %
  Then, in spherical \emph{coordinate} basis, the position of the particle is given by
  %
  \begin{equation*}
    \atvec\pos i = \tuple{\sxpos, \sypos, \szpos}\,.
  \end{equation*}
  %
  The particle velocity is thus
  %
  \begin{equation*}
    \atvec\vel i = d_t\atvec\pos i
                 = \tuple{\dt\sxpos, \dt\sypos, \dt\szpos}\,.
  \end{equation*}
  %
  Here's where the magic happens:
  %
  \begin{equation*}
    \metric\vat{\vel,\vel} = \atcovmet ij\,\tvec{\dt\pos}{i}\tvec{\dt\pos}{j}\,,
  \end{equation*}
  %
  where $\metric$ encodes the magic in its coefficients, $\atcovmet ij$, which for spherical coordinates are $\atcovmet ij = \diag\tuple{1, \parens{\sxpos}^2, \parens{\sxpos\sin\sypos}^2}$. Thus, the inner product of $\vel$ with itself becomes
  %
  \begin{equation*}
    \vel\iprod\vel = \parens{\dt\sxpos}^2 + \parens{\sxpos\dt\sypos}^2 + \parens{\sxpos\sin\sypos\dt\szpos}^2\,.
  \end{equation*}
  %
  Finally, the kinetic energy is given by replacing the last equation into \cref{eq:kineticenergyinvariant}:
  %
  \begin{equation*}
    2\ekin = \mass\br{\dt\sxpos^2 + \sxpos^2\,\dt\sypos^2 + \sxpos^2\sin^2\sypos\,\dt\szpos^2}\,.
  \end{equation*}
\end{solution}
