
\section{Particle mechanics}
%
\begin{definition}[Particle]
  A \lingo{particle} is an ideal minute fragment of matter bearing physical or chemical properties; \eg, volume or mass.
\end{definition}

\begin{definition}[Point particle]
  A \lingo{point particle} is zero-dimensional particle.
\end{definition}

\begin{note}
  As being zero-dimensional, a point particle lacks spatial extension. Thus, such a representation of an object is appropriate when size, shape, and structure are irrelevant in a given context.
\end{note}

\begin{definition}[Point mass]
  A \lingo{point mass} is a point particle with nonzero mass and no other properties or structure.
\end{definition}

\begin{exercise}
  Consider an object moving around a point. Find the object acceleration as it moves.
\end{exercise}

\begin{solution}
  Model the moving object as a point particle. Measure the particle position, $\pos$, with two \emph{local coordinates}: $\tuple{\pxpos, \pypos}$, where $\pxpos$ represents the distance from the particle to the point and $\pypos$ the angle the particle traces with respect to the polar axis. Thus, as given by the coordinate choice, the chosen positive orientation is \lingo{counterclockwise}. Assume the particle moving with this orientation.

In coordinate basis, find the differential of the particle position, then its velocity, and finally its acceleration.

Find the differential of the particle position, $d\pos$, by differentiating the particle position:
%
\begin{equation*}
  d\pos = \tuple{d\tvec\pos\pxpos, d\tvec\pos\pypos} 
        = \tuple{d\pxpos, d\pypos}\,.
\end{equation*}

Then, the particle velocity, $\vel = \dt\pos$, is
%
\begin{equation*}
  \vel = \tuple{\tvec\vel\pxpos, \tvec\vel\pypos}
       = \dt\pos
       = \tuple{\dt\pxpos, \dt\pypos}\,.
\end{equation*}

To calculate the particle acceleration, $\acc$, apply the \lingo{temporal covariant derivative} (\aka, Bianchi's derivative) to the particle velocity: $\acc = \bder\vel = \dbder\vel$, or, in index notation together with Einstein's summation convention,
%
\begin{equation*}
  \tvec\acc i = \tvec{\dbder\vel}{i} 
                = \tvec{\dt\vel}{i} + \skchris ijk\tvec\vel j\tvec{\dt\pos}{k}\,,
\end{equation*}
%
where the indices $i,j$ run from $\pxpos$ to $\pypos$; \ie,
%
\begin{align*}
  \tvec\acc\pxpos &= \tvec{\dt\vel}{\pxpos} + \skchris{\pxpos}{\pxpos}{\pxpos}\tvec\vel\pxpos\tvec{\dt\pos}{\pxpos} + \dotsb \qquad\text{and} \\
  \tvec\acc\pypos &= \tvec{\dt\vel}{\pypos} + \skchris{\pypos}{\pxpos}{\pypos}\tvec\vel\pypos\tvec{\dt\pos}{\pxpos} + \dotsb \,.
\end{align*}
%
Now, considering that the only nonzero Christoffel's symbols are $\skchris{\pxpos}{\pypos}{\pypos} = -\pxpos$ and $\skchris{\pypos}{\pxpos}{\pypos} = \skchris{\pypos}{\pypos}{\pxpos} = 1/r$ and denoting $\tvec{\dt\vel}{i}$ by $\tvec{\ddt\pos}{i}$, the particle acceleration becomes
%
\begin{equation*}
  \tvec\acc\pxpos = \ddt\pxpos - \pxpos\parens{\dt\pypos}^2 \qquad\text{and}\qquad
  \tvec\acc\pypos = \pxpos\ddt\pypos + \dt\pxpos\dt\pypos \,.
\end{equation*}
%
Since $\tvec\acc\pxpos$ and $\tvec\acc\pypos$ are in coordinate basis, they should be transformed to physical basis via the coefficients of the metric, giving
%
\begin{equation*}
  \tvec\acc\pxpos\sbtxt{phy} = \ddt\pxpos - \pxpos\parens{\dt\pypos}^2 \qquad\text{and}\qquad
  \tvec\acc\pypos\sbtxt{phy} = \pxpos\dt\pypos + \dt\pxpos\dt\pypos                 \,,
\end{equation*}
%
where $\tvec\acc\pxpos\sbtxt{phy}$ is called the \lingo{radial acceleration} component and $\tvec\acc\pypos\sbtxt{phy}$ the \lingo{circumferential acceleration} component.
%
\end{solution}
