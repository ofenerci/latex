
\section{Modern thermodynamics pearls!}
%
\subsubsection{Conservation of energy [p. 25-26]}
%
The first law of thermodynamics states that energy obeys a local conservation law. By this we mean something very specific:
%
\begin{quotation}
  Any decrease in the amount of energy in a given region of space must be exactly balanced by a simultaneous increase in the amount of energy in an adjacent region of space.
\end{quotation}
%
Note the adjectives \emph{simultaneous} and \emph{adjacent}. The laws of physics do not permit energy to disappear now and reappear later. Similarly the laws do not permit energy to disappear from here and reappear at some distant place. \emph{Energy is conserved right here, right now}.

It is usually possible to observe and measure the physical processes whereby energy is transported from one region to the next. This allows us to express the energy-conservation law as an equation:
%
\begin{equation*}
  \text{change in energy (inside boundary)} = \text{net flow of energy (inward minus outward across boundary)}\,.
\end{equation*}

The word flow in this expression has a very precise technical meaning, closely corresponding to one of the meanings it has in everyday life.

There is also a \lingo{global law of conservation of energy}: The total energy in the universe cannot change. The local law implies the global law but not conversely. The global law is interesting, but not nearly as useful as the local law, for the following reason: suppose I were to observe that some energy has vanished from my laboratory. It would do me no good to have a global law that asserts that a corresponding amount of energy has appeared \scare{somewhere} else in the universe. There is no way of checking that assertion, so I would not know and not care whether energy was being globally conserved. Also it would be very hard to reconcile a non-local law with the requirements of special relativity.

[...], there is an important distinction between the notion of conservation and the notion of constancy. Local conservation of energy says that the energy in a region is constant except insofar as energy flows across the boundary.


\subsubsection{General Case : Some Energy Not Available [p. 31]}
%
This sheds an interesting side-light on the energy-conservation law. As with most laws of physics, this law, by itself, does not tell you what will happen; it only tells you what \emph{cannot} happen: you cannot have any process that fails to conserve energy. To say the same thing another way: 
%
\begin{quotation}
  if something is prohibited by the energy-conservation law, the prohibition is absolute, whereas if something is permitted by the energy-conservation law, the permission is conditional, conditioned on compliance with all the other laws of physics. 
\end{quotation}
%
In particular, [...], you have to comply with all the laws, not just conservation of energy. You also have to conserve angular momentum. You also have to comply with the second law of thermodynamics.
