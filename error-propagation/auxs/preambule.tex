% preambule
% provides definitions, commands and environments

% ------------------------------------------------------------- Epigraph
%\docepigraph{Text}{Author}{Source}
\newcommand{\docepigraph}[3]
{
\makeatletter
  \begingroup\vspace{0.5cm}
    \begin{flushright}
      \begin{minipage}[H]{0.9\linewidth}
        \small\textsf{#1}% text...quote
        \begin{flushright}
          % author, reference
          {--- \sc\MakeLowercase{#2}}, {\sc\MakeLowercase{#3}}
        \end{flushright}
      \end{minipage}
    \end{flushright}
  \endgroup\vspace{0.5cm}
\makeatother
}

% -------------------------------------------------------- Verbatim file
% File name:
% Name, location, beg. line {hello.f90}{./main/hello.f90}{1}
% ------------------------------------------------------------
%\newcommand{\docfileverb}[3]%
%{%
%     \medskip%
%     \noindent\peqmayus{File name:} {\code{\color{webgreen} #1}}%
%
%     \begingroup%
%     \linenumbers*[#3]%
%     \small%
%     \verbatiminput{#2}%
%     \endgroup%
%}%

% ------------------------------------------------------------ Equations
\def\beq{\begin{equation*}}%
\def\eeq{\end{equation*}}%

% ----------------------------- Majuscules and minuscules, old ligatures
\newcommand{\mayus}[1]{{\spacedallcaps{#1}}}      % Capitals
\newcommand{\minus}[1]{{\spacedlowsmallcaps{#1}}} % Small caps

\newcommand{\oldlig}[1]%
{%
  {%
  \fontspec[Ligatures=Rare]{MinionPro-Regular}%
  \textit{% only noticeable with italics
     {#1}%
   }}%
}%

% -------------------------------------------------------- Ling. Formats
% Code
\newcommand{\code}[1]{\texttt{#1}} % in typewritter

% Quotation marks
\newcommand{\dfscare}[1]{«\,{#1}\,»} % double french scare
\newcommand{\scare}[1]{\dfscare{#1}}

% Linguistic form (foreign language words and so on)
\newcommand{\lingform}[1]{\textsl{#1}}
\newcommand{\lingo}[1]{\textit{#1}}
%
% Maths and physics
\newcommand{\fact}[1]{\oldlig{#1}}    % scientific facts
\newcommand{\theorem}[1]{\oldlig{#1}} % lebesgue's style for theorems

% Margin notes
\newcommand{\note}[1]{\mgraffito{#1}}

% ------------------------------------------------- Common abbreviations
% "sc." and "viz." introduce a clarification; "i.e.", an equivalence
%
\newcommand{\aka}{\lingform{aka}\xspace}                     % also known as
\newcommand{\abinitio}{\lingform{ab initio}\xspace}          % from first principles

\newcommand{\adhoc}{\lingform{ad hoc}\xspace}                % for this: designed with specific purpose

\newcommand{\aposteriori}{\lingform{a posteriori}\xspace}    % from what is after
\newcommand{\apriori}{\lingform{a priori}\xspace}            % from what is before
\newcommand{\apropos}{\lingform{á propos}\xspace}            % to the purpos, -(of) concerning
\newcommand{\ca}{\lingform{ca.}\xspace}                      % circa, approx
\newcommand{\confer}{\lingform{cf.}\xspace}                  % compare, confer
% if one folds hid arms, so does the other; if one crosses his legs, ditto
\newcommand{\ditto}{\lingform{ditto}\xspace}                 % dictussaid
\newcommand{\eg}{\lingform{e.g.}\xspace}                     % exempli gratia
\newcommand{\etc}{\lingform{\&c.}\xspace}                    % exempli gratia
\newcommand{\ibidem}{\lingform{ibid.}\xspace}                % in the same place
\newcommand{\ie}{\lingform{i.e.}\xspace}                     % id est
\newcommand{\interalia}{\lingform{i.a.}\xspace}              % among other things: Hemingway- author (i.a. 'The Sun Also Rises')
\newcommand{\nbene}[1]{\mgraffito{\lingform{N.B.}\,: {#1}}}  % pay attention
\newcommand{\quodvide}{\lingform{q.v.}\xspace}               % which to see
\newcommand{\quaevide}{\lingform{qq.v.}\xspace}              % plural of q.v.
\newcommand{\sic}{[\lingform{sic}]\xspace}                   % so or thus
\newcommand{\scilicet}{\lingform{sc.}\xspace}                % "one may know" same "videlicet"
\newcommand{\videinfra}{\lingform{v.i.}\xspace}              % see below
\newcommand{\videlicet}{\lingform{viz.}\xspace}              % that is to say, namely
\newcommand{\videsupra}{\lingform{v.s.}\xspace}              % see above
\newcommand{\viceversa}{\lingform{vice versa}\xspace}        % italics
\newcommand{\vs}{\lingform{vs.}\xspace}                      % versus

% vis-à-vis literally means 'face to face.' Avoid using it to mean 'about,
% concerning,' as in : he wanted to talk to me vis-à-vis next weekend.
% sense 'in contrast, comparison, or relation to,' however, vis-à-vis is 
% generally acceptable: : let us consider government regulations vis-à-vis 
% employment rates.
% preposition
% in relation to; with regard to : many agencies now have a unit to deal with
% women's needs vis-à-vis employment.
% * as compared with; as opposed to : the advantage for U.S. exports is the 
% value of the dollar vis-à-vis other currencies.
\newcommand{\visavis}{\lingform{vis-à-vis}\xspace}%

% Author's signature
%\newcommand{\authorSignature}{%
%   \begin{flushright}%
%       \begin{tabular}{m{5cm}}%
%          \\ \hline%
%           \centering\docAuthor \\%
%       \end{tabular}%
%    \end{flushright}%
%}%

% >>>>>>>>>>>>>>>>>>>>>>>>>>>>>>>>>>>>>>>>>>>>>>>>>>>>>>>>>>>>>>> Floats
% Floats width
\newcommand{\docfloatwidth}{0.95\textwidth}%

% Table headline
\newcommand{\tabhead}[1]{{\spacedlowsmallcaps{#1}}}%

% Table element centered
\newcommand{\TEC}[1]{\multicolumn{1}{c}{#1}}%

% ------------------------------------------------------------- PreTable
% \docpretable{position}{size}{cols}%
% position: bthH. size: 0.9\textwidth. cols: llcp{6mm}
% use: \docfloatwidth whenever possible!
% NOTE: does not include \toprule
\newcommand{\docpretable}[3]{%
    \begin{table}[#1]%
      \capstart\center\footnotesize%
        \begin{tabularx}{#2}%
          {#3}%
}%

% ------------------------------------------------------------ PostTable
% \end{tabularx}
% \docposttable{caption}{caption}{label}
% include: \end{tabularx}%
\newcommand{\docposttable}[3]{%
% \end{tabularx}%
      \normalsize\caption[#1]{#2}\label{#3}%
    \end{table}%
}%
% ------------------------------------------------------------- EndTable

% --------------------------------------------------------------- Figure
%
% position: bthH. size:width=0.5\textwidth. file:location+filename.pdf
% caption. label:fig:wec
% use: \docfloatwidth whenever possible!
% \docfigure{position}{size}{file}%
%   {caption}%
%   {caption}%
%   {label}%
\newcommand{\docfigure}[6]{%
    \begin{figure}[#1]%
      \capstart\center%
        {\center\includegraphics[#2]{#3}}%
      \caption[#4]{#5}%
    \label{#6}%
\end{figure}%
}% end figure
%
% ------------------------------------------------------------ EndFigure

% --------------------------------------------------------- Tikpz Figure
% \tikzgraph{pos}{size}{prelim}{figure code}{ref caption}{caption}{label}
% position: bthH
% size: 0.9\textwidth. Use: \docfloatwidth whenever possible!
% preliminaries: x=2.0cm, y=1.0cm,only marks,...
% refcaption: caption in TOC
% caption: figure caption in text
%\newcommand{\tikzgraph}[7]{% begin
%    \begin{figure}[#1]\capstart\centerline{%
%      \resizebox{#2}{!}{%
%        \begin{tikzpicture}[#3]%
%          {#4}%
%        \end{tikzpicture}%
%           }%
%         }%
%      \caption[#5]{#6}\label{#7}%
%    \end{figure}%
%}% end tikpzfigure
%%

% siunitx setup
%
\sisetup{
  input-symbols = \pi\tau,
  exponent-product = \cdot  % use cdot instead of times in sci notation
}
