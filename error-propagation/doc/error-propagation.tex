\chapter*{Propagation of errors in calculations}
%
[Adapted from \citep[p. 47]{oldenburg:2014}]


\newcommand{\period}{t}


\section*{Description}
%
Consider the quantities $a_k$s, with measured values $\mvalue a_k$s and measured uncertainty $\muncer a_k$s, and the quantities $b_l$s, with sample mean values $\smean b_l$s and sample standard deviations $\sstdev_{b_l}$s; \ie, the $\mvalue a_k$s arise from single measurements and the $\smean b_l$s from multiple measurements. Consider now a function $f$ that depends on the $a_k$s and on the $b_l$s. Then, the most likely value of $f$ is given by
%
\beq
    \mvalue y = f\vat{\mvalue a_k, \smean b_l}\,,
\eeq
%
whereas its maximum uncertainty by
%
\beq
    \muncer y = \ssum{k}{\abs\left(\ipd{a_k}y\right)\muncer a_k} 
                + \ssum{l}{\abs\left(\ipd{b_l}y\right)\sstdev_{b_l}}\,,
\eeq
%
where $\abs x$ represents the absolute value of $x$.


\section*{Example}
%
Determine the local free fall acceleration by the period of a mathematical pendulum of length $\length/\si{m} = 1.00 \pm 0.01$ and period $\period/\si{s} = 2.0062 \pm 0.0057$. The length was measured once and the period thousand times (Monte Carlo simulation).

The period of a math pendulum of length $\length$ is given by
%
\beq
    \period = 2\pi\sqrt{\dfrac{\length}{\grav}}\,,
\eeq
%
where $\grav$ represents the local free fall acceleration.

From the pendulum equation, isolate $\grav$ to have
%
\beq
    \grav = 4\pi^2\dfrac{\length}{\period^2}\,.
\eeq

Then, the max uncertainty of $\grav$ is
%
\beq
    \muncer\grav = \abs\left(\ipd{\length}\grav\right)\muncer\length 
                   + \abs\left(\ipd{\period}\grav\right)\sstdev_\period
                 = 4\pi^2\dfrac{1}{\period^2}\muncer\length + 8\pi^2\dfrac{\length}{\period^3}\sstdev_\period\,.
\eeq

Divide the last equation by the equation for $\grav$ to have the relative uncertainty (fractional change) of $\grav$
%
\beq
    \dfrac{\muncer\grav}{\grav} = \dfrac{\muncer\length}{\length} + 2\dfrac{\sstdev_\period}{\period}\,.
\eeq

Replace the given values in the $\grav$ equation and in the equation for its relative uncertainty to find
%
\begin{align*}
    \grav &= 4\pi^2\dfrac{\length}{\period^2} 
          = 4\pi^2\dfrac{1.00}{2.0062^2}
          = 9.808696222936447\,,\\
%
    \muncer\grav &= \grav\left(\dfrac{\muncer\length}{\length} + 2\dfrac{\sstdev_\period}{\period}\right)\\
                 &= 9.808696222936447\left(\dfrac{0.01}{1.00} + 2\dfrac{0.0057}{2.0062}\right)\\
                 &= 0.153823746668\,.
\end{align*}

Finally, the value for the free fall acceleration is $\grav/\si{m.s^{-2}} \sim \num{9.809 +- 0.154}$.

