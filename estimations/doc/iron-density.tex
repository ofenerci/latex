\chapter*{Iron mass density}
%
Reference iron mass density value: \SI{7.874}{g.cm^{-3}}, \citep{wiki:iron}.


\section*{Guesstimation}
%
Iron is heavier than water and lighter than mercury, so estimate its mass density $\dens$ value as
%
\beq
    \dens = \sqrt{\SI{1}{g/cm^3}\cdot\SI{14}{g/cm^3}}
          \sim\SI{3.741657386773941}{g/cm^3}\,,
\eeq
%
which yields our first estimate for the value of iron mass density as \SI{3}{g/cm^3}.


\section*{Mahajan equation}
%
According to \citep[p. 57]{sanjoy:2008}, mass density of elements can be estimated via their atomic masses, $A$:
%
\beq
    \dens/\si{g.cm^{-3}} = \dfrac{A}{18}\,.
\eeq

With this model, estimate the mass density of iron as $55/18 \sim \SI{3}{g.cm^{-3}}$.


\section*{Atomic radius}
%
The atomic radius of iron is \SI{126}{pm}, \citep{wiki:iron}. Round this value to \SI{e-8}{cm}.

Consider atoms to be cubes with sides equal to two times their atomic radius (one atomic diameter). Then, the volume of an iron atom will be $\left(2\cdot\SI{e-8}{cm}\right)^3\sim\SI{e-23}{cm^3}$. 

An iron atom has 56 nucleons (neutrons and protons), so its mass is 56 times the proton mass: $56\cdot\SI{2e-24}{g}\sim\SI{e-22}{g}$.

Thus, our iron mass density estimate becomes $\SI{e-22}{g}/\SI{e-23}{cm^3}\sim\SI{10}{g/cm^3}$.

Finally, by using the same model (cubic atom), by using all the exact values for the properties and by carrying all the decimals in the calculations, one finds an estimate of iron mass density of \SI{6.14}{g/cm^3}.
