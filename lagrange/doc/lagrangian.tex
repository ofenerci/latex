\chapter{Lagrangian mechanics}
%
[Based upon \citep{tong:2004}]


\section{The principle of least action}
%
Part of the power of the Lagrangian formulation over the Newtonian approach is that it does away with vectors in favor of more general coordinates. Let's write the positions of $\npart$ particles with coordinates $\ivec\pos i$, where $i = 1,\dotsc,3\npart$. Then Newton's equations read
%
\begin{equation}
  \icov{\dt\mom} i = -\ipd{\ivec\pos i}\epot\,,
\end{equation}
%
where $\icov\mom i = \mass_i\ivec{\dt\pos} i$. The \lingo{number of degrees of freedom} of the system is said to be $3\npart$. These parameterize a $3\npart$-dimensional space known as the \lingo{configuration space} $\cspace$. Each point in $\cspace$ specifies a configuration of the system; \ie, the positions of \emph{all} $\npart$ particles. \lingo{Time evolution} gives rise to a curve in $\cspace$, \vide \cref{fig:configurationspace}.
%
%%% Figure
%
% position: bthH. size:width=0.5\textwidth. file:location+filename.pdf
% caption. label:fig:wec
\docfigure{bt}{width=0.7\textwidth}{./graph/config-space.pdf}{Configuration space}
  {The path of particles in real space (on the left) and in configuration space (on the right).}
  {fig:configurationspace}


\subsection{Lagrangian}
%
Define the Lagrangian to be a function of the positions $\ivec\pos i$ and the velocities $\ivec{\dt\pos}{i}$ of all the particles, given by
%
\begin{equation}
  \elag = \ekin - \epot \implies
  \elag\vat{\ivec\pos i, \ivec{\dt\pos} i} = \ekin\vat{\dt\pos} - \epot\vat\pos\,,
\end{equation}
%
where $\ekin$ is the kinetic energy and $\epot$ the potential energy. Note the minus sign between energies! To describe the principle of least action, we consider all smooth paths $\ivec\pos i\vat t$ in $\cspace$ with fixed end points so that
%
\begin{equation}
  \ivec\pos i\vat{t\txt i} = \ivec\pos i\txt{initial} \qquad\text{and}\qquad
  %
  \ivec\pos i\vat{t\txt f} = \ivec\pos i\txt{final}\,.
\end{equation}
%
Of all these possible paths, only one is the true path taken by the system. Which one? To each path, let us assign a number called the action $\action$ defined as
%
\begin{equation}
  \action = \sint{t\txt i, t\txt f}{\elag} \implies
  %
  \action\vat{\ivec\pos i\vat t} = \lint{t\txt i, t\txt f}{\elag\vat{\ivec\pos i, \ivec{\dt\pos}i}}{\dx t}\,.
\end{equation}
%
The action is a functional; \ie, a function of the path, which is itself a function. The \lingo{principle of least action} is the following result:
%
\begin{quote}\begin{center}
%
\theorem{the actual path taken by the system is an extremum of $\action$.}
%
\end{center}\end{quote}
%
The requirement that $\action$ be extremum implies that the first-order change must be zero for any possible perturbation about the true evolution. This can be true only if
%
\begin{equation}
  \xeleq{\ivec\pos i}{\ivec{\dt\pos}{i}}{\elag} = 0
\end{equation}
%
for each $i = 1,\dotsc,3\npart$. These equations are known as \lingo{Lagrange's equations} (or \lingo{Euler-Lagrange's equations}). It can be proved that Lagrange's equations are equivalent to Newton's.

Let's now define \lingo{Euler-Lagrange's operator} $\elop{}{}$ to keep notation tidy:
%
\mnote{The musical symbol $\natural$, called \lingo{natural}, represents the unaltered pitch of a note.}
%
\begin{equation}
  \elop{\pos}{\dt\pos} \defas \xeleq{\pos}{\dt\pos}{}\,.
\end{equation}
%
Using this operator, Lagrange's equations can be rewritten as
%
\begin{equation}
  \elop{\ivec\pos i}{\ivec{\dt\pos}{i}}\elag = 0\,.
\end{equation}


\subsection{Remarks on the principle of the least action}
%
\begin{itemize}
%
\item This is an example of a variational principle.
%
\item The principle of \emph{least} action is a slight misnomer. The proof only requires that $\partial\action = 0$, and does not specify whether it is a maxima or minima of $\action$. So \lingo{Principle of stationary action} would be a more accurate, but less catchy, name. It is sometimes called \scare{Hamilton's principle}.
%
\item \emph{All} the fundamental laws of physics can be written in terms of an action principle. This includes electromagnetism, general relativity, the standard model of particle physics and attempts to go beyond the known laws of physics such as string theory. 
%
\item There is a beautiful generalization of the action principle to quantum mechanics.
%
\item Back to classical mechanics, there are two very important reasons for working with Lagrange's equations rather than Newton's. The first is that Lagrange's equations hold in any coordinate system, while Newton's are restricted to an inertial frame. The second is the ease with which we can deal with constraints in the Lagrangian system. We'll look at these two aspects in the next two subsections.
%
\end{itemize}


\subsection{Changing coordinate systems}
%
\theorem{Lagrange's equations hold in any coordinate system}. This follows immediately from the action principle, which is a statement about paths and \emph{not} about coordinates.
%
\mnote{Paths are \emph{coordinate-independent} geometric objects.}
%




