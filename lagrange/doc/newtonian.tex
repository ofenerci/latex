\chapter{Newtonian mechanics}
%
[Based upon \citep{tong:2004}]


\section{Newton's laws of motion}
%
\subsection{Introduction}
%
Newton's \book{Principia} sums up the fundamental principles of classical mechanics based on previous knowledge and his ideas. A particle's motion, for instance, can be found applying Newton's \lingo{three laws of motion}. These laws and the motto \fact{focus on forces} started a whole program of research for future scientists.
%
\mnote{Newton's second law of motion: give me a particle, tell me the forces applied on it and I'll tell you how it moves.}
%
Newton's method for finding particles' motion is:
%
\begin{itemize}
%
\item given a collection of particles, acted upon by a collection of forces, draw a nice diagram, with the particles as points and the forces as arrows;
%
\item added up the forces and 
%
\item apply Newton's famous $\force = \mass\acc$ to figure out where the particle's velocities are heading next. 
%
\end{itemize}

Post Newtonian researchers found out Newton's method unsatisfactory, since
%
\begin{itemize}
%
\item it's messy and inelegant; 
%
\item it's hard to model extended objects, rather than point particles; 
%
\item it obscures certain features of dynamics -- chaos theory took over 200 years to discover -- and 
%
\item it's unclear the relationship between Newton's classical laws and quantum physics.
%
\end{itemize}

To resolve these issues, modelers established a new formalism, bringing new perspectives on Newton's ideas by reformulating them using more powerful techniques.
%
\mnote{Better techniques result into an immediate practical advantage to quantify certain complicated phenomena with relative ease.}
%
Simultaneously, such a formalism provides an elegant viewpoint that reveals the basic principles underling Newton's familiar laws of motion: it pries open $\force = \mass\acc$ to reveal the structures and symmetries that lie beneath.

Moreover, the formalism has become the basis for \emph{all} of fundamental modern physics. Every theory of Nature is best described in the newly developed language. Finally, it also provides the bridge between the classical and the quantum world.

However, there are phenomena in Nature for which these formalism is not particularly useful: \lingo{dissipative systems}, for example, are not so well suited to these new techniques.


\subsection{Newtonian mechanics: single particle}
%
\lingo{Particle}: an object of insignificant size; \eg, an electron, a tennis ball or a planet. The validity of this statement depends on the context: to a first approximation, the earth can be treated as a particle when computing its orbit around the sun. But if you want to understand its spin, it must be treated as an extended object.

The motion of a particle of \lingo{mass} $\mass$ at the \lingo{position} $\pos$ is modeled by \lingo{Newton's Second Law}:
%
\begin{equation}\label{eq:newtonsecondlaw}
  \force = \dt\mom\,,
\end{equation}
%
wherein $\force = \force\vat{\pos,\dt\pos}$ represents the \lingo{force} that, in general, can depend on both the position $\pos$ and the velocity $\dt\pos$
%
\mnote{friction forces depend on velocity.}
%
and $\mom = \mass\dt\pos$ represents the \lingo{momentum}. \Cref{eq:newtonsecondlaw} reduces to $\force = \mass\acc$ when mass is constant, $\dx\mass = 0$.

\emph{The goal of classical dynamics}: given positions and velocities at an initial time $t = t_0$, integrate \cref{eq:newtonsecondlaw} to determine $\pos\vat t$ for all $t$, as long as $\force$ remains finite. 

\Cref{eq:newtonsecondlaw} holds only in an \lingo{inertial frame}: a frame where a \lingo{free particle} with constant mass travels in a straight line:
%
\mnote{free particle: particle subject to no force}
%
\begin{equation}\label{eq:straightlinemotion}
  \pos = \pos_0 + \vel t\,.
\end{equation}
%
\lingo{Newton's first law} states that such frames exist.

An inertial frame is \emph{not} unique: there are an infinite number of inertial frames. Let $\frm$ be an inertial frame. Then, there are ten \lingo{linearly independent transformations} $\frm\to\frm'$ such that $\frm'$ is also an inertial frame; \ie, if \cref{eq:straightlinemotion} holds in $\frm$, then it also holds in $\frm'$. These are
%
\begin{itemize}
%
\item three rotations: $\pos' = O\pos$, where $O$ is a $3\cprod 3$ orthogonal matrix;
%
\item three spatial translations: $\pos' = \pos + \pos_k$, for a constant position $\pos_k$;
%
\item three boosts: $\pos' = \pos + \vel_k t$, for a constant velocity $\vel_k$;
%
\item one time translation: $t' = t + t_k$, for a constant time $t_k$.
%
\end{itemize}
%
%If motion is uniform in $\frame$, it will also be uniform in $\frame'$. These transformations make up the \lingo{Galilean Group} under which Newton's laws are \lingo{invariant}.
%
\mnote{Invariant: if you have a particle moving in a straight line and apply any or all of the transformations to \cref{eq:straightlinemotion}, then you end up with a particle also moving in a straight line.}
%
They will be important later, where we will see that these symmetries of space and time are the underlying reason for conservation laws. 

As a parenthetical remark, recall from special relativity that \lingo{Einstein's laws of motion} are invariant under \lingo{Lorentz transformations}, which, together with translations, make up the \lingo{Poincaré group}. We can recover the Galilean group from the Poincaré group by taking the speed of light to infinity.


\subsubsection{Angular momentum}
%
We define the angular momentum $\amom$ of a particle and the torque $\tork$ acting upon it as
%
\begin{equation}\label{eq:amomandtork}
  \amom = \pos\cprod\mom 
  \qquad\text{and}\qquad
  \tork = \pos\cprod\force
\end{equation}

Note that, unlike linear momentum, both $\amom$ and $\tork$ depend on where we take the origin: \emph{we measure angular momentum with respect to a particular point}. Let us cross both sides of \cref{eq:newtonsecondlaw} with $\pos$. Using the fact that $\dt\pos$ is parallel to $\mom$, we can write $\iod t\parth{\pos\cprod\mom} = \pos\cprod\dt\mom$. Then we get a version of Newton's second law that holds for angular momentum:
%
\begin{equation}\label{eq:angularnewtonsecondlaw}
  \tork = \dt\amom\,.
\end{equation}


\subsection{Conservation laws}
%
From \cref{eq:newtonsecondlaw} and \cref{eq:angularnewtonsecondlaw}, two important conservation laws follow immediately:
%
\begin{itemize}
%
\item If $\force = 0$, then $\mom$ is constant throughout the motion and
%
\item if $\tork = 0$, then $\amom$ is constant throughout the motion.
%
\end{itemize}
%
Notice that $\tork = 0$ does not require $\force = 0$, but only $\pos\cprod\force = 0$. This means that $\force$ must be parallel to $\pos$. This is the definition of a \lingo{central force}. As written above in terms of forces and torques, these conservation laws appear trivial. Afterwards, we'll see how they arise as a property of the symmetry of space as encoded in the Galilean group.


\subsection{Energy}
%
Recall the definitions of energy. Firstly define the \lingo{kinetic energy} $\ekin$ as
%
\begin{equation}
  2\ekin = \mass\dt\pos\iprod\dt\pos\,.
\end{equation}
%
Suppose hereafter that the mass is constant. Compute the change of kinetic energy with time: $\iod t\ekin = \dt\mom\iprod\dt\pos = \force\iprod\dt\pos$. If the particle travels from position $\pos_1$ at time $t_1$ to position $\pos_2$ at time $t_2$, then this change in kinetic energy is given by
%
\begin{equation}\label{eq:energyworktheorem}
  \ekin\vat{t_2} - \ekin\vat{t_1} = \sint{t_1,t_2}{\xod\ekin t}\dx t
                                  = \lint{t_1,t_2}{\force\iprod\dt\pos}{\dx t}
                                  = \sint{\pos_1,\pos_2}{\force\iprod\dx\pos}\,,
\end{equation}
%
where the final expression involving the integral of the force over the path is called the \lingo{work done by the force}. So see that the \theorem{work done is equal to the change in kinetic energy}. From now on we will mostly focus on a very special type of force known as a \lingo{conservative force}: a force that depends only on position $\pos$ rather than velocity $\dt\pos$ and is such that the work done is independent of the path taken. In particular, for a closed path, the work done vanishes.
%
\begin{equation}
  \oint\force\iprod\dt\pos = 0 \iff \gder\cprod\force = 0\,.
\end{equation}
%
It is a deep property of Euclidean (flat) space $\nespace 3$ that this property implies we may write the force as
%
\begin{equation}
  \force = -\gder\epot
\end{equation}
%
for some potential $\epot\vat\pos$. Systems which admit a potential of this form include gravitational, electrostatic and interatomic forces. \theorem{When we have a conservative force, we necessarily have a conservation law for energy}. To see this, return to \cref{eq:energyworktheorem}, which now reads
%
\begin{equation}
  \ekin\vat{t_2} - \ekin\vat{t_1} = -\sint{\pos_1,\pos_2}{\gder\epot\iprod\dx\pos}
                                  = -\epot\vat{t_2} + \epot\vat{t_1}\,,
\end{equation}
%
or, rearranging things,
%
\begin{equation}
  \ekin\vat{t_1} + \epot\vat{t_1} = \ekin\vat{t_2} + \epot\vat{t_2}
                                  \equiv\ener
\end{equation}
%
So $\ener = \ekin + \epot$ is also a \lingo{constant of motion}. It is the energy. When the energy is considered to be a function of position $\pos$ and momentum $\mom$ it is referred to as the \lingo{Hamiltonian} $\eham$. Afterwards, we will be seeing much more of the Hamiltonian.


\subsection{Examples}
%
\begin{description}
%
\item[The Simple Harmonic Oscillator] This is a one-dimensional system with a force proportional to the distance $\xpos$ to the
origin: $\force\vat\xpos = -k\xpos$. This force arises from a potential $2\epot = k\xpos^2$. Since $\force \not= 0$, momentum is \emph{not} conserved (the object oscillates backwards and forwards) and, since the system lives in only one dimension, angular momentum is not defined. But energy $2\ener = \mass\dt\xpos^2 + k\xpos^2$ is conserved.
%
\item[The Damped Simple Harmonic Oscillator] We now include a friction term so that $\force\vat{\xpos, \dt\xpos} = -k\xpos - \gamma\dt\xpos$. Since $\force$ is not conservative,
%
\mnote{a conservative force depends only on position, rather than on position and velocity.}
%
energy is not conserved. This system loses energy until it comes to rest.
%
\item[Particle Moving Under Gravity] Consider a particle of mass $\mass$ moving in three dimensions under the gravitational pull of a much larger particle of mass $\mass'$. The force is $\force\pos^2 = -\uknewton\mass\mass'\uvec\pos$, which arises from the potential $\epot\pos = -\uknewton\mass\mass'$. Again, the linear momentum $\mom$ of the smaller particle is not conserved, but the force is both central and conservative, ensuring the particle's total energy $\ener$ and the angular momentum $\amom$ are conserved.
%
\end{description}
