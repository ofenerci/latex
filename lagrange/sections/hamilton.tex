\section{Hamiltonian}
The \lingo{Hamiltonian}, $\ham$, is defined by
\beq
\ham = \igmom{}\iprod\igvel{} - \lag\,.
\eeq

The equations of motion can be found by
\beq
\igfor{i}     = -\ipd{\igpos{i}}{\ham}\,,\quad
\igvel{i}     = \ipd{\igmom{i}}{\ham}\quad\text{and}\quad
\ipd{t}{\lag} = -\ipd{t}{\ham}\,. 
\eeq

\begin{example}
Find the equation of motion for a simple pendulum. Use the Hamilton formalism of mechanics.
\end{example}

\begin{solution}
A simple pendulum is a pendulum swinging through an inviscid fluid and consisting of a bob of mass $\mass$ hanging of a inflexible, massless, frictionless rod of length $\length$. To find the equation of motion, follow pendulum amplitude $\igpos\theta$.

For the pendulum, calculate the Langrangian:
\beq
\lag = \dfrac{1}{2}\mass\length^2\left(\igvel{\theta}{}\right)^2 + \mass\grav\length\cos\vat\theta\,.
\eeq
Next, find the generalized momentum as
\beq
\igmom{\theta} = \ipd{\igvel{\theta}{}}{\lag} = \mass\grav\length\cos\vat\theta\,.
\eeq
Then, isolate the generalized velocity:
\beq
\igvel{\theta}{} = \dfrac{\igmom{\theta}}{\mass\length^2}\,.
\eeq

Replace the generalized momentum and velocity in the definition of $\ham$ to have
\beq
\ham = \igmom{\theta}\igvel{\theta}{} 
       - \dfrac{1}{2}\mass\length^2\left(\igvel{\theta}{}\right)^2 
       - \mass\length\cos\vat\theta\,,
\eeq
but $\igvel{\theta}{} = \igmom{\theta}/\mass\length^2$, thus
\beq
\ham = \dfrac{\igmom{\theta}^2}{2\mass\length^2} - \mass\grav\length\cos\vat\theta\,.
\eeq
Then, the equations of motion are
\beq
\igfor{\theta}   = -\ipd{\theta}\ham = -\mass\length\sin\vat\theta\quad\text{and}\quad
\igvel{\theta}{} = \ipd{\igmom{\theta}}{\ham} = \dfrac{\igmom{\theta}}{\mass\length^2}\,.
\eeq
Finally, taking the derivative of the generalized velocity and replacing into the generalized momentum, one has the equation of motion for the simple pendulum
\beq
\igacc{\theta}{} = -\dfrac{\grav}{\length}\sin\vat\theta\,.\mqed
\eeq
\end{solution}


