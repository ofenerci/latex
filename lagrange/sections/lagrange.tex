\section{The classical Lagrangian}
Here we will define the Lagrangian formulation of the fundamental problem of classical mechanics: ``Given a potential, how do particles move?'' This section serves as a short review of Lagrangians in general, and the next section will specialize to focus on Keplerian orbits in the classical setting -- if we are to understand the changes to the motion of particles in general relativity (GR), it behooves us to recall the motion in the ``normal'' case. Our ultimate goal is to shift from the specific sorts of notations used in introductory cases (for example, spherical coordinates), to a more abstract notation appropriate to the study of particle motion in general relativity.

As we go, we will introduce some basic tensor operations, but there will be more of this to come in Chapter 3. We just need to become comfortable with summation notation for now.


\subsection{Lagrangian and equations of motion}
A Lagrangian is the integrand of an action -- while this is not the usual definition, it is, upon definition of action, more broadly applicable than the usual ``kinetic minus potential'' form. In classical mechanics, the Lagrangian leading to Newton's second law reads, in Cartesian coordinates:
\beq
\lag = \ken - \pen
     = \dfrac{1}{2}\mass\vel^2 - \pen\,.
\eeq

Variation provides the ordinary differential equation (ODE) structure of interest, a set of three second-order differential equations, the Euler-Lagrange equations of motion:
\beq
\eleq\pos\vel = 0\,.
\eeq
We define, now, the Euler-Lagrange operator by
\beq
\elop\pos\vel = \left(\ipd\pos - \iod t\ipd\vel\right)\,.
\eeq
Thus, the Euler-Lagrange equations can be written as
\beq
\elop\pos\vel\lag = 0\,.
\eeq


In Cartesian coordinates, with the Lagrangian from (1.2), the Euler-Lagrange equations reproduce Newton's second law given a potential $\pen$:
\beq
\mass\acc = -\grad\pen\,.
\eeq

The advantage of the action approach, and the Lagrangian in particular, is that the equations of motion can be obtained for any coordinate representation of the kinetic energy and potential. Although it is easy to define and verify the correctness of the Euler-Lagrange equations in Cartesian coordinates, they are not necessary to the formulation of valid equations of motion for systems in which Cartesian coordinates are less physically and mathematically useful.

The Euler-Lagrange equations, in the form (1.11), hold regardless of our association of $\pos$ with Cartesian coordinates. Suppose we move to cylindrical coordinates $\elset{s, \theta, z}$ defined by
\beq
\fpos 1 = s\cos\vat\theta\,,\quad
\fpos 2 = s\sin\vat\theta   \quad\text{and}\quad
\fpos 3 = z\,,
\eeq
then the Lagrangian in Cartesian coordinates can be transformed to cylindrical coordinates by making the replacement for $\elset{\fpos i}$ in terms of $\elset{s, \theta, z}$ (and associated substitutions for the Cartesian velocities):
\beq
\lag = \dfrac{1}{2}\mass\left(\dt s^2 + \dt\theta^2 + \dt z^2\right) - \pen\vat{s, \theta, z}\,.
\eeq
But, the Euler-Lagrange equations require no modification, the variational procedure that gave us (1.11) can be applied in the cylindrical coordinates, giving three equations of motion:
\beq
\elop{s}{\dt s}\lag          = 0\,,\quad
\elop{\theta}{\dt\theta}\lag = 0   \quad\text{and}\quad
\elop{z}{\dt z}\lag          = 0\,.
\eeq
The advantage is clear: coordinate transformation occurs once and only once, in the Lagrangian. If we were to start with Newton's second law, we would have three equations with acceleration and coordinates coupled together. The decoupling of these would, in the end, return (1.15).


\subsection{Examples}
In one dimension, we can consider the Lagrangian
\beq
2\lag = \mass\vel^2 - 2\kspring\left(\pos - a\right)^2
\eeq
appropriate to a spring potential with spring constant $\kspring$ and equilibrium spacing $a$. Then the Euler-Lagrange equations give:
\beq
\elop{\pos}{\vel}\lag = \mass\acc + \kspring\left(\pos - a\right) = 0\,.
\eeq
