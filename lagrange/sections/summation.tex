\section{Summation convention}
Traditionally, a vector $v$ in three dimensional space, $\nespace 3$, is written as
\beq
v = v_1\hat x + v_2\hat y + v_3\hat z\,,
\eeq
where $\elset{v_1, v_2, v_3}$ are the components of $v$ and $\elset{\hat x, \hat y, \hat z}$ the unit vectors pointing in the direction of the Cartesian axes.

An alternative form of writing $v$ is by using the summation notation:
\beq
v = \sum_{i = 1}^{3}\fvec vi\fbvec i\,,
\eeq
where $\elset{\fvec vi}$ are the components of $v$ onto the frame $\elset{\fbvec i}$. Note that the $i$ on $\fvec vi$ is an index, a place holder that indicates \emph{which} component, and it is \emph{not} an exponent!

The summation convention consists on dropping the summation symbol:
\beq
v = \fvec vi\fbvec i\,.\mqed
\eeq
