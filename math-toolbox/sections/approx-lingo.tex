\section{Some Lingo about Approximate Solutions}


\epigraph{The art of being wise is the art of knowing what to overlook.}{William James}{}


\subsection{Spherical Cow}
Spherical cow is a metaphor for highly simplified scientific models of complex real life phenomena.

The phrase comes from a joke about theoretical physicists:
\begin{quote}
Milk production at a dairy farm was low, so the farmer wrote to the local university, asking for help from academia. A multidisciplinary team of professors was assembled, headed by a theoretical physicist, and two weeks of intensive on-site investigation took place. The scholars then returned to the university, notebooks crammed with data, where the task of writing the report was left to the team leader. Shortly thereafter the physicist returned to the farm, saying to the farmer ``I have the solution, but it only works in the case of spherical cows in a vacuum''.
\end{quote}

The point of the joke is that physicists will often reduce a problem to the simplest form they can imagine in order to make calculations more feasible, even though such simplification may hinder the model's application to reality.


\subsection{Fermi Problem}
In science, particularly in physics or engineering education, a Fermi problem, Fermi question, or Fermi estimate is an estimation problem designed to teach dimensional analysis, approximation, and the importance of clearly identifying one's assumptions. Named after physicist Enrico Fermi, such problems typically involve making justified guesses about quantities that seem impossible to compute given limited available information.

Fermi was known for his ability to make good approximate calculations with little or no actual data, hence the name. One example is his estimate of the strength of the atomic bomb detonated at the Trinity test, based on the distance traveled by pieces of paper dropped from his hand during the blast. Fermi's estimate of 10 kilotons of TNT was remarkably close to the now-accepted value of around 20 kilotons, a difference of less than one order of magnitude.


\subsubsection{Examples of Fermi problems}
The classic Fermi problem, generally attributed to Fermi, is ``How many piano tuners are there in Chicago?'' A typical solution to this problem involves multiplying a series of estimates that yield the correct answer if the estimates are correct. For example, we might make the following assumptions:
\begin{itemize}
\item There are approximately 5,000,000 people living in Chicago.
\item On average, there are two persons in each household in Chicago.
\item Roughly one household in twenty has a piano that is tuned regularly.
\item Pianos that are tuned regularly are tuned on average about once per year.
\item It takes a piano tuner about two hours to tune a piano, including travel time.
\item Each piano tuner works eight hours in a day, five days in a week, and 50 weeks in a year.
\end{itemize}
From these assumptions, we can compute that the number of piano tunings in a single year in Chicago is:

(5,000,000 persons in Chicago) / (2 persons/household) × (1 piano/20 households) × (1 piano tuning per piano per year) = 125,000 piano tunings per year in Chicago.

We can similarly calculate that the average piano tuner performs:

(50 weeks/year)×(5 days/week)×(8 hours/day)/(2 hours to tune a piano) = 1000 piano tunings per year per piano tuner.

Dividing gives:

(125,000 piano tunings per year in Chicago) / (1000 piano tunings per year per piano tuner) = 125 piano tuners in Chicago.

A famous example of a Fermi-problem-like estimate is the Drake equation, which seeks to estimate the number of intelligent civilizations in the galaxy. The basic question of why, if there is a significant number of such civilizations, ours has never encountered any others is called the Fermi paradox.


\subsubsection{Advantages and scope}
Scientists often look for Fermi estimates of the answer to a problem before turning to more sophisticated methods to calculate a precise answer. This provides a useful check on the results: where the complexity of a precise calculation might obscure a large error, the simplicity of Fermi calculations makes them far less susceptible to such mistakes. (Performing the Fermi calculation first is preferable because the intermediate estimates might otherwise be biased by knowledge of the calculated answer.)

Fermi estimates are also useful in approaching problems where the optimal choice of calculation method depends on the expected size of the answer. For instance, a Fermi estimate might indicate whether the internal stresses of a structure are low enough that it can be accurately described by linear elasticity; or if the estimate already bears significant relationship in scale relative to some other value, for example, if a structure will be over-engineered to withstand loads several times greater than the estimate.

Although Fermi calculations are often not accurate, as there may be many problems with their assumptions, this sort of analysis does tell us what to look for to get a better answer. For the above example, we might try to find a better estimate of the number of pianos tuned by a piano tuner in a typical day, or look up an accurate number for the population of Chicago. It also gives us a rough estimate that may be good enough for some purposes: if we want to start a store in Chicago that sells piano tuning equipment, and we calculate that we need 10,000 potential customers to stay in business, we can reasonably assume that the above estimate is far enough below 10,000 that we should consider a different business plan (and, with a little more work, we could compute a rough upper bound on the number of piano tuners by considering the most extreme reasonable values that could appear in each of our assumptions).


\subsubsection{Explanation}
Fermi estimates generally work because the estimations of the individual terms are often close to correct, and overestimates and underestimates help cancel each other out. That is, if there is no consistent bias, a Fermi calculation that involves the multiplication of several estimated factors (such as the number of piano tuners in Chicago) will probably be more accurate than might be first supposed.

In detail, multiplying estimates corresponds to adding their logarithms; thus one obtains a sort of Wiener process or random walk on the logarithmic scale, which diffuses as $\sqrt{n}$ (in number of terms $n$). In discrete terms, the number of overestimates minus underestimates will have a binomial distribution. In continuous terms, if one makes a Fermi estimate of $n$ steps, with standard deviation $\sigma$ units on the log scale from the actual value, then the overall estimate will have standard deviation $\sigma\sqrt{n}$, since the standard deviation of a sum scales as $\sqrt{n}$ in the number of summands.

For instance, if one makes a 9-step Fermi estimate, at each step overestimating or underestimating the correct number by a factor of 2 (or with a standard deviation 2), then after 9 steps the standard error will have grown by a logarithmic factor of $\sqrt{9} = 3$, so $2^3 = 8$. Thus one will expect to be within 1/8 to 8 times the correct value – within an order of magnitude, and much less than the worst case of erring by a factor of (about 2.7 orders of magnitude). If one has a shorter chain or estimates more accurately, the overall estimate will be correspondingly better.


\subsection{Back-of-the-envelope calculation}
A back-of-the-envelope calculation is a rough calculation, typically jotted down on any available scrap of paper such as the actual back of an envelope. It is more than a guess but less than an accurate calculation or mathematical proof.

The defining characteristic of back-of-the-envelope calculations is the use of simplified assumptions.

A similar phrase is ``back of a napkin'', which is also used in the business world to describe sketching out a quick, rough idea of a business or product.


\subsection{Sanity testing}
A sanity test or sanity check is a basic test to quickly evaluate whether a claim or the result of a calculation can possibly be true. It is a simple check to see if the produced material is rational (that the material's creator was thinking rationally, applying sanity). The point of a sanity test is to rule out certain classes of obviously false results, not to catch every possible error. A rule-of-thumb may be checked to perform the test. The advantage of a sanity test, over performing a complete or rigorous test, is speed.

In arithmetic, for example, when multiplying by 9, using the divisibility rule for 9 to verify that the sum of digits of the result is divisible by 9 is a sanity test -- it will not catch every multiplication error, however it's a quick and simple method to discover many possible errors.

In computer science, a sanity test is a very brief run-through of the functionality of a computer program, system, calculation, or other analysis, to assure that part of the system or methodology works roughly as expected. This is often prior to a more exhaustive round of testing.

When talking about quantities in physics, the claim of a power output of a car cannot be \SI{700}{kJ} since that is a unit of energy, not power (energy per unit time).


\subsection{Heuristic}
Heuristic refers to experience-based techniques for problem solving, learning, and discovery. Where the exhaustive search is impractical, heuristic methods are used to speed up the process of finding a satisfactory solution; mental shortcuts to ease the cognitive load of making a decision. Examples of this method include using a rule of thumb, an educated guess, an intuitive judgment, or common sense.

The most fundamental heuristic is trial and error, which can be used in everything from matching nuts and bolts to finding the values of variables in algebra problems.

Here are a few other commonly used heuristics, from George Pólya's 1945 book, How to Solve It:
\begin{itemize}
\item If you are having difficulty understanding a problem, try drawing a picture.
\item If you can't find a solution, try assuming that you have a solution and seeing what you can derive from that (``working backward'').
\item If the problem is abstract, try examining a concrete example.
\item Try solving a more general problem first (the ``inventor's paradox'': the more ambitious plan may have more chances of success).
\end{itemize}

In engineering, a heuristic is an experience-based method that can be used as an aid to solve process design problems, varying from size of equipment to operating conditions. By using heuristics, time can be reduced when solving problems. Several methods are available to engineers. These include Failure mode and effects analysis and Fault tree analysis. The former relies on a group of qualified engineers to evaluate problems, rank them in order of importance and then recommend solutions. The methods of forensic engineering are an important source of information for investigating problems, especially by elimination of unlikely causes and using the weakest link principle. Because heuristics are fallible, it is important to understand their limitations. They are aids that facilitate quick estimates and preliminary process designs.


Heuristic (engineering): In engineering, heuristics are experience-based methods used to reduce the need for calculations pertaining to equipment size, performance, or operating conditions. Heuristics are fallible and do not guarantee a correct solution. It is important to understand their limitations when applying them to different equipment and processes. Though heuristics are limited, they may be of value. This is because they offer time saving approximations in preliminary process design.

Problem solving methods are intrinsic to forensic engineering methods, where failures are analyzed for the root cause or causes. Only when failures have been investigated with conclusive results can remedial action be taken with confidence.

Example: Storage Vessels: These heuristics were taken from Turton's ``Analysis, Synthesis, and Design of Chemical Processes''.
\begin{itemize}
\item Use vertical tanks on legs when the tank is less than \SI{3.8}{m^3}.
\item Use horizontal tanks on concrete supports when the tank is between 3.8 and \SI{38}{m^3},
\item Use vertical tanks on concrete pads when the tank is beyond \SI{38}{m^3},
\item Liquids subject to breathing losses may be stored in tanks with floating or expansion roofs for conservation.
\item Freeboard is 15\% below \SI{1.9}{m^3} and 10\% above \SI{1.9}{m^3}.
\item Thirty day capacity often is specified for raw materials and products, but depends on connecting transportation equipment schedules.
\end{itemize}


\subsection{Orders of approximation}
In science, engineering, and other quantitative disciplines, orders of approximation refer to formal or informal terms for how precise an approximation is, and to indicate progressively more refined approximations: in increasing order of precision, a zeroth order approximation, a first order approximation, a second order approximation, and so forth.

Formally, an $n$th order approximation is one where the order of magnitude of the error is at most $x^n$, or in terms of big $O$ notation, the error is $O\vat{x^n}$. In suitable circumstances, approximating a function by a Taylor polynomial of degree $n$ yields an $n$th order approximation, by Taylor's theorem: a first order approximation is a linear approximation, and so forth.


\subsection{Handwaving}
Handwaving arguments often include order-of-magnitude estimates and dimensional consistency. Competent, well-intentioned researchers and professors rely on handwaving when, given a limited time, a large result must be shown and minor technical details cannot be given much attention; \eg, ``It can be shown that $z$ is even''.

Back-of-the-envelope calculations are approximate ways to get an answer by over-simplification and are compatible with handwaving.


