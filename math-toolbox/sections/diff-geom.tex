\section{Differential Geometry}


\subsection{Comma and Semi-colon Derivatives}
Comma-derivative: The components of the gradient of the one-form $\dx A$ are denoted $\cder Ak$, or sometimes $\igder kA$, and are given by
\beq
\cder Ak = \igder kA = \xpd A{\scoord k} \,.
\eeq

Semi-colon or covariant derivative: The covariant derivative of a \emph{contravariant} tensor $\cntens Aa$ (also called the ``semicolon derivative'' since its symbol is a semicolon) is given by
\beq
\coder{\cntens Aa}{b} = \xpd{\cntens Aa}{\scoord b} + \chris abk\cntens Ak 
                      = \cder{\cntens Aa}{b} + \chris abk\cntens Ak \,,
\eeq
where $\chris kij$ is a Christoffel symbol, Einstein summation has been used in the last term, and $\cder{\cntens Aa}{b}$ is a comma derivative. The notation $\gder\iprod A$, which is a generalization of the symbol commonly used to denote the divergence of a vector function in three dimensions, is sometimes also used.

The covariant derivative of a \emph{covariant tensor} $\cotens Aa$ is
\beq
\coder{\cotens Aa}{b} = \xpd{\cotens Aa}{\scoord b} - \chris abk\cotens Ak \,,
\eeq


\subsection{Some Derivatives}
Consider a scalar field $f = f\vat{t, \scoord i}$ (a scalar-valued function of the position vector). Then, the total time derivative of the scalar field, denoted $\dt f$, is defined by 
\beq
\dt f = \igder tf + \igder if\dtcntens\pvec i\,.
\eeq

Partial time derivative operator:
\beq
\igder t = \xpd{}{t}\,.
\eeq

Partial spatial (coordinate) derivative operator:
\beq
\igder k = \igder{\scoord k} = \xpd{}{\scoord k}.
\eeq

Absolute derivative of a tensor field $T$ upon the parameter $t$:
\beq
\mder Tt = \abstder{T} = \dfrac{\fder T}{\dx t} = \dfrac{\gder T}{\dx t}\,.
\eeq


\subsection{Continuity Equation}
Recall that the most important equation in fluid dynamics, as well as in general continuum mechanics, is the celebrated equation of continuity, (we explain the symbols in the following text)
\beq
\igder t\rho + \div(\rho u) = 0\,.
\eeq

As a warm-up for turbulence, we will derive the continuity equation, starting from the mass conservation principle. Let $\dx m$ denote an infinitesimal mass of a fluid particle. Then, using the absolute time derivative operator $\abstder{} = \mder t{}$, the mass conservation principle reads
\beq
\abstder{\dx m} = 0\,.
\eeq

If we further introduce the fluid density $\rho = \dx m/\dx v$, where $\dx v$ is an infinitesimal volume of a fluid particle, then the mass conservation principle can be rewritten as
\beq
\abstder{\rho\dx v} = 0\,,
\eeq
which is the absolute derivative of a product, and therefore expands into
\beq
\dt\rho \dx v + \rho\abstder{\dx v} = 0\,.
\eeq

Now, as the fluid density $\rho = \rho\vat{\scoord k, t}$ is a function of both time $t$ and spatial coordinates $\scoord k$, for $k = 1,2,3$, that is, a scalar-field, its total time derivative $\dt\rho$ is defined by
\beq
\dt\rho = \igder t\rho + \igder{\scoord k}\rho\igder t\scoord k 
        = \igder t\rho + \coder\rho k \cntens uk\,,
\eeq
or, in vector form,
\beq
\dt\rho = \igder t\rho + \grad\rho\iprod u\,,
\eeq
where $u = \cntens uk = \cntens uk\vat{\scoord k, t}$ is the velocity vector-field of the fluid.

Regarding $\abstder{\dx v}$, the other term figuring in the absolute derivative of a product, we start by expanding an elementary volume $\dx v$ along the sides $\elset{\dx\scoord i_{(p)}, \dx\scoord j_{(q)}, \dx\scoord k_{(r)}}$ of an elementary parallelepiped, as
\beq
\dx v = \dfrac{1}{3!}\mkron{pqr}{ijk}\,\dx\scoord i_{(p)} \dx\scoord j_{(q)} \dx\scoord k_{(r)}\,,\qquad
    [i,j,k,p,q,r = 1,2,3]
\eeq
so that its absolute derivative becomes [maths here :)] which finally simplifies into
\beq
\abstder{\dx v} = \coder{\cntens uk}k\dx v = \div{(u)}\dx v\,,
\eeq
Substituting the products into the continuity equation gives
\beq
\abstder{\rho \dx v} = \left(\igder t\rho + \coder\rho k\cntens uk\right)\,\dx v 
                        + \rho\coder{\cntens uk}k\dx v 
                     = 0
\eeq

As we are dealing with arbitrary fluid particles, then $\dx v \neq 0$, so from the last equation follows
\beq
\igder t\rho + \coder\rho k\cntens uk + \rho\coder{\cntens uk}k =
    \igder t\rho + \coder{\left(\rho\cntens uk\right)}k = 0\,,
\eeq

The last equation is the covariant form of the continuity equation, which in standard vector notation becomes
\beq
\igder t\rho + \div(\rho u) = 0\,.
\eeq


\subsection{Differential Forms}
Consider a set of coordinates $\elset{\cnvec x1,\dotsc,\cnvec xn}$ on a manifold $\region M$ and consider the set
\beq
\tuple{\nbvec 1, \dotsc, \nbvec n} = \tuple{\xpd{}{\cnvec x1},\dotsc,\xpd{}{\cnvec xn}}
\eeq
be the basis for $T_m\region M$.

Consider the set
\beq
\tuple{\dbvec 1, \dotsc, \dbvec n} = \tuple{\dx\cnvec x1, \dotsc, \dx\cnvec xn}
\eeq
be the dual basis for $T_m^*\region M$.

At each point $m\in\region M$, we can write a 2-form as
\beq
\Omega_m\vat{v,w} = \Omega_{ij}\vat m\cnvec vi\cnvec wj\,,
\eeq
where
\beq
\Omega_{ij}\vat m = \tuple{\xpd{}{\cnvec xi}, \xpd{}{\cnvec xj}}\,.
\eeq


\subsection{Algebra of Differential Forms}
Consider the forms $\elset{\dx\cnvec xi, \dx\cnvec xj}$ and the scalar $\alpha\in\set R$, then
\begin{itemize}
\item addition is commutative: $\dx\cnvec xi + \dx\cnvec xj = \dx\cnvec xj + \dx\cnvec xi$;
\item multiplication by a scalar is commutative: $\alpha\dx\cnvec xi = \dx\cnvec xi\alpha$;
\item subtraction: $\dx\cnvec xi - \dx\cnvec xj = \dx\cnvec xi + (-1)\dx\cnvec xj$;
\item multiplication: $\dx\cnvec xi\dx\cnvec xj = 0$, when $i=j$, and $\dx\cnvec xi\dx\cnvec xj = -\dx\cnvec xj\dx\cnvec xi$, when $i\neq j$. Both conditions can be summarized by
\beq
\acom{\dx\cnvec xi}{\dx\cnvec xj} = 0\,;
\eeq
or, expanding the anti-commutator brackets:
\beq
\xacom{\dx\cnvec xi}{\dx\cnvec xj} = 0\,.
\eeq
\end{itemize}

