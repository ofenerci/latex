\section{Fourier Series}
In mathematics, a \lingo{Fourier series} decomposes periodic functions or periodic signals into the sum of a (possibly infinite) set of simple oscillating functions, namely sines and cosines (or complex exponentials). The study of Fourier series is a branch of Fourier analysis.

The heat equation is a partial differential equation. Prior to Fourier's work, no solution to the heat equation was known in the general case, although particular solutions were known if the heat source behaved in a simple way, in particular, if the heat source was a sine or cosine wave. These simple solutions are now sometimes called eigensolutions. Fourier's idea was 
\begin{quote}
to model a complicated heat source as a superposition (or linear combination) of simple sine and cosine waves and to write the solution as a superposition of the corresponding eigensolutions. This superposition or linear combination is called the Fourier series.
\end{quote}

Although the original motivation was to solve the heat equation, it later became obvious that the same techniques could be applied to a wide array of mathematical and physical problems and especially those involving linear differential equations with constant coefficients, for which the eigensolutions are sinusoids. The Fourier series has many such applications in electrical engineering, vibration analysis, acoustics, optics, signal processing, image processing, quantum mechanics, econometrics, thin-walled shell theory, \etc.


\subsection{Definition}
In this section, $f\vat x$ denotes a function of the real variable $x$. This function is usually taken to be periodic, of period $2\pi$, which is to say that $f\vat{x + 2\pi} = f\vat x$, for all real numbers $x$. We will attempt to write such a function as an infinite sum, or series, of simpler $2\pi$-periodic functions. We will start by using an infinite sum of sine and cosine functions on the interval $[-\pi,\pi]$, as Fourier did and we will then discuss different formulations and generalizations.


\subsection{Fourier's formula for $2\pi$-periodic functions using sines and cosines}
For a periodic function $f\vat x$ that is integrable on $[-\pi,\pi]$, the numbers
\begin{align*}
a_n &= \dfrac{1}{\pi}\int_{-\pi}^{\pi}\,f\vat x\,\cos\vat{nx}\,\dx x\,,\qquad n\geq 0\\
b_n &= \dfrac{1}{\pi}\int_{-\pi}^{\pi}\,f\vat x\,\sin\vat{nx}\,\dx x\,.\qquad n\geq 1
\end{align*}
are called the \lingo{Fourier coefficients of $f$}. One introduces the \lingo{Fourier partial sums of degree $n$ generated by $f$ at the point $x$}, often denoted by
\beq
\nfsum fNx = \dfrac{a_0}{2} 
             + \sum_{n = 1}^{N} \left( a_n\cos\vat{nx} + b_n\sin\vat{nx} \right)\,,\qquad N\geq 0 \,.
\eeq

The partial sums for $f$ are \lingo{trigonometric polynomials}. One expects that the functions $\nfsum fNx$ approximate the function $f$ and that the approximation improves as $N\to\infty$. The infinite sum
\beq
\fseries fx = \dfrac{a_0}{2} 
              + \sum_{n = 1}^{\infty} \left( a_n\cos\vat{nx} + b_n\sin\vat{nx} \right)\,,\qquad N\geq 0 \,.
\eeq
is called the \lingo{Fourier series of generated by $f$ at the point $x$}. These trigonometric functions can themselves be expanded, using multiple angle formulae.

The Fourier series does \emph{not} always converge and even when it does converge for a specific value $x_0$ of $x$, the sum of the series at $x_0$ may differ from the value $f\vat{x_0}$ of the function. It is one of the main questions in harmonic analysis to decide when Fourier series converge and when the sum is equal to the original function. If a function is square-integrable on the interval $[-\pi,\pi]$, then the Fourier series converges to the function at almost every point. In engineering applications, the Fourier series is generally presumed to converge everywhere except at discontinuities, since the functions encountered in engineering are more well behaved than the ones that mathematicians can provide as counter-examples to this presumption. In particular, the Fourier series converges absolutely and uniformly to $f\vat x$ whenever the derivative of $f$ (which may not exist everywhere) is square integrable.

It is possible to define Fourier coefficients for more general functions or distributions, in such cases convergence in norm or weak convergence is usually of interest.


\subsection{Properties}
We say that $f$ belongs to $C^k\vat T$ if $f$ is a $2\pi$-periodic function on $\set R$ which is $k$ times differentiable and its $k$th derivative is continuous.
\begin{itemize}
\item If $f$ is a $2\pi$-periodic \emph{odd} function, then $a_n = 0$ for all $n$.
\item If $f$ is a $2\pi$-periodic \emph{even} function, then $b_n = 0$ for all $n$.
\end{itemize}


\subsection{Examples}
We now give a Fourier series expansion of a very simple function. Consider a sawtooth wave
\begin{align*}
          f\vat x &= \dfrac{x}{\pi}&\text{for $-\pi < x < \pi$\,,}\\
f\vat{x + 2\pi k} &= f\vat x       &\text{for $-\infty < x < \infty$ and $k\in\set Z$\,.}
\end{align*}

In this case, the Fourier coefficients are given by
\begin{align*}
a_0 &= \dfrac{1}{\pi}\int_{-\pi}^{\pi}f\vat x\,\dx x = 0\,,\\
a_n &= \dfrac{1}{\pi}\int_{-\pi}^{\pi}f\vat x\cos\vat{nx}\,\dx x = 0\,,\qquad n > 0\\
b_n &= \dfrac{1}{\pi}\int_{-\pi}^{\pi}f\vat x\sin\vat{nx}\,\dx x 
     = -\dfrac{2}{n}\cos\vat{n\pi} + \dfrac{2}{\pi n^2}\sin\vat{n\pi}
     = 2\dfrac{(-1)^{n+1}}{n}\,,\qquad n\geq 1\,.
\end{align*}

It can be proven that the Fourier series converges to $f\vat x$ at every point $x$ where $f$ is differentiable and therefore:
\begin{align*}
f\vat x = \fseries fx &= \dfrac{a_0}{2} 
                          + \sum_{n = 1}^{\infty} \left( a_n\cos\vat{nx} + b_n\sin\vat{nx} \right)\,,\\
                      &= 2\sum_{n = 1}^{\infty} \dfrac{(-1)^{n+1}}{n}\,\sin\vat{nx}\qquad
                          \text{for $x-\pi\in 2\pi\set Z$\,.}
\end{align*}
When $x = \pi$, the Fourier series converges to 0, which is the half-sum of the left- and right-limit of $f$ at $x = \pi$. This is a particular instance of the Dirichlet theorem for Fourier series.

