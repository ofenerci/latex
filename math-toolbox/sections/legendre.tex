\section{Legendre Transform}
The \lingo{Legendre transform}, \aka Legendre transformation, is an involutive transformation on the real-valued convex functions of one real variable. Its generalisation to convex functions of affine spaces is sometimes called the Legendre-Fenchel transformation. It is commonly used in thermodynamics and to derive the Hamiltonian formalism of classical mechanics out of the Lagrangian formulation.

The Legendre transform exploits the equivalence between lines and points.

The Legendre transform is invertible by its own inverse: \ie, apply the Legendre transform to $f$ to find $\ltrans f$; then, apply the Legendre transform to $\ltrans f$ to arrive back to $f$.

[a graphical derivation of the legendre transform, Sam Kennerly]

The Legendre transform is a trick for representing a function in terms of its first derivative.

While mathematically rigorous descriptions are arguably unnecessary for many applications, some caution is necessary to avoid serious errors in practice. A few common sources of confusion are:
\begin{itemize}
\item failing to clearly state the necessary existence/uniqueness conditions, 
\item using notation which confuses numbers with functions and
\item misinterpreting the somewhat-ambiguous formula $px - f\vat x$.
\end{itemize}

The second error is especially popular with physicists. The symbol $y\vat x$ is often used to represent both ``the function $y\vat{}$'' and ``the value of $y$ at $x$''. This \emph{abuse of notation} is usually harmless, but it can be dangerous when \emph{change-of-variable techniques} are used. Here we will use $y$ to mean a number, $y\vat{}$ to mean a function, and $y\vat x$ to mean ``the output of $y\vat{}$ when given $x$ as an input''.


\subsection{Existence/uniqueness conditions for a Legendre transform}
Suppose all of the following statements are true:
\begin{enumerate}
\item A well-behaved function $f\vat{}$ is defined over some chunk $\set D$ of the real line.
\item For any $x\in\set D$, you know how to find $f\vat x$ and $f'\vat x$.
\item The graph of $f\vat x$ always curves upward: for any $x\in\set D$, then $f''\vat x > 0$.
\end{enumerate}

Condition 1 is deliberately vague; what do ``well-behaved'' and ``chunk'' mean? The point is: when using functions that fail common tests (\eg, continuity, non-singularity, smoothness), then be careful. A more rigorous treatment than the one provided here may be necessary.

Condition 2 simply requires that an explicit formula for $f\vat x$, its derivative can be found either by hand or by computer and that derivative is also well-behaved.

Condition 3 is \emph{not always} stated explicitly, but it should be. Legendre transformations behave very badly if the curvature of $f\vat{}$ changes sign as $x$ changes. (If $f''\vat x$ fails to exist at some points, see the subsection ``Convex functions and convex sets.'')

Suppose that instead of using $x$ as a variable, you would prefer a new variable $p$ such that $p\vat x = f'\vat x$. The Legendre transform produces a formula, in terms of $p$, for a new function $g\vat{}$. The transform is \lingo{invertible}, so knowing $g\vat p$ tells you everything about $f\vat x$.


\subsection{Geometric interpretation of the Legendre transform}
Plot $f\vat x$. At each point, imagine a line tangent to the plot. This line intersects $\tuple{x,f\vat x}$ and has slope $p = f'\vat x$. Any straight line with slope $p$ must look like this for some $g\in\set R$:
\beq
y\vat x = px - g\,.
\eeq
Here $g$ means ``the \emph{negative} $y$-intercept of the line tangent to $f\vat{}$ at the point $\tuple{x,f\vat x}$\,''. (We could have defined $g$ to be the positive $y$-intercept, but that's not the usual convention.)

Since $f''\vat x > 0$ everywhere, there is \emph{only one} tangent line for each possible slope $p$. Draw pictures to convince yourself that if a function always curves upward, it can't have two tangent lines with the same slope.

For each possible slope $p$, there is \emph{exactly one} tangent line. That tangent line has its $y$-intercept at $y = -g\vat p$. We want to find the function $g\vat{}$ that maps $p$'s to $g$'s.

The really useful thing about $g\vat{}$ is this: 
\begin{quote}
each point $\tuple{x,f\vat x}$ has exactly one ``evil twin'' point $\tuple{p,g\vat p}$. Knowing $g\vat{}$ then gives us complete information about the $f\vat{}$ and \vis.
\end{quote}


\subsection{Recipe to Find the Legendre Transform}\label{subsubsec:recipelegendretrans}
The recipe for the Legendre transform is:
\begin{enumerate}
\item Check that $f\vat{}$ satisfies the existence/uniqueness conditions.
%
\item Define a new function $p\vat{}$ such that $p\vat x\defby f'\vat x$. Then, invert $p\vat{}$ and call the result $x\vat{}$. 
%
\item Define $g$ to be the negative of the $y$-intercept of the line tangent to $f\vat{}$ at $x$:
\beq
g = p\vat x\,x - f\vat x\,.
\eeq
%
\item Use the formula for $x\vat p$ to write the $x$'s as functions of $p$. Call the result $g\vat p$:
\beq
g\vat p = p\,x\vat p - f\vat{x\vat p}\,.
\eeq
\end{enumerate}
Just be careful to remember what $x\vat p$ means: it is the value of $x$ at which the slope of $f'\vat{}$ is $f'\vat x = p$. Otherwise this equation won't make any sense.

Additionally, change the notation~\footnote{~The usage of $g$ and $p$ hides the transformation of $f$ and $x$. That's why, we prefer more explicit notation: $\ltrans f$ and $\cvar x$, respectively.} and wording to agree with more ``standard'' sources:
\begin{itemize}
\item Notation: denote the transformed function $f$ by $\ltrans f$ and denote the transformed variable $p$ by $\cvar x$.
\item Wording: consider a function $f$ with domain $\set D$ and consider $x\in\set D$. Then, call $\ltrans f$ the \lingo{Legendre transform of $f$} and call $\cvar x$ the \lingo{conjugate variable of $x$}.
\end{itemize}


\subsection{Examples}
Find the Legendre transform of $\fmap f{x}{x^2}$ with $x\in\set R$.

\begin{solution}
Follow the steps in \cref{subsubsec:recipelegendretrans} to find the solution:
\begin{itemize}
\item $f$ is well behaved, $f'$ is also well behaved and $f'' > 0$.
\item Define $p\vat x \defby f'\vat x = 2x$. Invert this to find $x\vat p$: $p\vat x = 2x\implies x\vat p = p/2$.
\item Define $g\vat x \defby p\vat x x - f\vat x$: $g\vat x = (2x)x - x^2 = x^2$.
\item Use $x\vat p = p/2$ to write the $x$'s as functions of $p$: $g\vat p = (p/2)^2 = p^2/4$.
\item The Legendre transform is finally: $\ltrans f\vat{\cvar p} = \cvar p^2/4$.
\end{itemize}
Notice how we have changed the function $f$ for its equivalent $\ltrans f$: analytically, the derivative $\cvar x$ has replaced the independent variable $x$. Geometrically, the slope (tangent) $\cvar x$ has replaced the $x$-coordinate; \ie, a line has replaced a point.
\end{solution}


\begin{example}
Invert the last Legendre transform.
\end{example}

\begin{solution}
To invert a Legendre transform means to apply the Legendre transform to an already transformed function. In the present case, transform the function $\fmap f{x}{x^2/4}$ with $x\in\set R$.

\begin{itemize}
\item $f$ is well behaved, $f'$ is also well behaved and $f'' > 0$.
\item Define $p\vat x \defby f'\vat x = x/2$. Invert this to find $x\vat p$: $p\vat x = x/2\implies x\vat p = 2p$.
\item Define $g\vat x \defby p\vat x x - f\vat x$: $g\vat x = (x/2)x - (x^2/4) = x^2/4$.
\item Use $x\vat p = 2p$ to write the $x$'s as functions of $p$: $g\vat p = (2p)^2/4 = p^2$.
\item The Legendre transform is finally: $\ltrans f\vat{\cvar p} = \cvar p^2$, which is the original function we started with in the last example!
\end{itemize}
\end{solution}


\begin{example}
Imagine a pendulum made of a very light, rigid rod of length $r$ with a dense, point-like blob of mass $m$ on one end. The other end is attached to a ball bearing which allows the pendulum to rotate \ang{360} in a vertical plane. Define $\theta$ to be the angle between the rod and a vertical line and set $\theta = 0$, when the blob is at maximum height. Choose positive $\theta$ to be clockwise or counter-clockwise. Ignore friction but don't ignore gravity. Find the Lagrangian of the system and, then, its Legrendre transform: the Hamiltonian of the system.
\end{example}

\begin{solution}
The (approximate) gravitational potential energy of this object is $V = mgy = mgr\cos\theta$. The (approximate) rotational kinetic energy is $K = i\omega^2/2 = mr^2\omega^2/2$, where $\omega$ is the pendulum's angular velocity. The Lagrangian describing the system is $L = K - V$; \ie,
\beq
L\vat{\theta, \omega} = K - V = \dfrac{mr^2\omega^2}{2} - mgr\cos\theta\,.
\eeq

The Hamiltonian of this system is found by Legendre-transforming $L$ to remove the variable $\omega$. (The variable $\theta$ comes along for the ride. For our purposes, $\theta$ can be thought of as a constant during the Legendre-transform process.) First, define $p\vat\omega = L'\vat\omega$:
\beq
p\vat\omega = L'\vat\omega = mr^2\omega\,.
\eeq
Is $L''\vat\omega > 0$ for all $\omega$? Since $L''\vat\omega = mr^2$ and $m > 0$, then it is. Note that $p$ has a physical interpretation as the pendulum's angular momentum $mr^2\omega = i\omega$.

Now invert $p\vat\omega$ to find $\omega\vat p = p/mr^2$. Define $g = p\vat\omega - L\vat\omega$ as usual, use $\omega\vat p$ to write everything in terms of $p$'s, and call the result $g\vat p$:
\beq
g\vat p = \dfrac{p^2}{2mr^2} + mgr\cos\theta\,.
\eeq

Remembering that $\theta$ is not really a constant, we should call it $g\vat{\theta,p}$. Also, traditional notation uses $H$ and $L$ instead of $g$ and $p$ for ``Hamiltonian'' and ``angular momentum''.
\beq
H\vat{\theta, L} = \dfrac{L^2}{2mr^2} + mgr\cos\theta\,.
\eeq
This is the pendulum's Hamiltonian function. It has a physical interpretation as the total (kinetic plus potential) energy of the pendulum in terms of angular position and momentum.

In most simple physical systems like this one, using a Legendre transform to find the particle's Hamiltonian seems like extra work for no clear benefit; why not just write $H = K + V$ in the first place? For many practical calculations, this is an excellent criticism. The method is primarily important for providing a theoretical motivation for quantum mechanics.
\end{solution}


