\section{Taylor Series}
In mathematics, a \lingo{Taylor series} is a representation of a function as an infinite sum of terms calculated from the values of the function's derivatives at a single point.

If the Taylor series is centered at zero, then that series is also called a \lingo{Maclaurin series}.

It is common practice to approximate a function by using a finite number of terms of its Taylor series. Taylor's theorem gives quantitative estimates on the error in this approximation. Any \emph{finite number of initial terms of the Taylor series generated by a function} is called a \lingo{Taylor polynomial}. The Taylor series of a function is the limit of that function's Taylor polynomials, provided that the limit exists. A function may not be equal to its Taylor series, even if its Taylor series converges at every point. A function that is equal to its Taylor series in an open interval (or a disc in the complex plane) is known as an \lingo{analytic function}.

\begin{definition}
Consider $f\vat x$ to be a real-valued function infinitely differentiable in a neighborhood of a real number $a\in\set R$. Then, define the \lingo{Taylor series generated by $f$ at the point $a$}, denoted $\tseries{f}{x}{a}$, by the power series
%
\begin{equation}\label{eq:deftaylorseries}
\tseries{f}{x}{a} = \sum_{k = 0}^{\infty}\,\dfrac{f^{(k)}\vat a}{k!}(x - a)^k\,,
\end{equation}
%
where $k!$ denotes the \lingo{factorial of $k$} and $f^{(k)}\vat a$ the $k$th derivative of $f$ evaluated at the point $a$. The derivative of order zero $f$ is defined to be $f$ itself and $(x - a)^0$ and $0!$ are both defined to be 1.
\end{definition}

Call any finite number of initial terms, say $n$, of the Taylor series of the function $f$ a \lingo{Taylor polynomial of degree $n$ generated by $f$ at the point $a$}, denoted $\ntpol{n}{f}{x}{a}$, by
\begin{equation}\label{eq:deftaylorpoly}
\ntpol{n}{f}{x}{a} = \sum_{k = 0}^{n}\,\dfrac{f^{(k)}\vat a}{k!}(x - a)^k\,,
\end{equation}
Also refer to the Taylor polynomial of degree $n$ generated by $f$ at the point $a$ as the \lingo{$n$-degree Taylor polynomial generated by $f$ at $a$}.

Finally, when $a = 0$, refer to Taylor series as \lingo{Maclaurin series}.


\subsection{Properties}
The Taylor operator $\ntpol{n}{}{}{}$ has the following properties:
\begin{itemize}
\item Linearity property: if $c_1$ and $c_2$ are constants, then 
\beq
\ntpol{n}{}{}{}(c_1 f + c_2 g) = c_1 \ntpol{n}{f}{}{} + c_2 \ntpol{n}{g}{}{}\,.
\eeq
%
\item Differentiation property: the derivative of a Taylor polynomial of $f$ is a Taylor polynomial of $f'$:
\beq
(\ntpol{n}{f}{}{})' = \ntpol{n-1}{f'}{}{}\,.
\eeq
%
\item Integration property: an indefinite integral of a Taylor polynomial of $f$ is a Taylor polynomial of and indefinite integral of $f$. More precisely, if $g\vat x = \int_{a}^{x}\,f\vat t\,\dx t$, then
\beq
\ntpol{n + 1}{g}{x}{a} = \int_{a}^{x}\,\ntpol{n}{f}{t}{}\,\dx t\,.
\eeq
%
\item The Maclaurin series of an even function includes only even powers.
%
\item The Maclaurin series of an odd function includes only odd powers.
%
\end{itemize}


\subsection{Geometry}
The 1-degree Taylor polynomial is the tangent line to $f\vat x$ at $x = a$:
\beq
\ntpol{1}{f}{x}{a} = f\vat a + f'\vat a\,(x - a)\,.
\eeq
This is often called the \lingo{linear approximation to $f\vat x$ near $x = a$}; \ie, the \lingo{tangent line to the graph}. Therefore, view Taylor polynomials as a generalization of \lingo{linear approximations}. In particular, the 2-degree Taylor polynomial is sometimes called the \lingo{quadratic approximation}, the 3-degree Taylor polynomial is the \lingo{cubic approximation} and so forth.


\subsection{Applications}
Use Taylor series and Taylor polynomials for three important applications:
\begin{enumerate}
\item to find the sum of a series;
\item to evaluate limits;
\item to approximate functions.
\end{enumerate}


\subsection{Taylor's Theorem}
Let $k\geq 1$ be an integer and let the function $\fdef f{\set R}{\set R}$ be $k$ times differentiable at the point $a\in\set R$. Then, there exists a function $\fdef{h_k}{\set R}{\set R}$ such that
\beq
f\vat x = f\vat a 
          + f'\vat a(x - a) 
          + \dfrac{1}{2!} f''\vat a(x - a)^2
          + \dotsb
          + \dfrac{1}{k!} f^{(k)}\vat a(x - a)^k
          + h_k\vat x(x - a)^k
\eeq
and $\lim_{x\to a}h_k\vat x = 0$\,. This is called the \lingo{Peano form of the remainder}.

The polynomial appearing in Taylor's theorem is the $k$-th order Taylor polynomial
\beq
P_k\vat x = f\vat a 
          + f'\vat a(x - a) 
          + \dfrac{1}{2!} f''\vat a(x - a)^2
          + \dotsb
          + \dfrac{1}{k!} f^{(k)}\vat a(x - a)^k
\eeq
of the function $f$ at the point $a$. The Taylor polynomial is the unique ``asymptotic best fit'' polynomial in the sense that if there exists a function $\fdef{h_k}{\set R}{\set R}$ and a $k$-th order polynomial $p$ such that
\beq
f\vat x = p\vat x + h_k\vat x (x - a)^k\,,\qquad \lim_{x\to a} h_k\vat x = 0\,,
\eeq
then $p = P_k$. Taylor's theorem describes the asymptotic behavior of the remainder term
\beq
R_k\vat x = f\vat x - P_k\vat x\,,
\eeq
which is the approximation error when approximating $f$ with its Taylor polynomial.


\subsection{Formulae for the Remainder}

\subsubsection{Mean-value forms of the remainder}
Let $\fdef f{\set R}{\set R}$ be $k + 1$ times differentiable on the open interval and continuous on the closed interval between $a$ and $x$. Then,
\beq
R_k\vat x = \dfrac{f^{(k + 1)}(\xi_L)}{(k + 1)!}(x - a)^{k + 1}\,,
\eeq
for some real number $\xi_L$ between $a$ and $x$. This is the \lingo{Lagrange form of the remainder}. Similarly,
\beq
R_k\vat x = \dfrac{f^{(k + 1)}(\xi_C)}{k!}(x - \xi_C)^k (x - a)\,,
\eeq
for some real number $\xi_C$ between $a$ and $x$. This is the \lingo{Cauchy form of the remainder}.


\subsubsection{Integral Form of the Remainder}
Let $f\vat k$ be absolutely continuous on the closed interval between $a$ and $x$. Then,
\beq
R_k\vat x = \int_{a}^{x} \dfrac{f^{(k + 1)}\vat t}{k!}(x - t)^k\,\dx t\,.
\eeq


\subsubsection{Estimates for the Remainder}
It is often useful in practice to be able to estimate the remainder term appearing in the Taylor approximation, rather than having a specific form of it. Suppose that $f$ is $(k+1)$-times continuously differentiable in an interval $I$ containing $a$. Suppose that there are real constants $q$ and $Q$ such that
\beq
q \leq f^{(k+1)}\vat x \leq Q\,,
\eeq
throughout $I$. Then, the remainder term satisfies the inequality
\beq
q\dfrac{(x-a)^{k+1}}{(k + 1)!} \leq R_k\vat x \leq Q\dfrac{(x-a)^{k+1}}{(k + 1)!}
\eeq
if $x > a$, and a similar estimate if $x < a$. This is a simple consequence of the Lagrange form of the remainder. In particular, if
\beq
\magn{f^{(k+1)}\vat x} \leq M
\eeq
on an interval $I = \, ]a-r, a+r[$ with some $r > 0$, then
\beq
\magn{R_k\vat x} \leq M\dfrac{\magn{x-a}^{k+1}}{(k+1)!}\leq M\dfrac{r^{k + 1}}{(k+1)!}
\eeq
for all $x\in\,]a-r, a+r[$. The second inequality is called a \lingo{uniform estimate}, because it holds uniformly for all $x$ on the interval $x\in\,]a-r, a+r[$.


\subsection{Examples}

\subsubsection{Taylor Series and Maclaurin series}
Calculate the Taylor series of $\sin x$ at $x = 0$.

\begin{solution}
Let $f\vat x = \sin x$. Then,
\begin{align*}
& f\vat x    = \sin x   \implies f\vat 0       = \sin 0  = 0\,,\\
& f'\vat x   = \cos x   \implies f'\vat 0      = \cos 0  = 1\,,\\
& f''\vat x  = -\sin x  \implies f''\vat 0     = -\sin 0 = 0\,,\\
& f'''\vat x = -\cos x  \implies f'''\vat 0    = -\cos 0 = -1\,,\\
& f^{iv}\vat x = \sin x \implies f^{iv}\vat 0 = \sin 0 = 0\,,\\
& \cdots
\end{align*}
Note that the fourth derivative takes us back to the start point, so these values repeat in a cycle of four as $0,1,0,-1$; $0,1,0,-1$ and so on, with only the odd powers of $x$ appearing in the polynomials. Besides, note that the function $\sin$ is infinitely differentiable and that all of its derivatives exist at $x = 0$. Therefore, plug these results into \cref{eq:deftaylorpoly} to find
\beq
\ntpol{2n + 1}{\sin}{x}{0} = x 
                            - \dfrac{x^3}{3!} 
                            + \dfrac{x^5}{5!} 
                            - \dfrac{x^7}{7!} 
                            + \dotsb 
                            = \sum_{n = 0}^{\infty} (-1)^n\dfrac{x^{2n + 1}}{(2n + 1)!}\,.
\eeq

Working on a similar fashion, find the Maclaurin series for the cosine function
\beq
\ntpol{2n}{\cos}{x}{0} = 1 
                         - \dfrac{x^2}{2!} 
                         + \dfrac{x^4}{4!} 
                         + \dotsb 
                       = \sum_{n = 0}^{\infty} (-1)^n\dfrac{x^{2n}}{(2n)!}\,.\mqed
\eeq
\end{solution}

\begin{note}
The calculation of Maclaurin series is eased by noting the parity of functions. For instance, the sine function is odd; that is, $-\sin x = \sin -x$. Thus, Maclaurin series generated by the sine function will have only odd powers. In this way, calculate only the odd powers of the series, not all.

Similarly, the cosine function is even: $\cos x = \cos -x$. Thus, Maclaurin series generated by the cosine function will have only even powers. Therefore, calculate only the even powers of the series.
\end{note}


Calculate the Maclaurin series of $e^x$.

\begin{solution}
Let $f\vat x = e^x$. Then,
\begin{align*}
& f\vat x   = e^x \implies f\vat 0   = e^0 = 1\,,\\
& f'\vat x  = e^x \implies f'\vat 0  = e^0 = 1\,,\\
& f''\vat x = e^x \implies f''\vat 0 = e^0 = 1\,,\\
& \cdots
\end{align*}
Since $e^x$ is infinitely differentiable and that all of its derivatives exist at 0, then, plug these results into \cref{eq:deftaylorseries} to find
\beq
\tseries{e}{x}{0} = 1 
                    + \dfrac{x}{1!} 
                    + \dfrac{x^2}{2!} 
                    + \dfrac{x^3}{3!} 
                    + \dotsb 
                  = \sum_{n = 0}^{\infty}\dfrac{x^{n}}{n!}\,.\mqed
\eeq
\end{solution}


\subsubsection{Sum of Series}
Find the sum of the following series:
\beq
\sum_{n = 0}^{\infty}\,\dfrac{1}{n!} = 1 + \dfrac{1}{1!} + \dfrac{2}{2!} + \dfrac{3}{3!} + \dotsb \,.
\eeq

\begin{solution}
Substitute $x = 1$ in the Taylor series generated by $e^x$ to find the sum of the given series:
\beq
\tseries{e}{1}{0} = 1 + \dfrac{1}{1!} + \dfrac{1}{2!} + \dfrac{1}{3!} + \dotsb = e \,.
\eeq

If $x = -1$ is substituted, then
\beq
\tseries{e}{-1}{0} = \sum_{n = 0}^{\infty}\,\dfrac{(-1)^n}{n!} = \dfrac{1}{e}\,.
\eeq
\end{solution}


\subsubsection{Limits}
Evaluate $\lim_{x\to 0} \dfrac{\sin x - x}{x^3}$.

\begin{solution}
Plug in the Taylor series generated by $\sin x$:
\begin{align*}
\lim_{x\to 0} \dfrac{\sin x - x}{x^3} 
    &= \lim_{x\to 0}\dfrac{\left(x - \dfrac{x^3}{3!} + \dfrac{x^5}{5!} - \dfrac{x^7}{7!} + \dotsb \right) - x}{x^3}\,,\\
    &= \lim_{x\to 0}\dfrac{-\dfrac{x^3}{3!} + \dfrac{x^5}{5!} - \dfrac{x^7}{7!} + \dotsb}{x^3}    \,,\\
    &= \lim_{x\to 0}\left(-\dfrac{1}{3!} + \dfrac{1}{5!}x^2 - \dfrac{1}{7!}x^4 + \dotsb \right)   \,,\\
    &= -\dfrac{1}{6}\,.\mqed
\end{align*}
\end{solution}


\subsubsection{Approximations}
Find the 5-degree Taylor polynomial generated by $\sin\theta$ at $\theta = 0$. Use this result to approximate $\sin 0.3$.

\begin{solution}
The 5-degree Taylor polynomial generated by $\sin\theta$ is given by
\beq
\ntpol{2n + 1}{\sin}{\theta}{0} = \sum_{k = 0}^{5}\,(-1)^n\dfrac{\theta^{2n + 1}}{(2n + 1)!}
                                = \theta - \dfrac{\theta^3}{3!} + \dfrac{\theta^5}{5!}
                                = \theta - \dfrac{\theta^3}{6} + \dfrac{\theta^5}{120}\,.
\eeq
Approximate $\sin 0.3$ by setting $\theta = 0.3$ in the last equation
\beq
\ntpol{2n + 1}{\sin}{\theta}{0} = 0.3 - \dfrac{0.3^3}{6} + \dfrac{0.3^5}{120} = \num{0.295 520 25}\,.\mqed
\eeq
\end{solution}
