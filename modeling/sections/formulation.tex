\section{Physical processes}
A swinging gravitational pendulum is an instance of a dynamics process, since a description of motion, based on its causes, is the final aim. 

There are two main cases to consider when studying the motion of a gravitational pendulum: undamped motion, where no frictional forces and no drag are taken into account, and damped motion, where frictional forces or drag are considered. In both cases, however, the interplay between kinetic energy and gravitational potential must be regarded, since it drives motion.

To uncover the relationships between the different physical quantities that affect the pendulum motion, one firstly has to propose them; then, join them as dimensionless quantities and, finally, use physical considerations and order of magnitude analysis to restrain the physical model to thus decrease complexity. The last step will pave the path to a precise mathematical model.


%%% LATEX vars used in the following
\newcommand{\angpos}{\theta}     % angular position
\newcommand{\angvel}{\dt\theta}  % angular velocity
\newcommand{\angacc}{\ddt\theta} % angular acceleration


\subsection{Dimensional and order of magnitude analyses}
First, one can hypothesize that the pendulum bob hangs by a massless, frictionless and inflexible rod. This implies that the system center of gravity will coincide with the bob center of gravity (massless rod), that no damping due to friction will happen and that the bob will trace a circular orbit of radius equal to the length of the rod (inflexible rod).

Since the problem belongs to dynamics, we choose the dimensional set $\elset{\phdim L, \phdim M, \phdim T}$, with a cardinality of three. Next, let us list the possible physical quantities that may influence the pendulum motion together with their symbols and dimensions, see \cref{tab:quantitiesgravpendulum}.
%
% ------------------------------------------------------------- PreTable
\docpretable{bt}{0.5\textwidth}{lcc}%
% ------------------------------------------------------------ PostTable
\toprule
Quantity    & Symbol    & Dimension \\
\midrule
pendulum amplitude        & $\angpos$          & $ 1$ \\
initial amplitude         & $\angpos_0$        & $ 1$ \\
time                      & $t$                & $\phdim T$ \\
free fall acceleration    & $\grav$            & $\phdim L/\phdim T^2$ \\
rod length                & $\length$          & $\phdim L$ \\
bob mass                  & $\mass\txt{bob}$   & $\phdim M$ \\
bob density               & $\dens\txt{bob}$   & $\phdim M/\phdim L^3$ \\
bob diameter              & $\diam\txt{bob}$   & $\phdim L$ \\
fluid density             & $\dens\txt{fl}$    & $\phdim M\phdim L^3$ \\
fluid dynamic viscosity   & $\dynvis\txt{fl}$  & $\phdim M/\phdim L\phdim T$ \\
\bottomrule
%
\end{tabularx}
\docposttable{Quantities for pendulum motion}
    {Physical quantities involved in the motion of a gravitational pendulum}
    {tab:quantitiesgravpendulum}
% ------------------------------------------------------------- EndTable

A first approach may be to model the pendulum amplitude by a function $f$ of the form
\beq
\angpos = f\vat{
            \angpos_0,
            t,
            \grav,
            \length,
            \mass\txt{bob},
            \dens\txt{bob},
            \diam\txt{bob},
            \dens\txt{fl},
            \dynvis\txt{fl}
}\,.
\eeq
This complex relationship can be organized by means of using dimensional analysis.

To begin with, since $\dens\txt{bob}$, $\mass\txt{bob}$ and $\diam\txt{bob}$ are related, we can keep two of the three quantities. We keep diameter and density. Then, note that there are 6 physical quantities and 3 independent dimensions. Thus, according to the Pi-theorem, the number of dimensionless quantities that can be formed is $9 - 3 = 6$. These quantities are
\beq
\kdim_1 = \angpos\,,\quad
\kdim_2 = \angpos_0\,,\quad
\kdim_3 = \dfrac{\diam}{\length}\,,\quad
\kdim_4 = \dfrac{\dens\txt{fl}}{\dens\txt{bob}}\,,\quad
\kdim_5 = t\sqrt{\dfrac{\grav}{\length}}\,,\quad
\kdim_6 = \dfrac{\dens\txt{fl}\diam\sqrt{\length\grav}}{\dynvis}\,.
\eeq
Again, according to the Pi-theorem, the mathematical function we seek is of the \emph{form}:
\beq
\angpos = \kdimf\vat{\angpos_0, 
            \dfrac{\diam}{\length}, 
            \dfrac{\dens\txt{fl}}{\dens\txt{bob}}, 
            t\sqrt{\dfrac{\grav}{\length}}, 
            \dfrac{\dens\txt{fl}\diam\sqrt{\length\grav}}{\dynvis}}\,.
\eeq

Now, we use order of magnitude analysis to further restrain the physical model so to simplify the mathematical model. We assume:
\begin{itemize}
\item $\length\gg\diam$, then $\diam/\length\to 0$. The validity of this assumption depends on one's tolerance; \eg, if the bob diameter is, say, $\SI{5}{cm}$ and the rod length $\SI{1}{m}$, then the ratio $\diam/\length = 0.05$, which might be discarded for some purposes.
%
\item $\dens\txt{bob}\gg\dens\txt{fl}$, then $\dens\txt{fl}/\dens\txt{bob}\to 0$. Say, for instance, that the bob is made of steel and swings through air at $\SI{15}{\celsius}$ and at sea level, then 
\beq
\dens\txt{air}/\dens\txt{steel} = \SI{1.225}{kg/m^3}/\SI{7750}{kg/m^3} \sim \num{0.158e-4}\,, 
\eeq
which can be ignored. However, if the steel bob swings through oil with $\dens\txt{oil} = \SI{850}{kg/m^3}$, then 
\beq
\dens\txt{oil}/\dens\txt{steel} = \SI{850}{kg/m^3}/\SI{7750}{kg/m^3} \sim \num{0.110}\,, 
\eeq
which may be important for some applications.
%
\item an inviscid fluid. An inviscid fluid is a fluid with no viscosity, then 
\beq
\dens\txt{fl}\diam\sqrt{\length\grav}/\dynvis = 0\,. 
\eeq
The validity of this assumption will also depend on the circumstances.
\end{itemize}

Then, after having restrained the physical model, we seek for a mathematical function of the form
\beq
\angpos = \kdimf\vat{
            \angpos_0, 
            t\sqrt{\dfrac{\grav}{\length}}
            }\,.
\eeq

The functional form of $\kdimf$ must be found by a more refined theoretical analysis or by experimentation. However, dimensional analysis reduced the complex model by passing from ten to six dimensionless quantities. Moreover, order of magnitude analysis further limited the model to two dimensionless quantities. The physical model is more restricted, but based on sensible assumptions. At the end, only confrontation of the model with experimental data will provide the accuracy of the reductions made.


\subsection{Mathematical interpretation}
The mathematical model wishes to find a function to predict the pendulum amplitude variation with time: $\angpos = \angpos\vat t$.

Now, based on the previous section, we can combine all the previous assumptions by defining a \lingo{simple gravitational pendulum} as a pendulum composed of a massive bob of mass $\mass$ hanging by a massless, frictionless and inflexible rod of length $\length$ attached to a frictionless pivot. Under the influence of gravitational interactions, the pendulum swings through an inviscid fluid of negligible density.

The last definition leads to a very restricted physical model. However, this will allow us to finally find a closed form for the pendulum amplitude.


\subsection{Mathematical model}
Consider a simple gravitational pendulum composed of a bob of mass $\mass$ and a rod of length $\length$. Let $\angpos$ be the amplitude of the pendulum for any time $t$. Let the initial angle be $\angpos_0$ and the initial velocity be $\angvel_0 = 0$. Finally, let $\grav$ be the free fall acceleration. Then, find the equation of motion for the pendulum.

%\begin{solution}
Using $\angpos$ as the generalized position and $\angvel$ as the generalized velocity, write the Lagrangian for the system:
\beq
\lag = \dfrac{1}{2}\mass\length^2\angvel^2 - \mass\grav\length\cos\vat{\angpos}\,.
\eeq
Find next the generalized momentum and its temporal change
\beq
\ipd{\angvel}{\lag} = \mass\length^2\angvel\implies
\iod{t}\ipd{\angvel}{\lag} = \mass\length^2\angacc\,.
\eeq
Calculate the generalized force:
\beq
\ipd{\angpos}{\lag} = -\mass\grav\length\sin\vat\angpos\,.
\eeq
Replace the generalized force and the temporal change of the generalized momentum in the Euler-Lagrange equation to find:
\beq
\elop{\angpos}{\angvel}\lag = \mass\length^2\angacc + \mass\grav\length\sin\vat\angpos = 0\,.
\eeq
Since $\mass\length > 0$, divide the last equation through $\mass\length^2$ to have
\beq
\angacc + \dfrac{\grav}{\length}\sin\vat\angpos = 0\,,
\eeq
object to the initial conditions $\angpos\vat 0 = \angpos_0$ and $\angvel\vat 0 = 0$.
%\end{solution}

Note that we deduced the equation by means of Lagrangian mechanics. Writing down the Lagrangian was possible due to the fact that the system was assumed to be conservative; \ie, neither friction nor drag were considered.


\subsection{Assumptions}
The equation of motion for the pendulum was found using the following assumptions:
