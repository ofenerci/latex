\section{Guide}
This section provides with guidelines to approach the mathematical modeling of physical phenomena.


\subsection{Problem background}
A mathematical model of a physical phenomenon begins with the \lingo{problem background}, a description or introduction to the object. It should contain 
\begin{itemize}
\item a \lingo{description} of the essential features of the physical process and 
\item an identification of the \lingo{objectives}, the key questions requiring answers.
\end{itemize}

Answering what, who, where, how and why questions guides to write down the description. Additionally, including graphical illustrations aids not only in the description, but in the definition of physical quantities and the establishment of hypotheses, as well.


\subsection{Problem formulation}
The \lingo{problem formulation} aims to:
\begin{itemize}
\item identify key physical processes;
\item interpret these processes mathematically;
\item establish a mathematical model -- governing equations and suitable initial and boundary conditions;
\item state clearly the assumptions.
\end{itemize}

The formulation must be based on sound physical principles, experimental facts or laws expressed in mathematical terms. As a guide, then, define the physical framework (geometry, kinematics, dynamics, thermal transfer and so on), state a dimensional set and then define the physical quantities, constants, parameters, coefficients and provide their dimensions in the chosen set.

A more formal approach is to begin with educated guessing, followed by dimensional analysis, order of magnitude analysis~\footnote{Order of magnitude analysis is preceded by dimensional analysis, since \emph{only} the comparison of \emph{dimensionless} quantities is meaningful!}, analysis of extreme cases, simplifications and ends with a restricted model. The end result may be less accurate to fit experimental data, but less complex and thus more understandable. If fitting is not satisfactory, one can relax simplifications, one at a time, until a desired, or required, agreement is found.


\subsection{Analysis}
Once the physical and mathematical models and their assumptions have been proposed, one regularly faces a set of equations, probably differential equations together with initial and boundary conditions. The next step is then to analyze the set by
\begin{itemize}
\item non-dimensionalize the equations, included initial and boundary conditions;
\item making analogies with other related problems or phenomena, as the case of mass, energy and momentum transport and
\item relying on analytic and numeric methods, obtaining solutions (results).
\end{itemize}

In the case of obtaining analytic solutions to differential equations, it is necessary to verify that they satisfy the differential equations object to the initial and boundary conditions.

As a final step, uncertainty analysis should be performed, to obtained ranges of validity, instead of punctual solutions.


\subsection{Results}
The final step of modeling physical phenomena is to present results, give conclusions and discussion them. Specifically, one should
\begin{itemize}
\item interpret results with respect to the original physical process and objectives;
\item verify assumptions by confronting the results with reference values or experimental data and
\item identify the solution limitations and possible extensions.
\end{itemize}
