\section{Simple harmonic motion}
In this section, we present a small account of the simple harmonic motion of a spring-mass system -- a simple harmonic oscillator.


\subsection{Background}
We present, in the following, a brief description of the harmonic motion, emphasizing in simple harmonic motion. Then, we set the main objectives of the current analysis.


\subsubsection{Description}
A \lingo{harmonic oscillator} is a system that, when displaced from its equilibrium position, experiences a restoring force, $\force$, proportional to a displacement, $\disp$:
\beq
\force = -\kspring\disp\,,
\eeq
where $\kspring$ as a strictly positive constant.

If $\force$ is the only force acting on the system, then the system is called a \lingo{simple harmonic oscillator} and its motion is said to be a \lingo{simple harmonic motion}. Note that the force depends only on the position, thus itcan be written as the gradient of a \lingo{potential}, $\pen$; \ie, as $\force = -\grad\pen$. Additionally, since there are no other forces present -- such as drag, buoyancy, gravity and so on --, mechanical energy is conserved.

An instance of a harmonic oscillator is a \lingo{spring-mass system}. In such a system, $\force$ is given by \lingo{Hooke's law} and $\kspring$ is a constant factor characteristic of the spring, its \lingo{stiffness}. Regularly, the system is set into motion by stretching or contracting the mass together with the spring a distance $\odisp_0$ from the mass equilibrium position, called the \lingo{amplitude}, the maximum displacement, with null initial velocity, $\ovel = 0$.


\subsubsection{Objective}
The goal is to obtain a closed form mathematical function to predict the amplitude for a simple harmonic oscillator.


\subsection{Physical processes}
We go now into a more physical and mathematical approach to the analysis of the simple harmonic oscillator.


\subsubsection{Educated guessing}\label{subsubsec:guessingsimpleoscillator}
As a first approach to analyze the spring-mass system, we estimate the oscillator~\footnote{~Hereafter, oscillator will refer to the spring-mass system.} period.

Consider an oscillator composed of a mass $\mass$ and a spring of stiffness $\kspring$. After having set into motion by displacing the oscillator a distance $\odisp_0$, it experiences a force $\force\sim\kspring\odisp_0$ by the spring, which tries to restore the oscillator to its equilibrium position. This force accelerates the oscillator at $\oacc\sim\kspring\odisp_0/\mass$. During a time $\oper$, the oscillator travels a distance $\oacc\oper^2\sim\kspring\odisp_0\oper^2/\mass$. On the other hand, to complete a cycle, the oscillator has to travel a distance $\odisp\sim 2\odisp_0\sim\odisp$. Now, equating both distances, one finds that $\kspring\odisp_0\oper^2/\mass\sim\odisp_0$, which leads finally to an estimate of the oscillator period
\beq
\oper\sim\sqrt{\dfrac{\mass}{\kspring}}\,.
\eeq

It can be seen that $\odisp$ dependency on $\kspring$ and $\mass$ is not linear. Moreover, it depends on the ratio $\kspring/\mass$. Finally, the last equation implies that simple harmonic motion is \lingo{isochronous}; \ie, the period and frequency are independent of the amplitude.


\subsubsection{Dimensional analysis}
We would like to find the form of a dimensionless function of the physical quantities affecting the oscillator (spring-mass) motion.

The problem belongs to mechanics, so we choose the dimensional set $\elset{\phdim F, \phdim L, \phdim T}$, with cardinality of three. Using this set, consider \cref{tab:quantitiesharmonicscillator} as the list of the hypothesized physical quantities affecting oscillator motion,
%
% ------------------------------------------------------------- PreTable
\docpretable{bt}{0.51\textwidth}{lcc}%
% ------------------------------------------------------------ PostTable
\toprule
\multicolumn{1}{c}{Quantity} & Symbol    & Dimension \\
\midrule
%
Displacement            & $\odisp$    & $\phdim L$ \\    
Initial displacement    & $\odisp_0$  & $\phdim L$ \\
Spring stiffness        & $\kspring$  & $\phdim F/\phdim L$ \\
System mass             & $\mass$     & $\phdim F\phdim T^2/\phdim L$ \\
Time                    & $t$         & $\phdim T$ \\
\bottomrule
%
\end{tabularx}
\docposttable{Quantities for harmonic oscillator motion}
    {Physical quantities involved in the motion of a harmonic oscillator}
    {tab:quantitiesharmonicscillator}
% ------------------------------------------------------------- EndTable

There are three base dimensions and five physical quantities, thus, according to the Pi-theorem, $5 - 3 = 2$ dimensionless quantities can be formed:
\begin{equation}\label{eq:dimlessquantitiessimpleoscillator}
\kdim_1 = \dfrac{\odisp}{\odisp_0}\qquad\text{and}\qquad
\kdim_2 = t\sqrt{\dfrac{\kspring}{\mass}}\,.
\end{equation}
Then, again by the Pi-theorem, we seek for a dimensionless function, $\kdimf$, of the form
\beq
\kdim_1 = \kdimf\vat{\kdim_2}\implies 
\dfrac{\odisp}{\odisp_0} = \kdimf\vat{t\sqrt{\dfrac{\kspring}{\mass}}}\,.
\eeq
The closed form of $\kdimf$ must be found by a more complex theoretical analysis.


\subsubsection{Mathematical model}
Since the problem involves forces, we use Newton's formulation of mechanics to find the equation of motion for the simple harmonic oscillator.

Consider a simple harmonic oscillator consisting of a mass $\mass$ connected to a spring of stiffness $\kspring$ set into motion by initially displacing the mass a distance $\odisp_0$ with no velocity. Then, find the equation of motion for the oscillator displacement $\odisp$ for any time $t$.

Apply Newton's second law of motion to the oscillator to find
\beq
\mass\oacc = -\kspring\odisp\,,
\eeq
where $\oacc$ is the oscillator acceleration produced by the restoring force $\force = -\kspring\odisp$.

Since $\mass > 0$, divide the last equation through $\mass$ to have
\beq
\oacc + \dfrac{\kspring}{\mass}\odisp = 0\,.
\eeq

Lastly, join the initial conditions to the last equation:
\begin{equation}\label{eq:eqmotionsimpleharmonicoscillator}
  \begin{cases}
      \oacc\vat t + \dfrac{\kspring}{\mass}\odisp\vat t = 0\,, \\
      \odisp\vat 0 = \odisp_0\,, \\
      \ovel\vat 0 = 0\,,
  \end{cases}
\end{equation}
which results in the equation of motion for the simple harmonic oscillator.


\subsection{Analysis}
Now, we solve \cref{eq:eqmotionsimpleharmonicoscillator} to find a closed form of $\odisp\vat t$.


\subsubsection{Non-dimensionalization}
Consider \cref{eq:eqmotionsimpleharmonicoscillator}. The independent quantity is $t$, the dependent one $\odisp$ and the parameters are $\kspring$ and $\mass$.

Non-dimensionalize $\odisp$ using $\kdim_1$ found in \cref{eq:dimlessquantitiessimpleoscillator}:
\beq
\scpq\odisp = \kdim_1 = \dfrac{\odisp}{\odisp_0}\implies
\odisp = \odisp_0\scpq\odisp\,. 
\eeq
Find the differentials of $\scpq\odisp$:
\beq
   \dx\odisp = \odisp_0\dx\scpq\odisp\qquad\text{and}\qquad
\ndx 2\odisp = \odisp_0\ndx 2\scpq\odisp\,.
\eeq

Then, non-dimensionalize $t$ using $\kdim_2$, also found in \cref{eq:dimlessquantitiessimpleoscillator}:
\beq
\scpq t = \kdim_2 = t\sqrt{\dfrac{\kspring}{\mass}}\implies
t = \scpq t\sqrt{\dfrac{\mass}{\kspring}}\,.
\eeq
Calculate the differentials of $\scpq t$:
\beq
\dx t = \dx\scpq t\sqrt{\dfrac{\mass}{\kspring}}\qquad\text{and}\qquad
\dx t^2 = \dx\scpq t^2\dfrac{\mass}{\kspring}\,.
\eeq

Replacing $\scpq\odisp$, $\scpq t$ and their differentials in the equation of motion and in its initial conditions gives
\begin{equation}\label{eq:dimlessequationmotionsimpleoscillator}
\begin{cases}
\scoacc\vat{\scpq t} + \natfreq^2\scpq\odisp\vat{\scpq t} = 0\,, \\
\scpq\odisp\vat 0 = 1\,,\\
\scovel\vat 0 = 0\,.
\end{cases}
\end{equation}
where the derivatives are to be taken with respect to $\scpq t$ and $\natfreq$ is defined as
\beq
\natfreq = \sqrt{\dfrac{\kspring}{\mass}}\,.
\eeq
The parameter $\natfreq$ is called the \lingo{oscillator natural frequency}.


\subsubsection{Analytic solution}
\Cref{eq:dimlessequationmotionsimpleoscillator} is a second-order, linear ordinary differential equation with constant coefficient whose solution is given by
\beq
\scpq\odisp = \cos\vat{\scpq t}\,,
\eeq
or, in dimensional form,
\beq
\odisp = \odisp_0\cos\vat{t\sqrt{\dfrac{\kspring}{\mass}}}
       = \odisp_0\cos\vat{t\natfreq}\,.
\eeq


\subsection{Results}
The oscillator natural frequency is related to the \lingo{temporal frequency} $\cycfreq$, , by
\beq
\natfreq = 2\pi\cycfreq\,.
\eeq
In turn, $\cycfreq$ is related to the oscillator period, $\oper$, by
\beq
\oper = \dfrac{1}{\cycfreq} 
      = \dfrac{2\pi}{\natfreq} 
      = 2\pi\sqrt{\dfrac{\mass}{\kspring}}\,,
\eeq
which agrees with what was guessed in \cref{subsubsec:guessingsimpleoscillator}; implying that simple harmonic motion is \lingo{isochronous}.
