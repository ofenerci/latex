\section{Results}
Although physically and mathematically restrained with respect to the original problem, an undamped pendulum,  \cref{eq:solsimppendsmallangle} does provide a closed form function to predict the amplitude variation of a simple gravitational pendulum with respect to time. 

Hereafter, we discuss this equation under physical grounds and confront its predictions with experimental data.


\subsection{Theoretical discussion}
In this section, we investigate whether \cref{eq:solsimppendsmallangle} satisfies the principles of momentum and energy conservation.

[consistency on the description of motion: when particle moves to the right, the force points to the left (grav. potential energy restores kinetic energy), and \vis.

circular motion: amplitude describes a circle, since inflexible rod]


\subsubsection{Analogies with other phenomena}\label{subsec:analogieswithotherphenomena}
In classical \lingo{simple harmonic motion}, the period of the motion, $\oscper$, is the time required for a complete oscillation and defined by
\beq
\oscper = \dfrac{2\pi}{\natfreq}\,,
\eeq
where $\natfreq$ is the motion \lingo{natural frequency}.

The motion of a simple gravitational pendulum, described by \cref{eq:solsimppendsmallangle}, is an instance of simple harmonic motion, where $\angpos_0$ is the semi-amplitude of the oscillation and where the natural frequency is
\beq
\natfreq = \sqrt{\dfrac{\grav}{\length}}\,.
\eeq
The period of the pendulum motion, for the outward and return, is thus
\begin{equation}\label{eq:simplependulumperiod}
\oscper = 2\pi\sqrt{\dfrac{\length}{\grav}}\,,
\end{equation}
which is Christiaan Huygens's law for the period.

Note that only under the small-angle approximation, the period is independent of the amplitude $\angpos_0$; \ie, \lingo{isochronism} -- the property Galileo discovered.


\subsubsection{Momentum conservation}
Momentum of a system is conserved if there are no interactions; \ie, $\dx\mom = 0$. Now, since we departed from the hypothesis that gravity acts on the simple gravitational pendulum, \cref{eq:solsimppendsmallangle} should \emph{not} preserve momentum~\footnote{~In the grand scheme of things, momentum \emph{is} conserved, but to see this, we would need to add Earth's momentum as well as the pendulum's.}.

To begin with, find the pendulum angular velocity by differentiating \cref{eq:dimlesssolsimppendsmallangle}, the dimensionless form of \cref{eq:solsimppendsmallangle}, with respect to $\scpq t$:
\begin{equation}\label{eq:angposandangvel}
\angvel = \iod{\scpq t}\angpos = -\angpos_0\sin\vat{\scpq t}\,.
\end{equation}
Next, replace the pendulum angular velocity in \cref{eq:genmomentumanditstempchange}, the pendulum momentum:
\beq
\igmom\angpos = \mass\length^2\angvel 
              = - \mass\length^2\angpos_0\sin\vat{\scpq t}\,.
\eeq
Finally, find the momentum time derivative:
\beq
\dtigmom\angpos = - \mass\length^2\angpos_0\cos\vat{\scpq t}\,.
\eeq
Since $\dtigmom\angpos$ is \emph{not} zero, momentum is \emph{not} conserved during motion. Thus, \cref{eq:solsimppendsmallangle} accords with physical principles.


\subsubsection{Energy conservation}
In Hamilton's formulation of mechanics, the Hamiltonian of a system equals the system total energy. Thus, if the total energy is conserved, then the Hamiltonian time derivative must be null. In the case of the simple pendulum, \cref{eq:solsimppendsmallangle} \emph{should} conserve total energy, since we hypothesized an undamped system.

Firstly, write down the Hamiltonian, $\ham$, of the system:
\beq
\ham = \ken + \pen 
     = \dfrac{1}{2}\mass\length^2\angvel^2 + \mass\grav\length\left(1 - \cos\vat{\angpos}\right)\,. 
\eeq
Replace $\left(1 - \cos\vat{\angpos}\right)$ by the first term of its Taylor series:
\beq
2\ham = \mass\length^2\angvel^2 + \mass\grav\length\angpos^2\,. 
\eeq
Plug \cref{eq:dimlesssolsimppendsmallangle} and \cref{eq:angposandangvel} into the last equation to have
\beq
2\ham = \mass\length\angpos_0^2\left(\length\sin^2\vat{\scpq t} + \grav\cos^2\vat{\scpq t}\right)\,.
\eeq
Finally, derivate the last equation with respect to $\scpq t$:
\beq
\iod{\scpq t}\ham = \mass\length\angpos_0^2
                     \left( 
                         \length\sin\vat{\scpq t}\cos\vat{\scpq t} 
                         - \grav\sin\vat{\scpq t}\cos\vat{\scpq t}
                     \right)\,.
\eeq
Since $\iod{\scpq t}\ham$ does \emph{not} equal zero, \cref{eq:solsimppendsmallangle} does \emph{not} satisfy the energy conservation principle. This discrepancy originates due to the small-angle approximation.


\subsection{Experimental data and reference values}
No experimentation was specifically done for writing the present document. However, some reference values were found in the internet [source!]. 


\subsubsection{Experiment}\label{sec:experiment}
In [source!], the experimental set-up consisted of a pendulum with a spherical, stainless-steel-made bob of mass $\mass\txt{bob} = \SI{100.0}{m}$ hanged of a stainless steel rod whose length was varied during experimentation; however, it was assumed to be inflexible and massless. The rod was connected to a well lubricated pivot. The pendulum was set into motion by displacing the bob an initial angle $\angpos_0 = \ang{10.00}$. Finally, the local free fall acceleration was measured to be $\grav = \SI{10.02}{m/s^2}$ and the pendulum surrounded by air at room temperature.

The gathered experimental together with the model predictions of \cref{eq:solsimppendsmallangle} are presented in \cref{fig:confrontingmodelexpdata}.
%
% ------------------------------------------------------------- Figure
% position: bthH. size:width=0.5\textwidth. file:location+filename
 \docfigure{bt}{width=0.5\textwidth}{./graphs/pendulum-exp-model.pdf}%
   {Pendulum: experiment \vs model}
   {Model values for the simple gravitational pendulum and experimental data gathered from a real gravitational pendulum}%
   {fig:confrontingmodelexpdata}%
% ------------------------------------------------------------- EndFigure


\subsubsection{Verification of assumptions}
In this installment, with the data presented in \cref{sec:experiment}, we verify the assumptions made in, which led to \cref{eq:solsimppendsmallangle}.

First, consider the \lingo{frictionless pivot} assumption. Proper lubrication of the pivot reduces friction. With this, the frictional torque term disappears. To support this assumption, consider that a dry and clean joint of steel pivot and steel rod has a friction coefficient of 0.80, while when lubricated 0.16. Thus, we can discard friction.

Next, consider a \lingo{massless, inflexible rod}. The rod may be build of a stainless steel, a very strong material. Although the diameter of the rod was not reported, we can assume that it was strong enough to support the bob without elongating.This implies that the ratio of masses, $\mass\txt{rod}/\mass\txt{bob}$, vanishes.

Again, the diameter of the bob was not reported. But, we can estimate it by considering the stainless steel density equal to $\SI{7750}{kg/m^3}$. Then, since the bob was spherical, the bob diameter would be
\beq
\diam = \sqrt[3]{\dfrac{6\mass\txt{bob}}{\pi\dens\txt{bob}}}
      = \sqrt[3]{\dfrac{6\times\num{100.00e-3}}{\pi\times\num{7750}}}
      = \SI{2.2910}{cm}\,.
\eeq
With this number, we can calculate the $\diam/length$ ratio for the smaller case of $\length$ analyzed: $\SI{100.00}{cm}$. Then, $\diam/\length = \num{0.02291}$, which can be ignored.

Then, consider the non-buoyant fluid assumption. If the system was at $\SI{15}{\celsius}$, at sea level, then $\dens\txt{air} = \SI{1.225}{kg/m^3}$. Hence, 
\beq
\dfrac{\dens\txt{air}}{\dens\txt{steel}} = \dfrac{\num{1.225}}{\num{7750}} 
                                         = \num{0.158e-4}\,, 
\eeq
which can be ignored.

Finally, consider the hypothesis of taken air as an inviscid fluid. Then, plug some typical values into $\kdim_9$:
\beq
\dfrac{\dens\txt{air}\diam\txt{bob}\sqrt{\length\grav}}{\dynvis}
    = \dfrac{1.225\times\num{2.9011e-2}\sqrt{1.00\times 9.80665}}{\num{1.983e-5}}
    \sim \num{5611} \,,
\eeq
See that inertial forces, $\dens\txt{air}\diam\txt{bob}\sqrt{\length\grav}$, are larger than viscous forces, $\dynvis$. Thus, discarding viscosity was allowed.

We can conclude the experimental set up was such that satisfied the model assumptions.

\subsubsection{Analysis of experimental results}
Using \cref{fig:confrontingmodelexpdata}, one finds that the \lingo{coefficient of determination}, $R^2$, between model figures and experimental data is $\num{0.9876}$, while the relative error $\SI{2.03}{\%}$. Both figures show agreement between model and experiment, thus \cref{eq:solsimppendsmallangle} is taken to correctly represent real pendulums.


\subsubsection{Conclusions}
Nevertheless, if more accuracy is required or less agreement is found when applying \cref{eq:solsimppendsmallangle} to a real pendulum, then the model can be extended by relaxing the assumptions made in \cref{subsec:mathinterpretation}. For instance, if the pendulum swings through a viscous fluid, such as a liquid, then the dimensionless quantities $\kdim_8$ and $\kdim_9$ should be included. Or, if there is little care in lubricating the pivot, then $\kdim_4$, $\kdim_5$ and $\kdim_6$ should be further studied.
