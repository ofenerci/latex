\section{Results}
Although physically and mathematically restrained with respect to the original problem, an undamped pendulum,  \cref{eq:solsimppendsmallangle} does provide a closed form function to predict the amplitude variation of a simple gravitational pendulum with respect to time. 

Hereafter, we discuss this equation under physical grounds -- conservation laws -- and confront its predictions with experimental data.


\subsection{Theoretical discussion}
In this section, we investigate whether \cref{eq:solsimppendsmallangle} satisfies the principles of momentum and energy conservation.

[consistency on the description of motion: when particle moves to the right, the force points to the left (grav. potential energy restores kinetic energy), and \vis.

circular motion: amplitude describes a circle, since inflexible rod]


\subsubsection{Momentum conservation}
Momentum of a system is conserved if there are no interactions; \ie, $\dx\mom = 0$. Now, since we departed from the hypothesis that gravity acts on the simple gravitational pendulum, \cref{eq:solsimppendsmallangle} should \emph{not} preserve momentum~\footnote{~In the grand scheme of things, momentum \emph{is} conserved, but to see this, we would need to add Earth's momentum as well as the pendulum's.}.

To begin with, find the pendulum angular velocity by differentiating \cref{eq:dimlesssolsimppendsmallangle}, the dimensionless form of \cref{eq:solsimppendsmallangle}, with respect to $\scpq t$:
\begin{equation}\label{eq:angposandangvel}
\angvel = \iod{\scpq t}\angpos = -\angpos_0\sin\vat{\scpq t}\,.
\end{equation}
Next, replace the pendulum angular velocity in \cref{eq:genmomentumanditstempchange}, the pendulum momentum:
\beq
\igmom\angpos = \mass\length^2\angvel 
              = - \mass\length^2\angpos_0\sin\vat{\scpq t}\,.
\eeq
Finally, find the momentum time derivative:
\beq
\dtigmom\angpos = - \mass\length^2\angpos_0\cos\vat{\scpq t}\,.
\eeq
Since $\dtigmom\angpos$ is \emph{not} zero, momentum is \emph{not} conserved during motion. Thus, \cref{eq:solsimppendsmallangle} accords with physical principles.


\subsubsection{Energy conservation}
In Hamilton's formulation of mechanics, the Hamiltonian of a system equals the system total energy. Thus, if the total energy is conserved, then the Hamiltonian time derivative must be null. In the case of the simple pendulum, \cref{eq:solsimppendsmallangle} \emph{should} conserve total energy, since we hypothesized an undamped system.

Firstly, write down the Hamiltonian, $\ham$, of the system:
\beq
\ham = \ken + \pen 
     = \dfrac{1}{2}\mass\length^2\angvel^2 + \mass\grav\length\left(1 - \cos\vat{\angpos}\right)\,. 
\eeq
Replace $\left(1 - \cos\vat{\angpos}\right)$ by the first term of its Taylor series:
\beq
2\ham = \mass\length^2\angvel^2 + \mass\grav\length\angpos^2\,. 
\eeq
Plug \cref{eq:dimlesssolsimppendsmallangle} and \cref{eq:angposandangvel} into the last equation to have
\beq
2\ham = \mass\length\angpos_0^2\left(\length\sin^2\vat{\scpq t} + \grav\cos^2\vat{\scpq t}\right)\,.
\eeq
Finally, derivate the last equation with respect to $\scpq t$:
\beq
\iod{\scpq t}\ham = \mass\length\angpos_0^2
                     \left( 
                         \length\sin\vat{\scpq t}\cos\vat{\scpq t} 
                         - \grav\sin\vat{\scpq t}\cos\vat{\scpq t}
                     \right)\,.
\eeq
Since $\iod{\scpq t}\ham$ does \emph{not} equal zero, \cref{eq:solsimppendsmallangle} does \emph{not} satisfy the energy conservation principle. This discrepancy originates due to the small-angle approximation.


\subsection{Experimental data and reference values}
Experimental data was not specifically gathered for writing the present document. However, some reference values were found in the internet [source!]. Such values together with the model predictions of \cref{eq:solsimppendsmallangle} are presented in \cref{fig:confrontingmodelexpdata}.
%
% ------------------------------------------------------------- Figure
% position: bthH. size:width=0.5\textwidth. file:location+filename
 \docfigure{bt}{width=0.5\textwidth}{./graphs/pendulum-exp-model.pdf}%
   {Pendulum: experiment \vs model}
   {Model values for the simple gravitational pendulum and experimental data gathered from a real gravitational pendulum}%
   {fig:confrontingmodelexpdata}%
% ------------------------------------------------------------- EndFigure

Using \cref{fig:confrontingmodelexpdata}, one finds that the \lingo{coefficient of determination}, $R^2$, between model figures and experimental data is $\num{0.9876}$, while the relative error $\SI{2.03}{\%}$. Both figures show agreement between model and experiment, thus \cref{eq:solsimppendsmallangle} is taken to correctly represent real pendula.

Nevertheless, if more accuracy is required or less agreement is found when applying \cref{eq:solsimppendsmallangle} to a real pendulum, then the model can be extended by relaxing the assumptions made in \cref{subsec:assumptions}. For instance, if the pendulum swings through a viscous fluid, such as a liquid, then the dimensionless quantities $\kdim_8$ and $\kdim_9$ should be included. Or, if there is little care in lubricating the pivot, then $\kdim_4$, $\kdim_5$ and $\kdim_6$ should be further studied.


\subsection{Analogies with other phenomena}\label{subsec:analogieswithotherphenomena}
In classical \lingo{simple harmonic motion}, the period of the motion, $\oscper$, is the time required for a complete oscillation and defined by
\beq
\oscper = \dfrac{2\pi}{\natfreq}\,,
\eeq
where $\natfreq$ is the motion \lingo{natural frequency}.

The motion of a simple gravitational pendulum, described by \cref{eq:solsimppendsmallangle}, is an instance of simple harmonic motion, where $\angpos_0$ is the semi-amplitude of the oscillation and where the natural frequency is
\beq
\natfreq = \sqrt{\dfrac{\grav}{\length}}\,.
\eeq
The period of the pendulum motion, for the outward and return, is thus
\begin{equation}\label{eq:simplependulumperiod}
\oscper = 2\pi\sqrt{\dfrac{\length}{\grav}}\,.
\end{equation}
The last equation is Christiaan Huygens's law for the period.

Note that only under the small-angle approximation, the period is independent of the amplitude $\angpos_0$; \ie, \lingo{isochronism} -- the property Galileo discovered.
