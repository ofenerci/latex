\section{Results}
Although physically and mathematically restrained with respect to the original problem, an undamped pendulum,  \cref{eq:solsimppendsmallangle} does provide a closed form function to predict the displacement of a simple gravitational pendulum with respect to time. 

Hereafter, we discuss this equation under physical grounds and confront its predictions with experimental data.


\subsection{Theoretical discussion}
In this section, we investigate the physical consequences of \cref{eq:solsimppendsmallangle}.

[consistency on the description of motion: when particle moves to the right, the force points to the left (grav. potential energy restores kinetic energy), and \vis.

circular motion: amplitude describes a circle, since inflexible rod]


\subsubsection{Analogies with other phenomena}\label{subsec:analogieswithotherphenomena}
In classical \lingo{simple harmonic motion}, the period of the motion, $\oscper$, is the time required for a complete oscillation and defined by
\beq
\oscper = \dfrac{2\pi}{\natfreq}\,,
\eeq
where $\natfreq$ is the motion \lingo{natural frequency}.

The motion of a simple gravitational pendulum, described by \cref{eq:solsimppendsmallangle}, is an instance of simple harmonic motion, where $\angpos_0$ is the semi-amplitude of the oscillation and where the natural frequency is
\beq
\natfreq = \sqrt{\dfrac{\grav}{\length}}\,.
\eeq
The period of the pendulum motion, for the outward and return, is thus
\begin{equation}\label{eq:simplependulumperiod}
\oscper = 2\pi\sqrt{\dfrac{\length}{\grav}}\,,
\end{equation}
which is Huygens's law for the period.

Note that \emph{only} under the small-angle approximation, the period is independent of the amplitude; \ie, \lingo{isochronism} -- the property Galileo discovered.


\subsubsection{Momentum conservation}
Momentum of a system is conserved if no forces act on the system, thus $\dt\mom = 0$ holds. Since we departed from the hypothesis that gravity drives the simple gravitational pendulum, \cref{eq:solsimppendsmallangle} should \emph{not} preserve momentum~\footnote{~In the grand scheme of things, momentum \emph{is} conserved, but to see this, we would need to add Earth's momentum to the pendulum's.}.

Find the pendulum angular velocity by differentiating \cref{eq:dimlesssolsimppendsmallangle}, the dimensionless form of \cref{eq:solsimppendsmallangle}, with respect to $\scpq t$:
\begin{equation}\label{eq:angposandangvel}
\angvel = \iod{\scpq t}\angpos = -\angpos_0\sin\vat{\scpq t}\,.
\end{equation}
Next, replace the pendulum angular velocity in \cref{eq:genmomentumanditstempchange}, the pendulum momentum:
\beq
\igmom\angpos = \mass\length^2\angvel 
              = - \mass\length^2\angpos_0\sin\vat{\scpq t}\,.
\eeq
Finally, calculate the momentum time derivative:
\beq
\dtigmom\angpos = - \mass\length^2\angpos_0\cos\vat{\scpq t}\,.
\eeq
Since $\dtigmom\angpos$ is \emph{not} overall zero, momentum is \emph{not} conserved during motion. Thus, \cref{eq:solsimppendsmallangle} physical considerations regarding momentum.


\subsubsection{Energy conservation}
In Hamilton's formulation of mechanics, the Hamiltonian of a system equals the system total energy. Thus, if the total energy is conserved, then the Hamiltonian time derivative must be null. Equivalently, it can be proved that~\footnote{~This theorem is useful for it saves computing the time derivative of the Hamiltonian.}
\begin{quote}
if the Hamiltonian does not explicitly depend on time, then total energy is conserved. 
\end{quote}
Particularly, in the case of the simple pendulum, \cref{eq:solsimppendsmallangle} \emph{should} conserve total energy, because it was deduced by hypothesizing an undamped system.

Firstly, write down the Hamiltonian, $\ham$, of the system:
\beq
\ham = \ken + \pen 
     = \dfrac{1}{2}\mass\length^2\angvel^2 + \mass\grav\length\left(1 - \cos\vat{\angpos}\right)\,. 
\eeq
Replace $\left(1 - \cos\vat{\angpos}\right)$ by the first term of its Taylor series:
\beq
2\ham = \mass\length^2\angvel^2 + \mass\grav\length\angpos^2\,. 
\eeq
Plug \cref{eq:dimlesssolsimppendsmallangle} and \cref{eq:angposandangvel} into the last equation to have
\beq
2\ham = \mass\length\angpos_0^2\left(\length\sin^2\vat{\scpq t} + \grav\cos^2\vat{\scpq t}\right)\,.
\eeq
Since the Hamiltonian $\ham$ \emph{does} depend on time, \cref{eq:solsimppendsmallangle} does \emph{not} satisfy the energy conservation principle. The small-angle approximation originates this discrepancy.


\subsection{Experimental data and reference values}
No experimentation was specifically made for the writing of the present document. However, some reference values were found in the internet [source!]. 


\subsubsection{Experiment}\label{sec:experiment}
In [source!], the experimental set-up consisted of a pendulum with a spherical, stainless-steel-made bob of mass $\mass\txt{bob} = \SI{100.0}{m}$ hanged of a stainless steel rod whose length was varied during experimentation; however, it was assumed to be inflexible and massless. The rod was connected to a well lubricated pivot. The pendulum was set into motion by displacing the bob an amplitude $\angpos_0 = \ang{10.00}$ from rest. Finally, the pendulum was encased and surrounded by air at room temperature.


\subsubsection{Verification of assumptions}
Before confronting experimental numbers with predictions of \cref{eq:solsimppendsmallangle}, it is necessary to verify that all the assumptions made to deduce \cref{eq:solsimppendsmallangle} are satisfied.

First, consider the \lingo{frictionless pivot} assumption. The experiment was done by properly lubricating the pivot-rod joint. So the assumption holds. Some useful numbers to back-up this assumption: a dry and clean joint of steel pivot and steel rod has a friction coefficient of 0.80, while when lubricated 0.16.

Consider the \lingo{massless, inflexible rod} assumption. [source!] does not report numbers to support this assumption. We take it as satisfied.

Again, the diameter of the bob was not reported. But, we can estimate it by considering the stainless steel density equal to $\SI{7750}{kg/m^3}$. Then, considering a spherically shaped bob, the diameter would be
\beq
\diam = \sqrt[3]{\dfrac{6\mass\txt{bob}}{\pi\dens\txt{bob}}}
      = \sqrt[3]{\dfrac{6\times\num{100.00e-3}}{\pi\times\num{7750}}}
      = \SI{2.2910}{cm}\,.
\eeq
With this number, we can calculate the $\diam/\length$ ratio for the smaller case of $\length$ analyzed: $\SI{100.00}{cm}$. Then, $\diam/\length = \num{0.02291}$, which can be ignored.

Then, consider the non-buoyant fluid assumption. If the system was at $\SI{15}{\celsius}$, at sea level, then $\dens\txt{air} = \SI{1.225}{kg/m^3}$. Hence, 
\beq
\dfrac{\dens\txt{air}}{\dens\txt{steel}} = \dfrac{\num{1.225}}{\num{7750}} 
                                         = \num{0.158e-4}\,, 
\eeq
which can be discarded.

Finally, consider if the hypothesis of taken air as an inviscid fluid was satisfied by plugging some typical values into $\kdim_9$:
\beq
\dfrac{\dens\txt{air}\diam\txt{bob}\sqrt{\length\grav}}{\dynvis}
    = \dfrac{1.225\times\num{2.9011e-2}\sqrt{1.00\times 9.80665}}{\num{1.983e-5}}
    \sim \num{5611} \,,
\eeq
See that inertial forces, $\dens\txt{air}\diam\txt{bob}\sqrt{\length\grav}$, are much larger than viscous forces, $\dynvis$. 

Since the experimental set-up was such that satisfied all the hypotheses leading to \cref{eq:solsimppendsmallangle}, we can, therefore, use this equation to model the simple pendulum in this case.


\subsubsection{Analysis of experimental results}
The gathered experimental together with the model predictions of \cref{eq:solsimppendsmallangle} are presented in \cref{fig:confrontingmodelexpdata}.
%
% ------------------------------------------------------------- Figure
% position: bthH. size:width=0.5\textwidth. file:location+filename
 \docfigure{bt}{width=0.5\textwidth}{./graphs/pendulum-exp-model.pdf}%
   {Pendulum: experiment \vs model}
   {Model values for the simple gravitational pendulum and experimental data gathered from a real gravitational pendulum}%
   {fig:confrontingmodelexpdata}%
% ------------------------------------------------------------- EndFigure

Using these data, \cref{fig:confrontingmodelexpdata}, one finds that the \lingo{coefficient of determination}, $R^2$, between model and experimental figures is $\num{0.9876}$, while the relative error $\SI{2.03}{\%}$. Both numbers show agreement between model and experiment, thus \cref{eq:solsimppendsmallangle} is taken to correctly represent real pendulums.


\subsubsection{Conclusions}
A closed mathematical function between the simple pendulum displacement and time was found, \cref{eq:solsimppendsmallangle}. This model agrees with the momentum conservation but not with energy conservation due to the small-angle approximation. However, when its predictions are confronted with experimental data, the model figures agree with physical reality if the assumptions leading to it are satisfied by experimental set-ups.

Finally, if more accuracy is required or in cases where less agreement is found when applying \cref{eq:solsimppendsmallangle} to a real pendulum, then the model can be extended by relaxing the assumptions made in \cref{subsec:mathinterpretation}. For instance, if the pendulum swings through a viscous fluid, such as a liquid, then the dimensionless quantities $\kdim_8$ and $\kdim_9$ should be included. Or, if there is little care in lubricating the pivot, then $\kdim_4$, $\kdim_5$ and $\kdim_6$ should be further studied.
