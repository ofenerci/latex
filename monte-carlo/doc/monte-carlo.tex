\chapter*{Monte Carlo Integration}
%
Adapted from \citep{fang:2014, amth142}


\section*{Average value of a sequence}
%
Consider a \lingo{sequence} of numbers $a = \bseq ak1n$. The \lingo{average value of the sequence}, $\mean a$, is the arithmetic average of the sequence series:
%
\beq
  \mean a = \dfrac{1}{n}\ssum k a_k\,.
\eeq
%
By analogy, extend this idea to find \emph{function} average values.


\section*{Average value of a function}
%
Consider a function $f$ integrable over the interval $a\leq x\leq b$. Then, estimate the function average value $\mean f\txt{est}$ by partitioning the interval into subintervals of width $\Dx x = \left(b - a\right)/n$, by picking a point $x_k$ in each subinterval, by calculating the function values $\elset{f\vat{x_k}}$ at each $x_k$ and by averaging such values:
%
\beq
  \mean f\txt{est} = \dfrac{1}{n}\ssum k f\vat{x_k}\,.
\eeq
%
Note that, as $n$ increases, the estimate improves -- a hint to work with calculus.

Multiply and divide thus the last equation by $\Dx x$, then use $n\Dx x = \left(b - a\right)$ to have
%
\beq
  \mean f\txt{est} = \dfrac{1}{b - a}\ssum k f\vat{x_k}\Dx x\,.
\eeq
%
Calculate next the average value of $f$ by taking the limit of the last equation, provided such a limit exists:
%
\beq
  \mean f = \dfrac{1}{b - a}\lim_{n\to\infty}\lsum{k = 1}{n} f\vat{x_k}\Dx x
          = \dfrac{1}{b - a}\int_{\ccint ab} f\vat x\dx x\,.
\eeq

Finally, define the \lingo{average value} of a function $f$ integrable over the interval $i = \ccint ab$ as~\nbene{functional  notation for the integral, \citep[p. 69]{apostol:1967}}
%
\beq
  \mean f \defas \dfrac{1}{b - a}\int_i f\,;
\eeq
%
that is, \definition{the average value of a function over an interval equals the integral of the function divided by the size of the interval}.


\section*{Monte-Carlo integration}
%
\lingo{Monte-Carlo integration} is a procedure to estimate a value for the integral of a function over an interval not by partitioning the interval and picking values at the subintervals, but by \lingo{randomly} picking numbers within the whole interval and with them calculating function values. The process of picking random numbers within the interval is called \lingo{random sampling}.

The process is the inverse to that of finding the average value using integration. In Monte-Carlo integration, the average value of a function is estimated first and then the value of the integral estimated by
%
\beq
  \int_i f = \mean f\left(b - a\right)\,.
\eeq
%
This procedure is formalized as follows.


\subsection*{Integration}
%
Suppose we wish to estimate the value of the integral $l = \int_{\ccint ab} f$ of a function $f$.

First, \emph{randomly} choose $n$ points $\elset{x_k}$ within $a\leq x\leq b$, use these to calculate the values of $f$, $\elset{f\vat{x_k}}$ and then estimate the average value of $f$:
%
\beq
  \mean f\txt{est} = \dfrac{1}{n}\lsum{k = 1}{n}{f\vat{x_k}}
\eeq
%
Estimate finally the value of the integral as
%
\beq
  l\txt{est} = \left(b - a\right)\mean f\txt{est}\,.
\eeq


\subsection*{Integration uncertainty}
%
The \lingo{central limit theorem} of probability theory provides with an estimate for the \lingo{uncertainty} in Monte-Carlo integration. 

Suppose the average value of a function $f$ is estimated by random sampling $n$ numbers, \aka the \lingo{sample size},
%
\beq
  \mean f\txt{est} = \dfrac{1}{n}\lsum{k = 1}{n} f\vat{x_k}\,.
\eeq

Then, the \lingo{variance} of the estimated average is
%
\beq
  \var\mean f\txt{est} = \dfrac{\stddev^2}{n}\,,
\eeq
%
where $\stddev$ is the variance of $f$.

Measure the uncertainty $\unc$ by the standard deviation:
%
\beq
  u = \dfrac{\stddev}{\sqrt{n}}\,.
\eeq

Note that the uncertainty goes to zero like $1/\sqrt{n}$\,; \ie, for example, to decrease the uncertainty by a factor of \num{1000}, increase the sample size by a factor of \num{1000000}.


\subsection*{Example}
%
Estimate the value of the integral~\nbene{The reference value is $\left(1 - \exp\vat{-2\pi}\right)/2\sim 0.4990663$.}
%
\beq
  l = \int_{\ccint{0}{\tau}}\exp\vat{-x}\sin\vat{x}\,\dx x\,,
\eeq
%
where $\tau\defas 2\pi$.

With the aid of computer generated (pseudo) random numbers (see \cref{app:code} for the computer source code), it was possible to estimate the value of the integral and its uncertainty as \num{0.49819 +- 0.00037}, for a sample size of \num{100000}. The result of the Monte-Carlo integration differs in 0.20\% from the reference value of \num{0.4990663}.

