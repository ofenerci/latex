%%% MATHS
%
%%% GENERAL COMMANDS
%
\newcommand{\beq}{\begin{equation*}}
\newcommand{\eeq}{\end{equation*}}
%
\newcommand{\eqtxt}[1]{\text{[{#1}]}} % text in equations
%
%\newcommand{\defby}{\equiv}   % defined by
\newcommand{\defby}{\coloneqq} % defined by
%\newcommand{\defby}{\doteq}   % defined by
\newcommand{\diff}{\Delta}     % difference operator
%
%\let\oldsqrt\sqrt
%\renewcommand{\sqrt}[1]{\left({#1}\right)^{\tfrac{1}{2}}} % sqrt
%
\newcommand{\txt}[1]{_\text{#1}}      % text as subscript in variables
%
%%% SETS
%
\newcommand{\set}[1]{\mathcal{#1}}                    % set
\newcommand{\elset}[1]{\left\lbrace{#1}\right\rbrace} % elements of a set
\newcommand{\setprop}[2]{\elset{{#1}:{#2}}}           % set with a property
\DeclareMathOperator{\sprod}{\otimes}                 % Cartesian (set) product
\DeclareMathOperator{\setprod}{\otimes}               % Cartesian (set) product
\newcommand{\nset}[2]{\set{#1}^{#2}}                  % n-dim set: Cartesian power
%
\DeclareMathOperator{\union}{\cup}                    % union of sets
\DeclareMathOperator{\inter}{\cap}                    % intersection of sets
%
\newcommand{\dgspace}[3]{{#1}^{#2}_{#3}} % dim-grade space
%
\newcommand{\espace}[1]{\nset{E}{#1}}  % n-dim Euclidean space
\newcommand{\mkspace}[1]{\nset{M}{#1}} % n-dim Minkowski space
%
\newcommand{\ga}{\set{G}}            % geometric algebra
\newcommand{\nga}[1]{\ga^{#1}}       % n-dim geometric algebra
\newcommand{\dgga}[2]{\ga^{#1}_{#2}} % dimension-grade GA
%
\newcommand{\tuple}[1]{\left[{#1}\right]} % tuple
%
%%% FUNCTIONS
%
\newcommand{\class}[1]{C^{#1}}              % class (differentiability)
\newcommand{\fdef}[3]{{#1}:{#2}\to{#3}}     % function definition
\newcommand{\fmap}[3]{{#1}:{#2}\mapsto{#3}} % function mapping
\newcommand{\vat}[1]{\!\tuple{#1}}          % (function) value at ()
\DeclareMathOperator{\fcomp}{\circ}         % function composition
%
\DeclareMathOperator{\bprod}{\ast} % a binary operator
%
\newcommand{\fder}{\mathrm{D}}         % derivative
\newcommand{\fpder}[1]{\partial_{#1}}  % partial derivative
%
%%% SEQUENCES AND SERIES
%
\newcommand{\seq}[3]{\elset{{#1}}_{#2}^{#3}} % sequence
\newcommand{\serie}[3]{\sum_{#1}^{#2}\,{#3}} % serie
%
\DeclareMathOperator{\fib}{fib} % fibonacci numbers
%
%%% GEOMETRIC OBJECTS
%
%%% no decoration! vectors on equal footing than other multivectors!
%\let\oldvec\vec
%\renewcommand{\vec}[1]{#1}               % vector (for the doc final version)
%\renewcommand{\vec}[1]{\bm{\mathrm{#1}}} % vector (for drafts)
%
\newcommand{\point}[1]{\mathcal{#1}} % point
\newcommand{\curve}[1]{\set{#1}}     % curve
\newcommand{\surf}[1]{\set{#1}}      % surface
\newcommand{\region}[1]{\set{#1}}    % region in space
\newcommand{\bound}{\partial}        % boundary of an object
%
\newcommand{\nvec}[1]{\hat{#1}}   % normal vector
\newcommand{\ovec}[1]{\check{#1}} % omitted vector from a product
\newcommand{\pscl}{i}             % pseudoscalar unit
%
\newcommand{\ortho}{\perp}                   % orthogonal or perpendicular
\newcommand{\perto}[2]{{#1}_{\perp{#2}}}     % to be perpendicular to 
\newcommand{\parto}[2]{{#1}_{\parallel{#2}}} % to be parallel to 
\newcommand{\ortto}[2]{{#1}_{\perp{#2}}}     % to be orthogonal to 
\newcommand{\colto}[2]{{#1}_{\parallel{#2}}} % to be collinear to
\newcommand{\projon}[2]{\parto{#1}{#2}}      % projection onto
\newcommand{\rejon}[2]{\ortto{#1}{#2}}       % rejection onto
%
\newcommand{\magn}[1]{\lvert{#1}\rvert} % magnitude, norm, length
\newcommand{\inv}[1]{{#1}^{-1}}         % inverse
\newcommand{\rev}[1]{{#1}^{\dagger}}    % reversion
\newcommand{\hdual}[1]{{}^*{#1}}        % Clifford-Hodge dual
%
\newcommand{\acom}[2]{\left[{#1},{#2}\right]_{+}} % anticommutator
\newcommand{\com}[2]{\left[{#1},{#2}\right]_{-}}  % commutator
%
\newcommand{\xacom}[2]{{#1}{#2} + {#2}{#1}} % expanded anticommutator
\newcommand{\xcom}[2]{{#1}{#2} - {#2}{#1}}  % expanded commutator
%
\newcommand{\step}[2]{\langle{#1}\rangle_{#2}} % step
\newcommand{\sstep}[1]{\step{#1}{0}}           % scalar step
%
\DeclareMathOperator{\Grade}{grade}   % grade
\newcommand{\grade}[2]{\step{#1}{#2}} % grade
\newcommand{\sgrade}[1]{\sstep{#1}}   % scalar grade
%
\newcommand{\sclif}[2]{{#1}_{\bar{#2}}} % clif with step
%
\newcommand{\even}[1]{{#1}_{+}} % even part
\newcommand{\odd}[1]{{#1}_{-}}  % odd part
%
\DeclareMathOperator{\gorm}{gorm}            % geometric norm ?
\newcommand{\xgorm}[1]{\sstep{\rev{#1}{#1}}} % expanded gorm
%
\newcommand{\metric}{g}    % metric 
\newcommand{\kron}{\delta} % kronecker delta
%
\DeclareMathOperator{\diag}{diag} % diagonal
\DeclareMathOperator{\sign}{sig}  % signature
%
%\DeclareMathOperator{\iprod}{\bm{\cdot}} % inner prod.
\DeclareMathOperator{\iprod}{\cdot}       % inner prod.
\DeclareMathOperator{\oprod}{\wedge}      % outer prod.
\DeclareMathOperator{\cprod}{\times}      % cross prod.
%
\newcommand{\cgprod}[2]{{#1}\iprod{#2} + {#1}\oprod{#2}} % cannon. decomp. g.prod.
%
%%% CALCULUS
%
\newcommand{\dx}{\text{d}}               % ordinary derivative
\newcommand{\dt}[1]{\dot{#1}}            % ordinary time derivative
\newcommand{\ddt}[1]{\ddot{#1}}          % ordinary second time derivative
\newcommand{\cder}[2]{{#1}_{,{#2}}}      % comma derivative
\newcommand{\coder}[2]{{#1}_{;{#2}}}     % semi-colon or covariant derivative
\newcommand{\mder}[2]{\text{D}_{#2}{#1}} % material derivative
\newcommand{\abstder}[1]{\dot{\overline{#1}}} % absolute time derivative
%
\newcommand{\chris}[3]{\Gamma^{#1}_{{#2}{#3}}} % Christoffel symbol
%
\newcommand{\xpd}[2]{\dfrac{\partial{#1}}{\partial{#2}}} % expanded partial derivative
\newcommand{\xod}[2]{\dfrac{\dx{#1}}{\dx{#2}}}           % expanded ordinary derivative
\newcommand{\xmd}[2]{\dfrac{\text{D}{#1}}{\text{D}{#2}}} % expanded material derivative
%
\newcommand{\nxod}[3]{\dfrac{\dx^{#1}{#2}}{\dx{#3}^{#1}}} % expanded n ordinary derivative
%
\newcommand{\nxpd}[3]{\dfrac{\partial^{#1}{#2}}{\partial{#3}^{#1}}} % expanded n partial derivative
%
%% Shifting a symbol vertically in Math mode
%% http://tex.stackexchange.com/questions/47016/shifting-a-symbol-vertically-in-math-mode
%% the command \bigtriangledown takes an optional argument to rise the symbol a given
%% vertical distance
%\usepackage{amsfonts}%
%\let\oldbigtriangledown\bigtriangledown
%\renewcommand{\bigtriangledown}[1][0pt]{%
%  \mathrel{\raisebox{#1}{$\oldbigtriangledown$}}%
%}
%%
%\DeclareMathOperator{\gder}{\bigtriangledown[2.0pt]} % geometric derivative
%\DeclareMathOperator{\lder}{\bigtriangleup}          % laplace (derivative) operator
%\DeclareMathOperator{\dalder}{\square}               % d'alembert (derivative) operator
%
\newcommand{\gder}{\nabla}    % geometric derivative
\newcommand{\lder}{\nabla^2}  % laplace (derivative) operator
\newcommand{\dalder}{\square} % d'alembert (derivative) operator

%
\newcommand{\dder}[2]{\gder_{#2}{#1}}  % directional derivative
%
%%% GRAD, DIV, CURL AND ALL THAT ;)
%
\DeclareMathOperator{\grad}{grad} % gradient
\let\olddiv\div
\DeclareMathOperator{\Div}{div}   % divergence
\renewcommand{\div}{\Div}
\DeclareMathOperator{\curl}{curl} % curl
\DeclareMathOperator{\rot}{rot}   % curl 2
\DeclareMathOperator{\lap}{lap}   % laplacian or laplace operator
%
%%% TENSORS
%
\newcommand{\tens}[1]{\mathsf{#1}}    % tensor
\newcommand{\slot}{-}                 % slot
\DeclareMathOperator{\tprod}{\otimes} % tensor product
\DeclareMathOperator{\tcont}{cont}    % contraction
\newcommand{\itcont}[1]{\tcont_{#1}}  % index contraction
\newcommand{\lct}{\epsilon}           % levi-civita tensor
%
\DeclareMathOperator{\tcomp}{comp}    % tensor components
\newcommand{\cotens}[2]{{#1}_{#2}}    % covariant tensor
\newcommand{\cntens}[2]{{#1}^{#2}}    % contravariant tensor
%
\newcommand{\dtcotens}[2]{\dt{#1}_{#2}} % covariant tensor time derivative
\newcommand{\dtcntens}[2]{\dt{#1}^{#2}} % contravariant tensor time derivative
%
%%% INDEX NOTATION
%
\newcommand{\fvec}{\gamma} % frame element, frame vector
%
\newcommand{\frm}[1]{\elset{\fvec_{#1}}}                 % frame
\newcommand{\ifrm}[3]{\elset{\fvec_{#1};{#2}\dotsc{#3}}} % indexed frame
%
\newcommand{\rfrm}[1]{\elset{\fvec^{#1}}} % (indexed) reciprocal frame
%
\newcommand{\ifvec}[1]{\fvec_{#1}} % ind. frame vector
\newcommand{\rfvec}[1]{\fvec^{#1}} % ind. reciprocal frame vector
%
\newcommand{\comp}[2]{{#1}^{#2}}   % (ind.) component in frame (contra.)
\newcommand{\rcomp}[2]{{#1}_{#2}}  % (ind.) component in reciprocal frame (co.)
%
\newcommand{\imet}[2]{\metric_{{#1}{#2}}} % metric coefficients in frame
\newcommand{\rmet}[2]{\metric^{{#1}{#2}}} % metric coefficients in reciprocal frame
\newcommand{\mmet}[2]{\metric^{#1}_{#2}}  % mixed metric coefficients
\newcommand{\ikron}[2]{\kron_{{#1}{#2}}}  % kronecker delta coefficients in frame
\newcommand{\rkron}[2]{\kron^{{#1}{#2}}}  % kronecker delta coefficients in reciprocal frame
\newcommand{\mkron}[2]{\kron^{#1}_{#2}}   % mixed kronecker coefficients delta
%
%%% calculus
%
%\newcommand{\igder}[1]{\gder_{#1}}   % indexed geometric derivative in reciprocal frame
\newcommand{\igder}[1]{\partial_{#1}} % indexed geometric derivative in reciprocal frame
\newcommand{\rgder}[1]{\partial^{#1}} % indexed geometric derivative in frame
%
%%% VARIOUS
%
\DeclareMathOperator{\given}{\vert}                     % A given B
\newcommand{\iverson}[1]{\left[{#1}\right]_{\text{iv}}} % Iverson brackets
\newcommand{\poisson}[1]{\left[{#1}\right]_{\text{pb}}} % Poisson brackets
\newcommand{\mtrx}[1]{\left[{#1}\right]}                % matrix representation
\newcommand{\avg}[1]{\left\langle{#1}\right\rangle}     % average
%
%%% TRANSFORMATIONS (TRANSFORMS)
%
%%% taylor series and polynomials
%
% T_nf(x;a) Taylor polynomial of degree n generated by f at the point a
\newcommand{\ntpol}[4]{T_{#1}{#2}\vat{{#3}\,;{#4}}}  %
% Taylor series of degree n generated by f at the point a
\newcommand{\tseries}[3]{\ntpol{\infty}{#1}{#2}{#3}} %
%
%%% fourier series
%
% Fourier partial sum of degree n generated by f at the point a
\newcommand{\nfsum}[3]{F_{#2}{#1}\vat{#3}}
% Fourier series generated by f
\newcommand{\fseries}[2]{\nfsum{#1}{\infty}{#2}}
%
%%% legendre transform
%
\newcommand{\ltrans}[1]{{#1}_{\star}} % legendre transform of a function
\newcommand{\cvar}[1]{{#1}_{\star}}   % conjugate variable
%