\section{First-order reaction}
Consider a batch reactor hosting a first-order-kinetics chemical reaction of the form
\beq
\ce{R ->[\krcoeff] P}\,,
\eeq
where $\ce R$ represents the reactant and $\ce P$ the product and $\krcoeff$ the reaction kinetic coefficient, $[1/\phdim T]$.

Let $t$ represent the chemical reaction duration, let $\conc r$ be $\ce R$ molar concentration and let $\conc p$ be \ce P molar concentration, $\dim\conc r = \dim\conc p = [\phdim N/\phdim L^3]$. Consider $\conc{r,0}$ to be the reactant concentration when $t = 0$. Then, model the reaction kinetics by applying the mass conservation principle to the reactor:
\beq
-\iod t\conc r = \krcoeff\conc r\qquad\text{and}\qquad
 \conc r\vat 0 = \conc{r0}\,,
\eeq
where the term $-\iod t\conc r$ represents the reactant consumption rate, $[\phdim N/\phdim T\phdim L^3]$.

To find the reactant concentration, solve the differential equation by separating variables and applying the initial condition to have
\beq
\conc r = \conc{r0}\exp\vat{-\krcoeff t}\,.
\eeq
Choose the set $\elset{\phdim N, \phdim L, \phdim T}$ as a dimensional system to non-dimensionalize the model by applying the scaling transformations
\beq
\conc r    = \scpq{\conc r}\conc{r0}\,,\qquad
\conc p    = \scpq{\conc r}\conc{r0}\qquad\text{and}\qquad
\krcoeff t = \scpq t\,,
\eeq
which yield
\beq
\scpq{\conc r} = \exp\vat{-\scpq t}\,.
\eeq

Next, to calculate the product concentration, apply the stoichiometric condition~\footnote{~In dimensional form, the stoichiometric condition is $\conc r + \conc p = \conc{r,0}$.}:
\beq
\scpq{\conc p} = 1 - \scpq{\conc r}\,.
\eeq
