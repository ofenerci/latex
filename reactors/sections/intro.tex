\section{Introduction}
Before beginning any calculation, a system must be clearly defined. This definition is done by drawing the \lingo{system boundaries}. If total inflows in and outflows are to be known, then a whole chemical plant is embedded by its system boundaries. Minor units, on the other hand, like unit operations, are chosen when system boundaries are drawn.

The second step is to distinguish between process and reactor calculations. \lingo{Process calculations} give an \emph{overview} of a system, for only inflows and outflows of all mass crossing a system boundaries are used; \ie, they are \lingo{black box} calculations. On the other hand, \lingo{reactor calculations} answer the question of \emph{why} a chemical reaction occurs to a certain extent, since they are based on mass balances \emph{and} kinetic laws. Not only does this help to design chemical reactors, but also to gain a deeper understanding of a reaction system inner workings.

Non-reactive and reactive systems: \lingo{Non-reactive systems} are systems where no chemical reaction takes place; \ie, the input and output streams flowing into the system will be in the same chemical form, but perhaps in a different phase or separated from other substances. All \lingo{separation process} are examples of non-reactive systems. \lingo{Reaction systems} how to design and operate reaction systems to raw material and transformed into a chemical product of any kind; \eg, microorganisms using \ce C, \ce N and \ce P in wastewater for their biochemical metabolism. In chemical plants, the exact composition of both reactants and products is known, whereas in more complex systems, like an ecosystem or the atmosphere, compositions are difficult or sometimes impossible to gather.

\lingo{elements, components and inert substances}. \lingo{Elements} are \lingo{conservative} within a system. They are neither consumed nor produced in a chemical process or physical separation. A \lingo{component} is a chemical compound of any kind; \eg, \ce{NO3-}, \ce{SO4^{2-}}, \ce{CO2} and benzene. In a steady-state, in non-reactive systems, a component mass inflow equals its outflow. However, in reaction systems, equality does not hold. Finally, an \lingo{inert substance} is one that does not react, even if surrounded by reactive substances. Water, for example, is an inert substance in many systems, as it merely acts as a solvent. In other cases, the \emph{assumption} of a substance being inert can simplify analysis, such as the case of substances that react to a small extent or is present is such large surpluses that the change in its total amount can be neglected.
