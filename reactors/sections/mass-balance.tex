\section{Mass balance}
Mass is a conservative quantity, hence, given a control volume $\vol$ of a system, the sum of mass flows entering the system will equal the sum exiting minus (plus) the consumed (generated) or accumulated fractions:
\beq
\begin{pmatrix}
\text{rate of mass} \\
\text{in}
\end{pmatrix}
+
\begin{pmatrix}
\text{rate of mass} \\
\text{out}
\end{pmatrix}
+
\begin{pmatrix}
\text{rate of mass} \\
\text{produced}
\end{pmatrix}
-
\begin{pmatrix}
\text{rate of mass} \\
\text{consumed}
\end{pmatrix}
=
\begin{pmatrix}
\text{rate of mass} \\
\text{accumulated}
\end{pmatrix}\,.
\eeq

The last statement represents the key point in \lingo{mass transfer}: analogously to the force balance in statics, the mass balance allows us to quantify and verify mass flows in our system.

Let us now apply this fundamental balance to some ideal examples.


\subsection{Ideal chemical reactors}

\subsubsection{Batch reactors}
A \lingo{batch reactor} is a non-continuous, perfectly mixed and closed vessel where a reaction takes place, see \cref{fig:bathreactor}.
%
% ------------------------------------------------------------- Figure
% position: bthH. size:width=0.5\textwidth. file:location+filename
 \docfigure{bt}{width=0.5\textwidth}{./figures/batch-reactor.pdf}%
   {Batch reactor}
   {Schema of a batch reactor}%
   {fig:bathreactor}%
% ------------------------------------------------------------- EndFigure

Given the reactor volume $\vol$ and the initial concentration of a chemical species $\ce A$, $\conc{A,0}$, inside the reactor, then the total mass of $\ce A$ inside the vessel will be $\mass_{\ce A} = \vol\conc{A,0}$. In the unit time, the concentration of $\ce A$ will change only by virtue of a chemical reaction. The mass balance quantifies this change in this case:
\beq
\flow\vol\conc{A,in} - \flow\vol\conc{A,out} \pm \int_\vol\rrate\dx\vol = \iod t\mass_{\ce A}\,,
\eeq
where $\rrate$ is the rate of generation ($+$) or depletion ($-$) of $\ce A$. Since the assumptions of no flow in or out of the reactor volume, $\flow\vol = 0$, and of constant reactor volume, $\dx\vol = 0$,
\beq
\iod t\mass_{\ce A} = \iod t\conc A\vol = \vol\iod t\conc A = \vol\rrate\,,
\eeq
where $\conc A = \conc A\vat t$ is the concentration of $\ce A$, inside the reactor, at any time $t$. Then,
\beq
\iod t\conc A = \rrate\,,
\eeq

The last differential equation is the \lingo{characteristic equation of a batch reactor}. Considering a \lingo{first-order reaction} ($\rrate = -\krcoeff\conc A$), then
\beq
\iod t\conc A = -\krcoeff\conc A\,,
\eeq
whose solution is
\beq
\dfrac{\conc A}{\conc{A,0}} = \exp\vat{-\krcoeff t}\,,
\eeq
or, in alternative forms,
\begin{align*}
\kdim_{\conc A} &= \exp\vat{-\kdim_t}    & \text{[using dimensionless quantities]} \\
 \scpq{\conc A} &= \exp\vat{-\scpq t}\,. & \text{[using scaled quantities]}
\end{align*}
The last equations offer ways to relate concentration and time. At any $t$, we can know the concentration of $\ce A$ in the reactor, given the reaction constant and the initial concentration $\conc{A,0}$.

For a \lingo{second-order reaction}~\footnote{~The algebraic passages will hereafter be omitted.} ($\rrate = -\krcoeff\conc A^2$),
\beq
\dfrac{\conc A}{\conc{A,0}} = \dfrac{1}{1 + \krcoeff\conc{A,0}t}\,.
\eeq

This procedure may be repeated for any order of reaction by substituting the expression for $\rrate$ in the
characteristic equation. 


\subsubsection{Continuous-stirred tank reactor}
A \lingo{continuous-stirred tank reactor}, CSTR, is a well-mixed vessel that operates at \lingo{steady-state} -- no mass accumulates in the reactor. The main assumption is that the concentration of the incoming fluid will become \lingo{instantaneously} equal to the outgoing upon entering the vessel, see \cref{fig:cstreactor}.
%
% ------------------------------------------------------------- Figure
% position: bthH. size:width=0.5\textwidth. file:location+filename
 \docfigure{bt}{width=0.5\textwidth}{./figures/cst-reactor.pdf}%
   {CSTR}
   {Schema of a continuous-stirred tank reactor}%
   {fig:cstreactor}%
% ------------------------------------------------------------- EndFigure

A CSTR differs from a batch only in the fact that it is not closed. Thus, the mass of a species $\ce A$ flowing in and flowing out of the reactor, terms in the mass balance, will not cancel:
\beq
\iod t\mass_{\ce A} = \flow\vol\left(\conc{A,in} - \conc{A,out}\right) + \int_\vol\rrate\dx\vol = 0\,.
\eeq
Note, additionally, that the volumetric inflow and outflow are equal $\flow{\vol\txt{in}} = \flow{\vol\txt{out}} = \flow\vol$ and that the term that does cancel is the accumulation, due to the steady state hypothesis. Solving the differential equation, one finds that
\beq
\conc{A,in} - \conc{A,out} + \rtime\rrate = 0\,,
\eeq
where $\rtime = \vol/\flow\vol$ is the average \lingo{hydraulic residence time}. The last equation represents the \lingo{characteristic equation for a CSTR}. Assuming a \lingo{first-order reaction}, the model then becomes
\beq
\dfrac{\conc{A,out}}{\conc{A,in}} = \dfrac{1}{1 + \rtime}\,.
\eeq


\subsubsection{Plug flow reactor}
A \lingo{plug flow reactor}, PFR, consists in a long, straight pipe in which the reactive fluid transits at steady- state (no accumulation). The main assumptions of this model are that the fluid is completely mixed in any cross- section at any point, but it experiences no axial mixing; \ie, contiguous cross-sections cannot exchange mass with each other, see \cref{fig:pfreactor}.
%
% ------------------------------------------------------------- Figure
% position: bthH. size:width=0.5\textwidth. file:location+filename
 \docfigure{bt}{width=0.9\textwidth}{./figures/pf-reactor.pdf}%
   {PFR}
   {Schema of a plug flow reactor}%
   {fig:pfreactor}%
% ------------------------------------------------------------- EndFigure

Operating a mass balance on the selected volume $\Dx\vol = \surf\Dx\length$, and assuming steady-state conditions, we obtain
\beq
\iod t\mass = \flow\vol\conc A\vat{t} - \flow\vol\conc A\vat{t + \Dx t} + \int_{\Dx\vol}\rrate\,\dx\vol = 0\,,
\eeq
hence,
\beq
\flow\vol\conc A\vat{t} - \flow\vol\conc A\vat{t + \Dx t} + \rrate\Dx\vol = 
\flow\vol\conc A\vat{t} - \flow\vol\conc A\vat{t + \Dx t} + \rrate\flow\vol\Dx t = 0
\implies
\dfrac{\Dx\conc A}{\Dx t} = \rrate\,.
\eeq
Considering an infinitesimally thin cross-sectional volume, its thickness will reduce to $\dx\length$, therefore:
\beq
\iod t\conc A = \rrate\,,
\eeq
which is the \lingo{characteristic equation of the plug flow reactor}. Considering a \lingo{first-order reaction}, $-\rrate = \krcoeff\conc A$, the concentration equation will be
\beq
\scpq{\conc A} = \exp\vat{-\scpq t}\,.
\eeq


\subsection{Non-ideal chemical reactors - Segregated flow analysis}
The non-ideality of industrial and natural processes led engineers to develop corrections to the ideal models, in order to use them with less restrictions. For this reason, it is defined a residence time distribution, which is a function that describes the evolution of the average instantaneous concentration versus the elapsed time. It is very convenient to express the \lingo{residence time distribution} as the normalized function $\eadist$, called the \lingo{exit age distribution},
\beq
\eadist\vat t = \dfrac{\conc A\vat t}{\int_0^\infty\conc A\vat t\,\dx t}\,,
\eeq
which, due to its definition, has its total area under the curve equal to unity:
\beq
\int_0^\infty\eadist\vat t\,\dx t = 0\,.
\eeq

\Cref{fig:eadistcurve} shows the evolution of $\eadist$ \vs $t$. The $\eadist$ curve is the distribution needed to account for non-ideal flow.
%
% ------------------------------------------------------------- Figure
% position: bthH. size:width=0.5\textwidth. file:location+filename
 \docfigure{bt}{width=0.5\textwidth}{./figures/eadist-curve.pdf}%
   {E curve}
   {Residence time distribution: exit age distribution curve}%
   {fig:eadistcurve}%
% ------------------------------------------------------------- EndFigure

Considering the definition of $\eadist$, the average residence time becomes
\beq
\rtime = \int_0^\infty t\eadist\vat t\,\dx t\,.
\eeq

A useful tool used in this field is the \lingo{cumulative residence time fraction} (or cumulative frequency) curve $\crtfrac$, defined as
\beq
\crtfrac\vat t = \int_0^\infty t\eadist\vat t\,\dx t 
               = \dfrac{\int_0^{t_i}\conc A\vat t\,\dx t}{\int_0^\infty\conc A\vat t\,\dx t}\,.
\eeq
The last equation shows that the $\crtfrac$ curve at $t = t_i$ is defined as the cumulative area under the $\eadist$ curve from 0 to $t_i$. This means that $\crtfrac$ represents the fraction of flow with a residence time less or equal than $t_i$. Combining the two last equations, we have
\beq
\rtime = \int_0^1 t\,\dx\crtfrac\,,
\eeq
which is the highlighted area in \cref{fig:flowcharcurves}. Note that the boundaries of the last integral must be 0 to 1, since the area under the $\eadist$ curve equals unity.

The reason why we introduce the use of these functions is to quantify the non-ideality of reactors. A classic example is the evaluation of the average residence time. According to the ideal reactor theory, $\rtime = \vol/\flow\vol$, where $\vol$ is the total volume of the reactor. In case dead zones are present in the vessel, the residence time distribution will not account for them, showing a decreased reactor volume. Hence, $\rtime$ calculated in both ways will give an estimate of the dead zone volume.

\Cref{fig:flowcharcurves} illustrates the characteristic curves for various flows.
%
% ------------------------------------------------------------- Figure
% position: bthH. size:width=0.5\textwidth. file:location+filename
 \docfigure{bt}{width=0.7\textwidth}{./figures/flow-char-curves.pdf}%
   {Distribution curves}
   {Characteristic curves for various flow types}%
   {fig:flowcharcurves}%
% ------------------------------------------------------------- EndFigure





