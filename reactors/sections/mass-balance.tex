\section{Mass balance}
Mass is a conservative quantity, hence given a control volume $\vol$ the sum of mass flows entering the system will equal the sum exiting minus (plus) the consumed (generated) or accumulated fractions:
\beq
\begin{pmatrix}
\text{rate of mass} \\
\text{in}
\end{pmatrix}
+
\begin{pmatrix}
\text{rate of mass} \\
\text{out}
\end{pmatrix}
+
\begin{pmatrix}
\text{rate of mass} \\
\text{produced}
\end{pmatrix}
-
\begin{pmatrix}
\text{rate of mass} \\
\text{consumed}
\end{pmatrix}
=
\begin{pmatrix}
\text{rate of mass} \\
\text{accumulated}
\end{pmatrix}\,.
\eeq

The last statement represents the key point in \lingo{mass transfer}: analogously to the force balance in statics, the mass balance allows us to quantify and verify mass flows in our system.

Let us now apply this fundamental balance to some ideal examples.


\subsection{Ideal Chemical Reactors}

\subsubsection{Batch reactors}
A \lingo{batch reactor} is a non-continuous, perfectly mixed and closed vessel where a reaction takes place. 

Given its volume $\vol$ and the initial internal concentration of a species $\ce A$, $\conc{A,0}$, the total mass will be $\mass = \vol\conc{A,0}$. In the unit time, the concentration will be able to change only in virtue of a chemical reaction. The mass balance quantifies this change, in this case:
\beq
\flow\vol\conc{A,in} - \flow\vol\conc{A,out} \pm \int_\vol\rrate\dx\vol = \iod t\mass\,,
\eeq
where $\rrate$ is the rate of generation (+) or depletion (-). Since the assumptions of no flow in or out of the reactor volume, $\flow\vol = 0$, and constant reactor volume $\vol$,
\beq
\iod t\mass = \iod t\conc A\vol = \vol\iod t\conc A = \vol\rrate\,,
\eeq
where $\conc A = \conc A\vat t$ is the concentration at any time inside the reactor. Then,
\beq
\iod t\conc A = \rrate\,,
\eeq

The last differential equation is the \lingo{characteristic equation} of a batch reactor. Considering a \lingo{first-order reaction}; \ie, ($\rrate = -\krcoeff\conc A$):
\beq
\iod t\conc A = -\krcoeff\conc A\,,
\eeq
whose solution is
\beq
\dfrac{\conc A}{\conc{A,0}} = \exp\vat{-\krcoeff t}\,.
\eeq
The last equation offers a relationship between concentration and time. At any point in time, then, we can know the inner concentration, known the reaction constant and the initial concentration.

For a \lingo{second-order reaction} ($\rrate = -\krcoeff\conc A^2$),
\beq
\dfrac{\conc A}{\conc{A,0}} = \dfrac{1}{1 + \krcoeff\conc{A,0}t}\,.
\eeq

This procedure may be repeated for any order of reaction, just substituting the expression for $\rrate$ in the
characteristic equation [9]. 

\begin{note}
The algebraic passages will heretofore be omitted.
\end{note}


\subsubsection{Continuous-stirred tank reactor}
A Continuous-Stirred Tank Reactor (CSTR) is a well-mixed vessel that operates at steady-state ($\flow{\vol\txt{in}} = \flow{\vol\txt{out}} = \flow\vol$). The main assumption in this case is that the concentration of the incoming fluid will become \lingo{instantaneously} equal to the outgoing upon entering the vessel.

A CSTR differs from a batch only in the fact that it is not closed. Thus, the mass flows in and out of the reactor in the mass balance will not cancel:
\beq
\iod t\mass = \flow\vol\left(\conc{A,in} - \conc{A,out}\right) + \int_\vol\rrate\dx\vol = 0\,.
\eeq
The mass balance in the last equation equals to 0, due to the \lingo{steady state hypothesis}; \ie, no mass accumulation in the reactor. Solving,
\beq
\conc{A,in} - \conc{A,out} + \rtime\txt h\rrate = 0\,,
\eeq
where $\rtime\txt h = \vol/\flow\vol$ is the average \lingo{hydraulic residence time}. The last equation represents the \lingo{characteristic equation} for a CSTR. Assuming a first-order reaction,
\beq
\dfrac{\conc{A,out}}{\conc{A,in}} = \dfrac{1}{1 + \rtime\txt h}\,.
\eeq





