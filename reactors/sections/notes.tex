\section{Notes on notation}

\subsection{Einstein summation convention}
Consider a vector $v$ living in three dimensional (Euclidean) space, $\nespace 3$, and consider a Cartesian coordinate system subjacent $\nespace 3$. Now, let $\elset{\fbvec x, \fbvec y, \fbvec z}$ be the unit vectors pointing in the directions of the Cartesian axes and $\elset{\fvec vx, \fvec vy, \fvec vz}$ the components of $v$ projected onto the axes. Then, $v$ is traditionally written as
\beq
v = \fbvec x\fvec vx + \fbvec y\fvec vy + \fbvec z \fvec vz\,.
\eeq

Consider, on the other hand, an alternative, more compact form of writing $v$ -- the sigma notation:
\beq
v = \sum_{i = 1}^3 \fbvec i\fvec vi\,,
\eeq
where the index $i$ runs from 1 to 3, the dimension of space.

Einstein summation convention consists on dispensing with the summation sign and its limits, leaving only the index:
\beq
v = \fbvec i\fvec vi\,.
\eeq

Geometric derivative: define the \lingo{geometric derivative}, denoted $\gder$, by
\beq
\gder = \rbvec i\igd i\,,
\eeq
where index notation and Einstein summation convention were used.

Gradient: let $\phi = \phi\vat\pos$, then define the \lingo{gradient of $\phi$}, denoted $\grad\phi$, by
\beq
\grad\phi = \rbvec i\igd i\phi\,.
\eeq

The Laplacian: let $\phi = \phi\vat\pos$, then the \lingo{Laplacian of $\phi$} is defined as (in Cartesian coordinates)
\beq
\lap\phi = \igd i\igd j\phi\,.
\eeq

