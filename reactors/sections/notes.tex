\section{Notes on notation}

\subsection{Einstein summation convention}
Consider a vector $v$ in three dimensional (Euclidean) space, $\nespace 3$, and consider a frame of orthonormal vectors $\elset{\fbvec x, \fbvec y, \fbvec z}$ for $\nespace 3$. Let $\elset{\rvec vx, \rvec vy, \rvec vz}$ be the components of $v$ onto the frame. Then, $v$ is traditionally written as
\beq
v = \fbvec x\rvec vx + \fbvec y\rvec vy + \fbvec z \rvec vz\,,
\eeq
or as in other similar fashion; \eg, the frame elements noted by $e$ instead of $\fbvec{}$; or, as in engineering, $\elset{\hat\imath, \hat\jmath, \hat k}$; then, $v = \hat{\imath}v_x + \hat{\jmath}v_y + \hat{z}v_z$.

Consider, on the other hand, an alternative, more compact form of writing $v$. Begin by indexing the frame elements as $\setbuild{\fbvec i}{i:1\to 3}$. Then, relabel the components of $v$ to $\setbuild{\fvec vi}{i:1\to 3}$, where $i$ is an index and \emph{not} an exponent. Finally, use the summation notation to express $v$:
\beq
v = \fbvec 1\fvec v1 + \fbvec 2\fvec v2 + \fbvec 3\fvec v3 
  = \sum_{i = 1}^3 \fbvec i\fvec vi\,.
\eeq

Although the summation notation helps to save some typing, the Einstein summation convention goes further: by agreeing with dispensing with the summation sign and its limits, leaving only the indexed variables -- in this case, the indexed components:
\beq
v = \fbvec i\fvec vi\,.
\eeq

Besides being more compact, the summation convention allows the expression of vectors in any $n$-th dimensional space, without any notational change; \eg, consider $u\in\nespace n$ and an orthonormal frame $\elset{\fbvec k}$ for $\nespace n$, then the components of $u$ onto the frame can be written as
\beq
u = \fbvec k\fvec uk\,,
\eeq
where $k$ runs from 1 to $\dim\nespace n = n$.


\subsection{Metric}
Consider $\nespace n$ and a frame $\elset{\fbvec i}$. Then, define the \lingo{metric of $\nespace n$}, denoted $\met$, by defining the metric coefficients, $\fmet ij$,
\beq
\met = \fmet ij = \fbvec i\iprod\fbvec j\,.
\eeq
For an orthonormal frame, \eg, a Cartesian frame, the metric becomes Kronecker delta
\beq
\fmet ij = \fkron ij = \iver{i = j} = \diag\tuple{1, 1, \cdots, 1}\,,
\eeq
where $\iver{i = j}$ are Iverson brackets.


\subsection{Inner product}
Consider two vectors $u$ and $v$ in $\nespace n$. Then, define the \lingo{inner product of $u$ and $v$}, denoted by $u\iprod v$, by
\beq
u\iprod v = \fbvec i\fvec ui\iprod\fbvec j\fvec vj
          = \fbvec i\iprod\fbvec j\fvec ui\fvec vj
          = \fmet ij\fvec ui\fvec vj\,.
\eeq
For a Cartesian frame, $\met = \kron$, thus
\beq
u\iprod v = \fmet ij\fvec ui\fvec vj
          = \fkron ij\fvec ui\fvec vj
          = \fvec ui\fvec vj\iver{i = j}\,.
\eeq


\subsection{Partial derivatives}
Consider an orthonormal frame $\elset{\fbvec i}$ for $\nespace n$. Then, denote the \lingo{spatial partial derivatives} by
\beq
\xpd{}{\fvec\pos i} = \igd{\pos i} 
                    = \igd i\,,
\eeq
where $\elset{\fvec\pos i}$ are the components of the position vector, $\pos$, onto the frame.


\subsection{Geometric derivative}
Consider $\nespace n$ and consider an orthonormal frame $\elset{\fbvec i}$ in $\nespace n$ whose all of its elements are nonzero. Then, define a reciprocal orthogonal frame $\elset{\rbvec i}$ by
\beq
\rbvec i = \inv{\fbvec i} = \dfrac{\fbvec i}{\fbvec i\fbvec i}\,.
\eeq

Next, define the \lingo{geometric derivative}, denoted $\gder$, by
\beq
\gder = \rbvec i\igd i\,,
\eeq
where the summation convention was used~\footnote{~In traditional notation, the geometric derivative would be written as $\gder = \rbvec i\igd i = \rbvec i\xpd{}{\fvec\pos i}$.}.


\subsection{Gradient}
Let $\phi$ be a \lingo{scalar field}~\footnote{~A scalar field is a function of the position, and possibly time, that returns a scalar.}, $\phi\vat\pos$, then define the \lingo{gradient of $\phi$}, denoted $\grad\phi$, by
\beq
\grad\phi = \rbvec i\igd i\phi\,.
\eeq


\subsection{Divergence}
Let $\Phi$ be a \lingo{vector field}~\footnote{~A vector field is a function of the position, and possibly time, that returns a vector.}, $\Phi\vat\pos$, then define the \lingo{divergence of $\Phi$}, denoted $\div\Phi$, by
\beq
\div\Phi = \gder\iprod\Phi 
         = \rbvec i\igd i\iprod\fbvec j\fvec\Phi j
         = \igd i\rbvec i\iprod\fbvec j\fvec\Phi j
         = \igd i\mmet ij\fvec\Phi j
         = \igd i\fvec\Phi i\,.
\eeq


\subsection{Laplacian}
Let $\phi$ be a scalar field, $\phi\vat\pos$, then define the \lingo{Laplacian of $\phi$}, denoted $\lap\phi$, by
\beq
\lap\phi = \div\grad\phi
         = \gder\iprod\gder\phi
         = \rbvec i\igd i\iprod\rbvec j\igd j\phi
         = \igd i\igd j\rbvec i\iprod\rbvec j\phi
         = \igd i\igd j\rmet ij\phi\,.
\eeq
Note that $\met$ cannot be factored out in front of $\igd i\igd j$, because $\rmet ij$ may vary with $\pos$ -- as it's the case of spherical coordinates.

In Cartesian frames, the metric equals Kronecker delta, which does not depend on $\pos$, so
\beq
\lap\phi = \igd i\igd j\rmet ij\phi
         = \rkron ij\igd i\igd j\phi
         = \igd i\igd j\phi\iver{i = j}\,.
\eeq


\subsection{Lagrangian mechanics}
Lagrangian:
\beq
\lag = \ken - \pen\,.
\eeq

Euler-Lagrange equation:
\beq
\eleq{\pos}{\dt\pos}\,,
\eeq
where the term $\iod t\igd{\dt\pos}\lag$ is called the \lingo{generalized momentum} and the term $\igd{\pos}\lag$ the \lingo{generalized force}.

Hamiltonian: $\ham$.
