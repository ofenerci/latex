\section{Notes on notation}

\subsection{Einstein summation convention}
Consider a vector $v$ living in three dimensional (Euclidean) space, $\nespace 3$, and consider a frame of orthonormal vectors $\elset{\fbvec x, \fbvec y, \fbvec z}$ for $\nespace 3$. Let now $\elset{\fvec vx, \fvec vy, \fvec vz}$ be the components of $v$ onto the frame. Then, $v$ is traditionally written as
\beq
v = \fbvec x\fvec vx + \fbvec y\fvec vy + \fbvec z \fvec vz\,,
\eeq
or as in other similar fashion~\footnote{~Like the frame elements noted by $e$ instead of $\fbvec{}$. Moreover, in engineering, the frame would be $\elset{\hat\imath, \hat\jmath, \hat k}$ and thus $v$ would be written as $v = \hat{\imath}v_x + \hat{\jmath}v_y + \hat{z}v_z$. Notice the inconsistency of the engineering notation!}.

Consider, on the other hand, an alternative, more compact form of writing $v$. Begin by indexing the frame elements to $\setbuild{\fbvec i}{i:1\to 3}$. Then, relabel the components of $v$ to $\setbuild{\fvec vi}{i:1\to 3}$. Finally, use the summation notation to express $v$:
\beq
v = \fbvec 1\fvec v1 + \fbvec 2\fvec v2 + \fbvec 3\fvec v3 
  = \sum_{i = 1}^3 \fbvec i\fvec vi\,.
\eeq

Although the summation notation helps to save typing, the Einstein summation convention goes one step further: by agreeing with dispensing with the summation sign and its limits, leaving only the indexed variables -- in this case, the indexed components:
\beq
v = \fbvec i\fvec vi\,.
\eeq

Besides being more compact, the summation convention allows the expression of vectors living in any $n$-th dimensional space, without any notational change; \eg, consider $u\in\nespace n$ and an orthonormal frame $\elset{\fbvec k}$, then the components of $u$ on the frame can be written as
\beq
u = \fbvec k\fvec uk\,,
\eeq
where $k$ runs from 1 to $\dim\nespace n = n$.


\subsection{Metric}
Consider $n$-th dimensional Euclidean space, $\nespace n$ and consider a frame $\elset{\fbvec i}$. Then, define the \lingo{metric of $\nespace n$}, denoted $\metric$, by the metric coefficients, $\fmet ij$,
\beq
\metric = \fmet ij = \fbvec i\iprod\fbvec j\,.
\eeq
For an orthonormal frame, \eg, a Cartesian frame, the metric becomes Kronecker delta
\beq
\fmet ij = \delta_{ij} = \diag\tuple{1, 1, \cdots, 1}\,.
\eeq


\subsection{Geometric derivative}
Consider $n$-th dimensional Euclidean space, $\nespace n$ and consider an orthogonal frame $\elset{\fbvec i}$ whose all of its elements are nonzero. Then, define a reciprocal orthogonal frame $\elset{\rbvec i}$ by
\beq
\rbvec i = \inv{\fbvec i} = \dfrac{\fbvec i}{\fbvec i\fbvec i}\,.
\eeq

On the other hand, define the \lingo{geometric derivative}, denoted $\gder$, by
\beq
\gder = \rbvec i\igd i\,,
\eeq
where the summation convention was used.

In traditional notation, the geometric derivative would be written as
\beq
\gder = \xpd{}{\fvec\pos i}
      = \igd{\pos i}
      = \igd i\,,
\eeq
where $\elset{\fvec\pos i}$ are the components of the position vector, $\pos$, on the orthogonal frame.


\subsection{Gradient}
With the geometric derivative and the summation convention, it becomes easier to note the gradient, divergence, curl and the Laplacian.

Let $\phi$ be a scalar field $\phi\vat\pos$, then define the \lingo{gradient of $\phi$}, denoted $\grad\phi$, by
\beq
\grad\phi = \rbvec i\igd i\phi\,.
\eeq


\subsection{Divergence}
Let $\Phi$ be a vector field $\Phi\vat\pos$, then define the \lingo{divergence of $\Phi$}, denoted $\div\Phi$, by
\beq
\div\Phi = \gder\iprod\Phi 
         = \rbvec i\igd i\iprod\fbvec j\fvec\Phi j
         = \rbvec i\iprod\fbvec j\igd i\fvec\Phi j
         = \metric^i_j\igd i\fvec\Phi j
         = \igd j\fvec\Phi j\,.         
\eeq

The Laplacian: let $\phi = \phi\vat\pos$, then the \lingo{Laplacian of $\phi$} is defined as (in Cartesian coordinates)
\beq
\lap\phi = \igd i\igd j\phi\,.
\eeq

