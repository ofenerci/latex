\chapter*{\docTitle}\label{sec:report}


\section*{Corrida de toros}
%
La corrida de toros es un espectáculo en que se \lingo{lidia}, a pie o a caballo, toros bravos en una \lingo{plaza de toros}.

En la corrida, los \lingo{toreros} son los protagonistas. Estos siguen un estricto protocolo, reglamentado por la estética, para lidiar toros. 

Las corridas pueden clasificarse según la edad y el \lingo{trapío} del toro: en \lingo{becerradas}, \lingo{novilladas} y \lingo{corridas} de toros propiamente dichas, las que se desarrollan a pie o a caballo -- \lingo{rejoneo}.


\section*{Breve historia}
%
Antiguamente, los \lingo{matadores} recorrían los pueblos de España, divirtiendo al público mediante la práctica del toreo a pie. A éstos les acompañaban los \lingo{pajes}: ayudantes de los matadores. 

En 1723, Felipe V prohibió el toreo a caballo a sus cortesanos. Los matatoros y pajes empezaron a torear por su cuenta en las ciudades más importantes y a desatar el entusiasmo del gran público.

En la segunda mitad del siglo XVIII, se produjeron en España cambios en su práctica que dieron lugar a las corridas modernas:
%
\begin{itemize}
%
\item el toreo a pie sustituye al de a caballo;
%
\item nacen las \lingo{ganaderías bravas} y se comienza a seleccionar toros;
%
\item se construyen las primeras plazas como edificios permanentes y
%
\item se escriben las primeras \lingo{tauromaquias}: reglas que fijan la técnica y las normas del arte de torear.
%
\end{itemize}


\section*{La Corrida: El Ritual del Toreo}
%
La corrida comienza con un desfile de los matadores seguidos del personal de la plaza. El espectáculo en sí se divide en tres partes, denominadas tercios.


\subsection*{Los participantes}
%
\begin{itemize}
%
\item El matador: quien realiza la parte principal de la \lingo{faena} y mata al toro con el \lingo{estoque}.
%
\item Banderilleros: quienes asisten al matador.
%
\item Picador: jinete que utiliza una vara larga para castigar al toro.
%
\item Presidente: persona que preside un festejo. Éste se encarga de ordenar el comienzo del festejo, los cambios de tercio y premiar a los matadores.
%
\end{itemize}


\subsection*{Tercio de Varas}
%
El toro entra en el \lingo{ruedo} donde será probado por el matador con el capote. Aquí el matador observa el comportamiento, las \lingo{embestidas} y la \lingo{bravura} del toro.

Después, entran dos picadores armados con una lanza y montados en caballos grandes, protegidos y con los ojos cubiertos. Cuando el toro ataca al caballo, el picador lo pica en el \lingo{morrillo}, joroba musculosa del cuello del toro. Con esto, se logra que el toro mantenga la cabeza baja y así que embista de una manera menos peligrosa y más fiable para el matador.


\subsection*{Tercio de banderillas}
%
En este tercio, los tres banderilleros tienen que poner banderillas en el toro. Estas lo debilitan y enfurecen, haciendo que embista fieramente.


\subsection*{Tercio de Muerte}
%
El matador entra en la arena con una \lingo{muleta}, sujetando a un palo en una mano y una espada en la otra. 

La faena termina con una serie de \lingo{pases} en los que el matador mata al toro de una \lingo{estocada}: clavando la espada entre sus hombros hacia su corazón.

Si el matador \emph{no} tiene éxito en la estocada, usa una espada para bajar la cabeza del toro para cortar la espina dorsal, produciendo su muerte instantánea.

El cuerpo del toro es luego arrastrado afuera por caballos. Si la faena ha sido buena, puede ser arrastrado por toda la plaza como un honor. Raramente \lingo{indultado}: el perdón de la muerte.

Si el matador ha hecho una buena faena, la multitud pide un premio para él: una oreja del toro. Si la faena es excepcional, se el dan las dos orejas.


\section*{Cogidas de Toreros}
%
Las corridas de toros se han vuelto mucho más seguras con el paso de los años, especialmente con protecciones para los caballos y los equipos médico para los toreros. Para un torero, el momento más peligroso es la entrada a matar ya que se hace sin protección; entonces, un error, le puede costar la vida. \lingo{Cogida} es la palabra usada cuando el toro \lingo{cornea} al matador.


\section*{Controversia y Protestas}
%
Las corridas son controvertidas en muchas partes del mundo, incluyendo a España, Portugal, Perú, México y Ecuador, llegando a su prohibición. Sus partidarios argumentan su tradición cultural y su atracción turística, mientras que los defensores de los derechos de los animales lo consideran como un deporte sangriento que tortura a toros y caballos.

Sin embargo, las corridas se considera como una parte integrante de la cultura hispánica. Es la realización del espíritu de España. Esto se demuestra con reflejos en la literatura (Lope de Vega, Garcia Lorca, Heminguay) y en la pintura (Goya, Picasso). En fin, los espectadores no van al festejo para ver las torturas del toro, pero van a ver la valentía y \scare{maestría} del matador. El toro bravo, en su turno, puede morir luchando con \scare{honor} y \scare{fama}.
