%%% MATHS
%
%%% GENERAL COMMANDS
%
\newcommand{\mqed}{\tag*{\qedsymbol}}             % math qed
\newcommand*{\txtqed}{\hfill\ensuremath{\square}} % text qed
%
\newcommand{\beq}{\begin{equation*}}
\newcommand{\eeq}{\end{equation*}}
%
\newcommand{\defby}{\doteq} % defined by
%
\newcommand{\txt}[1]{_\text{#1}} % subscript text
%
%%% SETS
%
\newcommand{\set}[1]{\mathcal{#1}}
\newcommand{\nset}[2]{\set{#2}^{#1}}  % n dim set
\newcommand{\espace}[1]{\nset{#1}{E}} % n dim. Euclidean space
\newcommand{\region}[1]{\set{#1}}
%
\newcommand{\elset}[1]{\left\lbrace{#1}\right\rbrace} % elements of a set
\newcommand{\tuple}[1]{\left[{#1}\right]}
%
%%% FUNCTIONS
%
\newcommand{\vat}[1]{\left[{#1}\right]} % function value at
\newcommand{\orderof}{O}                % order of (magnitude)
%
%%% GEOMETRIC OBJECTS
%
\newcommand{\bound}{\partial} % boundary
\newcommand{\surf}{s}         % surface
\newcommand{\vol}{v}          % volume
%
%%% GEOMETRIC ALGEBRA
%
\newcommand{\magn}[1]{\vert{#1}\vert} % magnitude
%
\newcommand{\iprod}{\cdot}  % inner product
\newcommand{\oprod}{\wedge} % outer product
%
\newcommand{\inv}[1]{{#1}^{-1}} % inverse
%
\newcommand{\uvec}[1]{\hat{#1}} % unit vector
%
%%% GEOMETRIC CALCULUS
%
\newcommand{\gder}{\nabla}   % geometric derivative
\newcommand{\lder}{\nabla^2} % laplace derivative
%
\DeclareMathOperator{\Div}{div} % divergence
\renewcommand{\div}{\Div}
\DeclareMathOperator{\grad}{grad} % gradient
\DeclareMathOperator{\lap}{lap}   % Laplace operator
%
\newcommand{\dirdev}[2]{\gder_{#1}{#2}} % directional derivative
%
%%% CALCULUS
%
\newcommand{\dx}{\text{d}} % differential operator
\newcommand{\Dx}{\Delta}   % difference operator
%
% derivatives
\newcommand{\dt}[1]{\dot{#1}}       % dot derivative
\newcommand{\ddt}[1]{\ddot{#1}}     % ddot derivative
\newcommand{\iod}[1]{\dx_{#1}}      % indexed ordinary derivative
\newcommand{\ipd}[1]{\partial_{#1}} % indexed partial derivative
\newcommand{\igder}[1]{\ipd{#1}}    % indexed geometric derivative
\newcommand{\cder}[2]{{#1}_{,#2}}   % comma derivative
%
\newcommand{\xod}[2]{\dfrac{\dx{#1}}{\dx{#2}}}           % expanded ordinary derivative
\newcommand{\xpd}[2]{\dfrac{\partial{#1}}{\partial{#2}}} % expanded partial derivative
\newcommand{\nxpd}[3]{\dfrac{\partial^{#3}{#1}}{\partial{#2}^{#3}}} % n expanded partial derivative
%
%%% INDEX NOTATION
%
\newcommand{\bvec}{\gamma}         % basis vector
\newcommand{\fbvec}[1]{\bvec_{#1}} % frame (basis) element
\newcommand{\rbvec}[1]{\bvec^{#1}} % reciprocal frame (basis) element
\newcommand{\met}{g}               % metric
%
\newcommand{\frm}[1]{\elset{\fbvec{#1}}}  % frame
\newcommand{\rfrm}[1]{\elset{\rbvec{#1}}} % reciprocal frame
%
\newcommand{\fmet}[2]{\met_{{#1}{#2}}}% metric in frame
\newcommand{\rmet}[2]{\met^{{#1}{#2}}}% metric in reciprocal frame
%
\newcommand{\upipd}[1]{\partial^{#1}} % up indexed partial derivative
\newcommand{\dnipd}[1]{\partial_{#1}} % down indexed partial derivative
%
\newcommand{\fvec}[2]{{#1}^{#2}} % frame (contravariant) components
\newcommand{\rvec}[2]{{#1}_{#2}} % reciprocal frame (covariant) components
%
\newcommand{\ufbvec}[1]{\uvec{\bvec}_{#1}} % unit frame basis vector
%
%%% MATRICES
%
\DeclareMathOperator{\diag}{diag} % diagonal
\DeclareMathOperator{\sig}{sig}   % signature
%
%%% BRACKETS
%
\newcommand{\iverson}[1]{\left[{#1}\right]\txt{Iv}} % Iverson brackets
