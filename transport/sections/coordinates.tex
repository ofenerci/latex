\section{Coordinate systems and index notation}

\subsection{Cartesian coordinate system}
Consider a region $\region V$ to be a part of an $n$ dimensional Euclidean space, $\espace n$. Consider, next, a Cartesian coordinate system; \ie, a coordinate system equipped with a frame $\frm k$ whose elements satisfy
\beq
\fbvec k\iprod\fbvec l = \fmet kl = \iverson{k = l}\,.
\eeq
If frame elements satisfy the last equation, then they are \lingo{orthogonal} to each other and \lingo{normal} (have unit length); thus, the frame $\frm k$ is said to be \lingo{othornormal}.

Express now the position of any point $\pos$ in $\region V$ as an $n$-tuple
\beq
\pos = \tuple{\fvec\pos 1, \dots, \fvec\pos n}\,,
\eeq
or as a linear combination of the frame elements
\beq
\pos = \fvec\pos k\fbvec k\,,
\eeq
where the elements of $\elset{\fvec\pos k}$ are called the \lingo{components of $\pos$ onto the $\frm k$ frame}. (Einstein summation convention in force!)

To find the components of $\pos$ on $\frm k$, apply
\beq
\fvec\pos k = \pos\iprod\fbvec k\,.
\eeq


\subsubsection{Reciprocal Cartesian coordinate system}
Consider now a reciprocal Cartesian coordinate system $\rfrm k$ whose elements are the inverse~\footnote{~The square of a vector $a$ is defined as $a^2 = aa$, which, by axiom, $a^2\in\espace 1$. The inverse of a nonzero vector $a$ is then defined as $\inv a = a/a^2$.} of the $\frm k$ elements:
\beq
\rbvec k = \inv{\fbvec k} = \fbvec k\,,
\eeq
since the frame elements are orthonormal; \viz, $\inv{\fbvec k} = \fbvec k/\fbvec k^2$, but $\fbvec k^2 = 1$.


\subsubsection{Geometric derivative, gradient, divergence, Laplacian}
Define the geometric derivative (operator) $\gder$ by its components as~\footnote{~In traditional notation, 
\beq
\gder = \rbvec k\xpd{}{\fvec\pos k} 
      = \uvec x\xpd{}{x} + \uvec y\xpd{}{y} + \uvec z\xpd{}{z}\,.
\eeq}
% end footnote
\beq
\gder = \rbvec k\ipd k\,.
\eeq
This means, treat the geometric derivative as another vector.

Consider a scalar field $\phi = \phi\vat\pos$. Then, define the gradient of $\phi$ by
\beq
\grad\phi = \gder\phi 
          = \rbvec k\igder k\phi\,.
\eeq

Consider next a vector field $\psi = \psi\vat\pos$. Then, define the divergence of $\psi$ by
\beq
\div\psi = \gder\iprod\psi 
         = \rbvec k\igder k\iprod\rbvec l\rvec\psi l 
         = \ipd k\rmet kl\rvec\psi l\,,
\eeq
where $\rmet kl = \rbvec k\iprod\rbvec l$.

Finally, define Laplace operator on a scalar field $\phi$ by
\beq
\lap\phi = \div\grad\phi 
         = \gder\iprod\gder\phi
         = \rbvec k\igder k\iprod\rbvec l\igder l\phi
         = \igder k\igder l\rmet kl\phi\,.
\eeq


\subsection{Alternative coordinate systems}
Consider next another, alternative, coordinate system $\frm{k'}$, related to the Cartesian coordinate by
\beq
\fvec\pos k = f\vat{\fvec{\pos}{k'}}\,.
\eeq

It is possible now to express the Cartesian frame $\frm k$ in the alternative coordinate system by using the \lingo{tangent vectors} $\frm{k'}$ \emph{defined} by~\footnote{~In traditional notation:
\beq
\fbvec{k'} = \ipd{k'}{\pos} = \xpd{\pos}{\fvec{\pos}{k'}}\,.
\eeq}
% end footnote
\beq
\fbvec{k'} = \ipd{k'}\pos\,.
\eeq
These tangent vectors need \emph{not} be orthogonal nor normal.

Onto this alternative frame $\frm{k'}$, any point position $\pos$ can be expressed as a linear combination of the alternative frame elements:
\beq
\pos = \fvec\pos{k'}\fbvec{k'}\,,
\eeq
where $\fvec\pos{k'} = \pos\iprod\fbvec{k'}$.

The metric coefficients for the alternative frame are found by
\beq
\fmet{k'}{l'} = \fbvec{k'}\iprod\fbvec{l'}\,.
\eeq


\subsection{Examples}
Examples of this section:

\begin{example}
In Cartesian coordinates, a particle position is given by $\pos = \tuple{x,y}$, where $x$ and $y$ are the coordinates of the particle. Express the position in polar coordinates.
\end{example}

\begin{solution}
Polar coordinates, $\rho$ and $\varphi$, are defined by
\beq
x = \rho\cos\vat\varphi\qquad\text{and}\qquad
y = \rho\sin\vat\varphi\,,
\eeq
where $\rho^2 = x^2 + y^2$ and $\varphi = \arctan2\vat{y,x}$, see \url{https://en.wikipedia.org/wiki/Polar_coordinates#Converting_between_polar_and_Cartesian_coordinates}.

Then, $\pos$ can be written as
\beq
\pos = \tuple{x,y} = \rho\tuple{\cos\vat\varphi, \sin\vat\varphi}\,.
\eeq

The tangent vectors are thus
\beq
\fbvec\rho = \xpd{\pos}{\rho} 
           = \tuple{\cos\vat\varphi, \sin\vat\varphi}
\implies
\ufbvec\rho = \dfrac{\fbvec\rho}{\magn{\fbvec\rho}} 
            = \tuple{\cos\vat\varphi, \sin\vat\varphi}
\eeq
in the $\rho$ direction and
\beq
\fbvec\varphi = \xpd{\pos}{\varphi} 
              = \rho\tuple{-\sin\vat\varphi, \cos\vat\varphi}
\implies
\ufbvec\varphi = \dfrac{\fbvec\varphi}{\magn{\fbvec\varphi}} 
               = \dfrac{\fbvec\varphi}{\rho}
               = \tuple{-\sin\vat\varphi, \cos\vat\varphi}
\eeq
in the $\varphi$ direction.

In Cartesian coordinates, unit vectors are constant in direction and in magnitude. However, in polar coordinates, neither unit vector is constant.

With the unit tangent vectors, $\pos$ can be written as
\beq
\pos = \tuple{x,y} 
     = \tuple{\rho\cos\vat\varphi, \rho\sin\vat\varphi}
     = \rho\tuple{\cos\vat\varphi, \sin\vat\varphi}
     = \rho\ufbvec\rho\,.\mqed
\eeq
\end{solution}


\begin{example}
Express a particle velocity in polar coordinates.
\end{example}

\begin{solution}
Since the particle position in polar coordinates can be written as $\pos = \rho\ufbvec\rho$, the particle velocity, $\dt\pos$, can be calculated by calculating the change of the particle position
\beq
\dx\pos = \dx\left(\rho\ufbvec\rho\right)
        = \dx\rho\ufbvec\rho + \rho\dx\ufbvec\rho\,.
\eeq
The change of $\ufbvec\rho$ is given by
\beq
\dx\ufbvec\rho = \tuple{-\sin\vat\varphi\,\dx\varphi, \cos\vat\varphi\,\dx\varphi}
               = \dx\varphi\tuple{-\sin\vat\varphi, \cos\vat\varphi}
               = \dx\varphi\ufbvec\varphi\,.
\eeq
Thus, the change of the particle position becomes
\beq
\dx\pos = \dx\left(\rho\ufbvec\rho\right)
        = \dx\rho\ufbvec\rho + \rho\dx\varphi\ufbvec\varphi\,.
\eeq

Finally, the change of the particle position per unit time, \ie, velocity, is
\beq
\dt\pos = \xod{\pos}{t} 
        = \xod\rho t\ufbvec\rho + \rho\xod\varphi t\ufbvec\varphi
        = \dt\rho\ufbvec\rho + \rho\dt\varphi\ufbvec\varphi\,,
\eeq
where $\dt\rho$ is the \lingo{radial velocity}, rate at which $\rho$ changes magnitude or stretches' and $\rho\dt\varphi$ the \lingo{circumferential velocity}, rate at which $\rho$ changes direction or swings.
\end{solution}
