\section{Dimensionless Quantities}
Dimensionless groups can be classified into three broad categories, the dimensionless fluxes, the ratios of convection and diffusion and the ratios of different types of diffusion. Before proceeding to define dimensionless groups, we first define the diffusion coefficients for mass, momentum and energy, in order to provide a definite basis for the discussion. Here, we restrict ourselves to the macroscopic definition.


\subsection{Mass, momentum and energy diffusivities}
The fundamental constitutive relations in transport processes are the Fick's law for mass diffusion, Fourier's law for heat conduction and Newton's law for viscosity. The respective diffusion coefficients can be defined as follows.
%
\begin{itemize}
\item When there is a concentration difference maintained across a slab of fluid, then there is transfer of mass from the surface at higher concentration to the surface at lower concentration. The mass flux (mass per unit area per unit time) is inversely proportional to the length and directly proportional to the difference in concentration or temperature across the material. If a concentration difference $\Dx\conc$, $\dim\conc = \phdim M/\phdim L^3$, is maintained between two ends of a slab of length $\length$, the mass flux $\flux\mass$, $\dim\flux\mass = \phdim M/\phdim L^2\phdim T$, is given by Fick's law:
\beq
\flux\mass = -\kdiff \dfrac{\Dx\conc}{\length}\,.
\eeq
Here, the negative sign indicates that mass is transferred from the region of higher concentration to the region of lower concentration.
%
\item For the transport of thermal energy, the energy flux is related to the temperature difference by Fourier's law. For a slab of material of length $\length$ with a temperature difference $\Dx\temp$ across the material, the energy flux $\flux\then$ is given by
\beq
\flux\then = -\kthcond\dfrac{\Dx\temp}{\length}\,,
\eeq
where $\kthcond$ is the thermal conductivity, and the negative sign indicates that thermal energy is transferred from the region of higher to the region of lower temperature.
%
\item The Newton's law of viscosity relates the shear stress $\shear$ (force per unit area at the wall) to the strain rate (change in velocity per unit length across the flow) for the simple shear flow of a fluid as shown in figure 1.1.2.
\beq
\shear_{xy} = \kinvis\dfrac{\Dx\vel_x}{\length}.
\eeq
It should be noted that there is no negative sign in Newton's law, in contrast to Fick's law and Fourier's law. This is due to the difference in convention with regard to the definition of stress in fluid mechanics and the definition of fluxes in transport phenomena. The shear stress $\shear_{xy}$ in Newton's law is defined as the force per unit area at a surface in the $x$ direction whose outward unit normal is in the $y$ direction. In contrast, the fluxes are defined as positive if they are directed into the volume. Therefore, the shear stress is actually the negative of the momentum flux. If the stress is defined to the the force per unit area acting at a surface whose inward unit normal is in the $x$ direction, then this would introduce a negative sign in Newton's equation. However, it is conventional in fluid mechanics to define the stress with reference to the outward unit normal to the surface. As we will see later, this difference in convention will not affect the balance equations that are finally obtained for the rate of change of momentum.

In this course, we will adopt the convention of defining the stress $\shear_{xy}$ as the force per unit area in the $x$ direction acting along a surface whose outward unit normal is in the $y$ direction and use the last equation for Newton's law of viscosity.
\end{itemize}

The diffusion coefficients are the proportionality constants in the relationship between the flux of a quantity (mass, thermal energy, momentum) and the driving force. The flux of a quantity (mass, thermal energy, momentum) is the amount of that quantity transferred per unit area per unit time. The driving force for a quantity (mass, thermal energy, momentum) is the gradient (change per unit distance) in the density (quantity per unit volume) of that quantity. So the transport equations can be written in the general form
\beq
\begin{pmatrix}
\text{transport of quantity} \\
\text{per unit area} \\
\text{per unit time}
\end{pmatrix} = 
\begin{pmatrix}
\text{diffusion} \\
\text{coefficient}
\end{pmatrix}
\dfrac{
    \begin{pmatrix}
    \text{change in density (per unit volume)} \\
    \text{of the quantity} \\
    \text{across the material}
    \end{pmatrix}
    }
    {\text{thickness of the material}}
\eeq
From dimensional analysis of the above equation, it is easy to see that the diffusion coefficients of all quantities have dimensions of $\phdim L^2/T$. These diffusion coefficients are defined from Fick's law, Fourier's law and Newton's law as follows.

\begin{enumerate}
\item From Fick's law, the diffusion coefficient $\kdiff$ is the ratio of mass flux (mass transported per unit area per unit time) and the gradient in the concentration (mass per unit volume). Therefore, the diffusivity of mass is just the diffusion coefficient $\kdiff$.

\item It is possible to define a diffusion coefficient for heat transfer as follows. The difference in temperature $\Dx\temp$ can be expressed in terms of the difference in the specific energy between the two sides as $\Dx\temp = \Dx\then/\dens\kshcap_v$, where $\Dx\then$ is the specific thermal energy (per unit volume) and $\kshcap_v$ the volumetric thermal capacity. With this, the equation for the thermal flux can be written as
\beq
\flux\then = \dfrac{\kthcond}{\dens\kshcap_v}\dfrac{\Dx\then}{\length}\,.
\eeq
It is obvious that the above equation has the same form as the mass flux equation, with a thermal diffusivity $\kdiff_h = \kthcond/\dens\kshcap_v$, which has $\dim\kdiff_h = \phdim L^2/\phdim T$.

\item The \lingo{momentum diffusivity} is the relation between the flux of momentum (rate of transport of momentum per unit area per unit time) and the difference in the momentum density (momentum per unit volume). Consider the layer of fluid shown in figure 1.1.2. Since the momentum of a parcel of fluid is the mass of that parcel multiplied by its velocity, the momentum density is the product of the mass density $\dens$ and velocity $\vel_x$. Therefore, the equation for the flux, expressed in terms of the momentum density, is
\beq
\shear_{xy} = \kinvis\dfrac{\Dx(\dens\vel_x)}{\length}\,,
\eeq
where $\kinvis$ is the momentum diffusivity. For an incompressible fluid with constant density, the momentum diffusivity is $\kinvis = \dynvis/\dens$. The momentum diffusivity, which has dimensions $\dim\kinvis = \phdim L^2/\phdim T$, is also referred to as the \lingo{kinematic viscosity}.
\end{enumerate}

The dimensionless quantities ratios of diffusivities are summarized in \autoref{tab:rationofdiffusivities}.
%
% ------------------------------------------------------------- PreTable
\docpretable{bt}{0.9\textwidth}{ccl}%
% position: bthH. size: 0.9\textwidth. cols: llcp{6mm}
% use: \docfloatwidth whenever possible!
% NOTE: does not include \toprule
\toprule
Dimensionless quantity & Definition & Ratio \\
\midrule
Reynolds number & $\kreynolds = \dens\vel\length/\dynvis$ & momentum convection to momentum diffusion \\
Prandtl number  & $\kprandtl = \kinvis/\kdiff_h$          & momentum diffusion to thermal diffusion   \\
Schmidt number  & $\kschmidt = \kinvis/\kdiff$            & momentum diffusion to mass diffusion      \\
Peclet number   & $\kpeclet = \vel\length/\kdiff_h$       & thermal convection to thermal diffusion \\
\bottomrule
% ------------------------------------------------------------ PostTable
\end{tabularx}
\docposttable{Dimensionless diffusivities ratios}{Dimensionless quantities that are ratios of diffusivities. The Prandtl number $\kprandtl = \kinvis/\kdiff_h = \kshcap_\press\dynvis/\kthcond$ is the ratio of momentum and thermal diffusivity and the Schmidt number $\kschmidt = \dynvis/\dens\kdiff = \kinvis/\kdiff$ the ratio of momentum and mass diffusivity.}{tab:rationofdiffusivities}
% include: \end{tabularx}%
% ------------------------------------------------------------ EndTable


\subsection{Ratio of convection and diffusion}
Convective transport takes place due to the mean flow of a fluid, even in the absence of a concentration difference. For example, if a fluid with concentration $\conc$ travels with velocity $\vel$ in a pipe, the total amount of mass transported per unit time is ($\conc\vel\area_p$), where $\area_p$ is the cross-sectional area of the tube. Therefore, the flux (mass transported by the fluid, per unit area perpendicular to the flow per unit time) is ($\conc\vel$). Consequently, the ratio between the rate of transport due to convective and diffusive effects is ($\vel\length/\kdiff$), where $\length$ is the length scale across which there is a change in the density of this quantity.

The dimensionless numbers which are ratios of convective and diffusive transport rates (\autoref{tab:rationofdiffusivities}) 
%
\begin{itemize}
\item the Reynolds number, $\kreynolds = \dens\vel\length/\dynvis = \vel\length/\kinvis$, the ratio of momentum convection and diffusion,

\item the Peclet number for mass transfer, $\kpeclet = \vel\length/\kdiff$, the ratio of mass convection and diffusion, and

\item the Peclet number for thermal transfer, $\kpeclet = \vel\length/\kdiff_h = \dens\kshcap_\press\vel\length/\kthcond$, where $\kshcap_\press$ is the specific thermal capacity at constant pressure.
\end{itemize}


\subsection{Dimensionless numbers in natural convection}
In natural convection, the driving force for convection is the body force caused by a variation in the density of the fluid, which is in turn caused by variation in temperature. In contrast to natural convection, the characteristic flow velocity is not known \apriori, but has to be determined from the driving force due to the density difference. If we assume there is a balance between the driving force and the viscous stresses in the fluid, dimensional analysis can be used to infer that the characteristic velocity should scale as $\chpq\vel\approx\force\diam/\dynvis$, where $\force$ is the driving force for convection per unit volume of fluid, $\diam$ is the diameter of the body or the characteristic length and $\dynvis$ is the dynamic viscosity.

The driving force for convection, per unit volume of the fluid, is proportional to the product of the density variation $\Dx\dens$ and the acceleration due to gravity $\grav$. The density variation is is proportional to $\dens\kthexp\Dx\temp$, where $\Dx\temp$ is the variation in temperature and $\kthexp$ is the coefficient of thermal expansion. Therefore, the force per unit area is proportional to $\dens\grav\kthexp\Dx\temp$. With this, the characteristic velocity for the flow is $\chpq\vel\approx\dens\diam^2\grav\kthexp\Dx\temp/\dynvis$. The \lingo{Grashof number} is the Reynolds number based upon this convection velocity,
\beq
\kgrashof = \dfrac{\dens\chpq\vel\diam}{\dynvis} 
          = \dfrac{\dens^2\diam^3\grav\kthexp\Dx\temp}{\dynvis^2}\,.
\eeq

In defining the Grashof number, we used the momentum diffusivity and the body force to obtain the characteristic velocity scale. An alternative definition is the \lingo{Rayleigh number}, where the thermal diffusivity, $\kthcond/\dens\kshcap_\press$, is used to determine the characteristic velocity. The characteristic velocity is then given by $\chpq\vel\approx\force\diam^2\kshcap_\press/\kthcond$, where $\force = \dens\kthexp\Dx\temp$ is the driving force for convection per unit volume of fluid. The Reynolds number based on this convection velocity and momentum diffusivity is the Rayleigh number,
\beq
\krayleigh = \dfrac{\dens^2\kshcap_\press\diam^3\grav\kthexp\Dx\temp}{\dynvis\kthcond}\,.
\eeq
The \lingo{Rayleigh-Benard} instability of a fluid layer heated from below occurs when the Rayleigh number increases beyond a critical value.


\subsection{Other dimensionless groups}
There are two important dimensionless groups involving surface tension, which are the \lingo{Weber number} and the \lingo{capillary number}. The Weber number is the ratio of inertial and surface tension forces,
\beq
\kweber = \dfrac{\dens\vel^2\length}{\surftens}\,,
\eeq
while the capillary number is the ratio of viscous and surface tension forces,
\beq
\kcapillar = \dfrac{\dynvis\vel}{\surftens}\,,
\eeq
where $\surftens$ is the surface tension, $\dim\surftens = \phdim F/\phdim L$.

In problems involving free interfaces, if the Reynolds number is low, inertial forces are negligible compared to viscous forces, and the capillary number has to be determined in order to assess the ratio of viscous and surface tension forces. At high Reynolds number, it is appropriate to determine the Weber number in order to examine the ratio of inertial and surface tension forces. In addition to the Weber and capillary numbers, the \lingo{Bond number} is a dimensionless group appropriate for situations where a free interface is under the influence of a gravitational field. This dimensionless group is defined as the ratio of gravitational and surface tension forces,
\beq
\kbond = \dfrac{\dens\grav\length^2}{\surftens}\,.
\eeq


\subsection{Dimensionless groups involving gravity}
The ratio of inertial and gravitational forces is given by the \lingo{Froud number},
\beq
\kfroud = \dfrac{\vel^2}{\grav\length}\,,
\eeq
In applications involving rotation of fluids, the Froud number is also the ratio of centrifugal and gravitational forces.


\subsection{Summary}
From the magnitudes of dimensionless quantities, one can immediately infer which effects are dominant and which are negligible. For instance, if the Reynolds number is small, then inertial effects are small compared to viscous effects. Thus, the effect of inertia can be neglected and fluid density is no longer an important parameter. However, it should be noted that though convective effects can be neglected with the Reynolds number or Peclet number is small, diffusive effects can not always be neglected even when these numbers are large, for reasons to do with the formation of momentum or thermal boundary layers near surfaces where these effects become important.


\subsection{Examples}
Some examples on the use of dimensional analysis and dimensionless quantities.

\begin{example}
A fluid flows through a pipe heated from the outside. The temperature of the wall of the pipe is higher than the average temperature of the fluid by a constant value $\Dx\temp$. The change in temperature of the fluid is due to the thermal transfer from the walls and not due to frictional heating generated by the flow. One would like to predict the rate of thermal transfer, per unit area of the wall of the pipe, so that the length of pipe required for the heat exchanger can be designed.
\end{example}

\begin{solution}
The first step is to collect the set of variables on which the thermal flux $\flux\then$ may depend. This can depend on the thermal properties of the fluid, the conductivity $\kthcond$, the specific thermal capacity $\kshcap_\press$, the difference in temperature between the fluid and the wall $\Dx\temp$, the flow properties of the fluid, the density $\dens$, dynamic viscosity $\dynvis$, the average fluid velocity $\vel$, the diameter of the pipe $\diam$ and the length of pipe $\length$. The fundamental dimensions that we can classify these quantities into are $\elset{\phdim M, \phdim L, \phdim T, \phdimtemp}$.

There are nine dimensionless groups and four dimensions in \autoref{tab:fluidpipesystem}, apparently necesitating five dimensionless groups. However, a further reduction is possible when \emph{there is no interconversion between thermal energy (being transferred) and mechanical energy (driving the flow)}. In this case, it is possible to consider heat energy as a dimension $\phdim H$, because it is different from mechanical energy. There are now five dimensions, $\elset{\phdim H, \phdim M, \phdim L, \phdim T, \phdimtemp}$, and a total of nine dimensional quantities in \autoref{tab:fluidpipesystem}. Therefore, there are only four dimensionless groups.
%
% ------------------------------------------------------------- PreTable
\docpretable{bt}{0.9\textwidth}{ccl}%
% position: bthH. size: 0.9\textwidth. cols: llcp{6mm}
% use: \docfloatwidth whenever possible!
% NOTE: does not include \toprule
\toprule
Quantity    & Symbol    & Dimension \\
\midrule
Thermal flux        & $\flux\then$        & $\phdim H/\phdim T\phdim L^2$ \\
Pipe diameter       & $\diam$             & $\phdim L$ \\
Pipe length         & $\length$           & $\phdim L$ \\
Average fluid velocity    & $\vel$        & $\phdim L/\phdim T$ \\
Fluid density       & $\dens$             & $\phdim M/\phdim L^3$ \\
Fluid viscosity     & $\dynvis$           & $\phdim M/\phdim L \phdim T$ \\
Fluid specific thermal capacity & $\kshcap_\press$ & $\phdim H/\phdim M\phdimtemp$ \\
Thermal conductivity   & $\kthcond$         & $\phdim H/\phdim L\phdim T\phdimtemp$  \\
Temperature difference & $\Dx\temp$         & $\phdimtemp$ \\
\bottomrule
% ------------------------------------------------------------ PostTable
\end{tabularx}
\docposttable{Thermal transfer in a fluid-pipe system}{Relevant quantities and their dimensions for the thermal energy transfer to a fluid flowing in a pipe.}{tab:fluidpipesystem}
% include: \end{tabularx}%
% ------------------------------------------------------------ EndTable

Of the dimensionless groups, the easiest one to is the ratio $\length/\diam$ of the length and diameter of the pipe. The dimensionless group containing $\flux\then$ is the \lingo{dependent} dimensionless quantity, which contains the dependent quantity, which has to be determined as a function of all the other \lingo{independent} quantities in the problem. For the independent variable $\flux\then$, the relation is of the form
\beq
\flux\then\diam^a\kthcond^b\Dx\temp^c\dynvis^d\dens^e = \kdim_1\,.
\eeq
The above relations can easily be solved to obtain $a = 1$, $b = c = -1$ and $d = e = 0$. Using these, the dimensionless group is
\beq
\kdim_1 = \knusselt = \dfrac{\flux\then\diam}{\kthcond\Dx\temp}\,.
\eeq
The previous dimensionless quantity is known as the \lingo{Nusselt number}. In thermal transfer at a boundary (surface) within a fluid, the Nusselt number is the ratio of convective to conductive thermal transfer across (normal to) the boundary. In this context, convection includes both advection and conduction.

Of the two other dimensionless groups, one can easily be identified as the Reynolds number, $\kreynolds = \dens\vel\diam/\dynvis$, the ratio of inertia and viscosity for this case. The second dimensionless group contains the specific thermal energy capacity, $\kshcap_\press$. Since $\kshcap_\press$ contains both the thermal energy and mass dimensions, the dimensionless quantities has to contain the thermal conductivity, $\kthcond$, as well as the (viscosity $\dynvis$ or density $\dens$). The dimensionless quantity constructed with the specific thermal energy, viscosity and conductivity is the Prandtl number,
\beq
\kprandtl = \dfrac{\kshcap_\press\dynvis}{\kthcond}\,.
\eeq

Therefore, the general expression for the average thermal flux can be written as
\beq
\dfrac{\flux\then\diam}{\kthcond\Dx\temp} = g\vat{
    \dfrac{\length}{\diam},
    \dfrac{\dens\vel\diam}{\dynvis},
    \dfrac{\kshcap_\press\dynvis}{\kthcond}
    }
\eeq
or, more compactly, as
\beq
\knusselt = g\vat{\dfrac{\length}{\diam}, \kreynolds, \kprandtl}\,.
\eeq

Dimensional analysis has certainly simplified the problem, since it is much easier to deal with relationships between four dimensionless quantities, rather than with nine dimensional quantities. However, it is not possible to obtain further simplification using dimensional analysis. There are two possible ways to further simplify the problem. One is to do further analytic calculations that incorporate the details of the thermal and mass transfer processes. The other is to perform experiments and then to obtain empirical correlations between the parameters. In the latter case, it is sufficient to consider the variation in the thermal flux for variations in the dimensionless quantities alone and it is not necessary to examine variations in individual quantities.
\end{solution}


\begin{example}
Consider a stirred tank reactor for heterogeneous catalysis, where the reactants and products are in solution and the catalyst is in the form of solid particles. The dissolved reactant $\ce R$ diffuses to the surface of the suspended catalyst particle, reacts at the surface and the product diffuses back into the fluid. It is necessary to determine the average flux of the reactant to the surface, given the difference in concentration $\Dx\conc = \conc_\infty - \conc_s$, where $\conc_\infty$ is the concentration in the bulk and $\conc_s$ is the surface concentration. There is also relative motion of characteristic velocity $\vel$ between the catalyst particle and the fluid, due to stirring.
\end{example}

\begin{solution}
The different dimensional quantities of relevance are presented in \autoref{tab:masstransfertoparticle}.
%
% ------------------------------------------------------------- PreTable
\docpretable{bt}{0.9\textwidth}{ccll}%
% position: bthH. size: 0.9\textwidth. cols: llcp{6mm}
% use: \docfloatwidth whenever possible!
% NOTE: does not include \toprule
\toprule
Quantity    & Symbol    & Dimension     & Modified dimension\\
\midrule
Mass flux                & $\flux\mass$ & $\phdim M/\phdim L^2\phdim T$ & $\phdim M_s/\phdim L^2\phdim T$ \\
Diffusion coefficient    & $\kdiff$     & $\phdim L^2/\phdim T$         & \\
Concentration difference & $\Dx\conc$   & $\phdim M/\phdim L^3$         & $\phdim M_s/\phdim L^3$ \\
Particle diameter        & $\diam$      & $\phdim L$                    & \\
Particle velocity        & $\vel$       & $\phdim L/\phdim T$           & \\
Fluid density            & $\dens$      & $\phdim M/\phdim L^3$         & \\
Fluid viscosity          & $\dynvis$    & $\phdim M/\phdim L\phdim T$   & \\
\bottomrule
% ------------------------------------------------------------ PostTable
\end{tabularx}
\docposttable{Mass transfer to a particle quantities}{Relevant quantities and their dimensions for the mass transfer to a particle.}{tab:masstransfertoparticle}
% include: \end{tabularx}%
% ------------------------------------------------------------ EndTable

The choice of dimensional quantities requires further discussion. It is clear that the average mass flux depends on the bulk concentration, the diffusion coefficient and the particle diameter. However, it is also necessary to include the fluid density, viscosity and the particle velocity relative to the fluid for the following reason. When the particle moves relative to the fluid, the generated flow pattern alters the distribution of the solute around the particles and thereby modifies the diffusion flux at the particle surface. The flow pattern, in turn, depends on the fluid viscosity, density and the flow velocity, and, therefore, the average mass flux could also depend on these.

In \autoref{tab:masstransfertoparticle}, there are seven dimensional quantities and three dimensions. On this basis, we would expect that there are four dimensionless quantities. However, a further simplification can be made by distinguishing between the mass dimension in the mass transport (the solute mass) and that in the flow dynamics (the mass of the total fluid). The flux and the diffusion coefficient contain the mass of the solute, whereas the mass of the fluid (solute plus solvent) appears in the density and fluid viscosity. If the solute concentration does not affect the fluid density and viscosity, we can make a distinction between the mass dimension for the solute, $\phdim M_s$, from the mass dimension for the fluid, $\phdim M$. In the modified dimensions shown in \autoref{tab:masstransfertoparticle}, the mass flux and the diffusion coefficient depend on the solute mass $\phdim M_s$, while the density and viscosity depend on the fluid mass $\phdim M$. There are now four dimensions, $\elset{\phdim M, \phdim M_s, \phdim L, \phdim T}$, and therefore there are only three dimensionless quantities.

The first dimensionless group can be constructed by non-dimensionalizing the flux by the diffusion coefficient, the concentration difference and the particle diameter:
\beq
\kdim_1 = \flux\mass(\Dx\conc)^a\kdiff^b\diam^c\,,
\eeq
The indices $a$, $b$ and $c$ are determined from dimensional consistency to provide the dimensionless flux, called the Sherwood number,
\beq
\ksherwood = \dfrac{\flux\mass\diam}{\kdiff\Dx\conc}\,.
\eeq

Two other dimensionless groups can be defined. One of the dimensionless numbers is the Reynolds number, $\kreynolds = \dens\vel\diam/\dynvis$, the ratio of fluid inertia and viscosity. The second dimensionless group can be defined in two ways. One possible definition is the Schmidt number, $\kschmidt = \dynvis/\dens\kdiff$, the dimensionless combination of the diffusivity, viscosity and density. The alternate dimensionless group is the Peclet number, $\kpeclet = \vel\diam/\kdiff$, constructed from the flow velocity, diameter and the mass diffusivity. If we use the Reynolds and Schmidt numbers, then the dimensionless flux can be expressed as
\beq
\kdim_1 = \dfrac{\flux\mass\diam}{\kdiff\Dx\conc} 
        = g\vat{
            \dfrac{\dens\vel\diam}{\dynvis},
            \dfrac{\dynvis}{\dens\kdiff}
        }
\eeq
or, equivalently,
\beq
\ksherwood = g\vat{\kreynolds, \kpeclet}\,. \mqed
\eeq
\end{solution}


\begin{example}
In designing new processes, it is not sufficient to study the the process on the laboratory scale, but to also study the exact industrial set up on a small scale before building a larger scale industrial apparatus. Dimensional analysis plays a very important role in industrial scale up. Scaling up cannot be done by multiplying all parameters by a given factor, but the dimensionless groups have to be kept a constant while scaling up. This will be illustrated using the example of a stirred tank reactor. In this, the fluid is stirred using an impeller of a certain shape, and the impeller is to be designed so that optimum mixing is achieved for minimum power.
\end{example}

\begin{solution}
For a given impeller shape, it is necessary to estimate the power consumption for stirring at a given frequency $\freq$. The power consumption will, in general, depend on the shape and dimension of the impeller, as well as the vessel, as well as other details such as baffles, etc. If we keep the relative ratios of the lengths of the impeller, vessel, baffles, etc., a constant, then there is only one length scale in the problem, which we will consider the impeller diameter $\diam$. In addition, the power can also depend on the density of the fluid, $\dens$, the fluid viscosity $\dynvis$ and the frequency of rotation $\freq$. An additional dependence arises on the acceleration due to gravity $\grav$. This is because sometimes during stirring, the interface of the fluid raises at the edges and lowers in the middle. This upward motion due to centrifugal forces is balanced by a downward force due to gravity and so gravity could also be an important factor. This is because it is important to ensure that the dimensions of the interface (curvature, extent of depression at the center) are also in the same proportion as the impeller diameter. When considering the interface, the surface tension (normally denoted by the symbol $\surftens$) is also a relevant parameter. The surface tension has dimensions of force/length of energy/area; \ie, $\phdim M/\phdim T^2$. The relevant parameters with their dimensional dependencies are shown in \autoref{tab:impellerinreactor}.
%
% ------------------------------------------------------------- PreTable
\docpretable{bt}{0.9\textwidth}{ccl}%
% position: bthH. size: 0.9\textwidth. cols: llcp{6mm}
% use: \docfloatwidth whenever possible!
% NOTE: does not include \toprule
\toprule
Quantity    & Symbol    & Dimension \\
\midrule
Power       & $\power$    & $\phdim M\phdim L^2/\phdim T^3$ \\
Frequency   & $\freq$     & $1/\phdim T$ \\
Diameter    & $\diam$     & $\phdim L$ \\
Density     & $\dens$     & $\phdim M/\phdim L^3$ \\
Viscosity   & $\dynvis$   & $\phdim M/\phdim L\phdim T$ \\
Gravity     & $\grav$     & $\phdim L/\phdim T^2$ \\
Surface tension & $\surftens$ & $\phdim M/\phdim T^2$ \\
\bottomrule
% ------------------------------------------------------------ PostTable
\end{tabularx}
\docposttable{Reactor-impeller quantities}{Relevant quantities and their dimensions for the calculation of the power required for the impeller in a reactor.}{tab:impellerinreactor}
% include: \end{tabularx}%
% ------------------------------------------------------------ EndTable

The dimensionless variables derived above have the following physical interpretations. The \lingo{power number}, $\kpower = \power/\freq^3\diam^5\dens$, is the dimensionless quantity that involves the dependent variable, the power, which has to be determined as a function of all the other independent variables. This gives the ratio of the power required to the work done by centrifugal forces. On the other hand, the \lingo{Reynolds number}, $\kreynolds = \dens\diam^2\freq/\dynvis$, is the ratio of centrifugal forces and viscous forces, or the ratio of convection and diffusion. The dimensionless group $\freq^2\diam/\grav$ is the \lingo{Froude number}, which gives the ratio of centrifugal and gravitational forces. The dimensionless group $\dens\freq\diam^3/\surftens$ is the \lingo{Weber number}, which is the ratio of centrifugal and surface tension forces.

Since there are seven variables and three dimensions, it is possible to create four dimensionless groups. Let us assume these four contain $\power$, $\dynvis$, $\grav$ and $\surftens$ expressed in terms of the other variables. Then, the four groups are ($\power/\freq^3\diam^5\dens$), ($\freq\diam^2\dens/\dynvis$), ($\freq^2\diam/\grav$) and ($\dens\freq^2\diam^3/\surftens$). Therefore, the expression for the power has to have the form:
\beq
\dfrac{\power}{\freq^3\diam^5\dens} = g\vat{
                                            \dfrac{\freq\diam^2\dens}{\dynvis},
                                            \dfrac{\freq^2\diam}{\grav},
                                            \dfrac{\dens\freq^2\diam^3}{\surftens}
                                            }\,.
\eeq


It is instructive to determine the order of magnitudes of the different dimensionless quantities in the problem. The density of the liquid is usually of the order of $\si{10^3}{kg/m^3}$, the viscosity of a very viscous fluid such as a polymer melt could be as high as \si{1}{kg/m.s} and the surface tension of a liquid-gas interface is, at maximum, $\si{0.1}{kg/s^2}$. If the frequency is of the order of $\SIrange{1}{10}{rev/s}$, the Froude number $\freq^2\diam/\grav\sim\numrange{0.1}{10}$, indicating that both centrifugal and gravitational forces are important in the present problem. The Weber number $\dens\freq^2\diam^3/\surftens\sim\numrange{104}{106}$, which is large, indicating that the surface tension effects are small when compared to inertial effects. Therefore, the effect of surface tension can be neglected in the present application. The Reynolds number $\dens\diam^2\freq\dynvis\sim\numrange{103}{104}$, which is large. Therefore, it might naively be expected that viscous effects can be neglected in comparison to inertial effects. However, as it can be seen using analysis of boundary layers, diffusion \emph{cannot} be neglected, because it is diffusive transport which is responsible for the transport of mass, momentum and energy at the bounding surfaces of the fluid. With the neglect of surface tension effects, the last relation reduces to
\beq
\dfrac{\power}{\freq^3\diam^5\dens} = g\vat{
                                            \dfrac{\freq\diam^2\dens}{\dynvis},
                                            \dfrac{\freq^2\diam}{\grav}
                                            }\,.
\eeq

In a scale up,
\begin{quote}
the dimensionless numbers have to be kept a constant.
\end{quote}
For example, we are interested in designing a reactor with an impeller diameter of $\si{1}{m}$ with a revolution of $\si{10}{rev/s}$ and the fluid in the reactor is water with density $\si{1000}{kg/m^3}$ and viscosity $\si{10^{-3}}{kg/m.s}$. In order to determine the performance, we design a smaller reactor with an impeller of size $\si{10}{cm}$. What is the fluid that should be used and what is the speed at which the reactor should operate? The speed of rotation can be determined from the consideration that the Froude number has to be a constant. If quantities for the big reactor are denoted with the subscript $b$ and those for the small reactor are denoted by the subscript $s$, then for the Froude number to be a constant, we require
\beq
\freq^2\txt b\diam\txt b = \freq^2\txt s\diam\txt s\,.
\eeq
Substituting the dimensions and the frequency of the big reactor, we get the impeller speed of the small reactor as $\si{316}{rev/s}$. The choice of fluid to be used in the small reactor is determined by the condition that the Reynolds number has to be a constant:
\beq
\dfrac{\dens\txt b\freq\txt b\diam^2\txt b}{\dynvis\txt b} = 
    \dfrac{\dens\txt s\freq\txt s\diam^2\txt s}{\dynvis\txt s}\,.
\eeq
Relating the frequency and diameters of the two reactors, we get
\beq
\dfrac{\dens\txt s}{\dynvis\txt s} = 31.6\dfrac{\dens\txt b}{\dynvis\txt b}\,.
\eeq
Finally since the Reynolds number and the Froude number are kept a constant between the two configurations, the Power number is also a constant
\beq
\dfrac{\power\txt b}{\freq^3\txt b\diam^5\txt b\dens\txt b} =
    \dfrac{\power\txt s}{\freq^3\txt s\diam^5\txt s\dens\txt s}\,.
\eeq
Therefore, the ratio of the power required in the two configurations is
\beq
\dfrac{\power\txt b}{\power\txt s} = 3160\dfrac{\dens\txt b}{\dens\txt s}\,.
\eeq
From the power requirement of the small reactor, the power estimate for the big reactor can be obtained using the above relation.
=======
%There are nine dimensionless groups and four dimensions in table 1.3, ap- parently necesitating five dimensionless groups. However, a further reduction is possible when there is no interconversion between heat energy (which is being transferred) and mechanical energy (which is driving the flow). In this case, it is possible to consider heat energy as a dimension H which is different from mechanical energy. There are now five dimensions, H, M, L, T and Υ, and a total of nine dimensional quantities in table 1.3. Therefore, there are only four dimensionless groups.

\end{solution}
