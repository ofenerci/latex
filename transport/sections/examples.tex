\section{Examples}
Some examples on the use of dimensional analysis and dimensionless quantities.


\begin{example}
A solid body, of definite geometric shape, is fixed in a stream of liquid and maintained at a definite temperature higher than the temperature of the liquid at points remote from the body. It's required to find the rate at which thermal energy is transferred from the body to the liquid.
\end{example}

\begin{solution}
The first step is to identify relevant transport properties and material properties; then, to choose a suitable dimensionally independent set, in the present case $\elset{\phdim E, \phdim, L, \phdim T, \phdimtemp}$; and finally to express all the relevant quantities using the chosen set. The result of this process is shown in \cref{tab:thermaltransferfrombodytofluid}.
%
% ------------------------------------------------------------- PreTable
\docpretable{bt}{0.9\textwidth}{ccl}%
% position: bthH. size: 0.9\textwidth. cols: llcp{6mm}
% use: \docfloatwidth whenever possible!
% NOTE: does not include \toprule
\toprule
Quantity    & Symbol    & Dimension \\
\midrule
Thermal transfer rate        & $\flow\then$    & $\phdim E/\phdim T$ \\
Linear dimension of the body & $\length$       & $\phdim L$ \\
Stream velocity              & $\vel$          & $\phdim L/\phdim T$ \\
Temperature difference       & $\temp$         & $\phdimtemp$ \\
Volumetric thermal capacity of the liquid & $\kshcap$ & $\phdim E/\phdim L^3\phdimtemp$ \\
Liquid thermal conductivity  & $\kthcond$      & $\phdim E/\phdim L\phdim T\phdimtemp$ \\
\bottomrule
% ------------------------------------------------------------ PostTable
\end{tabularx}
\docposttable{Body-fluid thermal transfer}{Relevant quantities and their dimensions for the calculation of the thermal transfer rate from a body to a fluid.}{tab:thermaltransferfrombodytofluid}
% include: \end{tabularx}%
% ------------------------------------------------------------ EndTable

The second step is to find out the number of dimensionless numbers, $n$, using the $\kdim$-theorem. In the present case, $n = 6 - 4 = 2$.

Then, form the first dimensionless quantity using the quantity being sought, $\flow\then$:
\beq
\kdim_1 = \dfrac{\flow\then}{\length\kthcond\temp}\,.
\eeq

By the same token, the second dimensionless quantity is found using the rest of the quantities that model the phenomenon:
\beq
\kdim_2 = \dfrac{\length\kshcap\vel}{\kthcond}\,.
\eeq

Then, the form of the model equation is given by
\beq
\dimfunc\vat{\kdim_1, \kdim_2} = 0 
    \implies \kdim_1 = f\vat{\kdim_2} 
    \implies \dfrac{\flow\then}{\length\kthcond\temp} = \dimfunc\vat{\dfrac{\length\kshcap\vel}{\kthcond}}\,.
\eeq

[seek known dimensionless quantities]

Finally, the model can be written as
\beq
\dfrac{\flow\then}{\length\kthcond\temp} = \dimfunc\vat{\dfrac{\length\kshcap\vel}{\kthcond}}\,.\mqed
\eeq
\end{solution}


\begin{example}
A fluid flows through a pipe heated from the outside. The temperature of the wall of the pipe is higher than the average temperature of the fluid by a constant value $\Dx\temp$. The change in temperature of the fluid is due to the thermal transfer from the walls and not due to frictional heating generated by the flow. One would like to predict the rate of thermal transfer, per unit area of the wall of the pipe, so that the length of pipe required for the heat exchanger can be designed.
\end{example}

\begin{solution}
The first step is to collect the set of variables on which the thermal flux $\flux\then$ may depend. This can depend on the thermal properties of the fluid, the conductivity $\kthcond$, the specific thermal capacity $\kshcap_\press$, the difference in temperature between the fluid and the wall $\Dx\temp$, the flow properties of the fluid, the density $\dens$, dynamic viscosity $\dynvis$, the average fluid velocity $\vel$, the diameter of the pipe $\diam$ and the length of pipe $\length$. The fundamental dimensions that we can classify these quantities into are $\elset{\phdim M, \phdim L, \phdim T, \phdimtemp}$.

There are nine dimensionless groups and four dimensions in \cref{tab:fluidpipesystem}, apparently necessitating five dimensionless groups. However, a further reduction is possible when \emph{there is no interconversion between thermal energy (being transferred) and mechanical energy (driving the flow)}. In this case, it is possible to consider heat energy as a dimension $\phdim H$, because it is different from mechanical energy. There are now five dimensions, $\elset{\phdim H, \phdim M, \phdim L, \phdim T, \phdimtemp}$, and a total of nine dimensional quantities in \cref{tab:fluidpipesystem}. Therefore, there are only four dimensionless groups.
%
% ------------------------------------------------------------- PreTable
\docpretable{bt}{0.9\textwidth}{ccl}%
% position: bthH. size: 0.9\textwidth. cols: llcp{6mm}
% use: \docfloatwidth whenever possible!
% NOTE: does not include \toprule
\toprule
Quantity    & Symbol    & Dimension \\
\midrule
Thermal flux        & $\flux\then$        & $\phdim H/\phdim T\phdim L^2$ \\
Pipe diameter       & $\diam$             & $\phdim L$ \\
Pipe length         & $\length$           & $\phdim L$ \\
Average fluid velocity    & $\vel$        & $\phdim L/\phdim T$ \\
Fluid density       & $\dens$             & $\phdim M/\phdim L^3$ \\
Fluid viscosity     & $\dynvis$           & $\phdim M/\phdim L \phdim T$ \\
Fluid specific thermal capacity & $\kshcap_\press$ & $\phdim H/\phdim M\phdimtemp$ \\
Thermal conductivity   & $\kthcond$         & $\phdim H/\phdim L\phdim T\phdimtemp$  \\
Temperature difference & $\Dx\temp$         & $\phdimtemp$ \\
\bottomrule
% ------------------------------------------------------------ PostTable
\end{tabularx}
\docposttable{Thermal transfer in a fluid-pipe system}{Relevant quantities and their dimensions for the thermal energy transfer to a fluid flowing in a pipe.}{tab:fluidpipesystem}
% include: \end{tabularx}%
% ------------------------------------------------------------ EndTable

Of the dimensionless groups, the easiest one to is the ratio $\length/\diam$ of the length and diameter of the pipe. The dimensionless group containing $\flux\then$ is the \lingo{dependent} dimensionless quantity, which contains the dependent quantity, which has to be determined as a function of all the other \lingo{independent} quantities in the problem. For the independent variable $\flux\then$, the relation is of the form
\beq
\flux\then\diam^a\kthcond^b\Dx\temp^c\dynvis^d\dens^e = \kdim_1\,.
\eeq
The above relations can easily be solved to obtain $a = 1$, $b = c = -1$ and $d = e = 0$. Using these, the dimensionless group is
\beq
\kdim_1 = \knusselt = \dfrac{\flux\then\diam}{\kthcond\Dx\temp}\,.
\eeq
The previous dimensionless quantity is known as the \lingo{Nusselt number}. In thermal transfer at a boundary (surface) within a fluid, the Nusselt number is the ratio of convective to conductive thermal transfer across (normal to) the boundary. In this context, convection includes both advection and conduction.

Of the two other dimensionless groups, one can easily be identified as the Reynolds number, $\kreynolds = \dens\vel\diam/\dynvis$, the ratio of inertia and viscosity for this case. The second dimensionless group contains the specific thermal energy capacity, $\kshcap_\press$. Since $\kshcap_\press$ contains both the thermal energy and mass dimensions, the dimensionless quantities has to contain the thermal conductivity, $\kthcond$, as well as the (viscosity $\dynvis$ or density $\dens$). The dimensionless quantity constructed with the specific thermal energy, viscosity and conductivity is the Prandtl number,
\beq
\kprandtl = \dfrac{\kshcap_\press\dynvis}{\kthcond}\,.
\eeq

Therefore, the general expression for the average thermal flux can be written as
\beq
\dfrac{\flux\then\diam}{\kthcond\Dx\temp} = \dimfunc\vat{
    \dfrac{\length}{\diam},
    \dfrac{\dens\vel\diam}{\dynvis},
    \dfrac{\kshcap_\press\dynvis}{\kthcond}
    }
\eeq
or, more compactly, as
\beq
\knusselt = \dimfunc\vat{\dfrac{\length}{\diam}, \kreynolds, \kprandtl}\,.
\eeq

Dimensional analysis has certainly simplified the problem, since it is much easier to deal with relationships between four dimensionless quantities, rather than with nine dimensional quantities. However, it is not possible to obtain further simplification using dimensional analysis. There are two possible ways to further simplify the problem. One is to do further analytic calculations that incorporate the details of the thermal and mass transfer processes. The other is to perform experiments and then to obtain empirical correlations between the parameters. In the latter case, it is sufficient to consider the variation in the thermal flux for variations in the dimensionless quantities alone and it is not necessary to examine variations in individual quantities.
\end{solution}


\begin{example}
Consider a stirred tank reactor for heterogeneous catalysis, where the reactants and products are in solution and the catalyst is in the form of solid particles. The dissolved reactant $\ce R$ diffuses to the surface of the suspended catalyst particle, reacts at the surface and the product diffuses back into the fluid. It is necessary to determine the average flux of the reactant to the surface, given the difference in concentration $\Dx\conc = \conc_\infty - \conc_s$, where $\conc_\infty$ is the concentration in the bulk and $\conc_s$ is the surface concentration. There is also relative motion of characteristic velocity $\vel$ between the catalyst particle and the fluid, due to stirring.
\end{example}

\begin{solution}
The different dimensional quantities of relevance are presented in \cref{tab:masstransfertoparticle}.
%
% ------------------------------------------------------------- PreTable
\docpretable{bt}{0.9\textwidth}{ccll}%
% position: bthH. size: 0.9\textwidth. cols: llcp{6mm}
% use: \docfloatwidth whenever possible!
% NOTE: does not include \toprule
\toprule
Quantity    & Symbol    & Dimension     & Modified dimension\\
\midrule
Mass flux                & $\flux\mass$ & $\phdim M/\phdim L^2\phdim T$ & $\phdim M_s/\phdim L^2\phdim T$ \\
Diffusion coefficient    & $\kdiff$     & $\phdim L^2/\phdim T$         & \\
Concentration difference & $\Dx\conc$   & $\phdim M/\phdim L^3$         & $\phdim M_s/\phdim L^3$ \\
Particle diameter        & $\diam$      & $\phdim L$                    & \\
Particle velocity        & $\vel$       & $\phdim L/\phdim T$           & \\
Fluid density            & $\dens$      & $\phdim M/\phdim L^3$         & \\
Fluid viscosity          & $\dynvis$    & $\phdim M/\phdim L\phdim T$   & \\
\bottomrule
% ------------------------------------------------------------ PostTable
\end{tabularx}
\docposttable{Mass transfer to a particle quantities}{Relevant quantities and their dimensions for the mass transfer to a particle.}{tab:masstransfertoparticle}
% include: \end{tabularx}%
% ------------------------------------------------------------ EndTable

The choice of dimensional quantities requires further discussion. It is clear that the average mass flux depends on the bulk concentration, the diffusion coefficient and the particle diameter. However, it is also necessary to include the fluid density, viscosity and the particle velocity relative to the fluid for the following reason. When the particle moves relative to the fluid, the generated flow pattern alters the distribution of the solute around the particles and thereby modifies the diffusion flux at the particle surface. The flow pattern, in turn, depends on the fluid viscosity, density and the flow velocity, and, therefore, the average mass flux could also depend on these.

In \cref{tab:masstransfertoparticle}, there are seven dimensional quantities and three dimensions. On this basis, we would expect that there are four dimensionless quantities. However, a further simplification can be made by distinguishing between the mass dimension in the mass transport (the solute mass) and that in the flow dynamics (the mass of the total fluid). The flux and the diffusion coefficient contain the mass of the solute, whereas the mass of the fluid (solute plus solvent) appears in the density and fluid viscosity. If the solute concentration does not affect the fluid density and viscosity, we can make a distinction between the mass dimension for the solute, $\phdim M_s$, from the mass dimension for the fluid, $\phdim M$. In the modified dimensions shown in \cref{tab:masstransfertoparticle}, the mass flux and the diffusion coefficient depend on the solute mass $\phdim M_s$, while the density and viscosity depend on the fluid mass $\phdim M$. There are now four dimensions, $\elset{\phdim M, \phdim M_s, \phdim L, \phdim T}$, and therefore there are only three dimensionless quantities.

The first dimensionless group can be constructed by non-dimensionalizing the flux by the diffusion coefficient, the concentration difference and the particle diameter:
\beq
\kdim_1 = \flux\mass(\Dx\conc)^a\kdiff^b\diam^c\,,
\eeq
The indices $a$, $b$ and $c$ are determined from dimensional consistency to provide the dimensionless flux, called the Sherwood number,
\beq
\ksherwood = \dfrac{\flux\mass\diam}{\kdiff\Dx\conc}\,.
\eeq

Two other dimensionless groups can be defined. One of the dimensionless numbers is the Reynolds number, $\kreynolds = \dens\vel\diam/\dynvis$, the ratio of fluid inertia and viscosity. The second dimensionless group can be defined in two ways. One possible definition is the Schmidt number, $\kschmidt = \dynvis/\dens\kdiff$, the dimensionless combination of the diffusivity, viscosity and density. The alternate dimensionless group is the Peclet number, $\kpeclet = \vel\diam/\kdiff$, constructed from the flow velocity, diameter and the mass diffusivity. If we use the Reynolds and Schmidt numbers, then the dimensionless flux can be expressed as
\beq
\kdim_1 = \dfrac{\flux\mass\diam}{\kdiff\Dx\conc} 
        = \dimfunc\vat{
            \dfrac{\dens\vel\diam}{\dynvis},
            \dfrac{\dynvis}{\dens\kdiff}
        }
\eeq
or, equivalently,
\beq
\ksherwood = \dimfunc\vat{\kreynolds, \kpeclet}\,. \mqed
\eeq
\end{solution}


\begin{example}
In designing new processes, it is not sufficient to study the the process on the laboratory scale, but to also study the exact industrial set up on a small scale before building a larger scale industrial apparatus. Dimensional analysis plays a very important role in industrial scale up. Scaling up cannot be done by multiplying all parameters by a given factor, but the dimensionless groups have to be kept a constant while scaling up. This will be illustrated using the example of a stirred tank reactor. In this, the fluid is stirred using an impeller of a certain shape, and the impeller is to be designed so that optimum mixing is achieved for minimum power.
\end{example}

\begin{solution}
For a given impeller shape, it is necessary to estimate the power consumption for stirring at a given frequency $\freq$. The power consumption will, in general, depend on the shape and dimension of the impeller, as well as the vessel, as well as other details such as baffles, etc. If we keep the relative ratios of the lengths of the impeller, vessel, baffles, etc., a constant, then there is only one length scale in the problem, which we will consider the impeller diameter $\diam$. In addition, the power can also depend on the density of the fluid, $\dens$, the fluid viscosity $\dynvis$ and the frequency of rotation $\freq$. An additional dependence arises on the acceleration due to gravity $\grav$. This is because sometimes during stirring, the interface of the fluid raises at the edges and lowers in the middle. This upward motion due to centrifugal forces is balanced by a downward force due to gravity and so gravity could also be an important factor. This is because it is important to ensure that the dimensions of the interface (curvature, extent of depression at the center) are also in the same proportion as the impeller diameter. When considering the interface, the surface tension (normally denoted by the symbol $\surftens$) is also a relevant parameter. The surface tension has dimensions of force/length of energy/area; \ie, $\phdim M/\phdim T^2$. The relevant parameters with their dimensional dependencies are shown in \cref{tab:impellerinreactor}.
%
% ------------------------------------------------------------- PreTable
\docpretable{bt}{0.9\textwidth}{ccl}%
% position: bthH. size: 0.9\textwidth. cols: llcp{6mm}
% use: \docfloatwidth whenever possible!
% NOTE: does not include \toprule
\toprule
Quantity    & Symbol    & Dimension \\
\midrule
Power       & $\power$    & $\phdim M\phdim L^2/\phdim T^3$ \\
Frequency   & $\freq$     & $1/\phdim T$ \\
Diameter    & $\diam$     & $\phdim L$ \\
Density     & $\dens$     & $\phdim M/\phdim L^3$ \\
Viscosity   & $\dynvis$   & $\phdim M/\phdim L\phdim T$ \\
Gravity     & $\grav$     & $\phdim L/\phdim T^2$ \\
Surface tension & $\surftens$ & $\phdim M/\phdim T^2$ \\
\bottomrule
% ------------------------------------------------------------ PostTable
\end{tabularx}
\docposttable{Reactor-impeller quantities}{Relevant quantities and their dimensions for the calculation of the power required for the impeller in a reactor.}{tab:impellerinreactor}
% include: \end{tabularx}%
% ------------------------------------------------------------ EndTable

The dimensionless variables derived above have the following physical interpretations. The \lingo{power number}, $\kpower = \power/\freq^3\diam^5\dens$, is the dimensionless quantity that involves the dependent variable, the power, which has to be determined as a function of all the other independent variables. This gives the ratio of the power required to the work done by centrifugal forces. On the other hand, the \lingo{Reynolds number}, $\kreynolds = \dens\diam^2\freq/\dynvis$, is the ratio of centrifugal forces and viscous forces, or the ratio of convection and diffusion. The dimensionless group $\freq^2\diam/\grav$ is the \lingo{Froude number}, which gives the ratio of centrifugal and gravitational forces. The dimensionless group $\dens\freq\diam^3/\surftens$ is the \lingo{Weber number}, which is the ratio of centrifugal and surface tension forces.

Since there are seven variables and three dimensions, it is possible to create four dimensionless groups. Let us assume these four contain $\power$, $\dynvis$, $\grav$ and $\surftens$ expressed in terms of the other variables. Then, the four groups are ($\power/\freq^3\diam^5\dens$), ($\freq\diam^2\dens/\dynvis$), ($\freq^2\diam/\grav$) and ($\dens\freq^2\diam^3/\surftens$). Therefore, the expression for the power has to have the form:
\beq
\dfrac{\power}{\freq^3\diam^5\dens} = \dimfunc\vat{
                                                \dfrac{\freq\diam^2\dens}{\dynvis},
                                                \dfrac{\freq^2\diam}{\grav},
                                                \dfrac{\dens\freq^2\diam^3}{\surftens}
                                                }\,.
\eeq


It is instructive to determine the order of magnitudes of the different dimensionless quantities in the problem. The density of the liquid is usually of the order of $\si{10^3}{kg/m^3}$, the viscosity of a very viscous fluid such as a polymer melt could be as high as \si{1}{kg/m.s} and the surface tension of a liquid-gas interface is, at maximum, $\si{0.1}{kg/s^2}$. If the frequency is of the order of $\SIrange{1}{10}{rev/s}$, the Froude number $\freq^2\diam/\grav\sim\numrange{0.1}{10}$, indicating that both centrifugal and gravitational forces are important in the present problem. The Weber number $\dens\freq^2\diam^3/\surftens\sim\numrange{104}{106}$, which is large, indicating that the surface tension effects are small when compared to inertial effects. Therefore, the effect of surface tension can be neglected in the present application. The Reynolds number $\dens\diam^2\freq\dynvis\sim\numrange{103}{104}$, which is large. Therefore, it might naively be expected that viscous effects can be neglected in comparison to inertial effects. However, as it can be seen using analysis of boundary layers, diffusion \emph{cannot} be neglected, because it is diffusive transport which is responsible for the transport of mass, momentum and energy at the bounding surfaces of the fluid. With the neglect of surface tension effects, the last relation reduces to
\beq
\dfrac{\power}{\freq^3\diam^5\dens} = \dimfunc\vat{
                                                \dfrac{\freq\diam^2\dens}{\dynvis},
                                                \dfrac{\freq^2\diam}{\grav}
                                                }\,.
\eeq

In a scale up,
\begin{quote}
the dimensionless numbers have to be kept a constant.
\end{quote}
For example, we are interested in designing a reactor with an impeller diameter of $\si{1}{m}$ with a revolution of $\si{10}{rev/s}$ and the fluid in the reactor is water with density $\si{1000}{kg/m^3}$ and viscosity $\si{10^{-3}}{kg/m.s}$. In order to determine the performance, we design a smaller reactor with an impeller of size $\si{10}{cm}$. What is the fluid that should be used and what is the speed at which the reactor should operate? The speed of rotation can be determined from the consideration that the Froude number has to be a constant. If quantities for the big reactor are denoted with the subscript $b$ and those for the small reactor are denoted by the subscript $s$, then for the Froude number to be a constant, we require
\beq
\freq^2\txt b\diam\txt b = \freq^2\txt s\diam\txt s\,.
\eeq
Substituting the dimensions and the frequency of the big reactor, we get the impeller speed of the small reactor as $\si{316}{rev/s}$. The choice of fluid to be used in the small reactor is determined by the condition that the Reynolds number has to be a constant:
\beq
\dfrac{\dens\txt b\freq\txt b\diam^2\txt b}{\dynvis\txt b} = 
    \dfrac{\dens\txt s\freq\txt s\diam^2\txt s}{\dynvis\txt s}\,.
\eeq
Relating the frequency and diameters of the two reactors, we get
\beq
\dfrac{\dens\txt s}{\dynvis\txt s} = 31.6\dfrac{\dens\txt b}{\dynvis\txt b}\,.
\eeq
Finally since the Reynolds number and the Froude number are kept a constant between the two configurations, the Power number is also a constant
\beq
\dfrac{\power\txt b}{\freq^3\txt b\diam^5\txt b\dens\txt b} =
    \dfrac{\power\txt s}{\freq^3\txt s\diam^5\txt s\dens\txt s}\,.
\eeq
Therefore, the ratio of the power required in the two configurations is
\beq
\dfrac{\power\txt b}{\power\txt s} = 3160\dfrac{\dens\txt b}{\dens\txt s}\,.
\eeq
From the power requirement of the small reactor, the power estimate for the big reactor can be obtained using the above relation.
\end{solution}


\begin{example}
Consider an elastic pendulum made by attaching a massless spring of elastic constant $\kspring$ a box of volume $\vol$ filled with a liquid of density $\dens$. The mass of the liquid is acted upon by gravity. We are required to find an expression for the time of oscillation.
\end{example}

\begin{solution}
A list of the quantities and their dimensions in the $\elset{\phdim F, \phdim L, \phdim T}$ system is presented in \cref{tab:boxspringsystem}.
%
% ------------------------------------------------------------- PreTable
\docpretable{bt}{0.9\textwidth}{ccl}%
% position: bthH. size: 0.9\textwidth. cols: llcp{6mm}
% use: \docfloatwidth whenever possible!
% NOTE: does not include \toprule
\toprule
Quantity    & Symbol    & Dimension \\
\midrule
Elastic constant    & $\kspring$    & $\phdim F/\phdim L$ \\
Oscillation time    & $t$           & $\phdim T$ \\
Volume of box       & $\vol$        & $\phdim L^3$ \\
Density of liquid   & $\dens$       & $\phdim F\phdim T^2/\phdim L^4$ \\
Acceleration of gravity & $\grav$   & $\phdim L/\phdim T^2$ \\
\bottomrule
% ------------------------------------------------------------ PostTable
\end{tabularx}
\docposttable{Mass-Box quantities}{Relevant quantities and their dimensions for the calculation of the time of oscillation of a box attached to a spring.}{tab:boxspringsystem}
% include: \end{tabularx}%
% ------------------------------------------------------------ EndTable

In \cref{tab:boxspringsystem}, it can be seen that $n = 5 - 3 = 2$, thus $\kdim$ quantities must be constructed. However, one may notice that the volume of the box times the density of the liquid provides the mass of the liquid. This observation reduces the number of quantities from 2 to 1, where the mass of the system has the dimensions $\dim\mass = \phdim F\phdim T^2/\phdim L$. With this new set, one can write the only dimensionless quantity modeling the phenomenon
\beq
\kdim = \dfrac{t}{\sqrt{\mass/\kspring}}\,.
\eeq

Using this quantity, the model equation becomes
\beq
\dimfunc\vat{\kdim} = 0\,,
\eeq
where $\kdim$ is the zero of the function $\dimfunc$. As function of the original quantities, density and volume, the model becomes
\beq
t = \kdim\sqrt{\dfrac{\vol\dens}{\kspring}}\,,
\eeq
where the value of the dimensionless quantity, $\kdim$, must be experimentally determined.
\end{solution}


\begin{example}
Let's apply the principles of dimensional analysis to a problem of kinetic theory of gases: to find the pressure exerted by a perfect gas. 
\end{example}

\begin{solution}
The atoms of the gas in kinetic theory are considered as perfect spheres, completely elastic, and of negligible dimension compared with their distance apart. The only constant with dimensions required in determining the behavior of each atom is therefore its mass. The behavior of the aggregate of atoms as also characterized by the density of the gas or the number of atoms per unit volume. The problem domain is mechanics and the pressure exerted by the gas is to be found by computing the change of momentum per unit time and per unit area of the atoms striking the walls of the enclosure. The mechanics system of units is thus indicated. But, in addition to the ordinary mechanical features, there is the element of temperature to be considered. How does temperature enter in writing down the equations of motion of the system? The answer is through the gas constant, which gives the average kinetic energy of each atom as a function of the temperature. The analysis of the problem is summarized in \cref{tab:pressurekintheoryofgases}.
%
% ------------------------------------------------------------- PreTable
\docpretable{bt}{0.9\textwidth}{ccl}%
% position: bthH. size: 0.9\textwidth. cols: llcp{6mm}
% use: \docfloatwidth whenever possible!
% NOTE: does not include \toprule
\toprule
Quantity    & Symbol    & Dimension \\
\midrule
Pressure exerted by gas    & $\press$    & $\phdim M/\phdim L\phdim T$ \\
Mass of the atom           & $\mass$     & $\phdim M$ \\
Number of atoms per unit volume    & $\npart$    & $1/\phdim L^3$ \\
Absolute temperature       & $\temp$     & $\phdimtemp$ \\
Gas constant per atom      & $\kgas$     & $\phdim M\phdim L^2/\phdim T^2\phdimtemp$ \\
\bottomrule
% ------------------------------------------------------------ PostTable
\end{tabularx}
\docposttable{Gas pressure quantities}{Relevant quantities and their dimensions for the calculation of the pressure exerted by an ideal gas in a box.}{tab:pressurekintheoryofgases}
% include: \end{tabularx}%
% ------------------------------------------------------------ EndTable

There are $n = 5 - 4 = 1$ dimensionless quantity to construct:
\beq
\kdim = \dfrac{\press}{\npart\kgas\temp}\,.
\eeq
Note that, in the model, the pressure exerted by the gas is independent on the mass of the individual atoms.
\end{solution}
