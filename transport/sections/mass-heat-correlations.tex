\section{Mass and Thermal Transfer Correlations}

\subsection{Dimensionless fluxes}
In mass and thermal transfer, non-dimensional fluxes are obtained by scaling the average flux by a factor
\beq
\dfrac{\text{diffusion coefficient}\times\text{(concentration or temperature difference)}}
      {\text{characteristic length}}\,.
\eeq

The dimensionless group for mass transfer is the \lingo{Sherwood number}
\beq
\ksherwood = \dfrac{\flux\mass}{\kdiff\Dx\conc/\length}\,.
\eeq
The ratio of the flux and the concentration difference, $\flux\mass/\Dx\conc$, is called the \lingo{mass transfer coefficient}, $k_m$. Therefore, the Sherwood number is also written as
\beq
\ksherwood = \dfrac{k_m\length}{\kdiff}\,.
\eeq

Analogous to the Sherwood number, the \lingo{Nusselt number} is a dimensionless thermal flux
\beq
\knusselt = \dfrac{\flux\then}{\kdiff_h\Dx\then/\length}\,,
\eeq
where $\kdiff_h$ is the thermal diffusivity and $\Dx\then$ is the difference in the energy density for a characteristic length $\length$. Since $\kdiff_h = \kthcond/\dens\kshcap_\press$ and $\then = \dens\kshcap_\press\temp$, the Nusselt number becomes
\beq
\knusselt = \dfrac{\flux\then}{\kthcond\Dx\temp/\length}\,.
\eeq
The ratio $\flux\then/\Dx\temp$ is also referred to as the \lingo{heat transfer coefficient}, $h$, and the Nusselt number is then written as
\beq
\knusselt = \dfrac{h\length}{\kthcond}\,.
\eeq

It is important to note that thermal and mass transfer coefficients, $h$ and $k_m$, are \emph{not} material properties. They do depend on the system geometry and flow conditions and can be calculated only after solving the specific transport problem. Therefore, they are derived quantities, on par with the Nusselt and Sherwood numbers. The calculation or experimentation required to obtain these is identical to that for the Nusselt and Sherwood numbers. Therefore, in this course, we shall work in terms of the Nusselt and Sherwood numbers alone and will avoid the confusion generated by introducing the thermal and mass transfer coefficients. In case a the thermal and mass transfer coefficients are necessary, they can be calculated from the Nusselt and Sherwood numbers from their definitions.

The length and velocity in dimensionless numbers for commonly encountered configurations are defined as follows.
\begin{enumerate}
\item For the flow past suspended particles, the velocity $\vel$ is the difference between the particle velocity and the free-stream fluid velocity and the length is usually the particle diameter $\diam$ (for spherical particles) or characteristic dimension.

\item For the flow past a flat plate, the length is the total length of the plate, while the velocity is the constant free-stream velocity far from the plate. The average fluxes are defined per unit surface area of the plate.

\item For the flow in tubes and channels, the length is the tube diameter or channel width, while the velocity is the average flow velocity through the tube or channel. The average fluxes are defined per unit surface area of the tube or channel.
\end{enumerate}


\subsection{Correlations}


