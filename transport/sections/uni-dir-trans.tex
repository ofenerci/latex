\section{Unidirectional transport}
In this chapter, we consider transport in which there is a variation in the mass, momentum and temperature fields in only one dimension. The analysis is considerably simplified in this case, since there is only one spatial coordinate to be considered. However, the examples solved here illustrate the basic principles of the solution of more complex problems in multiple dimensions, which involve shell balances to derive differential equations for the concentration, velocity and temperature fields, and then an integration procedure for determining the variations in the concentration, velocity and temperature.


\subsection{Solutions to the diffusion equation}

\subsubsection{Unsteady transport into an infinite fluid}

\paragraph{Mass transfer} 
Consider a flat surface in the $xy$ plane located at $z = 0$ through which a solvent diffuses into the fluid. The plane surface is assumed to be of infinite extent in the $xy$ plane, and the height of the fluid supported by the plane is also considered to be of infinite extent. Initially, the concentration of the solute in the solvent at the surface~\footnote{~$\conc_\infty$ does not mean infinite concentration, but it refers to the concentration at a very long distance. This is analogous to the case of a hot body being cooled by a fluid stream. The body temperature can be denoted by $\temp\txt b$ and the fluid temperature constant at a long distance can be denoted by $\temp_\infty$.} is $\conc_\infty$. At time $t = 0$, the solute concentration at the surface is instantaneously increased to $\conc_0$, and there is diffusion of the solute into the solvent. We would like to determine the variation of the concentration of the solute with time.

As the diffusion proceeds, there is transport of solute from the surface to the solvent, resulting in an increase in the concentration near the surface. However, the concentration far from the surface $z\to\infty$ remains unchanged at $\conc = \conc_\infty$. The conditions for the concentration field at the spatial boundaries and at the initial time are
\beq
\conc = \begin{cases*}
            \conc_\infty    & for all $t$ as $z\to\infty$;\\
            \conc_0         & for all $t > 0$ at $z = 0$;\\
            \conc_\infty    & for all $z> 0$ at $t = 0$\,.
        \end{cases*}
\eeq
The solution is simplified by non-dimensionalizing (scaling) the concentration:
\beq
\scpq\conc = \dfrac{\conc - \conc_\infty}{\conc_0 - \conc_\infty}\,.
\eeq
Note that, as in thermal transport, the driving force is not bare $\conc$, but a change thereof, $\Dx\conc$. Using such a non-dim, the conditions for $\scpq\conc$ become
\beq
\scpq\conc = \begin{cases*}
                0    & for all $t$ as $z\to\infty$;\\
                1    & for all $t > 0$ at $z = 0$;\\
                0    & for all $z> 0$ at $t = 0$\,.
             \end{cases*}
\eeq


\paragraph{Thermal transfer} 
The configuration consists of a flat surface in the $xy$ plane of infinite extent and a fluid of thermal conductivity $\kthcond$ and specific thermal capacity $\kshcap_\press$ in the half space $z > 0$. Initially, the temperature of the fluid and the surface is $\temp_\infty$. At time $t = 0$, the temperature of the surface is instantaneously increased to $\temp_0$. Determine the temperature as a function of time.

As conduction proceeds, there is a transport of energy from the surface to the fluid, resulting in an increase in the temperature of the fluid. However, the temperature at a large distance from the surface, $z\to\infty$, remains at $\temp = \temp_\infty$. Therefore, the conditions for the temperature at the spatial boundaries of the fluid and at initial time are
\beq
\temp = \begin{cases*}
            \temp_\infty    & for all $t$ as $z\to\infty$;\\
            \temp_0         & for all $t > 0$ at $z = 0$;\\
            \temp_\infty    & for all $z> 0$ at $t = 0$\,.
        \end{cases*}
\eeq
The solution is simplified by non-dimensionalizing (scaling) the temperature:
\beq
\scpq\temp = \dfrac{\temp - \temp_\infty}{\temp_0 - \temp_\infty}\,.
\eeq
The conditions for $\scpq\conc$ become
\beq
\scpq\temp = \begin{cases*}
                0    & for all $t$ as $z\to\infty$;\\
                1    & for all $t > 0$ at $z = 0$;\\
                0    & for all $z> 0$ at $t = 0$\,.
             \end{cases*}
\eeq

\paragraph{Momentum transfer}
The configuration consists of an infinite fluid in the $z > 0$ half space bounded by an infinite flat surface in the $xz$ plane. The fluid and the surface are initially at rest. At time $t = 0$, the plane is instantaneously moved with a constant velocity $\vel$ in the $x$ direction. Determine the fluid velocity as a function of time.

As time proceeds, the momentum that is transported from the surface diffuses through the fluid, resulting in fluid motion. However, the fluid at a large distance from the surface $z\to\infty$ remains at rest. If $\vel_x$ is the fluid velocity in the $x$ direction, it is convenient to define a non-dimensional fluid velocity $\scpq\vel_x = \vel_x/\vel$. The conditions for the non-dimensional fluid velocity at the spatial boundaries of the flow and at initial time are
\beq
\scpq\vel_x = \begin{cases*}
                0    & for all $t$ as $z\to\infty$;\\
                1    & for all $t > 0$ at $z = 0$;\\
                0    & for all $z> 0$ at $t = 0$\,.
             \end{cases*}
\eeq


\paragraph{Shell balance}
In all three cases, mass, thermal and momentum transfer, the boundary and initial conditions are identical in form. Since the transport in all three cases involves diffusion in the direction perpendicular to the surface, and no variation along the surface, it is anticipated that the differential equation governing the transport will also be identical in form. The differential equation for the concentration field is first derived using a shell balance, and the analogous equations for heat and momentum transfer are provided.

Consider a non-moving shell of thickness $\Dx z$ in the $z$ coordinate and of thickness $\Dx x$ and $\Dx y$ in the $xy$ plane. There is a transport of mass across the surfaces of the shell due to diffusion, which results in a change in the concentration in the shell. We consider the variation in the concentration within this control volume over a time interval $\Dx t$. Mass conservation requires that
\beq
\begin{pmatrix}
\text{accumulation of mass} \\
\text{in the shell}
\end{pmatrix}
= 
\begin{pmatrix}
\text{input of mass} \\
\text{into the shell}
\end{pmatrix}
-
\begin{pmatrix}
\text{output of mass} \\
\text{from the shell}
\end{pmatrix}
\eeq

The accumulation of mass in a time $\Dx t$ is given by
\beq
\begin{pmatrix}
\text{accumulation of mass} \\
\text{in the shell}
\end{pmatrix}
= 
\left(\conc\vat{\pos, t + \Dx t} - \conc\vat{\pos, t}\right)\Dx x\Dx y\Dx z\,.
\eeq

The mass flux at the surface at $z$ is given by
\beq
\flux\mass_z = -\kdiff\xpd\conc z\biggr\rvert_z
\eeq
and the mass entering the shell through the surface at $z$ in a time interval $\Dx t$ is given by the product of the mass flux, the area of transfer and the time interval $\Dx t$:
\beq
\begin{pmatrix}
\text{input of mass} \\
\text{into the shell}
\end{pmatrix}
=
- \Dx t\Dx x\Dx y\left(\kdiff\xpd\conc z\biggr\rvert_z\right)\,.
\eeq

In a similar manner, the mass leaving the surface at $z + \Dx z$ is given by
\beq
\begin{pmatrix}
\text{output of mass} \\
\text{from the shell}
\end{pmatrix}
=
- \Dx t\Dx x\Dx y\left(\kdiff\xpd\conc z\biggr\rvert_{z + \Dx z}\right)\,.
\eeq
Substituting the last equations in the mass balance equation and dividing by $\Dx t\Dx x\Dx y\Dx z$, we have
\beq
\dfrac{\conc\vat{\pos, t + \Dx t} - \conc\vat{\pos, t}}{\Dx t}
=
\dfrac{1}{\Dx z}
\left(
    \kdiff\xpd\conc z\biggr\rvert_{z + \Dx z} - 
    \kdiff\xpd\conc z\biggr\rvert_{z}
\right)\,.
\eeq
The limits $\Dx t\to 0$ and $\Dx z\to 0$ are taken to obtain a partial differential equation for the concentration field:
\beq
\xpd\conc t = \xpd{}{z}\left(\kdiff\xpd\conc z\right)\,.
\eeq
If the diffusion coefficient is independent of the spatial position, then the differential equation reduces to
\beq
\xpd\conc t = \kdiff\nxpd\conc z 2
\implies
\ipd t\conc = \kdiff\ipd{zz}\conc\,.
\eeq
The last equation can be scaled by using $\scpq\conc$:
\beq
\ipd t\scpq\conc = \kdiff\ipd{zz}\scpq\conc\,.
\eeq

A similar shell balance procedure can be carried out for thermal transfer. The equation for the temperature field is
\beq
\xpd{\scpq\temp} t = \xpd{}{z}\left(\kdiff\xpd{\scpq\temp}z\right)
\implies
\ipd t\scpq\temp = \ipd z\left(\ipd z\scpq\temp\right)\,.
\eeq

Though the final result for the momentum transfer process is exactly analogous to those of mass and thermal transport, the procedure is slightly different. We provide a brief outline of the calculation. First, note that there are now two directions in the problem. Since momentum is a vector, there is a direction associated with the momentum itself, which is the $x$ direction. The second is the direction of diffusion, which is the $z$ direction. The fundamental balance relation is
\beq
\begin{pmatrix}
\text{rate of change of}\\
\text{$x$ momentum} \\
\text{in the shell}
\end{pmatrix}
=
\begin{pmatrix}
\text{sum of forces} \\
\text{in the $x$ direction}
\end{pmatrix}\,.
\eeq
The rate of change of momentum in the differential volume of thickness $\Dx z$ about $z$ in a time interval $\Dx t$ is given by
\beq
\begin{pmatrix}
\text{rate of change of}\\
\text{$x$ momentum} \\
\text{in the shell}
\end{pmatrix}
=
\dfrac{\dens\Dx\vel_x\Dx x\Dx y\Dx z}{\Dx t}\,.
\eeq
The forces acting on the two surfaces at $z$ and $z + \Dx z$ are the products of the shear stress $\shear_{xz}$ and the surface area $\Dx x\Dx y$. It is important to keep track of the directions of the forces in this case, since the force is a vector. The shear stress $\shear_{xz}$ is defined as the force per unit area in the $x$ direction acting at a surface whose outward unit normal is in the positive $z$ direction. For the surface at $z + \Dx z$, the outward unit normal is in the $+z$ direction and, therefore, the force per unit area at this surface is $+\shear_{xz}|_{z + \Dx z}$. For the surface at $z$, the outward unit normal is in the $-z$ direction, and therefore the force per unit area at this surface is $-\shear_{xz}|_{z}$. Finally, the rate of change of momentum, which is the change in momentum per unit time, is given by $(\dens\Dx\vel_x)(\surf\Dx z)/\Dx t$. Therefore, the momentum balance equation is
\beq
\surf\Dx z\dfrac{\dens\Dx\vel_x}{\Dx t} =
\surf\left(\shear_{xz}\biggr\rvert_{z + \Dx z} - \shear_{xz}\biggr\rvert_{z}\right)\,.
\eeq
Dividing through by $\surf\Dx z$, we obtain
\beq
\dens\dfrac{\Dx\vel_x}{\Dx t} =
\dfrac{\shear_{xz}\biggr\rvert_{z + \Dx z} - \shear_{xz}\biggr\rvert_{z}}{\Dx z}\,.
\eeq
Taking the limit $\Dx t\to 0$ and $\Dx z\to 0$, we obtain the partial differential equation
\beq
\dens\ipd t\vel_x = \ipd z\shear_{xz}\,.
\eeq
The shear stress is given by the product of the viscosity and the gradient of the velocity
\beq
\shear_{xz} = \dynvis\lim_{\Dx z\to 0}\dfrac{\Dx\vel_x}{\Dx z} 
            = \dynvis\xpd{\vel_x}{z}
\implies
\shear_{xz} = \dynvis\ipd z\vel_x\,.
\eeq
With this, the governing equation for the velocity field becomes
\beq
\dens\ipd t\vel_x = \ipd z\dynvis\ipd z\vel_x\,.
\eeq
Note that the differential equation derived above has the same form as the concentration and energy diffusion equations, although it was derived from a force balance. This indicates that the diffusion process is the same for mass, momentum and energy. However, it should be noted that momentum could be transmitted by pressure forces in addition to viscous forces, and there is no analogue of pressure in mass and energy transport.

When the velocity $\vel_x$ is scaled by $\vel$, the velocity of the bottom surface, the momentum equation becomes
\beq
\ipd t\scpq\vel_x = \ipd z\kinvis\ipd z\scpq\vel_x\,,
\eeq
where $\kinvis = \dynvis/\dens$ is the momentum diffusivity (kinematic viscosity).


\paragraph{Solution}
The conservation equations for mass, thermal energy and momentum are summarized as
\begin{align*}
\ipd t\scpq\conc  &= \kdiff\ipd{zz}\scpq\conc\,,\\
\ipd t\scpq\temp  &= \kdiff\ipd{zz}\scpq\temp\,,\\
\ipd t\scpq\vel_x &= \kinvis\ipd{zz}\scpq\vel_x\,,
\end{align*}
with boundary conditions:
\beq
\scpq\conc = \scpq\temp = \scpq\vel_x =
\begin{cases*}
    0    & for all $t$ as $z\to\infty$;\\
    1    & for all $t > 0$ at $z = 0$;\\
    0    & for all $z> 0$ at $t = 0$\,.
\end{cases*}
\eeq
Since such conservation equations are identical in form and the boundary conditions are also identical, then the same solution procedure can be used for all these and the solutions for the concentration, velocity and temperature fields, expressed in terms of the dimensionless variables $\scpq\conc$, $\scpq\temp$ and $\scpq\vel_x$, turn out to be identical.

In order to solve the concentration equation with its boundary conditions, it is first important to realize that there no intrinsic length scale in the problem, because the fluid and the flat plate are of infinite extent. Since the concentration $\scpq\conc$ is dimensionless, there are only three dimensional variables $z$, $t$ and $\kdiff$ in the problem. These contain two dimensions, $\elset{\phdim L, \phdim T}$, and it is possible to construct only one dimensionless number, $\kdim = z/\kdiff t$. Therefore, just from dimensional analysis, it can be concluded that the concentration field does not vary independently with $z$ and $t$, but only on the combination $\kdim$. If this inference is correct, it should be possible to express the conservation equation in terms of the variable $\kdim$ alone. When $z$ and $t$ are expressed in terms of $\kdim$, the concentration equation becomes
\beq
-\left( \dfrac{z}{2\kdiff^{1/2}t^{3/2}} \right)\xpd{\scpq\conc}{\kdim} = 
\xod{}{t}\nxpd{\scpq\conc}{\kdim}{2}\,.
\eeq
After multiplying throughout by $t$, the equation for the concentration field reduces to
\beq
\dfrac{\kdim}{2}\xpd{\scpq\conc}{\kdim} + \nxpd{\scpq\conc}{\kdim}{2} = 0
\implies
\kdim\ipd\kdim\scpq\conc + 2\ipd{\kdim\kdim}\scpq\conc = 0\,.
\eeq
The last equation validates the inference that the non-dimensionalized concentration field is only a function of $\kdim$. It is also necessary to transform the boundary conditions into conditions for the $\kdim$ coordinate. The transformed boundary conditions are
\beq
\scpq\conc =
\begin{cases*}
    0    & as $\kdim\to\infty$;\\
    1    & at $\kdim = 0$;\\
    0    & as $\kdim\to\infty$\,.
\end{cases*}
\eeq
It is useful to note that the original conservation equation is a second order differential equation in $z$ and a first order differential equation in $t$, so it requires two boundary conditions in the $z$ coordinate and one initial condition. The conservation equation expressed in terms of $\kdim$ is a second order differential equation, which requires just two boundary conditions for $\kdim$. The last equation shows that one of the boundary conditions for $z\to\infty$ and the initial condition $t = 0$ turn out to be identical conditions for $\kdim\to\infty$ and, therefore, the transformation form $\tuple{z,t}$ produces no inconsistency in the boundary and initial conditions.

The equation $\kdim\ipd\kdim\scpq\conc + 2\ipd{\kdim\kdim}\scpq\conc = 0$ can be solved to obtain
\beq
\scpq\conc\vat\kdim = \psi_1 + \psi_2\int_{\kdim}^{\infty}\dx\kdim'\,\exp\vat{-\dfrac{\kdim^2}{2}}\,.
\eeq
The constants $\psi_1$ and $\psi_2$ are determined from the conditions $\scpq\conc = 1$ at $\kdim = 0$ and $\scpq\conc = 0$ for $\kdim = \infty$, to obtain
\beq
\scpq\conc\vat\kdim = \sqrt{\dfrac{2}{\pi}}\int_{\kdim}^{\infty}\dx\kdim'\,\exp\vat{-\dfrac{\kdim^2}{2}}\,,
\eeq
where $\pi$ is the geometrical constant.

From the solution, the last equation, it can be inferred that there is no intrinsic length scale in the system, but the length scale for the $z$ coordinate is a function of time and proportional to $\sqrt{\kdiff t}$ at time $t$. Thus, the length scale over which diffusion has taken place increases proportional to $t^{1/2}$. Thus, this equation will be a solution for the concentration diffusion in a configuration bounded by two plates, so long as the distance between the two plates, say $\length$, is large compared to $\sqrt{\kdiff t}$.


\subsubsection{Steady diffusion into a falling film}
This problem is a simplification of the actual diffusion in a falling film, which involves a combination of convection and diffusion. A thin film of fluid flows down a vertical surface. One side of the film is in contact with a gas which is soluble in the liquid, and as the liquid flows down, the gas is dissolved in the liquid. The concentration of gas in the liquid at the entrance is $\conc_\infty$, while the concentration of gas at the liquid-gas interface is $\conc_0$. The difference in concentration between the initial concentration in the liquid and the concentration at the interface acts as a driving force for diffusion. The $z$ coordinate is perpendicular to the gas-liquid interface, which is located at $z = 0$. We also assume that the penetration depth for the gas into the liquid (to be determined a little later) is small compared to the thickness of the liquid film, so that the boundary conditions in the $z$ coordinate given by
\beq
\scpq\conc\vat\kdim = \sqrt{\dfrac{2}{\pi}}\int_{\kdim}^{\infty}\dx\kdim'\,\exp\vat{-\dfrac{\kdim^2}{2}}
\eeq
are applicable. In addition, the boundary condition in the $x$ coordinate is $\conc = \conc_\infty$ at $x = 0$.


\subsubsection{Diffusion in bounded channels}
Next we consider the problem of diffusion in a channel bounded by two walls separated by a distance $\length$. Initially, the concentration of the fluid in the channel is equal to $\conc_\infty$. At initial time, the concentration of the solute on the wall at $z = 0$ is instantaneously increased to $\conc_0$, while the concentration on the surface at $z = \length$ remains equal to $\conc_\infty$. The problem is to find the concentration field as a function of time. Similar problems can be formulated for thermal energy and momentum transfer.

The concentration field is first expressed in terms of the scaled concentration field $\scpq\conc$ by 
\beq
\scpq\conc = \dfrac{\conc - \conc_\infty}{\conc_0 - \conc_\infty}\,.
\eeq
The diffusion equation, obtained by a shell balance as before, is given by 
\beq
\ipd t\scpq\conc = \kdiff\ipd{zz}\scpq\conc\,.
\eeq
However, there is a modification in the boundary conditions:
\beq
\scpq\conc =
\begin{cases*}
    0    & for all $t$ at $z = \length$;\\
    1    & for all $t > 0$ at $z = 0$;\\
    0    & for all $z> 0$ at $t = 0$\,.
\end{cases*}
\eeq
In this case, it is not possible to effect a reduction to a similarity form, because there is an additional length scale $\length$ in the problem and so the $z$ coordinate can be scaled by $\length$. A scaled $z$ coordinate is defined as $\scpq z = z/\length$ and the diffusion equation in terms of this coordinate is
\beq
\xpd{\scpq\conc} t = \dfrac{\kdiff}{\length^2}\nxpd{\scpq\conc}{\scpq z}{2}\,.
\eeq
The above equation suggests that it is appropriate to define a scaled time coordinate $\scpq t = \kdiff t/\length^2$ and the conservation equation in terms of this scaled time coordinate is
\beq
\xpd{\scpq\conc}{\scpq t} = \nxpd{\scpq\conc}{\scpq z}{2}
\implies
\ipd{\scpq t}{\scpq\conc} = \ipd{\scpq z\scpq z}{\scpq\conc}\,.
\eeq
